% Dissenting
% Breyer

\setcounter{page}{243}

  \textsc{Justice Breyer,} with whom \textsc{Justice Stevens} joins, dissenting.

  I agree with \textsc{Justice Kennedy} that context makes clear that Congress intended the phrase ``any other law enforcement officer'' to apply only to officers carrying out customs or excise duties. See 28 U.~S.~C. \S~2680(c). But I write separately to emphasize, as \textsc{Justice Kennedy}'s dissent itself makes clear, that the relevant context extends well beyond Latin canons and other such purely textual devices.

  As with many questions of statutory interpretation, the issue here is not the \emph{meaning} of the words. The dictionary meaning of each word is well known. Rather, the issue is the statute's \emph{scope.} What boundaries did Congress intend to set? To what circumstances did Congress intend the phrase, as used in \emph{this} statutory provision, to apply? The majority answers this question by referring to an amendment that creates an exception for certain forfeitures and by emphasizing the statutory word ``any.'' As to the amendment, I find \textsc{Justice Kennedy}'s counterargument convincing. See \emph{ante,} at 239--241 (dissenting opinion). And, in my view, the word ``any'' provides no help whatsoever.

  The word ``any'' is of no help because all speakers (including writers and legislators) who use general words such as ``all,'' ``any,'' ``never,'' and ``none'' normally rely upon context to indicate the limits of time and place within which they intend those words to do their linguistic work. And with the possible exception of the assertion of a universal truth, say, by a mathematician, scientist, philosopher, or theologian, such limits almost always exist. When I call out to my wife, \newpage  ``There isn't any butter,'' I do not mean, ``There isn't any butter in town.'' The context makes clear to her that I am talking about the contents of our refrigerator. That is to say, it is context, not a dictionary, that sets the boundaries of time, place, and circumstance within which words such as ``any'' will apply. See \emph{United States} v. \emph{Palmer,} 3 Wheat. 610, 631 (1818) (Marshall, C. J.) (``[G]eneral words,'' such as the word ``‘any','' must ``be limited'' in their application ``to those objects to which the legislature intended to apply them''); \emph{Small} v. \emph{United States,} 544 U.~S. 385, 388 (2005) (``The word ‘any' considered alone cannot answer'' the question ``whether the statutory reference ‘convicted in \emph{any} court' includes a conviction entered in a \emph{foreign} court''); \emph{Nixon} v. \emph{Missouri Municipal League,} 541 U.~S. 125, 132 (2004) (`` ‘[A]ny' '' means ``different things depending upon the setting''); \emph{United States} v. \emph{Alvarez-Sanchez,} 511 U.~S. 350, 357 (1994) (``[R]espondent errs in placing dispositive weight on the broad statutory reference to ‘any' law enforcement officer or agency without considering the rest of the statute'').

  Context, of course, includes the words immediately surrounding the phrase in question. And canons such as \emph{ejusdem generis} and \emph{noscitur a sociis} offer help in evaluating the significance of those surrounding words. Yet that help is limited. That is because other contextual features can show that Congress intended a phrase to apply more broadly than the immediately surrounding words by themselves suggest. See \emph{Circuit City Stores, Inc.} v. \emph{Adams,} 532 U. S. 105, 138--140 (2001) (\textsc{Souter,} J., dissenting) (finding ``good reasons'' not to apply \emph{ejusdem generis} because the statute's history and purposes make clear that the words ``any other class of workers'' in the phrase ``seamen, railroad employees, or any other class of workers'' refer, not just to other transportation workers, but to workers of all kinds including retail store clerks). It is because canons of construction are not ``conclusive'' and ``are often countered\dots by some maxim \newpage  pointing in a different direction.'' \emph{Id.,} at 115 (majority opinion). And it is because these particular canons simply crystallize what English speakers already know, namely, that lists often (but not always) group together items with similar characteristics. That is why we cannot, without comic effect, yoke radically different nouns to a single verb, \emph{e. g.,} ``He caught three salmon, two trout, and a cold.''

  In this case, not only the immediately surrounding words but also every other contextual feature supports \textsc{Justice Kennedy}'s conclusion. The textual context includes the location of the phrase within a provision that otherwise exclusively concerns customs and revenue duties. And the nontextual context includes several features that, taken together, indicate that Congress intended a narrow tortliability exception related to customs and excise.

  First, drafting history shows that the relevant portion of the bill that became the Federal Tort Claims Act concerned only customs and excise. Initially, the relevant provision of the bill exempted only claims ``arising in respect of the assessment or collection of any tax or customs duty.'' See, \emph{e. g.,} S. 4377, 71st Cong., 2d Sess., 4 (1930). In 1931, a Special Assistant to the Attorney General, Alexander Holtzoff, wrote additional draft language, namely, ``or the detention of any goods or merchandise by any officer of customs or excise or \emph{any other law enforcement officer.}'' Bill Draft, p. 2, reprinted in Report on Proposed Federal Tort Claims Bill p. 2 (1931) (emphasis added). Holtzoff, in a report to a congressional agency, said that the expanded language sought ``to include immunity from liability in respect of loss in connection with the detention of goods or merchandise by any officer of customs or excise.'' \emph{Id.,} at 16. Holtzoff explained that the language was suggested by a similar British bill that mentioned only customs and excise officials. \emph{Ibid.} (referring to the bill proposed in the Crown Proceedings Committee Report \S~11(5)(\emph{c}), pp. 17--18 (Apr. 1927) (Cmd. 2842) (``No proceedings shall lie under this section\dots for \newpage  or in respect of the loss of or any deterioration or damage occasioned to, or any delay in the release of, any goods or merchandise by reason of anything done or omitted to be done by any officer of customs and excise acting as such'')); see \emph{Kosak} v. \emph{United States,} 465 U.~S. 848, 857, n. 13 (1984) (While ``the ideas expressed [in Holtzoff's report] should not be given great weight in determining the intent of the Legislature,'' at least in some circumstances, ``it seems to us senseless to ignore entirely the views of [the provision's] draftsman''). And Members of Congress repeatedly referred to the exception as encompassing claims involving customs and excise functions. See, \emph{e. g.,} H. R. Rep. No. 2428, 76th Cong., 3d Sess., 5 (1940); S. Rep. No. 1196, 77th Cong., 2d Sess., 7 (1942); H. R. Rep. No. 2245, 77th Cong., 2d Sess., 10 (1942); H. R. Rep. No. 1287, 79th Cong., 1st Sess., 6 (1945); S. Rep. No. 1400, 79th Cong., 2d Sess., 33 (1946).

  Second, insofar as Congress sought, through the Act's exceptions, to preclude tort suits against the Government where ``adequate remedies were already available,'' \emph{Kosak, supra,} at 858; see S. Rep. No. 1400, at 33; H. R. Rep. No. 1287, at 6 (setting forth that purpose), a limited exception makes sense; a broad exception does not. Other statutes already provided recovery for plaintiffs harmed by federal officers enforcing customs and tax laws but not for plaintiffs harmed by all other federal officers enforcing most other laws. See \emph{Bazuaye} v. \emph{United States,} 83 F. 3d 482, 485--486 (CADC 1996) (detailing history).

  Third, the practical difference between a limited and a broad interpretation is considerable, magnifying the importance of the congressional silence to which \textsc{Justice Kennedy} points, see \emph{ante,} at 238. A limited interpretation of the phrase ``any other law enforcement officer'' would likely encompass only those law enforcement officers working, say, at borders and helping to enforce customs and excise laws. The majority instead interprets this provision to include the tens of thousands of officers performing unrelated tasks. \newpage  The Justice Department estimates that there are more than 100,000 law enforcement officers, not including members of the armed services. See, \emph{e. g.,} Dept. of Justice, Bureau of Justice Statistics Bulletin, B. Reaves, Federal Law Enforcement Officers, 2004, p. 1 (July 2006). And although the law's history contains much that indicates the provision's scope is limited to customs and excise, it contains \emph{nothing at all} suggesting an intent to apply the provision more broadly, indeed, to multiply the number of officers to whom it applies by what is likely one or more orders of magnitude. It is thus not the Latin canons, \emph{ejusdem generis} and \emph{noscitur a sociis,} that shed light on the application of the statutory phrase but \textsc{Justice Scalia}'s more pertinent and easily remembered English-language observation that Congress ``does not\dots hide elephants in mouseholes.'' \emph{Whitman} v. \emph{American Trucking Assns., Inc.,} 531 U. S. 457, 468 (2001).

  For these reasons, I dissent and I join \textsc{Justice Kennedy}'s dissent.
