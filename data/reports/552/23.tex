% Opinion of the Court
% Ginsburg

\setcounter{page}{26}

  \textsc{Justice Ginsburg} delivered the opinion of the Court.

  Petitioner James D. Logan pleaded guilty in a United States District Court to being a felon in possession of a firearm, in violation of 18 U.~S.~C. \S~922(g)(1). Logan's record as a recidivist, which included three relevant state convictions, led the District Court to impose a 15-year prison term, the minimum sentence mandated by the Armed Career Criminal Act (ACCA), 18 U.~S.~C. \S~924(e)(1) (2000 ed., Supp. V). For ACCA sentence-enhancement purposes, a prior conviction may be disregarded if the conviction ``has been expunged, or set aside,'' or the offender ``has been pardoned or has had civil rights restored.'' \S~921(a)(20) (2000 ed.). None of Logan's prior convictions have been expunged or set aside. Nor has he been pardoned for any past crime. And, bearing importantly on the instant petition, the three state-court convictions that triggered Logan's ACCA-enhanced sentence occasioned no loss of civil rights.

  Challenging his enhanced sentence, Logan presents this question: Does the ``civil rights restored'' exemption contained in \S~921(a)(20) encompass, and therefore remove from ACCA's reach, state-court convictions that at no time deprived the offender of civil rights? We hold that the \S~921(a)(20) exemption provision does not cover the case of an offender who retained civil rights at all times, and whose legal status, postconviction, remained in all respects unaltered by any state dispensation.

  Section 921(a)(20) sets out postconviction events---expungement, set aside, pardon, or restoration of civil rights---that extend to an offender a measure of forgiveness, relieving him from some or all of the consequences of his conviction. Congress might have broadened the \S~921(a)(20) exemption provision to cover convictions attended by no loss of civil rights. The national lawmakers, however, did not do so. Section 921(a)(20)'s failure to exempt convictions that do not revoke civil rights produces anomalies. But so does the extension of the \S~921(a)(20) exemption that Logan advances. \newpage  We are not equipped to say what statutory alteration, if any, Congress would have made had its attention trained on offenders who retained civil rights; nor can we recast \S~921(a)(20) in Congress' stead.

\section{I}

  Federal law generally prohibits the possession of a firearm by a person convicted of ``a crime punishable by imprisonment for a term exceeding one year.'' 18 U.~S.~C. \S~922(g)(1). Ordinarily, the maximum felon-in-possession sentence is ten years. See \S~924(a)(2). If the offender's prior criminal record includes at least three convictions for ``violent felon[ies]'' or ``serious drug offense[s],'' however, the maximum sentence increases to life, and ACCA mandates a minimum term of 15 years. \S~924(e)(1) (2000 ed., Supp. V).

  Congress defined the term ``violent felony'' to include specified crimes ``punishable by imprisonment for a term exceeding one year.'' \S~924(e)(2)(B) (2000 ed.). An offense classified by a State as a misdemeanor, however, may qualify as a ``violent felony'' for ACCA-enhancement purposes (or as a predicate for a felon-in-possession conviction under \S~922(g)) only if the offense is punishable by more than two years in prison. \S~921(a)(20)(B).

  In \emph{Dickerson} v. \emph{New Banner Institute, Inc.,} 460 U.~S. 103 (1983), we held that a State's expungement of a conviction did not nullify the conviction for purposes of the firearms disabilities Congress placed in \S\S~922(g)(1) and (h)(1). In so ruling, we noted that our decision would ensure greater uniformity in federal sentences. See \emph{id.,} at 119--120. Provisions for expungement ``var[ied] widely from State to State,'' we observed, \emph{id.,} at 120, and yielded ``nothing less than a national patchwork,'' \emph{id.,} at 122.

  In the Firearms Owners' Protection Act (FOPA), 100 Stat. 449, Congress amended \S~921(a)(20) in response to \emph{Dickerson}'s holding that, for purposes of federal firearms disabilities, state law did not determine the present impact of a \newpage  prior conviction. The amended provision excludes from qualification as a ``crime punishable by imprisonment for a term exceeding one year'' (or a misdemeanor under state law punishable by more than two years in prison):

    \begin{quote}

		``Any conviction which has been expunged, or set aside or for which a person has been pardoned or has had civil rights restored .~.~. unless such pardon, expungement, or restoration of civil rights expressly provides that the person may not ship, transport, possess, or receive firearms.'' 18 U.~S.~C. \S~921(a)(20).\footnotemark[1]

    \end{quote}

\noindent While \S~921(a)(20) does not define the term ``civil rights,'' courts have held, and petitioner agrees, that the civil rights relevant under the above-quoted provision are the rights to vote, hold office, and serve on a jury. See Brief for Petitioner 13, n. 10; cf. \emph{Caron} v. \emph{United States,} 524 U.~S. 308, 316 (1998).

\section{II}

  On May 31, 2005, police officers responded to a domestic disturbance complaint made by Logan's girlfriend, Asenath Wilson. App. 9, 12. Wilson told the officers, among other things, that she had seen Logan with a gun and that he usually kept it in the car. \emph{Id.,} at 9. Logan, who was with Wil\newpage son when the police arrived, consented to a search of his car. \emph{Id.,} at 11. In a hidden compartment behind the glove box, the officers found a 9-millimeter handgun. \emph{Id.,} at 9--10, 12.

\footnotetext[1]{FOPA, 100 Stat. 449, included a ``safety valve'' provision under which persons subject to federal firearms disabilities, including persons whose civil rights have not been restored, may apply to the Attorney General for relief from the disabilities. See 18 U.~S.~C. \S~925(c) (2000 ed., Supp. V). The relief provision has been rendered inoperative, however, for Congress has repeatedly barred the Attorney General from using appropriated funds ``to investigate or act upon [relief] applications.'' \emph{United States} v. \emph{Bean,} 537 U.~S. 71, 74--75 (2002) (internal quotation marks omitted). The bar on funding was renewed every year from 1992 through 2006. See \emph{id.,} at 75, n. 3 (1992 through 2002); Consolidated Appropriations Resolution, 2003, 117 Stat. 433; Consolidated Appropriations Act, 2004, 118 Stat. 53; Consolidated Appropriations Act, 2005, 118 Stat. 2859; Science, State, Justice, Commerce, and Related Agencies Appropriations Act, 2006, 119 Stat. 2290.}

  Logan pleaded guilty to the federal offense of possession of a firearm after having been convicted of a felony. \emph{Id.,} at 12. (In 1991, he had been convicted in an Illinois court of unlawful possession of a controlled substance. \emph{Id.,} at 9--10, 12.) The United States District Court for the Western District of Wisconsin sentenced Logan to imprisonment for 15 years, the mandatory minimum under ACCA. In imposing that enhanced sentence, the District Court took account of Logan's three Wisconsin misdemeanor battery convictions, each punishable by a maximum sentence of three years' imprisonment. \emph{Id.,} at 16--18.\footnotemark[2]

  Both in the District Court and on appeal, Logan argued that his Wisconsin misdemeanor convictions did not qualify as ACCA predicate offenses because they caused no loss of his civil rights. Rights retained, he urged, are functionally equivalent to rights revoked but later restored. If the exemption contained in \S~921(a)(20) covered the three statecourt misdemeanor convictions, Logan's maximum sentence, in lieu of the 15-year mandatory minimum under ACCA, would have been ten years, see \S~924(a)(2), and the United

\footnotetext[2]{Under Wisconsin law, misdemeanor battery is ordinarily punishable by a maximum term of nine months. See Wis. Stat. \S~940.19(1) (2005); \S~939.51(3). Logan was exposed to a three-year maximum term for each offense, however, because he was convicted as a ``repeater'' or ``habitual'' criminal. See App. 16--17; Wis. Stat. \S~939.62 (1999--2000).}

  Postdating Logan's battery convictions, Wisconsin prospectively reduced the maximum term for ``repeater'' misdemeanors to two years. See 2001 Wis. Act 109, \S~562 (Jan. 2002 special session) (amending Wis. Stat. \S~939.62). Misdemeanors committed in Wisconsin after this reduction no longer qualify as ``violent felonies'' under 18 U.~S.~C. \S~921(a)(20).

  Logan has never argued that his Wisconsin convictions should not count as ACCA predicates because they were punishable by more than two years' imprisonment solely because of his status as a recidivist offender. We express no opinion on this matter. Cf. \emph{United States} v. \emph{Rodriquez,} 464 F. 3d 1072 (CA9 2006), cert. granted, 551 U. S. 1191 (2007). \newpage  States Sentencing Guidelines would have indicated a sentence range of 37 to 46 months, see Brief for Petitioner 5. The District Court rejected Logan's argument, holding that the \S~921(a)(20) exemption provision ``applies only to defendants whose civil rights were both lost and restored pursuant to state statutes.'' App. in No. 05--CR--088--S--01 (WD Wis.), p. 11. Accordingly, the court sentenced Logan to imprisonment for 15 years. \emph{Id.,} at 12.

  The United States Court of Appeals for the Seventh Circuit affirmed, concluding that ``an offender whose civil rights have been neither diminished nor returned is not a person who ‘has had civil rights restored.' '' 453 F. 3d 804, 805 (2006). Logan's argument for treating retained rights the same way as restored rights, the appeals court observed, ``go[es] in the teeth of [\S~921(a)(20)'s] text.'' \emph{Ibid.}

  We granted certiorari, 549 U.~S. 1204 (2007), to resolve a split among the Circuits as to whether \S~921(a)(20)'s exception for ``civil rights restored'' should be interpreted to include civil rights retained at all times. Compare 453 F. 3d, at 809 (case below) (``civil rights restored'' does not include civil rights never revoked), and \emph{McGrath} v. \emph{United States,} 60 F. 3d 1005 (CA2 1995) (same), with \emph{United States} v. \emph{Indelicato,} 97 F. 3d 627, 631 (CA1 1996) (``civil rights restored'' includes civil rights never lost).

\section{III}

  Logan pleaded guilty to being a felon in possession of a firearm, in violation of \S~922(g)(1), and received a mandatory minimum 15-year sentence because he had at least three prior convictions for ``violent felon[ies].'' \S~924(e)(1) (2000 ed., Supp. V). He acknowledges his convictions in Wisconsin for three battery offenses that facially qualify as violent felonies under \S~921(a)(20)(B) (2000 ed.). See Brief for Petitioner 4--5. Thus the sole matter in dispute is whether Logan fits within the exemption from an ACCA-enhanced sentence for convictions ``expunged, or set aside'' or offend\newpage ers who ``ha[ve] been pardoned or ha[ve] had civil rights restored.'' \S~921(a)(20). None of Logan's battery convictions have been expunged, set aside, or pardoned. See 453 F. 3d, at 809. Under Wisconsin law, felons lose but can regain their civil rights and can gain the removal of firearms disabilities. See Wis. Stat. \S~6.03(1)(b) (Supp. 2006); Wis. Const., Art. XIII, \S~3(2); Wis. Stat. \S~756.02 (2001); \S~973.176(1) (2007). Persons convicted of misdemeanors, however, even if they are repeat offenders, generally retain their civil rights and are not subject to firearms disabilities.

  With this background in view, we turn to the proper interpretation of the \S~921(a)(20) exemption from ACCA-enhanced sentencing for offenders who have had their ``civil rights restored.'' Logan's misdemeanor convictions, we reiterate, did not result in any loss of the rights to vote, hold public office, or serve on juries. Should he nonetheless be ranked with offenders whose rights were terminated but later restored? The ordinary meaning of the word ``restored'' affords Logan no aid. In line with dictionary definitions,\footnotemark[3] the Court of Appeals stated: ``The word ‘restore' means to give back something that had been taken away.'' 453 F. 3d, at 805. Accord \emph{McGrath,} 60 F. 3d, at 1007 (``The ‘restoration' of a thing never lost or diminished is a definitional impossibility.''); cf. \emph{Indelicato,} 97 F. 3d, at 629 (``Clearly the ordinary reading of the word ‘restored' supports the government.'').

  The context in which the word ``restored'' appears in \S~921(a)(20) counsels adherence to the word's ordinary meaning. Words in a list are generally known by the company they keep. \emph{E. g., Dole} v. \emph{Steelworkers,} 494 U.~S. 26, 36 \newpage  (1990); \emph{Beecham} v. \emph{United States,} 511 U.~S. 368, 371 (1994). In \S~921(a)(20), the words ``civil rights restored'' appear in the company of the words ``expunged,'' ``set aside,'' and ``pardoned.'' Each term describes a measure by which the government relieves an offender of some or all of the consequences of his conviction. In contrast, a defendant who retains rights is simply left alone. He receives no statusaltering dispensation, no token of forgiveness from the government.

\footnotetext[3]{See, \emph{e. g.,} Webster's Third New International Dictionary 1936 (1993) (defining ``restore'' to mean ``give back (as something lost or taken away)''); American Heritage Dictionary 1486 (4th ed. 2000) (defining ``restore'' to mean ``bring back into existence or use; reestablish''); 13 Oxford English Dictionary 755 (2d ed. 1989) (defining ``restore'' to mean ``give back, [or] make return or restitution of (anything previously taken away or lost)'').}

  Opposing a plain-meaning approach to the language Congress enacted, Logan relies dominantly on the harsh results a literal reading could yield: Unless retention of rights is treated as legally equivalent to restoration of rights, less serious offenders will be subject to ACCA's enhanced penalties while more serious offenders in the same State, who have had civil rights restored, may escape heightened punishment. \emph{E. g.,} Reply Brief 8 (``[I]ndividuals who have committed more serious crimes than Petitioner may nonetheless have their rights restored, whereas misdemeanants who never lost their rights must suffer enhanced sentencing.''). Logan urges that this result---treating those who never lost their civil rights more harshly than those who lost, then regained, those rights---is not merely anomalous; it rises to the level of the absurd, particularly in States where restoration of civil rights is automatic and occurs immediately upon release from prison. See \emph{Caron,} 524 U. S., at 313 (automatic restoration of rights qualifies for \S~921(a)(20)'s exemption).

  Logan's argument, we note, overlooks \S~921(a)(20)'s ``unless'' clause. Under that provision, an offender gains no exemption from ACCA's application through an expungement, set-aside, pardon, or restoration of civil rights if the dispensation ``expressly provides that the [offender] may not ship, transport, possess, or receive firearms.'' Many States that restore felons' civil rights (or accord another measure of forgiveness) nonetheless impose or retain firearms disabilities.\newpage  See Brief for United States 30 (citing, \emph{inter alia,} La. Rev. Stat. Ann. \S~14:95.1(C) (West Supp. 2007), under which felons' firearms disabilities are lifted only after ten years and only if no further felony convictions intervene).\footnotemark[4] We further note that Wisconsin has addressed, and prospectively eliminated, the anomaly Logan asserts he encountered: Wisconsin no longer punishes misdemeanors by more than two years of imprisonment, and thus no longer has any misdemeanors that qualify as ACCA predicates. See \emph{supra,} at 29, n. 2.

  One can demur to Logan's argument that a literal reading of \S~921(a)(20) could produce anomalous results, for the resolution he proposes---reading into the exemption convictions under which civil rights are retained---would correct one potential anomaly while creating others. See \emph{McGrath,} 60 F. 3d, at 1009. Under Logan's proposed construction, the most dangerous recidivists in a State that does not revoke any offender's civil rights could fall within \S~921(a)(20)'s exemption. For example, Maine does not deprive any offenders of their civil rights. See Lodging for National Association of Criminal Defense Lawyers et al. as \emph{Amici Curiae} (NACDL Lodging), App. 1, pp. 23--24. As Logan would have us read \S~921(a)(20), all Maine crimes, including firstdegree murder, would be treated as crimes for which ``civil rights [have been] restored,'' while less serious crimes committed elsewhere would not.

  In \emph{McGrath,} the Second Circuit incisively identified Congress' response to \emph{Dickerson,} see \emph{supra,} at 27--28, as the cause of the multiple anomalies \S~921(a)(20) may produce:

\footnotetext[4]{Courts have divided on the question whether \S~921(a)(20)'s ``unless'' clause is triggered whenever state law provides for the continuation of firearm proscriptions, or only when the State provides individual notice to the offender of the firearms disabilities. Compare, \emph{e. g., United States} v. \emph{Cassidy,} 899 F. 2d 543, 549 (CA6 1990) (courts must look to ``the whole of state law''), with \emph{United States} v. \emph{Gallaher,} 275 F. 3d 784, 791, and n. 3 (CA9 2001) (individualized notice is required). We express no opinion on this issue.}

\begin{quote}

	\newpage  ``[Congress'] decision to have restoration triggered by events governed by state law insured anomalous results. The several states have considerably different laws governing pardon, expungement, and forfeiture and restoration of civil rights. Furthermore, states have drastically different policies as to when and under what circumstances such discretionary acts of grace should be extended.~.~.~. [Anomalies generated by \S~921(a)(20)] are the inevitable consequence of making access to the exemption depend on the differing laws and policies of the several states.'' 60 F. 3d, at 1009.

\end{quote}

\noindent Accord 453 F. 3d, at 807 (``When Congress replaced \emph{Dickerso[n]} .~.~. it ensured that similarly situated people would be treated differently---for states vary widely in which if any civil rights a convict loses and whether these rights are restored.''). See also M. Love, Relief from the Collateral Consequences of a Criminal Conviction: A State-by-State Resource Guide (2006), updated online at http://www.sentencingproject.org/PublicationDetails.aspx? PublicationID=486 (as visited Nov. 27, 2007, and in Clerk of Court's case file) (surveying state practices).

  Were we to accept Logan's argument, it bears emphasis, we would undercut \S~921(a)(20)(B), which places within ACCA's reach state misdemeanor convictions punishable by more than two years' imprisonment. Because state-law misdemeanors generally entail no revocation of civil rights,\footnotemark[5] Logan's proposed reading of the word ``restored'' to include ``retained'' would yield this curiosity: An offender would fall within ACCA's reach if his three prior offenses carried potential prison terms of over two years, but that same of\newpage fender would be released from ACCA's grip by virtue of his retention of civil rights. We are disinclined to say that what Congress imposed with one hand (exposure to ACCA) it withdrew with the other (exemption from ACCA).

\footnotetext[5]{See NACDL Lodging, App. 1 (compiling state laws); \emph{id.,} at 17--34 (indicating that Connecticut, Florida, Iowa, Louisiana, Nebraska, and New Hampshire revoke no misdemeanant's civil rights, but provide for punishment of certain crimes they classify as misdemeanors by prison terms exceeding two years).}

  We may assume, \emph{arguendo,} that when Congress revised \S~921(a)(20) in 1986, see \emph{supra,} at 27--28, it labored under the misapprehension that all offenders---misdemeanants as well as felons---forfeit civil rights, at least temporarily. Even indulging the further assumption that courts may repair such a congressional oversight or mistake,\footnotemark[6] we could hardly divine the revision the Legislature would favor. Perhaps Congress would choose to exempt offenders who never lost their civil rights. See \emph{McGrath,} 60 F. 3d, at 1009. But it is also plausible that Congress would remove the exemption for civil rights restoration as insufficiently indicative of official forgiveness. Or, Congress might elect to include restorations of civil rights along with expungements, set-asides, and pardons only if the restoration was nonautomatic, \emph{i. e.,} granted on a case-by-case basis. Homing in on the disparities resulting from diverse state legislation, see \emph{supra,} at 33--34 and this page, Congress might even revise \S~921(a)(20) to provide, in accord with \emph{Dickerson,} that federal rather than state law defines a conviction for purposes of \S\S~922 and 924. See 453 F. 3d, at 806--807.

\footnotetext[6]{But see \emph{Iselin} v. \emph{United States,} 270 U.~S. 245, 251 (1926) (``enlargement of [a statute] by [a] court, so that what was omitted, presumably by inadvertence, may be included within its scope .~.~. transcends the judicial function'').}

  In all events, a measure adopted ten years after \S~921(a)(20) was given its current shape cautions against any assumption that Congress did not mean to deny that exemption to offenders who retained their civil rights. In 1996, Congress enacted \S~922(g)(9), which outlaws possession of a firearm by anyone ``who has been convicted\dots of a misdemeanor crime of domestic violence.'' See Pub. L. 104--208, \newpage  Tit. VI, \S~658, 110 Stat. 3009--371 to 3009--372. Tailored to \S~922(g)(9), Congress adopted a definitional provision, corresponding to \S~921(a)(20), which reads:

\begin{quote}

	``A person shall not be considered to have been convicted of [a misdemeanor crime of domestic violence] if the conviction has been expunged or set aside, or is an offense for which the person has been pardoned or has had civil rights restored \emph{(if the law of the applicable jurisdiction provides for the loss of civil rights under such an offense)} unless the pardon, expungement, or restoration of civil rights expressly provides that the person may not ship, transport, possess, or receive firearms.'' 18 U.~S.~C. \S~921(a)(33)(B)(ii) (emphasis added).

\end{quote}

  Section 921(a)(33)(B)(ii) tracks \S~921(a)(20) in specifying expungement, set-aside, pardon, or restoration of rights as dispensations that can cancel lingering effects of a conviction. But the emphasized parenthetical qualification shows that the words ``civil rights restored'' do not cover a person whose civil rights were never taken away. See 453 F. 3d, at 808. Section 921(a)(33)(B)(ii) casts considerable doubt on Logan's hypothesis that, had Congress adverted to the issue when it drafted \S~921(a)(20), it would have placed in the same category persons who regained civil rights and persons who retained civil rights.

  Congress' enactment of \S~921(a)(33)(B)(ii) is also relevant to Logan's absurdity argument. See \emph{supra,} at 32. Statutory terms, we have held, may be interpreted against their literal meaning where the words ``could not conceivably have been intended to apply'' to the case at hand. \emph{Cabell} v. \emph{Markham,} 148 F. 2d 737, 739 (CA2) (L. Hand, J.), aff'd, 326 U.~S. 404 (1945); see \emph{Green} v. \emph{Bock Laundry Machine Co.,} 490 U.~S. 504, 511 (1989) (Federal Rule of Evidence 609(a)(1) ``can't mean what it says'' (internal quotation marks omitted)). In this case, it can hardly be maintained that Congress could not have meant what it said. Congress explic\newpage itly distinguished between ``restored'' and ``retained'' in \S~921(a)(33)(B)(ii). It is more than ``conceivable'' that the Legislature, albeit an earlier one, see \emph{supra,} at 27--28, meant to do the same in \S~921(a)(20).

  In sum, Congress framed \S~921(a)(20) to serve two purposes. See Tr. of Oral Arg. 28--29. It sought to qualify as ACCA predicate offenses violent crimes that a State classifies as misdemeanors yet punishes by a substantial term of imprisonment, \emph{i. e.,} more than two years. See \S~921(a)(20)(B). Congress also sought to defer to a State's dispensation relieving an offender from disabling effects of a conviction. See \emph{supra,} at 27--28. Had Congress included a retention-of-rights exemption, however, the very misdemeanors it meant to cover would escape ACCA's reach. See \emph{supra,} at 34--35. Logan complains of an anomalous result. Yet the solution he proposes would also produce anomalies. See \emph{supra,} at 33. Having no warrant to stray from \S~921(a)(20)'s text, we hold that the words ``civil rights restored'' do not cover the case of an offender who lost no civil rights.

\hrule

  For the reasons stated, the judgment of the Court of Appeals for the Seventh Circuit is

        \emph{Affirmed.}
