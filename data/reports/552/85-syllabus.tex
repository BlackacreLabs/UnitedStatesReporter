% Syllabus
% Reporter of Decisions

\setcounter{page}{85}

\noindent Under the statute criminalizing the manufacture and distribution of
cocaine, 21 U.~S.~C. \S~841, and the relevant Federal Sentencing
Guidelines, a drug trafficker dealing in crack cocaine is subject to
the same sentence as one dealing in 100 times more powder cocaine.
Petitioner Kimbrough pleaded guilty to four offenses: conspiracy to
distribute crack and powder; possession with intent to distribute
more than 50 grams of crack; possession with intent to distribute
powder; and possession of a firearm in furtherance of a drug-trafficking
offense. Under the relevant statutes, Kimbrough's plea subjected
him to a minimum prison term of 15 years and a maximum of life. The
applicable advisory Guidelines range was 228 to 270 months, or 19 to
22.5 years. The District Court found, however, that a sentence in
this range would have been greater than necessary to accomplish the
purposes of sentencing set forth in 18 U.~S.~C. \S~3553(a). In
making that determination, the court relied in part on its view that
Kimbrough's case exemplified the ``disproportionate and unjust effect
that crack cocaine guidelines have in sentencing.'' The court noted
that if Kimbrough had possessed only powder cocaine, his Guidelines
range would have been far lower: 97 to 106 months. Concluding that the
statutory minimum sentence was long enough to accomplish \S~3553(a)'s
objectives, the court sentenced Kimbrough to 15 years, or 180 months, in
prison. The Fourth Circuit vacated the sentence, finding that a sentence
outside the Guidelines range is \emph{per se} unreasonable when it is based
on a disagreement with the sentencing disparity for crack and powder
offenses.

\emph{Held:}

  1. Under \emph{United States} v. \emph{Booker,} 543 U.~S. 220, the
cocaine Guidelines, like all other Guidelines, are advisory only,
and the Fourth Circuit erred in holding the crack/powder disparity
effectively mandatory. A district judge must include the Guidelines
range in the array of factors warranting consideration, but the judge
may determine that, in the particular case, a within-Guidelines sentence
is ``greater than necessary'' to serve the objectives of sentencing,
\S~3553(a). In making that determination, the judge may consider the
disparity between the Guidelines' treatment of crack and powder
offenses. Pp. 94--110.

  (a) Crack and powder cocaine have the same physiological and
psy\newpage poses. The relevant statutes and Guidelines employ a 100-to-1
ratio that yields sentences for crack offenses three to six times longer
than those for offenses involving equal amounts of powder. Thus, a major
supplier of powder may receive a shorter sentence than a low-level
dealer who buys powder and converts it to crack. Pp. 94--100.

  (1) The crack/powder disparity originated in the Anti-Drug Abuse
Act of 1986 (1986 Act), which created a two-tiered scheme of fiveand
ten-year mandatory minimum sentences for drug manufacturing and
distribution offenses. Congress apparently adopted the 100-to-1 ratio
because it believed that crack, a relatively new drug in 1986, was
significantly more dangerous than powder. Thus, the 1986 Act's
fiveyear mandatory minimum applies to any defendant accountable for
5 grams of crack or 500 grams of powder, and its ten-year mandatory
minimum applies to any defendant accountable for 50 grams of crack
or 5,000 grams of powder. In developing Guidelines sentences for
cocaine offenses, the Sentencing Commission employed the statute's
weightdriven scheme, rather than its usual empirical approach based
on past sentencing practices. The statute itself specifies only two
quantities of each drug, but the Guidelines used the 100-to-1 ratio to
set sentences for a full range of drug quantities. Pp. 95--97.

  (2) Based on additional research and experience with the 100-to-1
ratio, the Commission later determined that the crack/powder
differential does not meet the objectives of the Sentencing Reform Act
and the 1986 Act. The Commission also found the disparity inconsistent
with the 1986 Act's goal of punishing major drug traffickers more
severely than low-level dealers, and furthermore observed that the
differential fosters a lack of confidence in the criminal justice system
because of a perception that it promotes an unwarranted divergence based
on race. Pp. 97--99.

  (3) The Commission has several times sought to achieve a reduction
in the crack/powder ratio. Congress rejected a 1995 amendment to the
Guidelines that would have replaced the 100-to-1 ratio with a 1-to-1
ratio, but directed the Commission to propose revision of the ratio
under the relevant statutes and Guidelines. Congress took no action
after the Commission's 1997 and 2002 reports recommended changing the
ratio. The Commission's 2007 report again urged Congress to amend the
1986 Act, but the Commission also adopted an ameliorating change in the
Guidelines. The modest amendment, which became effective on November 1,
2007, yields sentences for crack offenses between two and five times
longer than sentences for equal amounts of powder. The Commission thus
noted that it is only a partial remedy to the problems generated by the
crack/powder disparity. Pp. 99--100.\newpage 

  (b) The federal sentencing statute, as modified by \emph{Booker,}
requires a court to give respectful consideration to the Guidelines,
but ``permits the court to tailor the sentence in light of other
[\S~3553(a)] concerns as well,'' 543 U. S., at 245--246. The
Government contends that the Guidelines adopting the 100-to-1 ratio
are an exception to this general freedom and offers three arguments
in support of its position, each of which this Court rejects. Pp.
100--108.

  (1) The Government argues that the 1986 Act itself prohibits the
Commission and sentencing courts from disagreeing with the 100-to-1
ratio. This position lacks grounding in the statute, which, by its
terms, mandates only maximum and minimum sentences: A person convicted
of possession with intent to distribute five grams or more of crack
must be sentenced to a minimum of 5 years and a maximum of 40. A person
with 50 grams or more of crack must be sentenced to a minimum of ten
years and a maximum of life. The statute says nothing about appropriate
sentences within these brackets, and this Court declines to read any
implicit directive into the congressional silence. See \emph{Jama} v.
\emph{Immigration and Customs Enforcement,} 543 U.~S. 335, 341. Drawing
meaning from silence is particularly inappropriate here, because
Congress knows how to direct sentencing practices in express terms.
See, \emph{e. g.,} 28 U.~S.~C. \S~994(h). This cautious reading
of the 1986 Act draws force from \emph{Neal} v. \emph{United States,}
516 U.~S. 284, which involved different methods of calculating
lysergic acid diethylamide (LSD) weights: The method applicable in
determining statutory minimum sentences combined the weight of the
pure drug and its carrier medium, while the one controlling the
calculation of Guidelines ranges presumed a lower weight for the carrier
medium. This Court rejected the argument that the Guidelines and the
statute should be interpreted consistently, with the Guidelines'
presumptive-weight method controlling the mandatory minimum calculation.
Were the Government's current position correct, the Guidelines
involved in \emph{Neal} would be in serious jeopardy. The same reasons
alleged to justify reading into the 1986 Act an implicit command to the
Commission and sentencing courts to apply the 100-to-1 ratio to all
crack quantities could be urged in support of an argument that the 1986
Act requires the Commission to include the full weight of the carrier
medium in calculating LSD weights. Yet \emph{Neal} never questioned the
Guidelines' validity, and in fact endorsed the Commission's freedom
to adopt a new method. If the 1986 Act does not require the Commission
to adhere to the 1986 Act's method for determining LSD weights, it
does not require the Commission---or, after \emph{Booker,} sentencing
courts---to adhere to the 100-to-1 ratio for crack quantities other
than those triggering the statutory mandatory minimum sentences. Pp.
102--105.\newpage 

  (2) The Government also argues that Congress made clear, in
disapproving the Commission's 1995 proposed Guidelines amendment, that
the 1986 Act required the Commission and courts to respect the 100-to-1
ratio. But nothing in Congress' 1995 action suggested that crack
sentences must exceed powder sentences by a ratio of 100 to 1. To the
contrary, Congress required the Commission to recommend a revision of
the ratio. The Government argues that, by calling for recommendations
to change both the statute and the Guidelines, Congress meant to bar
any Guidelines alteration in advance of congressional action. But the
more likely reading is that Congress sought proposals to amend both
the statute and the Guidelines because the Commission's criticisms
of the 100-to-1 ratio concerned the exorbitance of the crack/ powder
disparity in both contexts. Moreover, as a result of the 2007 amendment,
which Congress did not disapprove or modify, the Guidelines now deviate
from the statute's 100-to-1 ratio, advancing a ratio that varies
(at different offense levels) between 25 to 1 and 80 to 1. Pp.
105--106.

  (3) Finally, the Government argues that if district courts are
free to deviate from the Guidelines based on disagreements with the
crack/powder ratio, ``unwarranted sentence disparities,'' 18
U.~S.~C. \S~3553(a)(6), will ensue. The Government claims that,
because sentencing courts remain bound by the 1986 Act's mandatory
minimum sentences, deviations from the 100-to-1 ratio could result in
sentencing ``cliffs'' around quantities triggering the mandatory
minimums. For example, a district court could grant a sizable downward
variance to a defendant convicted of distributing 49 grams of crack,
but would be required by the statutory minimum to impose a much higher
sentence for only 1 additional gram. The LSD Guidelines approved in
\emph{Neal,} however, create a similar risk of sentencing ``cliffs.''
The Government also maintains that, if district courts are permitted
to vary from the Guidelines based on their disagreement with the
crack/powder disparity, defendants will receive markedly different
sentences depending on the particular judge drawn for sentencing.
While uniformity remains an important sentencing goal, \emph{Booker}
recognized that some departures from uniformity were a necessary cost
of the remedy that decision adopted. And as to crack sentences in
particular, possible variations among district courts are constrained
by the 1986 Act's mandatory minimums. Moreover, to the extent that
the Government correctly identifies risks of ``unwarranted sentence
disparities'' within the meaning of \S~3353(a)(6), the proper solution
is for district courts to take account of sentencing practices in other
courts and the ``cliffs'' resulting from the statutory mandatory
minimum sentences and weigh these disparities \newpage  against the other
\S~3553(a) factors and any unwarranted disparities created by the
crack/powder ratio itself. Pp. 106--108.

  (c) \emph{Booker} rendered the Sentencing Guidelines advisory, 543 U.
S., at 245, but preserved a key role for the Sentencing Commission. In
the ordinary case, the Commission's recommendation of a sentencing
range will ``reflect a rough approximation of sentences that might
achieve \S3553(a)'s objectives.'' \emph{Rita} v. \emph{United States,}
551 U.~S. 338, 350. The sentencing judge, on the other hand, is
``in a superior position to find facts and judge their import under
\S~3553(a)'' in each particular case. \emph{Gall} v. \emph{United States,
ante,} at 51 (internal quotation marks omitted). In light of these
discrete institutional strengths, a district court's decision to vary
from the advisory Guidelines may attract greatest respect when the
sentencing judge finds a particular case ``outside the ‘heartland'
to which the Commission intends individual Guidelines to apply.''
\emph{Rita,} 551 U. S., at 351. On the other hand, while the Guidelines
are no longer binding, closer review may be in order when the sentencing
judge varies from the Guidelines based solely on the judge's view
that the Guidelines range ``fails properly to reflect \S~3553(a)
considerations'' even in a mine-run case. \emph{Ibid.} The crack
cocaine Guidelines, however, present no occasion for elaborative
discussion of this matter because those Guidelines do not exemplify the
Commission's exercise of its characteristic institutional role. Given
the Commission's departure from its empirical approach in formulating
the crack Guidelines and its subsequent criticism of the crack/powder
disparity, it would not be an abuse of discretion for a district court
to conclude when sentencing a particular defendant that the crack/powder
disparity yields a sentence ``greater than necessary'' to achieve
\S~3553(a)'s purposes, even in a mine-run case. Pp. 108--110.

  2. The 180-month sentence imposed on Kimbrough should survive
appellate inspection. The District Court began by properly calculating
and considering the advisory Guidelines range. It then addressed the
relevant \S~3553(a) factors, including the Sentencing Commission's
reports criticizing the 100-to-1 ratio. Finally, the court did not
purport to establish a ratio of its own, but appropriately framed its
final determination in line with \S~3553(a)'s overarching instruction
to ``impose a sentence sufficient, but not greater than necessary,''
to accomplish the sentencing goals advanced in \S~3553(a)(2). The
court thus rested its sentence on the appropriate considerations
and ``committed no procedural error,'' \emph{Gall, ante,} at 56.
Kimbrough's sentence was 4.5 years below the bottom of the Guidelines
range. But in determining that 15 years was the appropriate prison term,
the District Court properly homed in on the particular circumstances of
Kimbrough's case and accorded weight to the Sentencing Commission's
consistent and emphatic \newpage  position that the crack/powder disparity
is at odds with \S~3553(a). Giving due respect to the District
Court's reasoned appraisal, a reviewing court could not rationally
conclude that the 4.5-year sentence reduction Kimbrough received
qualified as an abuse of discretion. Pp. 110--111.

174 Fed. Appx. 798, reversed and remanded.

  \textsc{Ginsburg,} J., delivered the opinion of the Court, in which
\textsc{Roberts,} C. J., and \textsc{Stevens, Scalia, Kennedy, Souter,} and
\textsc{Breyer,} JJ., joined. \textsc{Scalia,} J., filed a concurring opinion,
\emph{post,} p. 112. \textsc{Thomas,} J., \emph{post,} p. 114, and \textsc{Alito, J.,}
\emph{post,} p. 116, filed dissenting opinions.

  \emph{Michael S. Nachmanoff} argued the cause for petitioner. With him on
the briefs were \emph{Frances H. Pratt, Geremy C. Kamens,} and \emph{Kenneth P.
Troccoli.}

  \emph{Deputy Solicitor General Dreeben} argued the cause for the United
States. With him on the brief were \emph{Solicitor General Clement,
Assistant Attorney General Fisher, Kannon K. Shanmugam, Nina Goodman,}
and \emph{Jeffrey P. Singdahlsen.\\[[*]]

^* Briefs of \emph{amici curiae} urging reversal were filed for the
American Civil Liberties Union by \emph{Adam B. Wolf, Graham A. Boyd,} and
\emph{Steven R. Shapiro;} for Federal Public and Community Defenders et
al. by \emph{Mark Osler, Carlos A. Williams, Henry J. Bemporad, Brett G.
Sweitzer,} and \emph{David L. McColgin;} for the NAACP Legal Defense and
Educational Fund, Inc., by \emph{Ian Heath Gershengorn, Theodore M. Shaw,
Jacqueline A. Berrien, Christina Swarns,} and \emph{Johanna Steinberg;}
for the National Association of Criminal Defense Lawyers by \emph{Miguel
A. Estrada, David Debold,} and \emph{Peter Goldberger;} and for the
Sentencing Project et al. by \emph{Matthew M. Shors} and \emph{Pammela Quinn.}
