% Syllabus
% Reporter of Decisions

\setcounter{page}{264}

  After this Court announced a ``new rule'' for evaluating the reliability of testimonial statements in criminal cases, see \emph{Crawford} v. \emph{Washington,} 541 U.~S. 36, 68--69, petitioner sought state postconviction relief, arguing that he was entitled to a new trial because admitting the victim's taped interview at his trial violated \emph{Crawford}'s rule. The Minnesota trial and appeals courts concluded that \emph{Crawford} did not apply retroactively under \emph{Teague} v. \emph{Lane,} 489 U.~S. 288. The State Supreme Court agreed, and also concluded that state courts are not free to give a decision of this Court announcing a new constitutional rule of criminal procedure broader retroactive application than that given by this Court.

\emph{Held:}

\emph{Teague} does not constrain the authority of state courts to give broader effect to new rules of criminal procedure than is required by that opinion. Pp. 269--291.

  (a) \emph{Crawford} announced a ``new rule''---as defined by \emph{Teague}---because its result ``was not \emph{dictated} by precedent existing at the time the defendant's conviction became final,'' \emph{Teague,} 489 U. S., at 301 (plurality opinion). It was not, however, a rule ``of [this Court's] own devising'' or the product of its own views about sound policy, \emph{Crawford,} 541 U. S., at 67. Pp. 269--271.

  (b) The Court first adopted a ``retroactivity'' standard in \emph{Linkletter} v. \emph{Walker,} 381 U.~S. 618, 629, but later rejected that standard for cases pending on direct review, \emph{Griffith} v. \emph{Kentucky,} 479 U.~S. 314, and on federal habeas review, \emph{Teague} v. \emph{Lane,} 489 U.~S. 288. Under \emph{Teague,} new constitutional rules of criminal procedure may not be applied retroactively to cases on federal habeas review unless they place certain primary individual conduct beyond the States' power to proscribe or are ``watershed'' rules of criminal procedure. \emph{Id.,} at 310 (plurality opinion). Pp. 271--275.

  (c) Neither \emph{Linkletter} nor \emph{Teague} explicitly or implicitly constrained the States' authority to provide remedies for a broader range of constitutional violations than are redressable on federal habeas. And \emph{Teague} makes clear that its rule was tailored to the federal habeas context and thus had no bearing on whether States could provide broader relief in their own postconviction proceedings. Nothing in Justice O'Connor's general nonretroactivity rule discussion in \emph{Teague} asserts or even intimates that her definition of the class eligible for relief under a new rule \newpage  should inhibit the authority of a state agency or state court to extend a new rule's benefit to a broader class than she defined. Her opinion also clearly indicates that \emph{Teague}'s general nonretroactivity rule was an exercise of this Court's power to interpret the federal habeas statute. Since \emph{Teague} is based on statutory authority that extends only to federal courts applying a federal statute, it cannot be read as imposing a binding obligation on state courts. The opinion's text and reasoning also illustrate that the rule was meant to apply only to federal courts considering habeas petitions challenging state-court criminal convictions. The federal interest in uniformity in the application of federal law does not outweigh the general principle that States are independent sovereigns with plenary authority to make and enforce their own laws as long as they do not infringe on federal constitutional guarantees. The \emph{Teague} rule was intended to limit federal courts' authority to overturn state convictions---not to limit a state court's authority to grant relief for violations of new constitutional law rules when reviewing its own State's convictions. Subsequent cases confirm this view. See, \emph{e. g., Beard} v. \emph{Banks,} 542 U.~S. 406, 412. Pp. 275--282.

  (d) Neither \emph{Michigan} v. \emph{Payne,} 412 U.~S. 47, nor \emph{American Trucking Assns., Inc.} v. \emph{Smith,} 496 U.~S. 167, cast doubt on the state courts' authority to provide broader remedies for federal constitutional violations than mandated by \emph{Teague.} Pp. 282--288.

  (e) No federal rule, either implicitly announced in \emph{Teague,} or in some other source of federal law, prohibits States from giving broader retroactive effect to new rules of criminal procedure. Pp. 288--290.

718 N. W. 2d 451, reversed and remanded.

  \textsc{Stevens,} J., delivered the opinion of the Court, in which \textsc{Scalia, Souter, Thomas, Ginsburg, Breyer,} and \textsc{Alito, JJ.,} joined. \textsc{Roberts,} C. J., filed a dissenting opinion, in which \textsc{Kennedy,} J., joined, \emph{post,} p. 291.

  \emph{Benjamin J. Butler} argued the cause for petitioner. With him on the briefs was \emph{Roy G. Spurbeck.}

  \emph{Patrick C. Diamond} argued the cause for respondent. With him on the brief were \emph{Lori Swanson,} Attorney General of Minnesota, \emph{Michael O. Freeman,} and \emph{Jean Burdorf.\\[[*]]

\footnotetext[*]{\emph{Jeffrey A. Lamken} and \emph{Pamela Harris} filed a brief for the National Association of Criminal Defense Lawyers as \emph{amicus curiae} urging reversal.

\emph{[Footnote * is continued on p. 266]\\\newpage 

\emph{Talis J. Colberg,} Attorney General of Alaska, and \emph{Timothy W. Terrell,} Assistant Attorney General, filed a brief for the State of Alaska et al. as \emph{amici curiae} urging affirmance.

Briefs of \emph{amici curiae} were filed for the State of Kansas et al. by \emph{Paul J. Morrison,} Attorney General of Kansas, \emph{Stephen R. McAllister,} Solicitor General, and \emph{Jared S. Maag,} Deputy Solicitor General, and by the Attorneys General for their respective States as follows: \emph{Troy King} of Alabama, \emph{Tom Miller} of Iowa, \emph{Michael A. Cox} of Michigan, \emph{W. A. Drew Edmondson} of Oklahoma, \emph{Greg Abbott} of Texas, \emph{Mark L. Shurtleff} of Utah, and \emph{Robert F. McDonnell} of Virginia; and for the American Civil Liberties Union et al. by \emph{Larry Yackle, Steven R. Shapiro,} and \emph{John Holdridge.}}
