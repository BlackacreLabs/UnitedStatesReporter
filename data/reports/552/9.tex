% Court
% Roberts

\setcounter{page}{12}

  \textsc{Chief Justice Roberts} delivered the opinion of the Court.

  The Railroad Revitalization and Regulatory Reform Act of 1976 prohibits States from discriminating against railroads by taxing railroad property more heavily than other commercial property in the State. Two decades ago, we held that this statute permits an aggrieved railroad to challenge a State's valuation of its property for tax purposes. \emph{Burlington Northern R. Co.} v. \emph{Oklahoma Tax Comm'n,} 481 U.~S. 454, 462 (1987). Because the railroad in that case challenged only the State's \emph{application} of its valuation methods, we expressly reserved the question whether a railroad may challenge the State's methods themselves. We answer that question today, and hold that railroads may challenge state methods for determining the value of railroad property, as well as how those methods are applied. The statute provides for nothing less.

\section{I}

  Congress enacted the Railroad Revitalization and Regulatory Reform Act in 1976. 90 Stat. 31.\footnotemark[1] Called the ``4--R Act'' for brevity, the law aimed to halt the economic decline of the rail industry by, among other means, barring ``discriminatory state taxation of railroad property.'' \emph{Burlington Northern, supra,} at 457; see also \emph{Department of Revenue of Ore.} v. \emph{ACF Industries, Inc.,} 510 U.~S. 332, 336 (1994). The 4--R Act prohibits four separate forms of discriminatory state taxation of railroads.\footnotemark[2] Only the first is at issue here: \newpage  States, the Act provides, may not ``[a]ssess rail transportation property at a value that has a higher ratio to the [property's] true market value\dots than the ratio'' between the assessed and true market values of other commercial and industrial property in the same taxing jurisdiction.49 U.~S.~C. \S~11501(b)(1). If the railroad ratio exceeds the ratio for other property by at least five percent, the district court may enjoin the tax. \S~11501(c).\footnotemark[3] \newpage 

\footnotetext[1]{The portion of the Act that concerns us here, \S~306, was originally codified at 49 U.~S.~C. \S~26c (1976 ed.). In 1978, Congress recodified it at 49 U.~S.~C. \S~11503 (1976 ed., Supp. II). Congress recodified it again in 1995, without substantive change, this time as \S~11501. For convenience, all references to the statute are to the text of \S~11501.}

\footnotetext[2]{Section 11501 reads, in relevant part:

``(b) The following acts unreasonably burden and discriminate against interstate commerce, and a State, subdivision of a State, or authority acting for a State or subdivision of a State may not do any of them:

``(1) Assess rail transportation property at a value that has a higher ratio to the true market value of the rail transportation property than the ratio that the assessed value of other commercial and industrial property in the same assessment jurisdiction has to the true market value of the other commercial and industrial property.

``(2) Levy or collect a tax on an assessment that may not be made    under paragraph (1) of this subsection.                                 

``(3) Levy or collect an ad valorem property tax on rail transportation property at a tax rate that exceeds the tax rate applicable to commercial and industrial property in the same assessment jurisdiction.

``(4) Impose another tax that discriminates against a rail carrier providing transportation subject to the jurisdiction of the Board under this part.''}

\footnotetext[3]{Section 11501(c) provides:

``Notwithstanding section 1341 of title 28 and without regard to the amount in controversy or citizenship of the parties, a district court of the United States has jurisdiction, concurrent with other jurisdiction of courts of the United States and the States, to prevent a violation of subsection (b) of this section. Relief may be granted under this subsection only if the ratio of assessed value to true market value of rail transportation property exceeds by at least 5 percent the ratio of assessed value to true market value of other commercial and industrial property in the same assessment jurisdiction. The burden of proof in determining assessed value and true market value is governed by State law. If the ratio of the assessed value of other commercial and industrial property in the assessment jurisdiction to the true market value of all other commercial and industrial property cannot be determined to the satisfaction of the district court through the random-sampling method known as a sales assessment ratio study (to be carried out under statistical principles applicable to such a study), the court shall find, as a violation of this section---

``(1) an assessment of the rail transportation property at a value that has a higher ratio to the true market value of the rail transportation prop \newpage  erty than the assessed value of all other property subject to a property tax levy in the assessment jurisdiction has to the true market value of all other commercial and industrial property; and

``(2) the collection of an ad valorem property tax on the rail transportation property at a tax rate that exceeds the tax ratio rate applicable to taxable property in the taxing district.''}


\noindent Petitioner CSX Transportation, Inc., is a freight rail carrier with multiple routes across the State of Georgia. As a consequence, it is subject to Georgia's ad valorem tax on real property. Under Georgia law, most commercial and industrial property is valued locally by county boards. Public utilities such as railroads, however, are initially valued by the State, which then certifies the proposed valuations to the county boards for adoption or alteration. In 2001, Georgia's State Board of Equalization, a respondent here, put CSX's ad valorem tax liability at \$4.6 million. A year later, the State's appraiser used a different combination of methodologies to determine the market value of CSX's in-state property.\footnotemark[4] The result was a significantly higher tax levy. The State estimated the railroad's 2002 market value at approximately \$7.8 billion, 472 F. 3d 1281, 1285 (CA11 2006), a 47 percent increase over the previous year. That brought the assessed value of CSX's Georgia property to \$514.9 million, for a final property tax bill of \$6.5 million. Brief for Petitioner 15.

  CSX filed suit in the United States District Court for the Northern District of Georgia, contending that the State's 2002 tax assessment violated the 4--R Act. The railroad alleged that Georgia had grossly overestimated the market \newpage  value of its in-state property while accurately valuing other commercial and industrial property in the State. The result, according to CSX, was that its rail property was taxed at a ratio of assessed-to-market value considerably more than five percent greater than the same ratio for the other property in the State.

\footnotetext[4]{Georgia assesses public utilities using the ``unit rule.'' Under this rule, ``an appraiser first determines the value of all assets of an entity, regardless of location,'' then multiplies ``by the percentage of the entity located within [the State] to determine what portion of the value of the company should be allocated to the state.'' 472 F. 3d 1281, 1283 (CA11 2006). The parties agree the unit rule is the appropriate rule for valuing CSX's property. There are, however, numerous methods available to value property under the unit rule, and many of these methods themselves have multiple variations.See \emph{id.,} at 1284.}

  To make its case, CSX submitted the testimony of its own expert appraiser, who relied on a combination of valuation methods different from those used by the appraiser for Georgia. The CSX appraiser calculated the 2002 market value of the railroad's property to be \$6 billion, not the \$7.8 billion figure used by the State. 472 F. 3d, at 1285--1286. CSX maintained that the state appraiser's valuation methodologies were flawed, and urged the District Court to accept the market value estimated by its expert as more accurate.

  The District Court refused to do so. Following a bench trial, the court ruled Georgia had not discriminated against CSX in violation of the 4--R Act because the State had used widely accepted valuation methods to arrive at its estimate of true market value. 448 F. Supp. 2d 1330, 1341 (ND Ga. 2005). In the judgment of the District Court, the Act ``does not generally allow a railroad to challenge the state's chosen methodology,'' as long as the State's methods are rational and not motivated by discriminatory intent. \emph{Ibid.}

  A divided panel of the Court of Appeals for the Eleventh Circuit affirmed. 472 F. 3d 1281. The majority reasoned that the ``text of the Act does not clearly state that railroads may challenge valuation methodologies,'' and that such a clear statement was required in light of the intrusion on state taxing prerogatives. \emph{Id.,} at 1289. Judge Fay dissented. \emph{Id.,} at 1292. Recognizing the division on this question among the Circuits, compare \emph{Consolidated Rail Corporation} v. \emph{Hyde Park,} 47 F. 3d 473, 481--482 (CA2 1995) (a railroad may challenge a State's valuation methodology), and \\Burlington Northern R. Co.} v. \emph{Department of Revenue of Wash.,} 23 F. 3d 239, 240--241 (CA9 1994) (same), with \emph{Chesa\newpage peake Western R. Co.} v. \emph{Forst,} 938 F. 2d 528, 531 (CA4 1991) (a railroad may not challenge a State's valuation methodology), and 472 F. 3d, at 1289 (case below), we granted certiorari, 550 U.~S. 968 (2007), and now reverse.

\section{II}

  ``[T]he language of \S~1150[1] plainly declares the congressional purpose.'' \emph{Burlington Northern,} 481 U. S., at 461. States may not tax railroad property at a ratio of assessedto-true-market value higher than the ratio for other commercial and industrial property in the same jurisdiction. In order to apply the Act, district courts must calculate the true market value of in-state railroad property. A court cannot undertake the comparison of ratios the statute requires without that figure at hand. We said as much in \emph{Burlington Northern\\: ``It is clear from [the Act's] language that in order to compare the actual assessment ratios, it is necessary to determine what the ‘true market values' are.''\\Ibid.}

  We do not see how a court can go about determining true market value if it may not look behind the State's choice of valuation methods. Georgia insists there is a clear and important distinction between valuation methodologies and their application. As the State would have it, the statute allows courts to question only the latter. We find no distinction between method and application in the language of the Act, and see no passage limiting district court factfinding in the manner the State proposes. The total lack of textual support for Georgia's position is not surprising. The dichotomy the State presses would eviscerate the statute by forcing courts to defer to the valuation estimate of the State, when discriminatory taxation by States was the very evil the Act aimed to ban.

  Georgia's position is untenable given the way market value is calculated. Valuation is not a matter of mathematics, as if the district court could prevent discriminatory taxation \newpage  simply by doublechecking the State's assessment equations. Rather, the calculation of true market value is an applied science, even a craft. Most appraisers estimate market value by employing not one methodology but a combination. These various methods generate a range of possible market values which the appraiser uses to derive what he considers to be an accurate estimate of market value, based on careful scrutiny of all the data available. Appraisal Institute, The Appraisal of Real Estate 49--50 (12th ed. 2001).

  Georgia's appraiser in the instant case, for example, used three different valuation techniques---the discounted cashflow approach, a market multiple approach, and a stock and debt approach. He derived five values from these three methods, ranging from \$8.126 billion to \$12.346 billion. After selecting a number at the low end of the range and then subtracting another \$400 million to account for intangible property not subject to ad valorem taxation, he settled on \$7.8 billion as his final estimate of the true market value.472 F. 3d, at 1284--1285.

  Appraisers typically employ a combination of methods because no one approach is entirely accurate, at least in the absence of an established market for the type of property at issue. The individual methods yield sometimes more, sometimes less reliable results depending on the peculiar features of the property evaluated. As the variation in the state appraiser's market-value range reveals, different methods can produce substantially different estimates. W. Kinnard, Income Property Valuation: Principles and Techniques of Appraising Income-Producing Real Estate 52 (1971).

  Given the extent to which the chosen methods can affect the determination of value, preventing courts from scrutinizing state valuation methodologies would render \S~11501 a largely empty command. It would force district courts to accept as ``true'' the market value estimated by the State, one of the parties to the litigation. States, in turn, would \newpage  be free to employ appraisal techniques that routinely overestimate the market worth of railroad assets. By then levying taxes based on those overestimates, States could implement the very discriminatory taxation Congress sought to eradicate. On Georgia's reading of the statute, courts would be powerless to stop them, and the Act would ultimately guarantee railroads nothing more than mathematically accurate discriminatory taxation. We do not find this interpretation compelling. Instead, we agree with Judge Fay in dissent below: ``Since the objective of any methodology is a determination of \\true market value,} a railroad should be allowed to challenge the method[s] used [by the State] in an attempt to prove that the result\dots was not the \emph{true market value} of its property.'' 472 F. 3d, at 1294.

  The State agrees that it may not be possible to fix true market value with any precision. But it draws a different conclusion from this premise. Because any number of estimates are plausible, Georgia argues, the court is as likely to get an accurate result by verifying the application of the State's methods---so long as they are broadly reasonable---as it is by employing another method altogether. The State warns that allowing railroads to introduce their own valuation estimates based on different methodologies will inevitably lead to a futile clash of experts, which courts will have no reasonable way to settle. At least one of the Courts of Appeals shares this concern. See \emph{Chesapeake Western,} 938 F. 2d, at 532 (``There is no absolute way to test the assertions of competing valuations .~.~.~'' (internal quotation marks and brackets omitted)).

  Congress was not similarly troubled. It directed courts to find true market value, however elusive. It made that value the objective benchmark for courts' evaluation of state taxes on railroad property. True market value may well not be a single, precise number, but Congress obviously believed it was susceptible to judicial inquiry and that some approximations were better than others. \newpage 


  Georgia's grim prophecies notwithstanding, the inquiry the statute mandates is not unfamiliar to courts. Valuation of property, though admittedly complex, is at bottom just ``an issue of fact about possible market prices,'' ,\emph{Suitum} v. \emph{Tahoe Regional Planning Agency,} 520 U.~S. 725, 741 (1997) an issue district courts are used to addressing. Railroad property is not frequently sold, but ``determinations of market value are routinely made in judicial proceedings without the benefit of a market transaction.'' \emph{Id.,} at 742.The District Court in this case made clear that it knew how to find true market value: ``In a more typical case, the court would look at both [the railroad expert's] appraisal and [the State's] appraisal to determine the true market value of [the railroad].'' 448 F. Supp. 2d, at 1338, n. 8.It refused to do so not because true market value is inherently elusive, but because it believed the Act did not allow it to question the State's methods.

  In light of the statute's directive making true market value a factual question to be determined by the district court, what Georgia is really asking for is a limitation on the types of evidence courts may consider as part of their factual inquiry. If Congress had wanted to impose such a limit by reserving to States the prerogative of selecting which valuation methods may be used, it surely could have done so. Out of deference to the States, for example, \S~11501(c) provides that ``[t]he burden of proof in determining\dots true market value [shall be] governed by State law.'' Congress could easily have included similar language insulating the State's chosen methodologies from judicial scrutiny. It did not. Like Oklahoma's argument in \\Burlington Northern,} Georgia's position in this case ultimately ``depends upon the addition of words to a statutory provision which is complete as it stands.'' 481 U. S., at 463.We decline to find distinctions in the statute where they do not exist, especially where, as here, those distinctions would thwart the law's operation. \newpage 

\section{III}

  Considering the clarity of the statute, we are tempted to leave the discussion at that. ``When we find the terms of a statute unambiguous, judicial inquiry is complete .~.~.~.'' \emph{Rubin} v. \emph{United States,} 449 U.~S. 424, 430 (1981).Georgia, however, lodges two objections to our interpretation, each of which merits a reply. First, the State argues that any interpretation of the Act allowing courts to question state valuation methods ignores the background principles of federalism against which the statute was enacted. The majority below expressed a similar concern. ``The selection of a valuation methodology,'' it ruled, ``is part of th[e] fundamental power of a state [to tax],'' ,472 F. 3d, at 1288 and should not be limited absent a clear statement from Congress. We have long held that the means States adopt to collect their taxes ``should be interfered with as little as possible.'' \emph{Dows} v. \emph{Chicago,} 11 Wall. 108, 110 (1871).But we are persuaded that allowing railroads to challenge a State's valuation methodologies has been clearly authorized by the terms of the 4--R Act.

  As an initial matter, we question Georgia's contention that its selection of valuation methodologies is an important state policy choice intimately connected to its tax power. Georgia does not prescribe any particular methodology as a matter of state law. Its appraisers use different methodologies in different combinations, as they see fit. See 472 F. 3d, at 1284--1285 (explaining that the state appraiser employed multiple methods and selected a value according to his best judgment). This suit, in fact, is the result of an individual appraiser's decision to employ a different combination of assessment techniques than that used by his immediate predecessors. The methods he selected were his choice, not the dictate of any state statute or regulation. \emph{Ibid.}

  But even if important questions of state policy are, as the Eleventh Circuit believed, ``intertwined with the selection of a valuation methodology,'' ,\emph{id.,} at 1288 judicial scrutiny of \newpage  those methodologies is authorized by the 4--R Act's clear command to find true market value. As we explained above, the power to calculate true market value necessarily includes the power to look behind a State's valuation methods. That the statute should vest this authority in the Nation's courts is hardly surprising, given Congress's conclusion that the States were assessing railroad property unfairly.

  Our decision in, \emph{Department of Revenue of Ore.} v. \emph{ACF} \\Industries, Inc.,} 510 U.~S. 332 (1994) is not to the contrary. That case concerned a different provision of the 4--R Act---namely, the command in \S~11501(b)(4) preventing a State from ``[i]mpos[ing] another tax that discriminates against a rail carrier providing transportation'' in the taxing jurisdiction. This bar on facially discriminatory taxes, we held, did not prevent a State from exempting certain nonrailroad property from otherwise generally applicable ad valorem taxes. \emph{Id.,} at 343.At the time the 4--R Act was adopted, a majority of States exempted one or more classes of business property from ad valorem taxation, ``including business inventories, raw materials used in textile manufacturing,\dots and mechanics tools,'' to name just a few. \emph{Id.,} at 344.The States had provided such property tax exemptions for years. In the face of this widespread and historical practice, we declined to read the 4--R Act to prohibit a type of tax exemption the text did not expressly mention. \emph{Ibid.}

  By contrast, we pointedly noted that the Act ``prohibit[s] discriminatory tax rates and assessment ratios in no uncertain terms\dots and set[s] forth precise standards for judicial scrutiny of challenged rate and assessment practices.'' \emph{Id.,} at 343. Georgia's claim that court review of state valuation methodologies is not authorized by a clear statement in the Act ignores the statute's explicit prohibition of discriminatory assessment ratios. A district court cannot accurately calculate or compare those ratios without determining true market value. Congress clearly permitted courts to question state valuation methodologies when it banned discrimi\newpage natory assessment ratios and made true market value a question to be litigated in federal court.

  Georgia also protests that our interpretation will destroy the States' discretion to choose their own valuation methodologies. We disagree. A State may use whatever method or methods it likes, so long as the result is not discriminatory. The Act does not prohibit the use of any valuation methodology. It prohibits discrimination. Far from requiring States to follow a particular method, we hold only that nothing in the statute prevents a railroad from attempting to show that the methods chosen by the State result in a discriminatory determination of true market value.

  The judgment of the Court of Appeals for the Eleventh Circuit is reversed.

\begin{flushright}\emph{It is so ordered.}\end{flushright}
