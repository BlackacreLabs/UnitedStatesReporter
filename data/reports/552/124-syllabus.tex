Type: Syllabus
Author: Reporter of Decisions

\setcounter{page}{214}

  The Federal Tort Claims Act (FTCA) waives the United States' sovereign immunity for claims arising out of torts committed by federal employees, see 28 U.~S.~C. \S~1346(b)(1), but, as relevant here, exempts from that waiver ``[a]ny claim arising in respect of the assessment or collection of any tax or customs duty, or the detention of any\dots property by any officer of customs or excise or any other law enforcement officer,'' \S~2680(c). Upon his transfer from an Atlanta federal prison to one in Kentucky, petitioner noticed that several items were missing from his personal property, which had been shipped to the new facility by the Federal Bureau of Prisons (BOP). Alleging that BOP officers had lost his property, petitioner filed this suit under, \emph{inter alia,} the FTCA, but the District Court dismissed that claim as barred by \S~2680(c). Affirming, the Eleventh Circuit rejected petitioner's argument that the statutory phrase ``any officer of customs or excise or any other law enforcement officer'' applies only to officers enforcing customs or excise laws.

\emph{Held:}

  Section 2680(c)'s text and structure demonstrate that the broad phrase ``any other law enforcement officer'' covers all law enforcement officers. Petitioner's argument that \S~2680(c) is focused on preserving sovereign immunity only for officers enforcing customs and excise laws is inconsistent with the statute's language. ``Read naturally, the word ‘any' has an expansive meaning, that is, ‘one or some indiscriminately of whatever kind.''' \emph{United States} v. \emph{Gonzales,} 520 U.~S. 1, 5. For example, in considering a provision imposing an additional sentence that was not to run concurrently with ``any other term of imprisonment,'' 18 U.~S.~C. \S~924(c)(1), the \emph{Gonzales} Court held that, notwithstanding the subsection's initial reference to federal drug trafficking crimes, the expansive word ``any'' and the absence of restrictive language left ``no basis in the text for limiting'' the phrase ``any other term of imprisonment'' to federal sentences. 520 U. S., at 5. To similar effect, see \emph{Harrison} v. \emph{PPG Industries, Inc.,} 446 U. S. 578, 588--589, in which the Court held that there was ``no indication whatever that Congress intended'' to limit the ``expansive language'' `` ‘any other final action' '' to particular kinds of agency action. The reasoning of \emph{Gonzales} and \emph{Harrison} applies equally to 28 U.~S.~C. \S~2680(c): Congress' use of ``any'' to modify ``other law enforcement officer'' is most naturally read to mean \newpage  ences to ``tax or customs duty'' and ``officer[s] of customs or excise'' indicate an intent to preserve immunity for claims arising from an officer's enforcement of tax and customs laws. The text also indicates, however, that Congress intended to preserve immunity for claims arising from the detention of property, and there is no indication of any intent that immunity for those claims turns on the type of law being enforced. Recent amendments to \S~2680(c) restoring the sovereign immunity waiver for officers enforcing \emph{any} federal forfeiture law, see \S~2680(c)(1), support the Court's conclusion by demonstrating Congress' view that, prior to the amendments, \S~2680(c) covered \emph{all} law enforcement officers. Against this textual and structural evidence, petitioner's reliance on the canons of statutory construction \emph{ejusdem generis} and \emph{noscitur a sociis} and on the rule against superfluities is unconvincing. The Court is unpersuaded by petitioner's attempt to create ambiguity where the statute's structure and text suggest none. Had Congress intended to limit \S~2680(c)'s reach as petitioner contends, it easily could have written ``any other law enforcement officer \emph{acting in a customs or excise capacity.}'' Instead, it used the unmodified, all-encompassing phrase ``any other law enforcement officer.'' This Court must give effect to the text Congress enacted.Pp. 217--228.

204 Fed. Appx. 778, affirmed.

  \textsc{Thomas,} J., delivered the opinion of the Court, in which \textsc{Roberts,} C. J., and \textsc{Scalia, Ginsburg,} and \textsc{Alito,} JJ., joined. \textsc{Kennedy, J.,} filed a dissenting opinion, in which \textsc{Stevens, Souter,} and \textsc{Breyer, JJ.,} joined, \emph{post,} p. 228. \textsc{Breyer,} J., filed a dissenting opinion, in which \textsc{Stevens,} J., joined, \emph{post,} p. 243.

  \emph{Jean-Claude André´,} by appointment of the Court, 551 U. S. 1186, argued the cause for petitioner. With him on the briefs were \emph{Michael G. Smith, Peter K. Stris, Shaun P. Martin,} and \emph{Brendan Maher.}

  \emph{Kannon K. Shanmugam} argued the cause for respondents. With him on the brief were \emph{Solicitor General Clement, Assistant Attorney General
Keisler, Deputy Solicitor General Kneedler,} and \emph{Mark B. Stern.}
