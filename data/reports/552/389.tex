% Opinion of the Court
% Kennedy

\setcounter{page}{392}

  \textsc{Justice Kennedy} delivered the opinion of the Court.

  This case arises under the Age Discrimination in Employment Act
of 1967 (ADEA or Act), 81 Stat. 602, as amended, \newpage  29 U. S.
C. \S621 \emph{et seq.} When an employee files ``a charge alleging
unlawful [age] discrimination'' with the Equal Employment Opportunity
Commission (EEOC), the charge sets the Act's enforcement mechanisms
in motion, commencing a waiting period during which the employee cannot
file suit. The phrase, ``a charge alleging unlawful discrimination,''
is used in the statute, \S~626(d), and ``charge'' appears in the
agency's implementing regulations; but it has no statutory definition.
In deciding what constitutes a charge under the Act the Courts of
Appeals have adopted different definitions. As a result, difficulties
have arisen in determining when employees may seek relief under the ADEA
in courts of competent jurisdiction.

  As a cautionary preface, we note that the EEOC enforcement mechanisms
and statutory waiting periods for ADEA claims differ in some respects
from those pertaining to other statutes the EEOC enforces, such as Title
VII of the Civil Rights Act of 1964, 78 Stat. 253, as amended, 42
U.~S.~C. \S~2000e \emph{et seq.,} and theAmericans with Disabilities
Act of 1990, 104 Stat. 327, as amended, 42 U.~S.~C. \S~12101 \emph{et
seq.} While there may be areas of common definition, employees and
their counsel must be careful not to apply rules applicable under one
statute to a different statute without careful and critical examination.
Cf. \emph{General Dynamics Land Systems, Inc.} v. \emph{Cline,} 540 U. S.
581, 586--587 (2004). This is so even if the EEOC forms and the same
definition of charge apply in more than one type of discrimination case.

\section{I}

  Petitioner, Federal Express Corporation (FedEx), provides mail pickup
and delivery services to customers worldwide. In 1994 and 1995, FedEx
initiated two programs, designed, it says, to make its 45,000-strong
courier network more productive. The programs, ``Best Practice Pays''
(BPP) and ``Minimum Acceptable Performance Standards'' \newpage  (MAPS),
tied the couriers' compensation and continued employment to certain
performance benchmarks, for instance the number of stops a courier makes
per day.

  Respondents are 14 current and former FedEx couriers over the age of
40. They filed suit in the United States District Court for the Southern
District of New York on April 30, 2002, claiming, \emph{inter alia,}
that BPP and MAPS violate the ADEA. Asserting that their claims were
typical of many couriers nationwide, respondents sought to represent a
plaintiffs' class of all couriers over the age of 40 who were subject
to alleged acts of age discrimination by FedEx. The suit maintains that
BPP and MAPS were veiled attempts to force older workers out of the
company before they would be entitled to receive retirement benefits.
FedEx, it is alleged, used the initiatives as a pretext for harassing
and discriminating against older couriers in favor of younger ones.

  The immediate question before us is the timeliness of the suit filed
by one of the plaintiffs below, Patricia Kennedy, referred to here as
``respondent.'' Petitioner moved to dismiss respondent's action,
contending respondent had not filed her charge with the EEOC at least
60 days before filing suit, as required by 29 U.~S.~C. \S~626(d).
Respondent countered that she filed a valid charge on December 11, 2001,
by submitting EEOC Form 283.

  The agency labels Form 283 an ``Intake Questionnaire.'' Respondent
attached to the questionnaire a signed affidavit describing the alleged
discriminatory employment practices in greater detail. The District
Court determined these documents were not a charge and granted the
motion to dismiss. No. 02 Civ. 3355(LMM) (SDNY, Oct. 9, 2002), App.
to Pet. for Cert. 39a. An appeal followed, and the Court of Appeals
for the Second Circuit reversed. See 440 F. 3d 558, 570 (2006).
We granted certiorari to consider whether respondent's filing was a
charge, 551 U.~S. 1102 (2007), and we now affirm. \newpage 

\section{II}

  This case presents two distinct questions: What is a charge as the
ADEA uses that term? And were the documents respondent filed in December
2001 a charge?

\subsection{A}

  The relevant statutory provision states:

      ``No civil action may be commenced by an individual under
    [the ADEA] until 60 days after a charge alleging unlawful
    discrimination has been filed with the Equal Employment Opportunity
    Commission~.~.~.~.

    * * *

    ``Upon receiving such a charge, the Commission shall promptly
    notify all persons named in such charge as pro spective defendants
    in the action and shall promptly seek to eliminate any alleged
    unlawful practice by infor mal methods of conciliation, conference,
    and persuasion.''29 U.~S.~C. \S~626(d).

  The Act does not define charge. While EEOC regulations give some
content to the term, they fall short of a comprehensive definition. The
agency has statutory authority to issue regulations, see \S~628; and
when an agency invokes its authority to issue regulations, which then
interpret ambiguous statutory terms, the courts defer to its reasonable
interpretations.See \emph{Chevron U. S. A. Inc.} v. \emph{Natural Resources}
\emph{Defense Council, Inc.,} 467 U.~S. 837, 843--845 (1984). The
regulations the agency has adopted---so far as they go---are reasonable
constructions of the term charge. There is little dispute about this.
The issue is the guidance the regulations give.

  One of the regulations, 29 CFR \S~1626.3 (2007), is entitled
``Other definitions.'' It says: ``\emph{charge} shall mean a statement
filed with the Commission by or on behalf of an aggrieved person which
alleges that the named prospective defendant \newpage  has engaged in
or is about to engage in actions in violation of the Act.'' Section
1626.8(a) identifies five pieces of information a ``charge should
contain'': (1)--(2) the names, addresses, and telephone numbers of
the person making the charge and the charged entity; (3) a statement
of facts describing the alleged discriminatory act; (4) the number
of employees of the charged employer; and (5) a statement indicating
whether the charging party has initiated state proceedings. The
next subsection, \S~1626.8(b), however, seems to qualify these
requirements by stating that a charge is ``sufficient'' if it meets
the requirements of \S~1626.6---\\i. e.,} if it is ``in writing
and\dots name[s] the prospective respondent and\dots generally
allege[s] the discriminatory act(s).''

  Even with the aid of the regulations the meaning of charge remains
unclear, as is evident from the differing positions of the parties now
before us and in the Courts of Appeals. Petitioner contends an Intake
Questionnaire cannot be a charge unless the EEOC acts upon it. On the
other hand some Courts of Appeals, including the Court of Appeals for
the Second Circuit, take a position similar to the Government's in
this case, that an Intake Questionnaire can constitute a charge if it
expresses the filer's intent to activate the EEOC's enforcement
processes. See, \emph{e. g., Steffen} v. \emph{Meridian Life Ins. Co.,} 859
F. 2d 534, 542 (CA7 1988). A third view, which seems to accord with
respondent's position, is that all completed Intake Questionnaires
are charges.See, \emph{e. g., Casavantes} v. \emph{California State Univ.,
Sacramento,} 732 F. 2d 1441, 1443 (CA9 1984).

\subsection{B}

  In support of her position that the Intake Questionnaire she filed,
taken together with the attached six-page affidavit, meets the
regulatory definition of a charge, respondent places considerable
emphasis on what might be described as the regulations' catchall or
saving provision, 29 CFR \S~1626.8(b). This seems to require only
a written document \newpage  with a general allegation of discriminatory
conduct by a named employer. Respondent points out that, when
read together, \S\S~1626.8(b) and 1626.6 say that a ``charge is
sufficient when the Commission receives\dots awritten statement''
that ``name[s] the [employer] and\dots generally allege[s] the
discriminatory act(s).'' Respondent views this language as unequivocal
and sees no basis for requiring that a charge contain any additional
information.

  The EEOC's view, as expressed in the Government's \emph{amicus}
brief, however, is that the regulations identify certain requirements
for a charge but do not provide an exhaustive definition. As such, not
all documents that meet the minimal requirements of \S~1626.6 are
charges.

  The question, then, becomes how to interpret the scope of
the regulations. Just as we defer to an agency's reasonable
interpretations of the statute when it issues regulations in the first
instance, see \emph{Chevron, supra,} the agency is entitled to further
deference when it adopts a reasonable interpretation of regulations it
has put in force.See \emph{Auer} v. \emph{Robbins,} 519 U.~S. 452 (1997).
Under \emph{Auer,} we accept the agency's position unless it is `` ‘
``plainly erroneous or inconsistent with the regulation.'' ' ''
\emph{Id.,} at 461 (quoting \emph{Robertson} v. \emph{Methow Valley Citizens
Council,} 490 U.~S. 332, 359 (1989)).

  In accord with this standard we accept the agency's position that
the regulations do not identify all necessary components of a charge;
and it follows that a document meeting the requirements of \S~1626.6
is not a charge in every instance. The language in \S\S~1626.6 and
1626.8 cannot be viewed in isolation from the rest of the regulations.
True, the structure of the regulations is less than clear. But the
relevant provisions are grouped under the title, ``Procedures---Age
Discrimination in Employment Act.'' A permissible reading is that the
regulations identify the procedures for filing a charge but do not state
the full contents a charge document must contain. This is the agency's
position, and we defer to it under \emph{Auer.} \newpage 


\subsection{C}

  This does not resolve the case. While we agree with the Government
that the regulations do not state all the elements a charge must
contain, the question of what additional elements are required remains.
On this point the regulations are silent.

  The EEOC submits that the proper test for determining whether a
filing is a charge is whether the filing, taken as a whole, should be
construed as a request by the employee for the agency to take whatever
action is necessary to vindicate her rights. Brief for United States
as \emph{Amicus Curiae} 15. The EEOC has adopted this position in the
Government's \emph{amicus} brief and in various internal directives it
has issued to its field offices over the years. See 1 EEOC Compliance
Manual \S~2.2(b), p. 2:0001 (Aug. 2002); Memorandum from Elizabeth M.
Thornton, Director, Office of Field Programs, EEOC, to All District,
Area, and Local Office Directors et al. (Feb. 21, 2002), online at
http://www.eeoc.gov/charge/memo2-21-02.html (hereinafter Thornton
Memo) (all Internet materials as visited Feb. 21, 2008, and available
in Clerk of Court's case file); Memorandum from Nicholas M. Inzeo,
Director, Office of Field Programs, EEOC, to All District, Field,
Area, and Local Office Directors et al. (Aug. 13, 2007), online at
http://www.eeoc.gov/charge/memo-8-13-07.html. The Government asserts
that this request-to-act requirement is a reasonable extrapolation
of the agency's regulations and that, as a result, the agency's
position is dispositive under \emph{Auer.}

  The Government acknowledges the regulations do not, on their face,
speak to the filer's intent. To the extent the request-to-act
requirement can be derived from the text of the regulations, it
must spring from the term charge. But, in this context, the term
charge is not a construct of the agency's regulations. It is a term
Congress used in the underlying statute that has been incorporated
in the regulations by the agency. Thus, insofar as they speak to the
fil\newpage er's intent, the regulations do so by repeating language
from the underlying statute. It could be argued, then, that this case
can be distinguished from \emph{Auer.\\See \emph{Gonzales} v. \emph{Oregon,}
546 U.~S. 243, 257 (2006) (the ``near equivalence of the statute and
regulation belies [the case for] \emph{Auer} deference''); \emph{Christensen}
v. \emph{Harris County,} 529 U.~S. 576, 588 (2000) (an agency cannot
\emph{``\\under the guise of interpreting a regulation\dots create \emph{de
facto} a new regulation'').

  It is not necessary to hold that \emph{Auer} deference applies to
the agency's construction of the term charge as it is used in the
regulations, however. For even if \emph{Auer} deference is inapplicable, we
would accept the agency's proposed construction of the statutory term,
and we turn next to the reasons for this conclusion.

\subsection{D}

  In our view the agency's policy statements, embodied in its
compliance manual and internal directives, interpret not only the
regulations but also the statute itself. Assuming these interpretive
statements are not entitled to full \emph{Chevron} deference, they
do reflect `` ‘a body of experience and informed judgment to
which courts and litigants may properly resort for guidance.' ''
\emph{Bragdon} v. \emph{Abbott,} 524 U.~S. 624, 642 (1998) (quoting
\emph{Skidmore} v. \emph{Swift\&Co.,} 323 U.~S. 134, 139--140 (1944)). As
such, they are entitled to a ``measure of respect'' under the less
deferential \emph{Skidmore} standard.\\Alaska Dept. of Environmental
Conservation} v. \emph{EPA,} 540 U.~S. 461, 487, 488 (2004); \emph{United
States} v. \emph{Mead Corp.,} 533 U.~S. 218, 227--239 (2001).

  Under \emph{Skidmore,} we consider whether the agency has applied its
position with consistency. \emph{Mead Corp., supra,} at 228; \emph{Good
Samaritan Hospital} v. \emph{Shalala,} 508 U.~S. 402, 417 (1993). Here,
the relevant interpretive statement, embodied in the compliance manual
and memoranda, has been binding on EEOC staff for at least five years.
See Thornton Memo, \emph{supra.} True, as the Government concedes, the
agency's implementation of this policy has been uneven. See Brief
for \newpage  United States as \emph{Amicus Curiae} 25. In the very case
before us the EEOC's Tampa field office did not treat respondent's
filing as a charge, as the Government now maintains it should have done.
And, as a result, respondent filed suit before the agency could initiate
a conciliation process with the employer.

  These undoubted deficiencies in the agency's administration of the
statute and its regulatory scheme are not enough, however, to deprive
the agency of all judicial deference. Some degree of inconsistent
treatment is unavoidable when the agency processes over 175,000
inquiries a year. \emph{Id.,} at 19, n. 10. And although one of the
policy memoranda the Government relies upon was circulated after we
granted certiorari, the position the document takes is consistent
with the EEOC's previous directives. We see no reason to assume the
agency's position---that a charge is filed when the employee requests
some action---was framed for the specific purpose of aiding a party in
this litigation.Cf. \emph{Bowen} v. \emph{Georgetown Univ. Hospital,} 488
U.~S. 204, 212--213 (1988).

  The EEOC, moreover, has drawn our attention to the need to define
charge in a way that allows the agency to fulfill its distinct statutory
functions of enforcing antidiscrimination laws and disseminating
information about those laws to the public. Cf. \emph{Barnhart} v.
\emph{Walton,} 535 U.~S. 212, 225 (2002) (noting that deference is
appropriate in ``matters of detail related to [an agency's]
administration'' of a statute). The agency's duty to initiate
informal dispute resolution processes upon receipt of a charge is
mandatory in the ADEA context. See 29 U.~S.~C. \S~626(d) (``[T]he
Commission\dots shall promptly seek to eliminate any alleged
unlawful practice by informal methods of conciliation, conference, and
persuasion''); Cf. \emph{Lopez} v. \emph{Davis,} 531 U.~S. 230, 241 (2001)
(noting that Congress' use of the term `` ‘shall' '' indicates
an intent to ``impose discretionless obligations''). Yet, at the
same time, Congress intended the agency to serve an ``educational''
function. See Civil Rights Act of 1964, \S~705(i), 78 Stat. 259;
\newpage  \emph{id.,} \S~705(g)(3) (noting that the Commission shall
have the power to ``furnish to persons subject to this title such
technical assistance as they may request''). Providing answers to the
public's questions is a critical part of the EEOC's mission; and it
accounts for a substantial part of the agency's work. Of about 175,000
inquiries the agency receives each year, it dockets around 76,000 of
these as charges. Brief for United States as \emph{Amicus Curiae} 19,
n. 10. Even allowing for errors in the classification of charges and
noncharges, it is evident that many filings come from individuals who
have questions about their rights and simply want information.

  For efficient operations, and to effect congressional intent, the
agency requires some mechanism to separate information requests from
enforcement requests. Respondent's proposed standard, that a charge
need contain only an allegation of discrimination and the name of the
employer, falls short in this regard. Were that stripped-down standard
to prevail, individuals who approach the agency with questions could end
up divulging enough information to create a charge. This likely would
be the case for anyone who completes an Intake Questionnaire---which
provides space to indicate the name and address of the offending
employer and asks the individual to answer the question, ``What action
was taken against you that you believe to be discrimination?'' App.
to Pet. for Cert. 43a. If an individual knows that reporting this
minimal information to the agency will mandate the agency to notify her
employer, she may be discouraged from consulting the agency or wait
until her employment situation has become so untenable that conciliation
efforts would be futile. The result would be contrary to Congress'
expressed desire that the EEOC act as an information provider and try to
settle employment disputes through informal means.

  For these reasons, the definition of charge respondent
advocates---\\i. e.,} that it need conform only to 29 CFR \S~1626.6---
is in considerable tension with the structure and purposes of the ADEA.
The agency's interpretive position---the \newpage  request-to-act
requirement---provides a reasonable alternative that is consistent
with the statutory framework. No clearer alternatives are within our
authority or expertise to adopt; and so deference to the agency is
appropriate under \emph{Skidmore.} We conclude as follows: In addition to
the information required by the regulations, \emph{i. e.,} an allegation
and the name of the charged party, if a filing is to be deemed a charge
it must be reasonably construed as a request for the agency to take
remedial action to protect the employee's rights or otherwise settle a
dispute between the employer and the employee.

  Some Courts of Appeals have referred to a `` ‘manifest intent'
'' test, under which, in order to be deemed a charge, the filing must
demonstrate ``an individual's intent to have the agency initiate its
investigatory and conciliatory processes.'' 440 F. 3d, at 566 (case
below); see also \emph{Wilkerson} v. \emph{Grinnell orp.\\, 270 F. 3d 1314,
1319 (CA11 2001); \emph{Steffen,} 859 F. 2d, at 543; \emph{Bihler} v. \emph{Singer
Co.,} 710 F. 2d 96, 99 (CA3 1983). If this formulation suggests the
filer's state of mind is somehow determinative, it misses the point.
If, however, it means the filing must be examined from the standpoint of
an objective observer to determine whether, by a reasonable construction
of its terms, the filer requests the agency to activate its machinery
and remedial processes, that would be in accord with our conclusion.

  It is true that under this permissive standard a wide range of
documents might be classified as charges. But this result is consistent
with the design and purpose of the ADEA. Even in the formal litigation
context, \emph{pro se} litigants are held to a lesser pleading standard
than other parties.See \emph{Estelle} v. \emph{Gamble,} 429 U.~S. 97, 106
(1976) (\\Pro se} pleadings are to be ``liberally construed'').
In the administrative context now before us it appears \emph{pro se}
filings may be the rule, not the exception. The ADEA, like Title VII,
sets up a ``remedial scheme in which laypersons, rather than lawyers,
are expected to initiate the process.'' \emph{EEOC} v. \emph{Commercial
Of\newpage fice Products Co.,} 486 U.~S. 107, 124 (1988); see also
\emph{Oscar Mayer \& Co.} v. \emph{Evans,} 441 U.~S. 750, 756 (1979) (noting
the ``common purpose'' of Title VII and the ADEA). The system
must be accessible to individuals who have no detailed knowledge of
the relevant statutory mechanisms and agency processes. It thus is
consistent with the purposes of the Act that a charge can be a form,
easy to complete, or an informal document, easy to draft. The agency's
proposed test implements these purposes.

  Reasonable arguments can be made that the agency should adopt a
standard giving more guidance to filers, making it clear that the
request to act must be stated in quite explicit terms. A rule of that
sort might yield more consistent results. This, however, is a matter
for the agency to decide in light of its experience and expertise in
protecting the rights of those who are covered by the Act. For its
decisions in this regard the agency is subject to the oversight of
the political branches.Cf. \emph{National Cable \& Telecommunications}
\emph{Assn.} v. \emph{Brand X Internet Services,} 545 U.~S. 967, 980 (2005)
(``Filling these gaps [in ambiguous statutes] involves difficult policy
choices that agencies are better equipped to make than courts'').
We find no reason in this case to depart from our usual rule: Where
ambiguities in statutory analysis and application are presented, the
agency may choose among reasonable alternatives.

\subsection{E}

  Asserting its interest as an employer, petitioner urges us to
condition the definition of charge, and hence an employee's ability
to sue, upon the EEOC's fulfilling its mandatory duty to notify the
charged party and initiate a conciliation process. In petitioner's
view, because the Commission must act ``[u]pon receiving such a
charge,'' 29 U.~S.~C. \S~626(d), its failure to do so means the
filing is not a charge.

  The agency rejects this view, as do we. As a textual matter, the
proposal is too artificial a reading of the statute to accept. The
statute requires the aggrieved individual to file \newpage  a charge
before filing a lawsuit; it does not condition the individual's right
to sue upon the agency taking any action. \emph{Ibid.} (``No civil
action may be commenced by an individual under [the ADEA] until 60
days after a charge alleging unlawful discrimination has been filed
with the Equal Employment Opportunity Commission''); Cf. \emph{Edelman}
v. \emph{Lynchburg College,} 535 U.~S. 106, 112--113 (2002) (rejecting
the argument that a charge is not a charge until the filer satisfies
Title VII's oath or affirmation requirement). The filing of a
charge, moreover, determines when the Act's time limits and procedural
mechanisms commence. It would be illogical and impractical to make the
definition of charge dependent upon a condition subsequent over which
the parties have no control.Cf. \emph{Logan} v. \emph{Zimmerman Brush Co.,}
455 U.~S. 422, 444 (1982) (Powell, J., concurring in judgment).

\section{III}

  Having determined that the agency acted within its authority in
formulating the rule that a filing is deemed a charge if the document
reasonably can be construed to request agency action and appropriate
relief on the employee's behalf, the question is whether the filing
here meets this test. The agency says it does, and we agree. The
agency's determination is a reasonable exercise of its authority to
apply its own regulations and procedures in the course of the routine
administration of the statute it enforces.

  Respondent's completed intake form contained all of the information
outlined in 29 CFR \S~1626.8, including: the employee's name,
address, and telephone number, as well as those of her employer; an
allegation that she and other employees had been the victims of ``age
discrimination''; the number of employees who worked at the Dunedin,
Florida, facility where she was stationed; and a statement indicating
she had not sought the assistance of any government agency regarding
this matter.See App. 265.\newpage 

  Petitioner maintains the filing was still deficient because
it contained no request for the agency to act. Were the Intake
Questionnaire the only document before us we might agree its handwritten
statements do not request action. The design of the form in use in
2001, moreover, does not give rise to the inference that the employee
requests action against the employer. Unlike EEOC Form 5, the Intake
Questionnaire is not labeled a ``Charge of Discrimination,'' see
\emph{id.,} at 275. In fact the wording of the questionnaire suggests
the opposite: that the form's purpose is to facilitate ``pre-charge
filing counseling'' and to enable the agency to determine whether it
has jurisdiction over ``potential charges.'' \emph{Id.,} at 265.
There might be instances where the indicated discrimination is so clear
or pervasive that the agency could infer from the allegations themselves
that action is requested and required, but the agency is not required to
treat every completed Intake Questionnaire as a charge.

  In this case, however, the completed questionnaire filed in December
2001 was supplemented with a detailed six-page affidavit. At the end of
the last page, respondent asked the agency to ``[p]lease force Federal
Express to end their age discrimination plan so we can finish out our
careers absent the unfairness and hostile work environment created
within their application of \emph{Best Practice/High-Velocity Culture
Change.}'' \emph{Id.,} at 273. This is properly construed as a
request for the agency to act.

  Petitioner says that, in context, the statement is ambiguous. It
points to respondent's accompanying statement that ``I have been
given assurances by an Agent of the U. S. Equal Employment Opportunity
Commission that this Affidavit will be considered confidential by the
United States Government and will not be disclosed as long as the case
remains open unless it becomes necessary for the Government to produce
the affidavit in a formal proceeding.'' \emph{Id.,} at 266. Petitioner
argues that if respondent intended the affidavit to \newpage  be kept
confidential, she could not have expected the agency to treat it as
a charge. This reads too much into the assurance of nondisclosure.
Respondent did not request the agency to avoid contacting her employer.
She stated only her understanding that the affidavit itself would
be kept confidential. Even then, she gave consent for the agency to
disclose the affidavit in a ``formal proceeding.'' Furthermore,
respondent checked a box on the Intake Questionnaire giving consent
for the agency to disclose her identity to the employer. \emph{Id.,} at
265. Here the combination of the waiver and respondent's request
in the affidavit that the agency ``force'' the employer to stop
discriminating against her were enough to bring the entire filing within
the definition of charge we adopt here.

  Petitioner notes that respondent did file a Form 5 (a formal charge)
with the EEOC but only after she filed her complaint in the District
Court. This shows, petitioner argues, that respondent did not intend the
earlier December 2001 filing to be a charge; otherwise, there would have
been no reason for the later filing. What matters, however, is whether
the documents filed in December 2001 should be interpreted as a request
for the agency to act. Postfiling conduct does not nullify an earlier,
proper charge.

  Documents filed by an employee with the EEOC should be construed,
to the extent consistent with permissible rules of interpretation, to
protect the employee's rights and statutory remedies. Construing
ambiguities against the drafter may be the more efficient rule to
encourage precise expression in other contexts; here, however, the rule
would undermine the remedial scheme Congress adopted. It would encourage
individuals to avoid filing errors by retaining counsel, increasing both
the cost and likelihood of litigation.

\section{IV}

  The Federal Government interacts with individual citizens through
all but countless forms, schedules, manuals, and \newpage  worksheets.
Congress, in most cases, delegates the format and design of these
instruments to the agencies that administer the relevant laws and
processes. An assumption underlying the congressional decision to
delegate rulemaking and enforcement authority to the agency, and the
consequent judicial rule of deference to the agency's determinations,
is that the agency will take all efforts to ensure that affected parties
will receive the full benefits and protections of the law. Here, because
the agency failed to treat respondent's filing as a charge in the
first instance, both sides lost the benefits of the ADEA's informal
dispute resolution process.

  The employer's interests, in particular, were given short shrift,
for it was not notified of respondent's complaint until she filed
suit. The court that hears the merits of this litigation can attempt
to remedy this deficiency by staying the proceedings to allow an
opportunity for conciliation and settlement. True, that remedy would be
imperfect. Once the adversary process has begun a dispute may be in a
more rigid cast than if conciliation had been attempted at the outset.

  This result is unfortunate, but, at least in this case, unavoidable.
While courts will use their powers to fashion the best relief
possible in situations like this one, the ultimate responsibility for
establishing a clearer, more consistent process lies with the agency.
The agency already has made some changes to the charge-filing process.
See Brief for United States as \emph{Amicus Curiae} 3, n. 2 (noting that
the Intake Questionnaire form respondent filed has been replaced with
a reworded form). To reduce the risk of further misunderstandings
by those who seek its assistance, the agency should determine, in the
first instance, what additional revisions in its forms and processes are
necessary or appropriate.

  The judgment of the Court of Appeals is affirmed.

\begin{flushright}\emph{It is so ordered.}\end{flushright}
