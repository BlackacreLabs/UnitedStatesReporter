% Syllabus
% Reporter of Decisions

\setcounter{page}{576}

  This is the third original action between New Jersey and Delaware
involving the boundary along the Delaware River (or River) separating
the two States. The first action was settled by a compact the two
States approved in 1905, and Congress ratified in 1907 (1905 Compact or
Compact). See \emph{New Jersey} v. \emph{Delaware,} 205 U.~S. 550 \emph{(New
Jersey} v. \emph{Delaware I).} The 1905 Compact addressed fishing rights
but did not define the interstate boundary line. Two provisions of the
Compact sowed the seeds for further litigation. Article VII provided:
``Each State may, on its own side of the river, continue to exercise
riparian jurisdiction of every kind and nature.'' But Article VIII
added: ``Nothing herein\dots shall affect the territorial limits,
rights, or jurisdiction of either State of, in, or over the Delaware
River, or the ownership of the subaqueous soil thereof, except as
herein expressly set forth.'' The second action, resolved by this
Court in 1934, conclusively determined the location of the interstate
boundary: Delaware owned ``the river and the subaqueous soil'' within
a twelve-mile circle centered on New Castle, Del., ``up to [the]
low water mark on the easterly or New Jersey side''; south of the
twelve-mile circle, the middle of the River's main ship channel marked
the boundary. \emph{New Jersey} v. \emph{Delaware,} 291 U.~S. 361, 385
\emph{(New Jersey} v. \emph{Delaware II).}

  The current controversy was sparked by the Delaware Department of
Natural Resources and Environmental Control's (DNREC) refusal to
grant British Petroleum permission to construct a liquefied natural
gas (LNG) unloading terminal projected to extend beyond New Jersey's
shore some 2,000 feet into Delaware territory. DNREC determined that,
under Delaware's Costal Zone Act (DCZA), the proposed terminal would
be an ``offshore bulk product transfer facilit[y]'' as well as a
``heavy industry use,'' both prohibited by the DCZA. New Jersey
commenced this action, seeking a declaration that Article VII of
the 1905 Compact gave it exclusive regulatory authority over all
projects appurtenant to its shores, including wharves extending past
the low-water mark on New Jersey's side into Delaware territory.
Delaware's answer asserted that, under, \emph{inter alia,} Article VIII
of the Compact and \emph{New Jersey} v. \emph{Delaware II,} it had regulatory
authority, undiminished by Article VII, over structures located within
its borders. On cross-motions for summary judgment, the Special Master
filed a report recommending \newpage  to New Jersey by Article VII is not
exclusive and that Delaware has overlapping jurisdiction, within the
twelve-mile circle, to regulate improvements outshore of the low-water
mark on the New Jersey side of the River. New Jersey filed exceptions.
\emph{Held:} Article VII of the 1905 Compact did not secure to New Jersey
\emph{exclusive} jurisdiction over all riparian improvements commencing
on its shores; New Jersey and Delaware have overlapping authority to
regulate riparian structures and operations of extraordinary character
extending outshore of New Jersey's domain into territory over which
Delaware is sovereign. Pp. 609--623.

  (a) The Court rejects New Jersey's argument that Article VII, which
accords each State ``riparian jurisdiction of every kind and nature,''
bars Delaware from any encroachment upon New Jersey's authority over
improvements extending from New Jersey's shore. Pp. 609--615.

  (1) The novel term ``riparian jurisdiction,'' as used in Article
VII, is properly read as a limiting modifier and does not mean
``exclusive jurisdiction.'' ``[R]iparian jurisdiction'' has never
been a legal term of art, and appears to be a verbal formulation
the 1905 Compact negotiators devised specifically for Article VII.
Elsewhere in the 1905 Compact---most notably, in Article VIII---the
more familiar term ``jurisdiction'' or ``exclusive jurisdiction''
appears. Attributing to ``riparian jurisdiction'' the same meaning
as ``jurisdiction'' unmodified, or equating the novel term with the
formulation ``exclusive jurisdiction,'' would deny operative effect
to each word in the Compact. See \emph{United States} v. \emph{Menasche,}
348 U.~S. 528, 538--539. Presumably drafted in recognition of the
still-unresolved boundary dispute, Article VIII requires an express
statement in the Compact in order to ``affect the territorial .~.~.
jurisdiction of either State\dots over the Delaware River.'' The
Court resists reading the uncommon term ``riparian jurisdiction,''
even when aggrandized by the phrase ``of every kind and nature,''
as effectuating a transfer to New Jersey of Delaware's entire
``territorial\dots jurisdiction\dots over [the portion of] the
Delaware River [in question].'' Pp. 610--612.

  (2) A riparian landowner ordinarily enjoys the right to build a
wharf to access navigable waters far enough to permit the loading and
unloading of ships. But that right, New Jersey agrees, is subject to
state regulation for the protection of the public. New Jersey sees
itself, however, as the only State empowered to regulate, for the
benefit of the public, New Jersey landowners' exercise of riparian
rights. Commonly, the State that grants riparian rights also has
regulatory authority over their exercise. But the 1905 Compact's
negotiators faced an unusual situation: As long as the boundary issue
remained unsettled, they could not know which State was sovereign within
the twelve-mile circle be\newpage yond New Jersey's shore. They likely
knew, however, that ``[t]he rights of a riparian owner [seeking to
wharf out into] a navigable stream\dots are governed by the law of
the state in which the stream is situated.'' \emph{Weems Steamboat Co.
of Baltimore} v. \emph{People's Steamboat Co.,} 214 U.~S. 345, 355.
With the sovereignty issue reserved by the 1905 Compact for another
day, it is difficult to gainsay the Special Master's conclusion that
Article VII's reference to ``riparian jurisdiction'' did not mean
``exclusive jurisdiction.'' Endeavoring to harmonize Article VII with
the boundary determination, the Special Master concluded that Article
VII's preservation to each State of ``riparian jurisdiction'' gave
New Jersey control of the riparian rights ordinarily and usually enjoyed
by landowners on New Jersey's shore. But once the boundary line at low
water is passed, the Special Master further concluded, New Jersey's
regulatory authority is qualified. Just as New Jersey cannot grant land
belonging to Delaware, New Jersey cannot authorize activities that go
beyond the exercise of ordinary and usual riparian rights in the face of
contrary regulation by Delaware. Pp. 612--615.

  (b) An 1834 compact between New Jersey and New York establishing the
two States' common Hudson River boundary casts informative light on
the 1905 New Jersey-Delaware Compact. Similar to the boundary settled
in \emph{New Jersey} v. \emph{Delaware II,} the 1834 accord located the New
Jersey-New York boundary at ``the low water-mark on the\dots New
Jersey side [of the Hudson River,]'' 4 Stat. 710. Unlike the 1905
Compact, however, the 1834 agreement expressly gave New Jersey ``the
\emph{exclusive right} of property in and to\dots land under water''
and ``\\the exclusive jurisdiction of and over the wharves, docks,
and improvements\dots on the shore of the said state} .~.~.,''
\emph{ibid.} (emphasis added). Comparable language is noticeably
absent in Article VII of the 1905 Compact, while other provisions of
the Compact appear to have been adopted almost verbatim from the 1834
New Jersey-New York accord. New Jersey, therefore, could hardly claim
ignorance that Article VII could have been but was not drafted to grant
it ``exclusive jurisdiction'' (not merely ``riparian jurisdiction'')
over wharves and other improvements extending from its shore into
navigable waters. Pp. 615--617.

  (c) \emph{Virginia} v. \emph{Maryland,} 540 U.~S. 56, 75---in which
this Court held that a Maryland-Virginia boundary settlement gave
Virginia ``sovereign authority, free from regulation by Maryland,
to build improvements appurtenant to [Virginia's] shore and to
withdraw water from the [Potomac] River''---provides scant support
for New Jersey's claim. As the Special Master explained, the result
in \emph{Virginia} v. \emph{Maryland} turned on the unique language of the
1785 compact and 1877 arbitration award there involved. The 1785
compact addressed only ``the right [of the \newpage  citizens of each
State] to build wharves and improvements regardless of which State
ultimately was determined to be sovereign over the River,'' \emph{id.,}
at 69. Concerning the States themselves, the 1877 arbitration award
that settled the boundary was definitive. See \emph{id.,} at 75. By
recognizing in that award Virginia's right, ``\emph{qua} sovereign,''
``to use the River beyond low-water mark,'' \emph{id.,} at 72,
the arbitrators manifested their intention to safeguard Virginia's
authority to construct riparian improvements outshore of the low-water
mark free from regulation by Maryland. By contrast, neither the 1905
Compact nor \emph{New Jersey} v. \emph{Delaware II} purported to give New
Jersey all regulatory oversight (as opposed to only ``riparian
jurisdiction''). Pp. 617--618.

  (d) Delaware's claim to regulatory authority is further supported
by New Jersey's acceptance (until the present controversy) of
Delaware's jurisdiction over water and land within its domain to
preserve the quality and prevent deterioration of its coastal areas.
When New Jersey sought federal approval for its coastal management
program, it made the representation---fundamentally inconsistent with
its position here---that any New Jersey project extending beyond
mean low water within the twelve-mile circle would require coastal
permits from both States. DNREC, with no objection from New Jersey,
had previously rejected as a prohibited bulk transfer facility an
earlier request to build an LNG terminal extending from New Jersey
into Delaware. DNREC issued permits for each of the three structures
extending from New Jersey into Delaware built between 1969 and 2006, one
of them undertaken by New Jersey itself. Even during the pendency of
this action, New Jersey applied to Delaware for renewal of the permit
covering the portion of New Jersey's project that extended into
Delaware. Pp. 618--621.

  (e) Nowhere does Article VII ``expressly set forth,'' in Article
VIII's words, Delaware's lack of any governing authority over
territory within the State's own borders. The Special Master correctly
determined that Delaware's pre-1971 ``hands off'' policy regarding
coastal development did not signal that the State never could or
never would assert any regulatory authority over structures using
its subaqueous land. In the decades since Delaware, pursuant to the
DCZA, began to manage its waters and submerged lands, the State has
followed a consistent course: Largely with New Jersey's cooperation,
Delaware has checked proposed structures and activity extending beyond
New Jersey's shore into Delaware's domain in order to protect the
natural environment of its coastal areas. P. 622.

  (f) Given the authority over riparian rights preserved for New
Jersey by the 1905 Compact, Delaware may not impede ordinary and
usual exercises of the right of riparian owners to wharf out from New
Jersey's shore. The project British Petroleum sought to construct
and operate, \newpage  however, goes well beyond the ordinary or usual.
Delaware's classification of the proposed LNG unloading terminal as a
``heavy industry use'' and a ``bulk product transfer facilit[y]''
under the DCZA has not been, and hardly could be, challenged as
inaccurate. Consistent with the scope of Delaware's retained
police power to regulate certain riparian uses, it was within that
State's authority to prohibit construction of the LNG facility. Pp.
622--623.

\noindent Delaware's authority to deny British Petroleum permission to construct
the proposed LNG terminal confirmed; New Jersey's exceptions
overruled; and the Special Master's proposed decree entered with
modifications consistent with the Court's opinion.

  \textsc{Ginsburg,} J., delivered the opinion of the Court, in which
\textsc{Roberts,} C. J., and \textsc{Kennedy, Souter,} and \textsc{Thomas,} JJ.,
joined, and in which \textsc{Ste vens,} J., joined as to paragraphs 1(c), 2,
3, and 4 of the Decree. \textsc{Stevens,} \textsc{J.,} filed an opinion concurring
in part and dissenting in part, \emph{post,} p. 624. \textsc{Scalia,} J.,
filed a dissenting opinion, in which \textsc{Alito,} J., joined, \emph{post,}
p. 628. \textsc{Breyer,} J., took no part in the consideration or decision
of the case.

  \emph{H. Bartow Farr III} argued the cause for plaintiff. With him on the
brief were \emph{Anne Milgram,} Attorney General of New Jersey, \emph{Rachel
J. Horowitz} and \emph{Barbara L. Conklin,} Deputy Attorneys General,
\emph{Gerard Burke,} Assistant Attorney General, and \emph{John R. Renella,
William E. Andersen, Amy C. Donlon, Dean Jablonski,} and \emph{Eileen P.
Kelly,} Deputy Attorneys General.

  \emph{David C. Frederick} argued the cause for defendant. With him on
the brief were \emph{Joseph R. Biden III,} Attorney General of Delaware,
\emph{Kevin P. Maloney, Scott H. Angstreich, Scott K. Attaway, Priya
R. Aiyar, Collins J. Seitz, Jr., Matthew F. Boyer,} and \emph{Max B.
Walton.\\[[*]]

^* \emph{Stuart A. Raphael} and \emph{Sona Rewari} filed a brief for BP
America Inc. et al. as \emph{amici curiae.}
