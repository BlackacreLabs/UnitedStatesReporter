% Concurring in the Judgment
% Ginsburg

\setcounter{page}{84}

  \textsc{Justice Ginsburg,} concurring in the judgment.

  It is better to receive than to give, the Court holds today, at least
when the subject is guns. Distinguishing, as the Court does, between
trading a gun for drugs and trading drugs for a gun, for purposes of
the 18 U.~S.~C. \S~924(c)(1) enhancement, makes scant sense to me. I
join the Court's judgment, however, because I am persuaded that the
Court took a wrong turn in \emph{Smith} v. \emph{United States,} 508 U. S.
223 (1993), when it held that trading a gun for drugs fits within
\S~924(c)(1)'s compass as ``us[e]'' of a firearm ``during and in
relation to any~.~.~.~.drug trafficking crime.'' For reasons well
stated by \textsc{Justice Scalia} in his dissenting opinion in \emph{Smith,}
508 U. S., at 241, I would read the word ``use'' in \S~924(c)(1) to
mean use as a weapon, not use in a bartering transaction. Accordingly, I
would overrule \emph{Smith,} and thereby render our precedent both coherent
and consistent with normal usage. Cf. \emph{Henslee} v. \emph{Union Planters
Nat. Bank \& Trust Co.,} 335 U.~S. 595, 600 (1949) (Frankfurter, J.,
dissenting) (``Wisdom too often never comes, and so one ought not to
reject it merely because it comes late.'').
