% Dissenting
% Stevens

\setcounter{page}{8}

  \textsc{Justice Stevens,} with whom \textsc{Justice Ginsburg} joins, dissenting.

  There is an obvious distinction between time limits that go to
the very initiation of a petition, and time limits that create an
affirmative defense that can be waived. Compare the majority and
dissenting opinions in \emph{John R. Sand \& Gravel Co.} v. \emph{United
States,} 457 F. 3d 1345 (CA Fed. 2006), cert. granted, 550 U. S.
968 (2007). The time limit under consideration in \emph{Pace} v.
\emph{DiGuglielmo,} 544 U.~S. 408 (2005), was of the former kind---as
the Court's opinion expressly noted. See \emph{id.,} at 417 (discussing
``time limits, which go to the very initiation of a petition and a
court's ability to consider that petition''). The time limit at
issue in this case is of the latter, distinguishable kind---as the Court
of Appeals correctly stated.480 F. 3d 1089, 1090 (CA11 2007) (holding
that \emph{Pace} did not address statutory tolling for ``a statute of
limitations that operated as an affirmative defense'').

  It is true that there is language in the majority opinion in \emph{Pace}
that is broad enough to cover both kinds of limitations provisions, but
only the former (those that do not operate as affirmative defenses) can
even arguably provide a reasonable basis for concluding that an untimely
petition has not been ``properly filed'' within the meaning of the
Antiterrorism and Effective Death Penalty Act of 1996 (AEDPA), 110
Stat. 1214.[[*]] I therefore respectfully dissent.


^* I continue to believe, as stated in my dissent in \emph{Pace,} 544 U.
S., at 427, that state timeliness bars that operate like procedural
bars (for example, those that require the courts to consider enumerated
exceptions) should not determine whether a state postconviction petition
is ``properly filed'' under AEDPA. Even accepting \emph{Pace,} however,
this case is distinguishable and should not be summarily reversed.
