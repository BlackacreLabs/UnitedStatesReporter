% Court
% Roberts

\setcounter{page}{183}

  \textsc{Chief Justice Roberts} delivered the opinion of the Court.

  Under the Internal Revenue Code, individuals may subtract from their adjusted gross income certain itemized deductions, but only to the extent the deductions exceed 2% of adjusted gross income. A trust may also claim those deductions, also subject to the 2% floor, except that costs incurred in the administration of the trust, which would not have been incurred if the trust property were not held by a trust, may be deducted without regard to the floor. In the case of individuals, investment advisory fees are subject to the 2% floor; the question presented is whether such fees are also subject to the floor when incurred by a trust. We hold that they generally are and therefore affirm the judgment below, albeit for different reasons than those given by the Court of Appeals. \newpage 

\section{I}

  The Internal Revenue Code imposes a tax on the ``taxable income'' of both individuals and trusts. 26 U.~S.~C. \S~1(a). The Code instructs that the calculation of taxable income begins with a determination of ``gross income,'' capaciously defined as ``all income from whatever source derived.'' \S~61(a). ``Adjusted gross income'' is then calculated by subtracting from gross income certain ``above-the-line'' deductions, such as trade and business expenses and losses from the sale or exchange of property. \S~62(a). Finally, taxable income is calculated by subtracting from adjusted gross income ``itemized deductions''---also known as ``below-the-line'' deductions---defined as all allowable deductions other than the ``above-the-line'' deductions identified in \S~62(a) and the deduction for personal exemptions allowed under \S~151 (2000 ed. and Supp. V). \S~63(d) (2000 ed.).

  Before the passage of the Tax Reform Act of 1986, 100 Stat. 2085, below-the-line deductions were deductible in full. This system resulted in significant complexity and potential for abuse, requiring ``extensive [taxpayer] recordkeeping with regard to what commonly are small expenditures,'' as well as ``significant administrative and enforcement problems for the Internal Revenue Service.'' H. R. Rep. No. 99--426, p. 109 (1985).

  In response, Congress enacted what is known as the ``2% floor'' by adding \S~67 to the Code. Section 67(a) provides that ``the miscellaneous itemized deductions for any taxable year shall be allowed only to the extent that the aggregate of such deductions exceeds 2 percent of adjusted gross income.'' The term ``miscellaneous itemized deductions'' is defined to include all itemized deductions other than certain ones specified in \S~67(b). Investment advisory fees are deductible pursuant to 26 U.~S.~C. \S~212. Because \S~212 is not listed in \S~67(b) as one of the categories of expenses that may be deducted in full, such fees are ``miscellaneous itemized deduc\newpage tions'' subject to the 2% floor. 26 CFR \S~1.67--1T(a)(1)(ii) (2007).

  Section 67(e) makes the 2% floor generally applicable not only to individuals but also to estates and trusts,\footnotemark[1] with one exception relevant here. Under this exception, ``the adjusted gross income of an estate or trust shall be computed in the same manner as in the case of an individual, except that\dots the deductions for costs which are paid or incurred in connection with the administration of the estate or trust and which would not have been incurred if the property were not held in such trust or estate\dots shall be treated as allowable'' and not subject to the 2% floor. \S~67(e)(1).

  Petitioner Michael J. Knight is the trustee of the William L. Rudkin Testamentary Trust, established in the State of Connecticut in 1967. In 2000, the Trustee hired Warfield Associates, Inc., to provide advice with respect to investing the Trust's assets. At the beginning of the tax year, the Trust held approximately \$2.9 million in marketable securities, and it paid Warfield \$22,241 in investment advisory fees for the year. On its fiduciary income tax return for 2000, the Trust reported total income of \$624,816, and it deducted in full the investment advisory fees paid to Warfield. After conducting an audit, respondent Commissioner of Internal Revenue found that these investment advisory fees were miscellaneous itemized deductions subject to the 2% floor. The Commissioner therefore allowed the Trust to deduct the investment advisory fees, which were the only claimed deductions subject to the floor, only to the extent that they exceeded 2% of the Trust's adjusted gross income. The discrepancy resulted in a tax deficiency of \emph{\$4,448.}

  The Trust filed a petition in the United States Tax Court seeking review of the assessed deficiency. It argued that the Trustee's fiduciary duty to act as a ``prudent investor'' \newpage  under the Connecticut Uniform Prudent Investor Act, Conn. Gen. Stat. \S\S~45a--541a to 45a--541\emph{l} (2007),\footnotemark[2] required the Trustee to obtain investment advisory services, and therefore to pay investment advisory fees. The Trust argued that such fees are accordingly unique to trusts and therefore fully deductible under 26 U.~S.~C. \S~67(e)(1). The Tax Court rejected this argument, holding that \S67(e)(1) allows full deductibility only for expenses that are not commonly incurred outside the trust setting. Because investment advisory fees are commonly incurred by individuals, the Tax Court held that they are subject to the 2\% floor when incurred by a trust. \emph{Rudkin Testamentary Trust} v. \emph{Commissioner,} 124 T. C. 304, 309--311 (2005).

\footnotetext[1]{Because this case is only about trusts, we generally refer to trusts throughout, but the analysis applies equally to estates.}

  The Trust appealed to the United States Court of Appeals for the Second Circuit. The Court of Appeals concluded that, in determining whether costs such as investment advisory fees are fully deductible or subject to the 2% floor, \S~67(e) ``directs the inquiry toward the counterfactual condition of assets held individually instead of in trust,'' and requires ``an objective determination of whether the particular cost is one that is peculiar to trusts and one that individuals are incapable of incurring.'' 467 F. 3d 149, 155, 156 (2006). The court held that because investment advisory fees were ``costs of a type that \emph{could} be incurred if the property were held individually rather than in trust,'' deduction of such fees by the Trust was subject to the 2% floor. \emph{Id.,} at 155--156.

\footnotetext[2]{Forty-four States and the District of Columbia have adopted versions of the Uniform Prudent Investor Act. See 7B U. L. A. 1--2 (2006) (listing States that have enacted the Uniform Prudent Investor Act). Five of the remaining six States have adopted their own versions of the prudent investor standard. See Del. Code Ann., Tit. 12, \S~3302 (1995 ed. and 2006 Supp.); Ga. Code Ann. \S~53--12--287 (1997); La. Rev. Stat. Ann. \S~9:2127 (West 2005); Md. Est. \& Trusts Code Ann. \S~15--114 (Lexis 2001); S. D. Codified Laws \S~55--5--6 (2004). Kentucky, the only remaining State, applies the prudent investor standard only in certain circumstances. See Ky. Rev. Stat. Ann. \S~286.3--277 (Lexis 2007 Cum. Supp.); \S\S~386.454(1), 386.502 (Supp. 2007).\newpage }

  The Courts of Appeals are divided on the question presented. The Sixth Circuit has held that investment advisory fees are fully deductible. \emph{O'Neill} v. \emph{Commissioner,} 994 F. 2d 302, 304 (1993). In contrast, both the Fourth and Federal Circuits have held that such fees are subject to the 2% floor, because they are ``commonly'' or ``customarily'' incurred outside of trusts. See \emph{Scott} v. \emph{United States,} 328 F. 3d 132, 140 (CA4 2003); \emph{Mellon Bank, N. A.} v. \emph{United States,} 265 F. 3d 1275, 1281 (CA Fed. 2001). The Court of Appeals below came to the same conclusion, but as noted announced a more exacting test, allowing ``full deduction only for those costs that \emph{could not} have been incurred by an individual property owner.'' 467 F. 3d, at 156 (emphasis added). We granted the Trustee's petition for certiorari to resolve the conflict, 551 U. S. 1144 (2007), and now affirm.

\section{II}

  ``We start, as always, with the language of the statute.'' \emph{Williams} v. \emph{Taylor,} 529 U.~S. 420, 431 (2000). Section 67(e) sets forth a general rule: ``[T]he adjusted gross income of [a] .~.~. trust shall be computed in the same manner as in the case of an individual.'' That is, trusts can ordinarily deduct costs subject to the same 2% floor that applies to individuals' deductions. Section 67(e) provides for an exception to the 2% floor when two conditions are met. First, the relevant cost must be ``paid or incurred in connection with the administration of the\dots trust.'' \S~67(e)(1). Second, the cost must be one ``which would not have been incurred if the property were not held in such trust.'' \emph{Ibid.}

  In applying the statute, the Court of Appeals below asked whether the cost at issue \emph{could} have been incurred by an individual.\footnotemark[3] This approach flies in the face of the statutory \newpage  language. The provision at issue asks whether the costs ``would not have been incurred if the property were not held'' in trust, \emph{ibid.,} not, as the Court of Appeals would have it, whether the costs ``could not have been incurred'' in such a case, 467 F. 3d, at 156. The fact that an individual could not do something is one reason he would not, but not the only possible reason. If Congress had intended the Court of Appeals' reading, it easily could have replaced ``would'' in the statute with ``could,'' and presumably would have. The fact that it did not adopt this readily available and apparent alternative strongly supports rejecting the Court of Appeals' reading.\footnotemark[4] Moreover, if the Court of Appeals' reading were correct, it is not clear why Congress would have included in the stat\newpage ute the first clause of \S~67(e)(1). If the only costs that are fully deductible are those that \emph{could} not be incurred outside the trust context---that is, that could \emph{only} be incurred by trusts---then there would be no reason to place the further condition on full deductibility that the costs be ``paid or incurred in connection with the administration of the .~.~. trust,'' \S~67(e)(1). We can think of no expense that could be incurred exclusively by a trust but would nevertheless \emph{not} be ``paid or incurred in connection with'' its administration.

\footnotetext[3]{The Solicitor General embraces this position in this Court, arguing that the Court of Appeals' approach represents the best reading of the statute and establishes an easily administrable rule. See Brief for Respondent 17--20, 22. Indeed, after the Court of Appeals' decision, the Commissioner \newpage adopted that court's reading of the statute in a proposed regulation. See Section 67 Limitations on Estates or Trusts, 72 Fed. Reg. 41245 (2007) (notice of proposed rulemaking) (a trust-related cost is exempted from the 2% floor only if ``an individual \emph{could not} have incurred that cost in connection with property not held in an estate or trust'' (emphasis added)). The Government did not advance this argument before the Court of Appeals. See Brief for Appellee in No. 05--5151--AG (CA2), pp. 3--4, 22--24. In fact, the notice of proposed rulemaking appears to be the first time the Government has ever taken this position, and we are the first Court to which the argument has been made in a brief. See Brief for United States in \emph{Mellon Bank, N. A.} v. \emph{United States,} No. 01--5015 (CA Fed.), p. 27 (``[I]f a trustrelated administrative expense is also customarily or habitually incurred outside of trusts, then it is subject to the two-percent floor''); Brief for United States in \emph{Scott} v. \emph{United States,} No. 02--1464 (CA4), p. 27 (same).}

\footnotetext[4]{In pressing the Court of Appeals' approach, the Solicitor General argues that ``to say that a team would not have won the game if it were not for the quarterback's outstanding play is to say that the team could not have won without the quarterback.'' Brief for Respondent 19. But the Solicitor General simply posits the truth of a proposition---that the team would not have won the game if it were not for the quarterback's outstanding play---and then states its equivalent. The statute, in contrast, does not posit any proposition. Rather, it asks a question: whether a particular cost \emph{would} have been incurred if the property were held by an individual instead of a trust.}

  The Trustee argues that the exception in \S~67(e)(1) ``establishes a straightforward causation test.'' Brief for Petitioner 22. The proper inquiry, the Trustee contends, is ``whether a particular expense of a particular trust or estate was caused by the fact that the property was held in the trust or estate.'' \emph{Ibid.} Investment advisory fees incurred by a trust, the argument goes, meet this test because these costs are caused by the trustee's obligation ``to obtain advice on investing trust assets in compliance with the Trustees' particular fiduciary duties.'' \emph{Ibid.} We reject this reading as well.

  On the Trustee's view, the statute operates only to distinguish costs that are incurred by virtue of a trustee's fiduciary duties from those that are not. But all (or nearly all) of a trust's expenses are incurred because the trustee has a duty to incur them; otherwise, there would be no reason for the trust to incur the expense in the first place. See G. Bogert \& G. Bogert, Law of Trusts and Trustees \S~801, p. 134 (2d rev. ed. 1981) (``[T]he payment for expenses must be reasonably necessary to facilitate administration of the trust''). As an example of a type of trust-related expense that would be subject to the 2% floor, the Trustee offers ``expenses for routine maintenance of real property'' held by a trust. Brief for Petitioner 23. But such costs would appear to be fully deductible under the Trustee's own reading because a trustee is obligated to incur maintenance expenses in light of the fiduciary duty to maintain trust property. See 1 Re\newpage statement (Second) of Trusts \S~176, p. 381 (1957) (``The trustee is under a duty to the beneficiary to use reasonable care and skill to preserve the trust property'').

  Indeed, the Trustee's formulation of its argument is circular: ``Trust investment advice fees are caused by the fact the property is held in trust.'' Brief for Petitioner 19. But ``trust investment advice fees'' are only aptly described as such because the property is held in trust; the statute asks whether such costs would be incurred by an individual if the property were not. Even when there is a clearly analogous category of costs that would be incurred by individuals, the Trustee's reading would exempt most or all trust costs as fully deductible merely because they derive from a trustee's fiduciary duty. Adding the modifier ``trust'' to costs that otherwise would be incurred by an individual surely cannot be enough to escape the 2% floor.

  What is more, if the Trustee's position were correct, then only the first clause of \S~67(e)(1)---providing that the cost be ``incurred in connection with the administration of the\dots trust''---would be necessary. The statute's second, limiting condition---that the cost also be one ``which would not have been incurred if the property were not held in such trust''---would do no work; we see no difference in saying, on the one hand, that costs are ``caused by'' the fact that the property is held in trust and, on the other, that costs are incurred ``in connection with the administration'' of the trust. Thus, accepting the Trustee's approach ``would render part of the statute entirely superfluous, something we are loath to do.'' \emph{Cooper Industries, Inc.} v. \emph{Aviall Services, Inc.,} 543 U.~S. 157, 166 (2004).

  The Trustee's reading is further undermined by our inclination, ``[i]n construing provisions\dots in which a general statement of policy is qualified by an exception, [to] read the exception narrowly in order to preserve the primary operation of the provision.'' \emph{Commissioner} v. \emph{Clark,} 489 U.~S. 726, 739 (1989). As we have said, \S~67(e) sets forth a general \newpage  rule for purposes of the 2\% floor established in \S~67(a): ``For purposes of this section, the adjusted gross income of an estate or trust shall be computed in the same manner as in the case of an individual.'' Under the Trustee's reading, \S~67(e)(1)'s exception would swallow the general rule; most (if not all) expenses incurred by a trust would be fully deductible. ``Given that Congress has enacted a general rule .~.~.~,we should not eviscerate that legislative judgment through an expansive reading of a somewhat ambiguous exception.'' \emph{Ibid.}

  More to the point, the statute by its terms does not ``establis[h] a straightforward causation test,'' Brief for Petitioner 22, but rather invites a hypothetical inquiry into the treatment of the property were it held outside a trust. The statute does not ask whether a cost was incurred \emph{because} the property is held by a trust; it asks whether a particular cost ``would not have been incurred if the property were not held in such trust,'' \S~67(e)(1). ``Far from examining the nature of the cost at issue from the perspective of whether it was caused by the trustee's duties, the statute instead looks to the counterfactual question of whether \emph{individuals} would have incurred such costs in the \emph{absence} of a trust.'' Brief for Respondent 9.

  This brings us to the test adopted by the Fourth and Federal Circuits: Costs incurred by trusts that escape the 2\% floor are those that would not ``commonly'' or ``customarily'' be incurred by individuals. See \emph{Scott,} 328 F. 3d, at 140 (``Put simply, trust-related administrative expenses are subject to the 2\% floor if they constitute expenses commonly incurred by individual taxpayers''); \emph{Mellon Bank,} 265 F. 3d, at 1281 (\S67(e) ``treats as fully deductible only those trust-related administrative expenses that are unique to the administration of a trust and not customarily incurred outside of trusts''). The Solicitor General also accepts this view as an alternative reading of the statute. See Brief for Respondent 20--21. We agree with this approach.\starpage 

  The question whether a trust-related expense is fully deductible turns on a prediction about what would happen if a fact were changed---specifically, if the property were held by an individual rather than by a trust. In the context of making such a prediction, when there is uncertainty about the answer, the word ``would'' is best read as ``express[ing] concepts such as custom, habit, natural disposition, or probability.'' \emph{Scott, supra,} at 139. See Webster's Third New International Dictionary 2637--2638 (1993); American Heritage Dictionary 2042, 2059 (3d ed. 1996). The Trustee objects that the statutory text ``does not ask whether expenses are ‘customarily' incurred outside of trusts,'' Reply Brief for Petitioner 15, but that is the direct import of the language in context. The text requires determining what would happen if a fact were changed; such an exercise necessarily entails a prediction; and predictions are based on what would customarily or commonly occur. Thus, in asking whether a particular type of cost ``would \emph{not} have been incurred'' if the property were held by an individual, \S~67(e)(1) excepts from the 2\% floor only those costs that it would be \emph{un}common (or unusual, or unlikely) for such a hypothetical individual to incur. 

\section{III}

  Having decided on the proper reading of \S~67(e)(1), we come to the application of the statute to the particular question in this case: whether investment advisory fees incurred by a trust escape the 2\% floor. 
  It is not uncommon or unusual for individuals to hire an investment
adviser. Certainly the Trustee, who has the burden of establishing its
entitlement to the deduction, has not demonstrated that it is. See
\emph{INDOPCO, Inc.} v. \emph{Commissioner,} 503 U.~S. 79, 84 (1992) (noting
the `` ‘familiar rule' that ‘an income tax deduction is a matter
of legislative grace and that the burden of clearly showing the right to
the claimed deduction is on the taxpayer' '' (quoting \emph{Interstate
Transit Lines} v. \emph{Commissioner,} 319 U.~S. 590, 593 (1943)));
\newpage  Tax Court Rule 142(a)(1) (stating that the ``burden of proof
shall be upon the petitioner,'' with certain exceptions not relevant
here). The Trustee's argument is that individuals cannot incur trust
investment advisory fees, not that individuals do not commonly incur
investment advisory fees.

  Indeed, the essential point of the Trustee's argument is that he engaged an investment adviser because of his fiduciary duties under Connecticut's Uniform Prudent Investor Act, Conn. Gen. Stat. \S~45a--541a(a) (2007). The Act eponymously requires trustees to follow the ``prudent investor rule.'' See n. 2, \emph{supra.} To satisfy this standard, a trustee must ``invest and manage trust assets \emph{as a prudent investor would,} by considering the purposes, terms, distribution requirements and other circumstances of the trust.'' \S~45a-- 541b(a) (emphasis added). The prudent investor standard plainly does not refer to a prudent \emph{trustee}; it would not be very helpful to explain that a trustee should act as a prudent trustee would. Rather, the standard looks to what a prudent investor with the same investment objectives handling his own affairs would do---\emph{i. e.,} a prudent individual investor. See Restatement (Third) of Trusts (Prudent Investor Rule) Reporter's Notes on \S~227, p. 58 (1990) (``The prudent investor rule of this Section has its origins in the dictum of Harvard College v. Amory, 9 Pick. (26 Mass.) 446, 461 (1830), stating that trustees must ‘observe how men of prudence, discretion, and intelligence manage their own affairs, not in regard to speculation, but in regard to the permanent disposition of their funds, considering the probable income, as well as the probable safety of the capital to be invested' ''). See also, \emph{e. g., In re Musser's Estate,} 341 Pa. 1, 9--10, 17 A. 2d 411, 415 (1941) (noting the ``general rule'' that ``a trustee must exercise such prudence and diligence in conducting the affairs of the trust as men of average diligence and discretion would employ in their own affairs''). And we have no reason to doubt the Trustee's claim that a hypothetical prudent investor in his position would have solicited investment advice, \newpage  just as he did. Having accepted all this, it is quite difficult to say that investment advisory fees ``would not have been incurred''---that is, that it would be unusual or uncommon for such fees to have been incurred---if the property were held by an individual investor with the same objectives as the Trust in handling his own affairs.

  We appreciate that the inquiry into what is common may not be as easy in other cases, particularly given the absence of regulatory guidance. But once you depart in the name of ease of administration from the language chosen by Congress, there is more than one way to skin the cat: The Trustee raises administrability concerns in support of his causation test, Reply Brief for Petitioner 6, but so does the Government in explaining why it prefers the Court of Appeals' approach to the one it has successfully advanced before the Tax Court and two Federal Circuits. Congress's decision to phrase the pertinent inquiry in terms of a prediction about a hypothetical situation inevitably entails some uncertainty, but that is no excuse for judicial amendment of the statute. The Code elsewhere poses similar questions---such as whether expenses are ``ordinary,'' see \S\S~162(a), 212; see also \emph{Deputy, Administratrix} v. \emph{Du Pont,} 308 U.~S. 488, 495 (1940) (noting that ``[o]rdinary has the connotation of normal, usual, or customary'')---and the inquiry is in any event what \S~67(e)(1) requires.

  As the Solicitor General concedes, some trust-related investment advisory fees may be fully deductible ``if an investment advisor were to impose a special, additional charge applicable only to its fiduciary accounts.'' Brief for Respondent 25. There is nothing in the record, however, to suggest that Warfield charged the Trustee anything extra, or treated the Trust any differently than it would have treated an individual with similar objectives, because of the Trustee's fiduciary obligations. See App. 24--27. It is conceivable, moreover, that a trust may have an unusual investment objective, or may require a specialized balancing of the \newpage  interests of various parties, such that a reasonable comparison with individual investors would be improper. In such a case, the incremental cost of expert advice beyond what would normally be required for the ordinary taxpayer would not be subject to the 2\% floor. Here, however, the Trust has not asserted that its investment objective or its requisite balancing of competing interests was distinctive. Accordingly, we conclude that the investment advisory fees incurred by the Trust are subject to the 2% floor.

  The judgment of the Court of Appeals is affirmed.

\begin{flushright}\emph{It is so ordered.}\end{flushright}
