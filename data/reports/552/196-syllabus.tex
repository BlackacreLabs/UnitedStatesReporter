% Syllabus
% Reporter of Decisions

\setcounter{page}{196}

  Under New York's current Constitution, State Supreme Court Justices are elected in each of the State's judicial districts. Since 1921, New York's election law has required parties to select their nominees by a convention composed of delegates elected by party members. An individual running for delegate must submit a 500-signature petition collected within a specified time. The convention's nominees appear automatically on the general-election ballot, along with any independent candidates who meet certain statutory requirements. Respondents filed suit, seeking, \emph{inter alia,} a declaration that New York's convention system violates the First Amendment rights of challengers running against candidates favored by party leaders and an injunction mandating a direct primary election to select Supreme Court nominees. The Federal District Court issued a preliminary injunction, pending the enactment of a new state statutory scheme, and the Second Circuit affirmed.

\emph{Held:}

\noindent New York's system of choosing party nominees for the State Supreme Court does not violate the First Amendment. Pp. 202--209.

  (a) Because a political party has a First Amendment right to limit its membership as it wishes, and to choose a candidate-selection process that will in its view produce the nominee who best represents its political platform, a State's power to prescribe party use of primaries or conventions to select nominees for the general election is not without limits. \emph{California Democratic Party} v. \emph{Jones,} 530 U.~S. 567, 577. However, respondents, who claim their own associational right to join and have influence in the party, are in no position to rely on the right that the First Amendment confers on political parties. Pp. 202--204.

  (b) Respondents' contention that New York's electoral system does not assure them a fair chance of prevailing in their parties' candidateselection process finds no support in this Court's precedents. Even if \emph{Kusper} v. \emph{Pontikes,} 414 U.~S. 51, 57, which acknowledged an individual's associational right to vote in a party primary without undue stateimposed impediment, were extended to cover the right to run in a party primary, the New York law's signature and deadline requirements are entirely reasonable. A State may demand a minimum degree of sup\newpage port for candidate access to a ballot, see \emph{Jenness} v. \emph{Fortson,} 403 U.~S. 431, 442. P. 204.

  (c) Respondents' real complaint is that the convention process following the delegate election does not give them a realistic chance to secure their party's nomination because the party leadership garners more votes for its delegate slate and effectively determines the nominees. This says no more than that the party leadership has more widespread support than a candidate not supported by the leadership. Cases invalidating ballot-access requirements have focused on the requirements themselves, and not on the manner in which political actors function under those requirements. \emph{E. g., Bullock} v. \emph{Carter,} 405 U.~S. 134. Those cases do not establish an individual's constitutional right to have a ``fair shot'' at winning a party's nomination. Pp. 204--207.

  (d) Respondents' argument that the existence of entrenched ``oneparty rule'' in the State's general election demands that the First Amendment be used to impose additional competition in the parties' nominee-selection process is a novel and implausible reading of the First Amendment. Pp. 207--209.

462 F. 3d 161, reversed.

\textsc{Scalia,} J., delivered the opinion of the Court, in which \textsc{Roberts,} C. J., and \textsc{Stevens, Souter, Thomas, Ginsburg, Breyer,} and \textsc{Alito,} JJ., joined. \textsc{Stevens,} J., filed a concurring opinion, in which \textsc{Souter,} J., joined, \emph{post,} p. 209. \textsc{Kennedy,} J., filed an opinion concurring in the judgment, in which \textsc{Breyer,} J., joined as to Part II, \emph{post,} p. 209.

  \emph{Theodore B. Olson} argued the cause for petitioners New York State Board of Elections et al. With him on the briefs were \emph{Matthew D. McGill, Michael S. Diamant, Todd D. Valentine, Randy M. Mastro,} and \emph{Jennifer L. Conn. Andrew J. Rossman} argued the cause for petitioners New York County Democratic Committee et al. With him on the briefs were \emph{Steven M. Pesner, James P. Chou, James E. d'Auguste,} \emph{Vincenzo A. DeLeo, Edward P. Lazarus, Carter G. Phillips, Thomas C. Goldstein, Joseph L. Forstadt, Ernst H. Rosenberger, Burton N. Lipshie, David A. Sifre,} and \emph{Arthur W. Greig. Andrew M. Cuomo,} Attorney General of New York, \emph{Barbara D. Underwood,} Solicitor General, \emph{Benjamin N. Gutman,} Deputy Solicitor General, and \emph{Denise A. Hartman,} As\newpage sistant Solicitor General, filed briefs for petitioner the State of New York.

  \emph{Frederick A. O. Schwarz, Jr.,} argued the cause for respondents. With him on the brief were \emph{Burt Neuborne, Deborah Goldberg, Kent A. Yalowitz,} and \emph{Paul M. Smith.}[[*]]

\footnotetext[*]{Briefs of \emph{amici curiae} urging reversal were filed for the Asian American Bar Association of New York by \emph{Steven B. Shapiro} and \emph{Vincent T. Chang}; for the Mid-Manhattan Branch of the NAACP et al. by \emph{Gene C. Schaerr, Steffen N. Johnson, Linda T. Coberly, and Michael J. Friedman}; and for the Republican National Committee by \emph{H. Christopher Bartolomucci.}

Briefs of \emph{amici curiae} urging affirmance were filed for the City of New York et al. by \emph{Preeta D. Bansal, Barry Kamins, Michael A. Cardozo, Victor A. Kovner}, and \emph{Kathryn Grant Madigan}; for the American Civil Liberties Union et al. by \emph{Arthur N. Eisenberg} and \emph{Steven R. Shapiro}; for the Asian American Legal Defense and Education Fund et al. by \emph{Mariann Meier Wang}; for the Cato Institute et al. by \emph{Erik S. Jaffe}; for Guy-Uriel E. Charles et al. by \emph{Ellen D. Katz, pro se}; for the New York County Lawyers' Association by \emph{Stephanie G. Wheeler} and \emph{Bradley P. Smith}; for the Washington Legal Foundation by \emph{Daniel J. Popeo} and \emph{Richard A. Samp}; for John Dunne by \emph{Andrew H. Schapiro}; for Charles J. Hynes by \emph{Paul A. Engelmayer}; for Edward I. Koch by \emph{Mr. Koch, pro se}, and \emph{Bruce S. Kaplan}; for Thomas Mann et al. by \emph{Daniel R. Ortiz, J. Gerald Hebert,} and \emph{Paul S. Ryan}; and for Former New York State Judges et al. by \emph{Holly K. Kulka} and \emph{Jonathan R. Dowell.}}
