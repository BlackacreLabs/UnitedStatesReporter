% Court
% Roberts

\setcounter{page}{497}

  \textsc{Chief Justice Roberts} delivered the opinion of the Court.

  The International Court of Justice (ICJ), located in the Hague,
is a tribunal established pursuant to the United Nations Charter to
adjudicate disputes between member states. In the \emph{Case Concerning
Avena and Other Mexican Nationals} (\\Mex.} v. \emph{U. S.\\), 2004 I.
C. J. 12 (Judgment of Mar. 31) (\emph{Avena}), that tribunal considered
a claim brought by Mexico against the United States. The ICJ held
that, based on violations of the Vienna Convention, 51 named Mexican
nation\newpage  als were entitled to review and reconsideration of their
state-court convictions and sentences in the United States. This was so
regardless of any forfeiture of the right to raise Vienna Convention
claims because of a failure to comply with generally applicable state
rules governing challenges to criminal convictions.

  In \emph{Sanchez-Llamas} v. \emph{Oregon,} 548 U.~S. 331
(2006)---issued after \emph{Avena} but involving individuals who were not
named in the \emph{Avena} judgment---we held that, contrary to the ICJ's
determination, the Vienna Convention did not preclude the application
of state default rules. After the \emph{Avena} decision, President George
W. Bush determined, through a Memorandum for the Attorney General (Feb.
28, 2005), App. to Pet. for Cert. 187a (Memorandum or President's
Memorandum), that the United States would ``discharge its international
obligations'' under \emph{Avena} ``by having State courts give effect to
the decision.''

  Petitioner José Ernesto Medellín, who had been convicted and
sentenced in Texas state court for murder, is one of the 51 Mexican
nationals named in the \emph{Avena} decision. Relying on the ICJ's
decision and the President's Memorandum, Medellín filed an
application for a writ of habeas corpus in state court. The Texas Court
of Criminal Appeals dismissed Medellín's application as an abuse
of the writ under state law, given Medellín's failure to raise his
Vienna Convention claim in a timely manner under state law. We granted
certiorari to decide two questions. \emph{First,} is the ICJ's judgment
in \emph{Avena} directly enforceable as domestic law in a state court
in the United States? \emph{Second,} does the President's Memorandum
independently require the States to provide review and reconsideration
of the claims of the 51 Mexican nationals named in \emph{Avena} without
regard to state procedural default rules? We conclude that neither
\emph{Avena} nor the President's Memorandum constitutes directly
enforceable federal law that pre-empts state limitations on the \newpage 
filing of successive habeas petitions. We therefore affirm the decision
below.

\section{I}

\subsection{A}

  In 1969, the United States, upon the advice and consent of the
Senate, ratified the Vienna Convention on Consular Relations (Vienna
Convention or Convention), Apr. 24, 1963, [1970] 21 U. S. T. 77, T. I.
A. S. No. 6820, and the Optional Protocol Concerning the Compulsory
Settlement of Disputes to the Vienna Convention (Optional Protocol
or Protocol), Apr. 24, 1963, [1970] 21 U. S. T. 325, T. I. A. S. No.
6820. The preamble to the Convention provides that its purpose is to
``contribute to the development of friendly relations among nations.''
21 U. S. T., at 79; \emph{Sanchez-Llamas, supra,} at 337. Toward
that end, Article 36 of the Convention was drafted to ``facilitat[e]
the exercise of consular functions.'' Art. 36(1), 21 U. S. T., at
100. It provides that if a person detained by a foreign country ``so
requests, the competent authorities of the receiving State shall,
without delay, inform the consular post of the sending State'' of such
detention, and ``inform the [detainee] of his righ[t]'' to request
assistance from the consul of his own state. Art. 36(1)(b), \emph{id.,}
at 101.

  The Optional Protocol provides a venue for the resolution of disputes
arising out of the interpretation or application of the Vienna
Convention. Art. I, 21 U. S. T., at 326. Under the Protocol,
such disputes ``shall lie within the compulsory jurisdiction of the
International Court of Justice'' and ``may accordingly be brought
before the [ICJ]\dots by any party to the dispute being a Party to
the present Protocol.'' \emph{Ibid.}

  The ICJ is ``the principal judicial organ of the United Nations.''
United Nations Charter, Art. 92, 59 Stat. 1051, T. S. No. 993
(1945). It was established in 1945 pursuant to the United
Nations Charter. The ICJ Statute---annexed to the \newpage  U. N.
Charter---provides the organizational framework and governing procedures
for cases brought before the ICJ. Statute of the International Court
of Justice (ICJ Statute), 59 Stat. 1055, T. S. No. 993 (1945).

  Under Article 94(1) of the U. N. Charter, ``[e]ach Member of the
United Nations undertakes to comply with the decision of the [ICJ] in
any case to which it is a party.'' 59 Stat. 1051. The ICJ's
jurisdiction in any particular case, however, is dependent upon the
consent of the parties. See Art. 36, \emph{id.,} at 1060. The ICJ
Statute delineates two ways in which a nation may consent to ICJ
jurisdiction: It may consent generally to jurisdiction on any question
arising under a treaty or general international law, Art. 36(2),
\emph{ibid.\\, or it may consent specifically to jurisdiction over a
particular category of cases or disputes pursuant to a separate treaty,
Art. 36(1), \emph{ibid.} The United States originally consented to the
general jurisdiction of the ICJ when it filed a declaration recognizing
compulsory jurisdiction under Art. 36(2) in 1946. The United States
withdrew from general ICJ jurisdiction in 1985. See U. S. Dept. of
State Letter and Statement Concerning Termination of Acceptance of ICJ
Compulsory Jurisdiction (Oct. 7, 1985), reprinted in 24 I. L. M. 1742
(1985). By ratifying the Optional Protocol to the Vienna Convention,
the United States consented to the specific jurisdiction of the ICJ with
respect to claims arising out of the Vienna Convention. On March 7,
2005, subsequent to the ICJ's judgment in \emph{Avena,} the United States
gave notice of withdrawal from the Optional Protocol to the Vienna
Convention. Letter from Condoleezza Rice, Secretary of State, to Kofi
A. Annan, Secretary-General of the United Nations.

\section{B}

  Petitioner José Ernesto Medellín, a Mexican national, has lived in
the United States since preschool. A member of the \newpage  ``Black and
Whites'' gang, Medellín was convicted of capital murder and sentenced
to death in Texas for the gang rape and brutal murders of two Houston
teenagers.

  On June 24, 1993, 14-year-old Jennifer Ertman and 16-yearold Elizabeth
Pena were walking home when they encountered Medellín and several
fellow gang members. Medellín attempted to engage Elizabeth in
conversation. When she tried to run, petitioner threw her to the ground.
Jennifer was grabbed by other gang members when she, in response to her
friend's cries, ran back to help. The gang members raped both girls
for over an hour. Then, to prevent their victims from identifying them,
Medellín and his fellow gang members murdered the girls and discarded
their bodies in a wooded area. Medellín was personally responsible for
strangling at least one of the girls with her own shoelace.

  Medellín was arrested at approximately 4 a.m. on June 29, 1993.
A few hours later, between 5:54 and 7:23 a.m., Medellín was given
\emph{Miranda} warnings; he then signed a written waiver and gave a
detailed written confession. App. to Brief for Respondent 32--36.
Local law enforcement officers did not, however, inform Medellín of
his Vienna Convention right to notify the Mexican consulate of his
detention. Brief for Petitioner 6--7. Medellín was convicted of
capital murder and sentenced to death; his conviction and sentence were
affirmed on appeal. \emph{Medellín} v. \emph{State,} No. 71,997 (Tex. Crim.
App., May 16, 1997), App. to Brief for Respondent 2--31.

  Medellín first raised his Vienna Convention claim in his first
application for state postconviction relief. The state trial court held
that the claim was procedurally defaulted because Medellín had failed
to raise it at trial or on direct review. The trial court also rejected
the Vienna Convention claim on the merits, finding that Medellín had
``fail[ed] to show that any non-notification of the Mexican authorities
im\newpage pacted on the validity of his conviction or punishment.''
\emph{Id.,} at 62.\footnotemark[1] The Texas Court of Criminal Appeals affirmed.
\emph{Id.,} at 64--65.

  Medellín then filed a habeas petition in Federal District Court.
The District Court denied relief, holding that Medellín's Vienna
Convention claim was procedurally defaulted and that Medellín had
failed to show prejudice arising from the Vienna Convention violation.
See \emph{Medellín} v. \emph{Cockrell,} Civ. Action No. H--01--4078 (SD
Tex., June 26, 2003), App. to Brief for Respondent 66, 86--92.

  While Medellín's application for a certificate of appealability was
pending in the Fifth Circuit, the ICJ issued its decision in \emph{Avena.}
The ICJ held that the United States had violated Article 36(1)(b) of the
Vienna Convention by failing to inform the 51 named Mexican nationals,
including Medellín, of their Vienna Convention rights. 2004 I. C.
J., at 53--55. In the ICJ's determination, the United States
was obligated ``to provide, by means of its own choosing, review
and reconsideration of the convictions and sentences of the \newpage 
[affected] Mexican nationals.'' \emph{Id.,} at 72, ¶ 153(9). The
ICJ indicated that such review was required without regard to state
procedural default rules. \emph{Id.,} at 56--57.


^1 The requirement of Article 36(1)(b) of the Vienna Convention that the
detaining state notify the detainee's consulate ``without delay''
is satisfied, according to the ICJ, where notice is provided within
three working days. \emph{Avena,} 2004 I. C. J. 12, 52, ¶ 97 (Judgment
of Mar. 31). See \emph{Sanchez-Llamas} v. \emph{Oregon,} 548 U.~S. 331, 362
(2006) (\textsc{Ginsburg,} J., concurring in judgment). Here, Medellín
confessed within three hours of his arrest---before there could be a
violation of his Vienna Convention right to consulate notification. App.
to Brief for Respondent 32--36. In a second state habeas application,
Medellín sought to expand his claim of prejudice by contending that
the State's noncompliance with the Vienna Convention deprived him of
assistance in developing mitigation evidence during the capital phase of
his trial. This argument, however, was likely waived: Medellín had the
assistance of consulate counsel during the preparation of his \emph{first}
application for state postconviction relief, yet failed to raise this
argument at that time. See Application for Writ of Habeas Corpus in
\emph{Ex parte Medellín,} No. 675430--A (Tex. Crim. App., Mar. 26, 1998),
pp. 25--31. In light of our disposition of this case, we need not
consider whether Medellín was prejudiced in any way by the violation of
his Vienna Convention rights.

  The Fifth Circuit denied a certificate of appealability.
\emph{Medellín} v. \emph{Dretke,} 371 F. 3d 270, 281 (2004). The court
concluded that the Vienna Convention did not confer individually
enforceable rights. \emph{Id.,} at 280. The court further ruled that it
was in any event bound by this Court's decision in \emph{Breard} v.
\emph{Greene,} 523 U.~S. 371, 375 (1998) (\\per curiam\\), which held
that Vienna Convention claims are subject to procedural default rules,
rather than by the ICJ's contrary decision in \emph{Avena.} 371 F. 3d,
at 280.

  This Court granted certiorari. \emph{Medellín} v. \emph{Dretke,} 544
U.~S. 660, 661 (2005) (\\per curiam\\) (\\Medellín I\\). Before
we heard oral argument, however, President George W. Bush issued his
Memorandum for the United States Attorney General, providing:

    ``I have determined, pursuant to the authority vested in me as
    President by the Constitution and the laws of the United States of
    America, that the United States will discharge its international
    obligations under the decision of the International Court of Justice
    in [\emph{Avena}], by hav ing State courts give effect to the decision
    in accordance with general principles of comity in cases filed by
    the 51 Mexican nationals addressed in that decision.'' App. to
    Pet. for Cert. 187a.

  Medellín, relying on the President's Memorandum and the ICJ's
decision in \emph{Avena,} filed a second application for habeas relief
in state court. \emph{Ex parte Medellín,} 223 S. W. 3d 315, 322--323
(Tex. Crim. App. 2006). Because the state-court proceedings might have
provided Medellín with the review and reconsideration he requested, and
because his claim for federal relief might otherwise have been barred,
we dismissed his petition for certiorari as improvidently granted.
\emph{Medellín I, supra,} at 664.\newpage 

  The Texas Court of Criminal Appeals subsequently dismissed
Medellín's second state habeas application as an abuse of the writ.
223 S. W. 3d, at 352. In the court's view, neither the \emph{Avena}
decision nor the President's Memorandum was ``binding federal
law'' that could displace the State's limitations on the filing of
successive habeas applications. 223 S. W. 3d, at 352. We again
granted certiorari. 550 U.~S. 917 (2007).

\section{II}

  Medellín first contends that the ICJ's judgment in \emph{Avena}
constitutes a ``binding'' obligation on the state and federal courts
of the United States. He argues that ``by virtue of the Supremacy
Clause, the treaties requiring compliance with the \emph{Avena} judgment
are \emph{already} the ‘Law of the Land' by which all state and federal
courts in this country are ‘bound.' '' Reply Brief for Petitioner
1. Accordingly, Medellín argues, \emph{Avena} is a binding federal rule of
decision that pre-empts contrary state limitations on successive habeas
petitions.

  No one disputes that the \emph{Avena} decision---a decision that flows
from the treaties through which the United States submitted to ICJ
jurisdiction with respect to Vienna Convention disputes---constitutes
an \emph{international} law obligation on the part of the United States.
But not all international law obligations automatically constitute
binding federal law enforceable in United States courts. The question
we confront here is whether the \emph{Avena} judgment has automatic
\emph{domestic} legal effect such that the judgment of its own force
applies in state and federal courts.

  This Court has long recognized the distinction between treaties that
automatically have effect as domestic law, and those that---while
they constitute international law commitments---do not by themselves
function as binding federal law. The distinction was well explained by
Chief Justice Marshall's opinion in \emph{Foster} v. \emph{Neilson,} 2
Pet. 253, 315 (1829), \newpage  overruled on other grounds, \emph{United
States} v. \emph{Percheman,} 7 Pet. 51 (1833), which held that a
treaty is ``equivalent to an act of the legislature,'' and hence
self-executing, when it ``operates of itself without the aid of
any legislative provision.'' \emph{Foster, supra,} at 314. When,
in contrast, ``[treaty] stipulations are not self-executing they
can only be enforced pursuant to legislation to carry them into
effect.'' \emph{Whitney} v. \emph{Robertson,} 124 U.~S. 190, 194 (1888).
In sum, while treaties ``may comprise international commitments
.~.~. they are not domestic law unless Congress has either enacted
implementing statutes or the treaty itself conveys an intention that it
be ‘self-executing' and is ratified on these terms.'' \emph{Igartu´
a-De La Rosa} v. \emph{United States,} 417 F. 3d 145, 150 (CA1 2005) (en
banc) (Boudin, C. J.).\footnotemark[2]

  A treaty is, of course, ``primarily a compact between independent
nations.'' \emph{Head Money Cases,} 112 U.~S. 580, 598 (1884).
It ordinarily ``depends for the enforcement of its provisions on
the interest and the honor of the governments which are parties to
it.'' \emph{Ibid.\\; see also The Federalist No. 33, p. 207 (J. Cooke
ed. 1961) (A. Hamilton) (comparing laws that individuals are ``bound
to observe'' as ``the \emph{supreme law} of the land'' with ``a mere
treaty, dependent on the good faith of the parties''). ``If these
[interests] fail, its infraction becomes the subject of international
negotiations and reclamations~.~.~.~. It is obvious that with all
this the judicial courts have nothing to do and can give no redress.''
\emph{Head Money Cases, supra,} at 598. Only ``[i]f the treaty
contains stipulations which are self-executing, that is, require no
legislation to make them operative, [will] they have the force

^2 The label ``self-executing'' has on occasion been used to convey
different meanings. What we mean by ``self-executing'' is that the
treaty has automatic domestic effect as federal law upon ratification.
Conversely, a ``non-self-executing'' treaty does not by itself give
rise to domestically enforceable federal law. Whether such a treaty
has domestic effect depends upon implementing legislation passed by
Congress. \newpage  and effect of a legislative enactment.'' \emph{Whitney,
supra,} at 194.\footnotemark[3]

  Medellín and his \emph{amici} nonetheless contend that the Optional
Protocol, U. N. Charter, and ICJ Statute supply the ``relevant
obligation'' to give the \emph{Avena} judgment binding effect in the
domestic courts of the United States. Reply Brief for Petitioner
5--6.\footnotemark[4] Because none of these treaty sources creates binding
federal law in the absence of implementing legislation, and because it
is uncontested that no such legislation exists, we conclude that the
\emph{Avena} judgment is not automatically binding domestic law.

\subsection{A}

  The interpretation of a treaty, like the interpretation of a statute,
begins with its text. \emph{Air France} v. \emph{Saks,} 470 \newpage  U. S.
392, 396--397 (1985). Because a treaty ratified by the United States
is ``an agreement among sovereign powers,'' we have also considered as
``aids to its interpretation'' the negotiation and drafting history
of the treaty as well as ``the postratification understanding'' of
signatory nations. \emph{Zicherman} v. \emph{Korean Air Lines Co.,} 516
U.~S. 217, 226 (1996); see also \emph{United States} v. \emph{Stuart,} 489 U.
S. 353, 365--366 (1989); \emph{Choctaw Nation} v. \emph{United States,} 318
U.~S. 423, 431--432 (1943).

^3 Even when treaties are self-executing in the sense that they create
federal law, the background presumption is that ``[i]nternational
agreements, even those directly benefiting private persons, generally
do not create private rights or provide for a private cause of action
in domestic courts.'' 2 Restatement (Third) of Foreign Relations Law
of the United States \S~907, Comment \emph{a,} p. 395 (1986) (hereinafter
Restatement). Accordingly, a number of the Courts of Appeals have
presumed that treaties do not create privately enforceable rights
in the absence of express language to the contrary. See, \emph{e. g.,
United States} v. \emph{Emuegbunam,} 268 F. 3d 377, 389 (CA6 2001);
\emph{United States} v. \emph{Jimenez-Nava,} 243 F. 3d 192, 195 (CA5 2001);
\emph{United States} v. \emph{Li,} 206 F. 3d 56, 60--61 (CA1 2000) (en banc);
\emph{Goldstar} (\emph{Panama}) \emph{S. A.} v. \emph{United States,} 967 F. 2d 965,
968 (CA4 1992); \emph{Canadian Transp. Co.} v. \emph{United States,} 663 F.
2d 1081, 1092 (CADC 1980); \emph{Mannington Mills, Inc.} v. \emph{Congoleum
Corp.,} 595 F. 2d 1287, 1298 (CA3 1979).

^4 The question is whether the \emph{Avena} judgment has binding effect
in domestic courts under the Optional Protocol, ICJ Statute, and U.
N. Charter. Consequently, it is unnecessary to resolve whether the
Vienna Convention is itself ``self-executing'' or whether it grants
Medellín individually enforceable rights. See Reply Brief for
Petitioner 5 (disclaiming reliance on the Vienna Convention). As in
\emph{Sanchez-Llamas,} 548 U. S., at 342--343, we thus assume, without
deciding, that Article 36 grants foreign nationals ``an individually
enforceable right to request that their consular officers be notified of
their detention, and an accompanying right to be informed by authorities
of the availability of consular notification.''

  As a signatory to the Optional Protocol, the United States agreed to
submit disputes arising out of the Vienna Convention to the ICJ. The
Protocol provides: ``Disputes arising out of the interpretation or
application of the [Vienna] Convention shall lie within the compulsory
jurisdiction of the International Court of Justice.'' Art. I, 21 U.
S. T., at 326. Of course, submitting to jurisdiction and agreeing to
be bound are two different things. A party could, for example, agree
to compulsory nonbinding arbitration. Such an agreement would require
the party to appear before the arbitral tribunal without obligating the
party to treat the tribunal's decision as binding. See, \emph{e. g.,}
North American Free Trade Agreement, U. S.-Can.-Mex., Art. 2018(1),
Dec. 17, 1992, 32 I. L. M. 605, 697 (1993) (``On receipt of the final
report of [the arbitral panel requested by a Party to the agreement],
the disputing Parties shall agree on the resolution of the dispute,
which normally shall conform with the determinations and recommendations
of the panel'').

  The most natural reading of the Optional Protocol is as a bare grant
of jurisdiction. It provides only that ``[d]isputes arising out of
the interpretation or application of the [Vienna] Convention shall
lie within the compulsory jurisdiction of the International Court of
Justice'' and ``may accordingly be brought before the [ICJ] .~.~.
by any party to the dispute being a Party to the present Protocol.''
Art. I, 21 U. S. T., at 326. The Protocol says nothing about the
effect of an ICJ decision and does not itself commit signatories to
\newpage  comply with an ICJ judgment. The Protocol is similarly silent as
to any enforcement mechanism.

  The obligation on the part of signatory nations to comply with ICJ
judgments derives not from the Optional Protocol, but rather from
Article 94 of the U. N. Charter---the provision that specifically
addresses the effect of ICJ decisions. Article 94(1) provides that
``[e]ach Member of the United Nations \emph{undertakes to comply} with
the decision of the [ICJ] in any case to which it is a party.'' 59
Stat. 1051 (emphasis added). The Executive Branch contends that the
phrase ``undertakes to comply'' is not ``an acknowledgement that an
ICJ decision will have immediate legal effect in the courts of U. N.
members,'' but rather ``a \emph{commitment} on the part of U. N. members
to take \emph{future} action through their political branches to comply
with an ICJ decision.'' Brief for United States as \emph{Amicus Curiae}
in \emph{MedellínI,} O. T. 2004, No. 04--5928, p. 34.

  We agree with this construction of Article 94. The Article is not
a directive to domestic courts. It does not provide that the United
States ``shall'' or ``must'' comply with an ICJ decision, nor
indicate that the Senate that ratified the U. N. Charter intended to
vest ICJ decisions with immediate legal effect in domestic courts.
Instead, ``[t]he words of Article 94\dots call upon governments to
take certain action.'' \emph{Committee of United States Citizens Living
in Nicaragua} v. \emph{Reagan,} 859 F. 2d 929, 938 (CADC 1988) (quoting
\emph{Diggs} v. \emph{Richardson,} 555 F. 2d 848, 851 (CADC 1976); internal
quotation marks omitted). See also \emph{Foster,} 2 Pet., at 314, 315
(holding a treaty non-self-executing because its text---`` ‘all
.~.~. grants of land\dots shall be ratified and confirmed'
''---did not ``act directly on the grants'' but rather ``pledge[d]
the faith of the United States to pass acts which shall ratify and
confirm them''). In other words, the U. N. Charter reads like
``a compact between independent nations'' that ``depends for the
enforcement of its provisions on the interest and the \newpage  honor of
the governments which are parties to it.'' \emph{Head Money Cases,}
112 U. S., at 598.\footnotemark[5] The remainder of Article 94 confirms that
the U. N. Charter does not contemplate the automatic enforceability of
ICJ decisions in domestic courts.\footnotemark[6] Article 94(2)---the enforcement
provision---provides the sole remedy for noncompliance: referral to
the United Nations Security Council by an aggrieved state. 59 Stat.
1051.

  The U. N. Charter's provision of an express diplomatic---that is,
nonjudicial---remedy is itself evidence that ICJ judgments were not
meant to be enforceable in domestic courts. See \emph{Sanchez-Llamas,}
548 U. S., at 347. And even this ``quintessentially \emph{international}
remed[y],'' \emph{id.,} at 355, is not absolute. First, the Security
Council must ``dee[m] necessary'' the issuance of a recommendation
or measure to effectuate the judgment. Art. 94(2), 59 Stat. 1051.
Second, as the President and Senate were undoubtedly aware in
subscribing to the U. N. Charter and Optional Protocol, the \newpage 
United States retained the unqualified right to exercise its veto of any
Security Council resolution.

^5 We do not read ``undertakes'' to mean that `` ‘ ``[t]he United
States\dots shall be at liberty to make respecting th[e] matter, such
laws as they think proper.'' ' '' \emph{Post,} at 554(\textsc{Breyer,}
J., dissenting) (quoting \emph{Todok} v. \emph{Union State Bank of Harvard,}
281 U.~S. 449, 453, 454 (1930) (holding that a treaty with Norway
did \emph{not} ``operat[e] to override the law of [Nebraska] as to the
disposition of homestead property'')). Whether or not the United
States ``undertakes'' to comply with a treaty says nothing about what
laws it may enact. The United States is \emph{always} ``at liberty to
make\dots such laws as [it] think[s] proper.'' \emph{Id.,} at 453.
Indeed, a later-in-time federal statute supersedes inconsistent treaty
provisions. See, \emph{e. g., Cook} v. \emph{United States,} 288 U.~S. 102,
119--120 (1933). Rather, the ``undertakes to comply'' language
confirms that further action to give effect to an ICJ judgment was
contemplated, contrary to the dissent's position that such judgments
constitute directly enforceable federal law, without more. See also
\emph{post,} at 533--535 (\textsc{Stevens,} J., concurring in judgment).

^6 Article 94(2) provides in full: ``If any party to a case fails to
perform the obligations incumbent upon it under a judgment rendered by
the Court, the other party may have recourse to the Security Council,
which may, if it deems necessary, make recommendations or decide upon
measures to be taken to give effect to the judgment.'' 59 Stat.
1051.

  This was the understanding of the Executive Branch when the President
agreed to the U. N. Charter and the declaration accepting general
compulsory ICJ jurisdiction. See, \emph{e. g.,} The Charter of the
United Nations for the Maintenance of International Peace and Security:
Hearings before the Senate Committee on Foreign Relations, 79th Cong.,
1st Sess., 124--125 (1945) (``[I]f a state fails to perform its
obligations under a judgment of the [ICJ], the other party may have
recourse to the Security Council''); \emph{id.,} at 286 (statement of Leo
Pasvolsky, Special Assistant to the Secretary of State for International
Organizations and Security Affairs) (``[W]hen the Court has rendered a
judgment and one of the parties refuses to accept it, then the dispute
becomes political rather than legal. It is as a political dispute
that the matter is referred to the Security Council''); A Resolution
Proposing Acceptance of Compulsory Jurisdiction of International Court
of Justice: Hearings on S. Res. 196 before the Subcommittee of the
Senate Committee on Foreign Relations, 79th Cong., 2d Sess., 142 (1946)
(statement of Charles Fahy, State Dept. Legal Adviser) (while parties
that accept ICJ jurisdiction have ``a moral obligation'' to comply
with ICJ decisions, Article 94(2) provides the exclusive means of
enforcement).

  If ICJ judgments were instead regarded as automatically enforceable
domestic law, they would be immediately and directly binding on state
and federal courts pursuant to the Supremacy Clause. Mexico or the
ICJ would have no need to proceed to the Security Council to enforce
the judgment in this case. Noncompliance with an ICJ judgment through
exercise of the Security Council veto---always regarded as an option by
the Executive and ratifying Senate during and after consideration of the
U. N. Charter, Optional Protocol, and ICJ Statute---would no longer be
a viable alternative. \newpage  There would be nothing to veto. In light
of the U. N. Charter's remedial scheme, there is no reason to believe
that the President and Senate signed up for such a result.

  In sum, Medellín's view that ICJ decisions are automatically
enforceable as domestic law is fatally undermined by the enforcement
structure established by Article 94. His construction would eliminate
the option of noncompliance contemplated by Article 94(2), undermining
the ability of the political branches to determine whether and how to
comply with an ICJ judgment. Those sensitive foreign policy decisions
would instead be transferred to state and federal courts charged
with applying an ICJ judgment directly as domestic law. And those
courts would not be empowered to decide whether to comply with the
judgment---again, always regarded as an option by the political
branches---any more than courts may consider whether to comply with
any other species of domestic law. This result would be particularly
anomalous in light of the principle that ``[t]he conduct of the foreign
relations of our Government is committed by the Constitution to the
Executive and Legislative---‘the political'---Departments.''
\emph{Oetjen} v. \emph{Central Leather Co.,} 246 U.~S. 297, 302 (1918).

  The ICJ Statute, incorporated into the U. N. Charter, provides further
evidence that the ICJ's judgment in \emph{Avena} does not automatically
constitute federal law judicially enforceable in United States courts.
Art. 59, 59 Stat. 1062. To begin with, the ICJ's ``principal
purpose'' is said to be to ``arbitrate particular disputes between
national governments.'' \emph{Sanchez-Llamas, supra,} at 355 (citing
59 Stat. 1055). Accordingly, the ICJ can hear disputes only between
nations, not individuals. Art. 34(1), \emph{id.,} at 1059 (``Only states
[\\i. e.,} countries] may be parties in cases before the [ICJ]'').
More important, Article 59 of the statute provides that ``[t]he
decision of the [ICJ] has \emph{no binding force} except between the
parties and in respect of that particular case.'' \newpage  \emph{Id.,} at
1062 (emphasis added).The dissent does not explain how Medellín, an
individual, can be a party to the ICJ proceeding.

  Medellín argues that because the \emph{Avena} case involves him, it
is clear that he---and the 50 other Mexican nationals named in the
\emph{Avena} decision---should be regarded as parties to the \emph{Avena}
judgment. Brief for Petitioner 21--22. But cases before the ICJ
are often precipitated by disputes involving particular persons or
entities, disputes that a nation elects to take up as its own. See,
\emph{e. g., Case Concerning the Barcelona Traction, Light \& Power Co.}
(\\Belg.} v. \emph{Spain}), 1970 I. C. J. 3 (Judgment of Feb. 5) (claim
brought by Belgium on behalf of Belgian nationals and shareholders);
\emph{Case Concerning the Protection of French Nationals and Protected
Persons in Egypt} (\\Fr.} v. \emph{Egypt}), 1950 I. C. J. 59 (Order of
Mar. 29) (claim brought by France on behalf of French nationals and
protected persons in Egypt); \emph{Anglo-Iranian Oil Co. Case} (\\U. K.}
v. \emph{Iran}), 1952 I. C. J. 93, 112 (Judgment of July 22) (claim brought
by the United Kingdom on behalf of the Anglo-Iranian Oil Company).
That has never been understood to alter the express and established
rules that only nation-states may be parties before the ICJ, Art.
34, 59 Stat. 1059, and---contrary to the position of the dissent,
\emph{post,} at 559---that ICJ judgments are binding only between those
parties, Art. 59, 59 Stat. 1062.\footnotemark[8]

^7 Medellín alters this language in his brief to provide that the ICJ
Statute makes the \emph{Avena} judgment binding ``in respect of [his]
particular case.'' Brief for Petitioner 22 (internal quotation marks
omitted). Medellín does not and cannot have a case before the ICJ
under the terms of the ICJ Statute.

^8 The dissent concludes that the ICJ judgment is binding federal law
based in large part on its belief that the Vienna Convention overrides
contrary state procedural rules. See \emph{post,} at 555--557, 559.
But not even Medellín relies on the Convention. See Reply Brief for
Petitioner 5 (disclaiming reliance). For good reason: Such reliance is
foreclosed by the decision of this Court in \emph{Sanchez-Llamas,} 548
U. S., at 351 (holding that \newpage  the Convention does not preclude
the application of state procedural bars); see also \emph{id.,} at 363
(\textsc{Ginsburg,} J., concurring in judgment). There is no basis for
relitigating the issue. Further, to rely on the Convention would
elide the distinction between a treaty---negotiated by the President
and signed by Congress---and a judgment rendered pursuant to those
treaties.\newpage 

  It is, moreover, well settled that the United States' interpretation
of a treaty ``is entitled to great weight.'' \emph{Sumitomo Shoji
America, Inc.} v. \emph{Avagliano,} 457 U.~S. 176, 184--185 (1982); see
also \emph{El Al Israel Airlines, Ltd.} v. \emph{Tsui Yuan Tseng,} 525 U.~S.
155, 168 (1999). The Executive Branch has unfailingly adhered to its
view that the relevant treaties do not create domestically enforceable
federal law. See Brief for United States as \emph{Amicus Curiae} 4,
27--29.\footnotemark[9]

  The pertinent international agreements, therefore, do not provide for
implementation of ICJ judgments through direct enforcement in domestic
courts, and ``where a treaty does not provide a particular remedy,
either expressly or implicitly, it \newpage  is not for the federal
courts to impose one on the States through lawmaking of their own.''
\emph{Sanchez-Llamas,} 548 U. S., at 347.

^9 In interpreting our treaty obligations, we also consider the
views of the ICJ itself, ``giv[ing] respectful consideration to the
interpretation of an international treaty rendered by an international
court with jurisdiction to interpret [the treaty].'' \emph{Breard}
v. \emph{Greene,} 523 U.~S. 371, 375 (1998) (\\per curiam\\); see
\emph{Sanchez-Llamas, supra,} at 355--356. It is not clear whether
that principle would apply when the question is the binding force
of ICJ judgments themselves, rather than the substantive scope of a
treaty the ICJ must interpret in resolving disputes. Cf. \emph{Phillips
Petroleum Co.} v. \emph{Shutts,} 472 U.~S. 797, 805 (1985) (``[A]
court adjudicating a dispute may not be able to predetermine the res
judicata effect of its own judgment''); 18 C. Wright, A. Miller, \&
E. Cooper, Federal Practice and Procedure \S~4405, p. 82 (2d ed.
2002) (``The first court does not get to dictate to other courts
the preclusion consequences of its own judgment''). In any event,
nothing suggests that the ICJ views its judgments as automatically
enforceable in the domestic courts of signatory nations. The \emph{Avena}
judgment itself directs the United States to provide review and
reconsideration of the affected convictions and sentences \emph{``by means
of its own choosing.''} 2004 I. C. J., at 72, ¶ 153(9) (emphasis
added). This language, as well as the ICJ's mere suggestion that the
``judicial process'' is best suited to provide such review, \emph{id.,}
at 65--66, confirm that domestic enforceability in court is not part
and parcel of an ICJ judgment.

\subsection{B}

  The dissent faults our analysis because it ``looks for the wrong
thing (explicit textual expression about selfexecution) using the
wrong standard (clarity) in the wrong place (the treaty language).''
\emph{Post,} at 562. Given our obligation to interpret treaty
provisions to determine whether they are self-executing, we have to
confess that we do think it rather important to look to the treaty
language to see what it has to say about the issue. That is after all
what the Senate looks to in deciding whether to approve the treaty.

  The interpretive approach employed by the Court today---resorting
to the text---is hardly novel. In two early cases involving an
1819 land-grant treaty between Spain and the United States, Chief
Justice Marshall found the language of the treaty dispositive. In
\emph{Foster,} after distinguishing between self-executing treaties (those
``equivalent to an act of the legislature'') and non-self-executing
treaties (those ``the legislature must execute''), Chief Justice
Marshall held that the 1819 treaty was non-self-executing. 2 Pet.,
at 314. Four years later, the Supreme Court considered another claim
under the same treaty, but concluded that the treaty was self-executing.
See \emph{Percheman,} 7 Pet., at 87. The reason was not because the
treaty was sometimes self-executing and sometimes not, but because
``the language of'' the Spanish translation (brought to the Court's
attention for the first time) indicated the parties' intent to ratify
and confirm the land grant ``by force of the instrument itself.''
\emph{Id.,} at 89.

  As against this time-honored textual approach, the dissent proposes
a multifactor, judgment-by-judgment analysis that would ``jettiso[n]
relative predictability for the open-ended rough-and-tumble of
factors.'' \emph{Jerome B. Grubart, Inc.} v. \emph{Great Lakes Dredge \&
Dock Co.,} 513 U.~S. 527, 547 (1995). \newpage  The dissent's novel
approach to deciding which (or, more accurately, when) treaties give
rise to directly enforceable federal law is arrestingly indeterminate.
Treaty language is barely probative. \emph{Post,} at 549 (``[T]he
absence or presence of language in a treaty about a provision's
self-execution proves nothing at all''). Determining whether treaties
themselves create federal law is sometimes committed to the political
branches and sometimes to the judiciary. \emph{Post,} at 549--550. Of
those committed to the judiciary, the courts pick and choose which shall
be binding United States law---trumping not only state but other federal
law as well---and which shall not. \emph{Post,} at 550--562. They do
this on the basis of a multifactor, ``context-specific'' inquiry.
\emph{Post,} at 549. Even then, the same treaty sometimes gives rise to
United States law and sometimes does not, again depending on an ad hoc
judicial assessment. \emph{Post,} at 550--562.

  Our Framers established a careful set of procedures that
must be followed before federal law can be created under the
Constitution---vesting that decision in the political branches, subject
to checks and balances. U. S. Const., Art. I, \S~7. They also
recognized that treaties could create federal law, but again through the
political branches, with the President making the treaty and the Senate
approving it. Art. II, \S~2. The dissent's understanding of the
treaty route, depending on an ad hoc judgment of the judiciary without
looking to the treaty language---the very language negotiated by the
President and approved by the Senate---cannot readily be ascribed to
those same Framers.

  The dissent's approach risks the United States' involvement in
international agreements. It is hard to believe that the United States
would enter into treaties that are sometimes enforceable and sometimes
not. Such a treaty would be the equivalent of writing a blank check to
the judiciary. Senators could never be quite sure what the treaties on
which they were voting meant. Only a judge could say for sure and only
at some future date. This uncertainty could \newpage  hobble the United
States' efforts to negotiate and sign international agreements.

  In this case, the dissent---for a grab bag of no less than seven
reasons---would tell us that this \emph{particular} ICJ judgment is federal
law. \emph{Post,} at 549--562. That is no sort of guidance. Nor is
it any answer to say that the federal courts will diligently police
international agreements and enforce the decisions of international
tribunals only when they \emph{should be} enforced. \emph{Ibid.} The
point of a non-self-executing treaty is that it ``addresses itself to
the political, \emph{not} the judicial department; and the legislature
must execute the contract before it can become a rule for the Court.''
\emph{Foster, supra,} at 314 (emphasis added); \emph{Whitney,} 124 U. S., at
195. See also \emph{Foster, supra,} at 307 (``The judiciary is not that
department of the government, to which the assertion of its interests
against foreign powers is confided''). The dissent's contrary
approach would assign to the courts---not the political branches---the
primary role in deciding when and how international agreements will
be enforced. To read a treaty so that it sometimes has the effect of
domestic law and sometimes does not is tantamount to vesting with the
judiciary the power not only to interpret but also to create the law.

\subsection{C}

  Our conclusion that \emph{Avena} does not by itself constitute binding
federal law is confirmed by the ``postratification understanding''
of signatory nations. See \emph{Zicherman,} 516 U. S., at 226. There
are currently 47 nations that are parties to the Optional Protocol and
171 nations that are parties to the Vienna Convention. Yet neither
Medellín nor his \emph{amici} have identified a single nation that treats
ICJ judgments as binding in domestic courts.\footnotemark[10] In determining that
the \newpage  Vienna Convention did not require certain relief in United
States courts in \emph{Sanchez-Llamas,} we found it pertinent that the
requested relief would not be available under the treaty in any other
signatory country. See 548 U. S., at 343--344, and n. 3. So too
here the lack of any basis for supposing that any other country would
treat ICJ judgments as directly enforceable as a matter of its domestic
law strongly suggests that the treaty should not be so viewed in our
courts.


^10 The best that the ICJ experts as \emph{amici curiae} can come up with
is the contention that local Moroccan courts have referred to ICJ
judgments as ``dispositive.'' Brief for ICJ Experts as \emph{Amici
Curiae} 20, n. 31. Even the ICJ experts do not cite a case so
holding, and Moroccan practice is at\newpage  best inconsistent, for at
least one local Moroccan court has held that ICJ judgments are not
binding as a matter of municipal law. See, \emph{e. g., Mackay Radio \&
Tel. Co.} v. \emph{Lal-La Fatma Bent si Mohamed el Khadar,} [1954] 21
Int'l L. Rep. 136 (Tangier, Ct. App. Int'l Trib.) (holding that
ICJ decisions are not binding on Morocco's domestic courts); see
also ``\\Socobel''} v. \emph{Greek State,} [1951] 18 Int'l L. Rep. 3
(Belg., Trib. Civ. de Bruxelles) (holding that judgments of the ICJ's
predecessor, the Permanent Court of International Justice, were not
domestically enforceable).


  Our conclusion is further supported by general principles of
interpretation. To begin with, we reiterated in \emph{Sanchez-Llamas} what
we held in \emph{Breard,} that `` ‘absent a clear and express statement
to the contrary, the procedural rules of the forum State govern the
implementation of the treaty in that State.' '' 548 U. S., at 351
(quoting \emph{Breard,} 523 U. S., at 375). Given that ICJ judgments may
interfere with state procedural rules, one would expect the ratifying
parties to the relevant treaties to have clearly stated their intent
to give those judgments domestic effect, if they had so intended. Here
there is no statement in the Optional Protocol, the U. N. Charter, or
the ICJ Statute that supports the notion that ICJ judgments displace
state procedural rules.

  Moreover, the consequences of Medellín's argument give pause. An
ICJ judgment, the argument goes, is not only binding domestic law but
is also unassailable. As a result, neither Texas nor this Court may
look behind a judgment and quarrel with its reasoning or result. (We
already know, from \emph{Sanchez-Llamas,} that this Court disagrees with
both \newpage  the reasoning and result in \emph{Avena.\\) Medellín's
interpretation would allow ICJ judgments to override otherwise binding
state law; there is nothing in his logic that would exempt contrary
federal law from the same fate. See, \emph{e. g., Cook} v. \emph{United
States,} 288 U.~S. 102, 119 (1933) (later-in-time selfexecuting treaty
supersedes a federal statute if there is a conflict). And there is
nothing to prevent the ICJ from ordering state courts to annul criminal
convictions and sentences, for any reason deemed sufficient by the ICJ.
Indeed, that is precisely the relief Mexico requested. \emph{Avena,} 2004
I. C. J., at 58--59.

  Even the dissent flinches at reading the relevant treaties to give
rise to self-executing ICJ judgments in all cases. It admits that
``Congress is unlikely to authorize automatic judicial enforceability
of \emph{all} ICJ judgments, for that could include some politically
sensitive judgments and others better suited for enforcement by other
branches.'' \emph{Post,} at 560. Our point precisely. But the lesson
to draw from that insight is hardly that the judiciary should decide
which judgments are politically sensitive and which are not.

  In short, and as we observed in \emph{Sanchez-Llamas,} ``[n]othing in
the structure or purpose of the ICJ suggests that its interpretations
were intended to be conclusive on our courts.'' 548 U. S., at 354.
Given that holding, it is difficult to see how that same structure and
purpose can establish, as Medellín argues, that \emph{judgments} of the
ICJ nonetheless were intended to be conclusive on our courts. A judgment
is binding only if there is a rule of law that makes it so. And the
question whether ICJ judgments can bind domestic courts depends upon the
same analysis undertaken in \emph{Sanchez-Llamas} and set forth above.

  Our prior decisions identified by the dissent as holding a number
of treaties to be self-executing, see \emph{post,} at 545--546, and
Appendix A, stand only for the unremarkable proposition that some
international agreements are self-executing and others are not. It is
well settled that the ``[i]nterpreta\newpage tion of [a treaty] .~.~.
must, of course, begin with the language of the Treaty itself.''
\emph{Sumitomo Shoji America, Inc.,} 457 U. S., at 180. As a result,
we have held treaties to be selfexecuting when the textual provisions
indicate that the President and Senate intended for the agreement to
have domestic effect.

  Medellín and the dissent cite \emph{Comegys} v. \emph{Vasse,} 1 Pet.
193 (1828), for the proposition that the judgments of international
tribunals are automatically binding on domestic courts. See \emph{post,}
at 546; Reply Brief for Petitioner 2; Brief for Petitioner 19--20.
That case, of course, involved a different treaty than the ones at
issue here; it stands only for the modest principle that the terms of
a treaty control the outcome of a case. \footnotemark[11] We do not suggest that
treaties can never afford binding domestic effect to international
tribunal judgments---only that the U. N. Charter, the Optional Protocol,
and the ICJ Statute do not do so. And whether the treaties underlying a
judgment are self-executing so that the judgment is directly enforceable
as domestic law in our courts is, of course, a matter for this Court to
decide. See \emph{Sanchez-Llamas, supra,} at 353--354.

\subsection{D}

  Our holding does not call into question the ordinary enforcement of
foreign judgments or international arbitral \newpage  agreements. Indeed,
we agree with Medellín that, as a general matter, ``an agreement to
abide by the result'' of an international adjudication---or what he
really means, an agreement to give the result of such adjudication
domestic legal effect---can be a treaty obligation like any other, so
long as the agreement is consistent with the Constitution. See Brief
for Petitioner 20. The point is that the particular treaty obligations
on which Medellín relies do not of their own force create domestic law.

^11 The other case Medellín cites for the proposition that the
judgments of international courts are binding, \emph{La Abra Silver
Mining Co.} v. \emph{United States,} 175 U.~S. 423 (1899), and the
cases he cites for the proposition that this Court has routinely
enforced treaties under which foreign nationals have asserted rights,
similarly stand only for the principle that the terms of a treaty
govern its enforcement. See Reply Brief for Petitioner 4, 5, and n.
2. In each case, this Court first interpreted the treaty prior to
finding it domestically enforceable. See, \emph{e. g., United States}
v. \emph{Rauscher,} 119 U.~S. 407, 422--423 (1886) (holding that the
treaty required extradition only for specified offenses); \emph{Hopkirk}
v. \emph{Bell,} 3 Cranch 454, 458 (1806) (holding that the treaty of peace
between Great Britain and the United States prevented the operation of a
state statute of limitations on British debts).

  The dissent worries that our decision casts doubt on some 70-odd
treaties under which the United States has agreed to submit disputes
to the ICJ according to ``roughly similar'' provisions. See
\emph{post,} at 540--541, 552--553. Again, under our established
precedent, some treaties are self-executing and some are not, depending
on the treaty. That the judgment of an international tribunal might
not automatically become domestic law hardly means the underlying
treaty is ``useless.'' See \emph{post,} at 553; cf. \emph{post,} at 548
(describing the British system in which treaties ``virtually always
requir[e] parliamentary legislation''). Such judgments would still
constitute international obligations, the proper subject of political
and diplomatic negotiations. See \emph{Head Money Cases,} 112 U. S., at
598. And Congress could elect to give them wholesale effect (rather
than the judgment-by-judgment approach hypothesized by the dissent,
\emph{post,} at 560) through implementing legislation, as it regularly
has. See, \emph{e. g.,} Foreign Affairs Reform and Restructuring Act
of 1998, \S~2242, 112 Stat. 2681--822, note following 8 U.~S.~C.
\S~1231 (directing the ``appropriate agencies'' to ``prescribe
regulations to implement the obligations of the United States under
Article 3'' of the Convention Against Torture and Other Forms of Cruel,
Inhuman or Degrading Treatment or Punishment); see also \emph{infra,} at
521--522 (listing examples of legislation implementing international
obligations).

  Further, that an ICJ judgment may not be automatically enforceable
in domestic courts does not mean the particular \newpage  underlying
treaty is not. Indeed, we have held that a number of the ``Friendship,
Commerce, and Navigation'' Treaties cited by the dissent, see
Appendix B, \emph{post,} are selfexecuting---based on ``the language of
the[se] Treat[ies].'' See \emph{Sumitomo Shoji America, Inc., supra,} at
180, 189--190. In \emph{Kolovrat} v. \emph{Oregon,} 366 U.~S. 187, 191,
196 (1961), for example, the Court found that Yugoslavian claimants
denied inheritance under Oregon law were entitled to inherit personal
property pursuant to an 1881 Treaty of Friendship, Navigation, and
Commerce between the United States and Serbia. See also \emph{Clark} v.
\emph{Allen,} 331 U.~S. 503, 507--511, 517--518 (1947) (finding that the
right to inherit real property granted German aliens under the Treaty
of Friendship, Commerce and Consular Rights with Germany prevailed
over California law). Contrary to the dissent's suggestion, see
\emph{post,} at 547, neither our approach nor our cases require that a
treaty provide for self-execution in so many talismanic words; that is
a caricature of the Court's opinion. Our cases simply require courts
to decide whether a treaty's terms reflect a determination by the
President who negotiated it and the Senate that confirmed it that the
treaty has domestic effect.

  In addition, Congress is up to the task of implementing
non-self-executing treaties, even those involving complex commercial
disputes. Cf. \emph{post,} at 560 (\textsc{Breyer,} J., dissenting). The
judgments of a number of international tribunals enjoy a different
status because of implementing legislation enacted by Congress. See,
\emph{e. g.,} 22 U.~S.~C. \S~1650a(a) (``An award of an arbitral
tribunal rendered pursuant to chapter IV of the [Convention on the
Settlement of Investment Disputes] shall create a right arising under
a treaty of the United States. The pecuniary obligations imposed by
such an award shall be enforced and shall be given the same full
faith and credit as if the award were a final judgment of a court of
general jurisdiction of one of the several States''); 9 U.~S.~C.
\S\S~201--208 (``The [U. N.] Convention on the Recogni\newpage tion
and Enforcement of Foreign Arbitral Awards of June 10, 1958, shall be
enforced in United States courts in accordance with this chapter,''
\S~201). Such language demonstrates that Congress knows how to accord
domestic effect to international obligations when it desires such a
result.\footnotemark[12]

  Further, Medellín frames his argument as though giving the \emph{Avena}
judgment binding effect in domestic courts simply conforms to the
proposition that domestic courts generally give effect to foreign
judgments. But Medellín does not ask us to enforce a foreign-court
judgment settling a typical commercial or property dispute. See,
\emph{e. g., Hilton} v. \emph{Guyot,} 159 U.~S. 113 (1895); \emph{United
States} v. \emph{Arredondo,} 6 Pet. 691 (1832); see also Uniform Foreign
Money-Judgments Recognition Act \S~1(2), 13 U. L. A., pt. 2, p. 44
(2002) (`` ‘[F]oreign judgment' means any judgment of a foreign
state granting or denying recovery of a sum of money''). Rather,
Medellín argues that the \emph{Avena} judgment has the effect of enjoining
the operation of state law. What is more, on Medellín's view,
the judgment would force the State to take action to ``review and
reconside[r]'' his case. The general rule, however, is that judgments
of foreign courts awarding injunctive relief, even as to private
parties, let alone sovereign States, ``are not generally entitled to
enforcement.'' See 1 Restatement \S~481, Comment \emph{b,} at 595.

  In sum, while the ICJ's judgment in \emph{Avena} creates an
international law obligation on the part of the United States, it
does not of its own force constitute binding federal law \newpage 
that pre-empts state restrictions on the filing of successive habeas
petitions. As we noted in \emph{Sanchez-Llamas,} a contrary conclusion
would be extraordinary, given that basic rights guaranteed by our own
Constitution do not have the effect of displacing state procedural
rules. See 548 U. S., at 360. Nothing in the text, background,
negotiating and drafting history, or practice among signatory nations
suggests that the President or Senate intended the improbable result
of giving the judgments of an international tribunal a higher status
than that enjoyed by ``many of our most fundamental constitutional
protections.'' \emph{Ibid.}

^12 That this Court has rarely had occasion to find a treaty
non-selfexecuting is not all that surprising. See \emph{post,} at 545
(\textsc{Breyer,} J., dissenting). To begin with, the Courts of Appeals
have regularly done so. See, \emph{e. g., Pierre} v. \emph{Gonzales,} 502 F.
3d 109, 119--120 (CA2 2007) (holding that the United Nations Convention
Against Torture and Other Cruel, Inhuman or Degrading Treatment or
Punishment is non-self-executing); \emph{Singh} v. \emph{Ashcroft,} 398 F.
3d 396, 404, n. 3 (CA6 2005) (same); \emph{Beazley} v. \emph{Johnson,} 242
F. 3d 248, 267 (CA5 2001) (holding that the International Covenant on
Civil and Political Rights is non-self-executing). Further, as noted,
Congress has not hesitated to pass implementing legislation for treaties

\section{III}

  Medellín next argues that the ICJ's judgment in \emph{Avena} is
binding on state courts by virtue of the President's February 28, 2005
Memorandum. The United States contends that while the \emph{Avena} judgment
does not of its own force require domestic courts to set aside ordinary
rules of procedural default, that judgment became the law of the land
with precisely that effect pursuant to the President's Memorandum
and his power ``to establish binding rules of decision that preempt
contrary state law.'' Brief for United States as \emph{Amicus Curiae} 5.
Accordingly, we must decide whether the President's declaration alters
our conclusion that the \emph{Avena} judgment is not a rule of domestic law
binding in state and federal courts.\footnotemark[13]

\subsection{A}

  The United States maintains that the President's constitutional
role ``uniquely qualifies'' him to resolve the sensitive \newpage 
foreign policy decisions that bear on compliance with an ICJ decision
and ``to do so expeditiously.'' Brief for United States as \emph{Amicus
Curiae} 11, 12. We do not question these propositions. See, \emph{e. g.,
First Nat. City Bank} v. \emph{Banco Nacional de Cuba,} 406 U.~S. 759,
767 (1972) (plurality opinion) (The President has ``the lead role
.~.~. in foreign policy''); \emph{American Ins. Assn.} v. \emph{Garamendi,}
539 U.~S. 396, 414 (2003) (Article II of the Constitution places with
the President the `` ‘vast share of responsibility for the conduct
of our foreign relations' '' (quoting \emph{Youngstown Sheet \& Tube
Co.} v. \emph{Sawyer,} 343 U.~S. 579, 610--611 (1952) (Frankfurter,
J., concurring))). In this case, the President seeks to vindicate
United States interests in ensuring the reciprocal observance of the
Vienna Convention, protecting relations with foreign governments,
and demonstrating commitment to the role of international law. These
interests are plainly compelling.

^13 The dissent refrains from deciding the issue, but finds it
``difficult to believe that in the exercise of his Article II powers
pursuant to a ratified treaty, the President can \emph{never} take action
that would result in setting aside state law.'' \emph{Post,} at 564.
We agree. The questions here are the far more limited ones of whether
he may unilaterally create federal law by giving effect to the judgment
of this international tribunal pursuant to this non-self-executing
treaty, and, if not, whether he may rely on other authority under the
Constitution to support the action taken in this partic

  Such considerations, however, do not allow us to set aside first
principles. The President's authority to act, as with the exercise
of any governmental power, ``must stem either from an act of Congress
or from the Constitution itself.'' \emph{Youngstown, supra,} at 585;
\emph{Dames \& Moore} v. \emph{Regan,} 453 U.~S. 654, 668 (1981).

  Justice Jackson's familiar tripartite scheme provides the accepted
framework for evaluating executive action in this area. First, ``[w]hen
the President acts pursuant to an express or implied authorization of
Congress, his authority is at its maximum, for it includes all that
he possesses in his own right plus all that Congress can delegate.''
\emph{Youngstown,} 343 U. S., at 635 (concurring opinion). Second,
``[w]hen the President acts in absence of either a congressional grant
or denial of authority, he can only rely upon his own independent
powers, but there is a zone of twilight in which he and Congress may
have concurrent authority, or in which its distribution is uncertain.''
\emph{Id.,} at 637. In this circumstance, Presidential authority
can derive support from ``congressional inertia, indifference or
quiescence.'' \emph{Ibid.} \newpage  Finally, ``[w]hen the President
takes measures incompatible with the expressed or implied will of
Congress, his power is at its lowest ebb,'' and the Court can sustain
his actions ``only by disabling the Congress from acting upon the
subject.'' \emph{Id.,} at 637--638.

\subsection{B}

  The United States marshals two principal arguments in favor of the
President's authority ``to establish binding rules of decision that
preempt contrary state law.'' Brief for United States as \emph{Amicus
Curiae} 5. The Solicitor General first argues that the relevant
treaties give the President the authority to implement the \emph{Avena}
judgment and that Congress has acquiesced in the exercise of such
authority. The United States also relies upon an ``independent''
international dispute-resolution power wholly apart from the asserted
authority based on the pertinent treaties. Medellín adds the additional
argument that the President's Memorandum is a valid exercise of his
power to take care that the laws be faithfully executed.

\subsubsection{1}

  The United States maintains that the President's Memorandum is
authorized by the Optional Protocol and the U. N. Charter. Brief for
United States as \emph{Amicus Curiae} 9. That is, because the relevant
treaties ``create an obligation to comply with \emph{Avena,}'' they
``\emph{implicitly} give the President authority to implement that
treaty-based obligation.'' \emph{Id.,} at 11 (emphasis added). As
a result, the President's Memorandum is well grounded in the first
category of the \emph{Youngstown} framework.

  We disagree. The President has an array of political and diplomatic
means available to enforce international obligations, but unilaterally
converting a non-self-executing treaty into a self-executing one is
not among them. The responsibility for transforming an international
obligation arising from a non-self-executing treaty into domestic law
falls to \newpage  Congress. \emph{Foster,} 2 Pet., at 315; \emph{Whitney,}
124 U. S., at 194; \emph{Igartu´ a-De La Rosa,} 417 F. 3d, at 150.
As this Court has explained, when treaty stipulations are ``not
self-executing they can only be enforced pursuant to legislation to
carry them into effect.'' \emph{Whitney, supra,} at 194. Moreover,
``[u]ntil such act shall be passed, the Court is not at liberty to
disregard the existing laws on the subject.'' \emph{Foster, supra,} at
315.

  The requirement that Congress, rather than the President, implement
a non-self-executing treaty derives from the text of the Constitution,
which divides the treaty-making power between the President and the
Senate. The Constitution vests the President with the authority to
``make'' a treaty. Art. II, \S~2. If the Executive determines
that a treaty should have domestic effect of its own force, that
determination may be implemented in ``mak[ing]'' the treaty, by
ensuring that it contains language plainly providing for domestic
enforceability. If the treaty is to be self-executing in this respect,
the Senate must consent to the treaty by the requisite two-thirds vote,
\emph{ibid.,} consistent with all other constitutional restraints.

  Once a treaty is ratified without provisions clearly according it
domestic effect, however, whether the treaty will ever have such effect
is governed by the fundamental constitutional principle that ``
‘[t]he power to make the necessary laws is in Congress; the power to
execute in the President.' '' \emph{Hamdan} v. \emph{Rumsfeld,} 548 U. S.
557, 591 (2006) (quoting \emph{Ex parte Milligan,} 4 Wall. 2, 139 (1866)
(opinion of Chase, C. J.)); see U. S. Const., Art. I, \S~1 (``All
legislative Powers herein granted shall be vested in a Congress of the
United States''). As already noted, the terms of a non-selfexecuting
treaty can become domestic law only in the same way as any other
law---through passage of legislation by both Houses of Congress,
combined with either the President's signature or a congressional
override of a Presidential veto. See Art. I, \S7. Indeed, ``the
President's power to see that \newpage  the laws are faithfully executed
refutes the idea that he is to be a lawmaker.'' \emph{Youngstown,} 343
U. S., at 587.

  A non-self-executing treaty, by definition, is one that was ratified
with the understanding that it is not to have domestic effect of its
own force. That understanding precludes the assertion that Congress has
implicitly authorized the President---acting on his own---to achieve
precisely the same result. We therefore conclude, given the absence of
congressional legislation, that the non-self-executing treaties at issue
here did not ``express[ly] or implied[ly]'' vest the President with
the unilateral authority to make them selfexecuting. See \emph{id.,}
at 635 (Jackson, J., concurring). Accordingly, the President's
Memorandum does not fall within the first category of the \emph{Youngstown}
framework.

  Indeed, the preceding discussion should make clear that the
non-self-executing character of the relevant treaties not only refutes
the notion that the ratifying parties vested the President with the
authority to unilaterally make treaty obligations binding on domestic
courts, but also implicitly prohibits him from doing so. When the
President asserts the power to ``enforce'' a non-self-executing
treaty by unilaterally creating domestic law, he acts in conflict with
the implicit understanding of the ratifying Senate. His assertion of
authority, insofar as it is based on the pertinent non-selfexecuting
treaties, is therefore within Justice Jackson's third category, not
the first or even the second. See \emph{id.,} at 637--638.

  Each of the two means described above for giving domestic effect
to an international treaty obligation under the Constitution---for
making law---requires joint action by the Executive and Legislative
Branches: The Senate can ratify a self-executing treaty ``ma[de]''
by the Executive, or, if the ratified treaty is not self-executing,
Congress can enact implementing legislation approved by the President.
It should not be surprising that our Constitution does not contemplate
vesting such power in the Executive alone. As Madison ex\newpage plained
in The Federalist No. 47, under our constitutional system of checks and
balances, ``[t]he magistrate in whom the whole executive power resides
cannot of himself make a law.'' J. Cooke ed., p. 326 (1961). That
would, however, seem an apt description of the asserted executive
authority unilaterally to give the effect of domestic law to obligations
under a non-self-executing treaty.

  The United States nonetheless maintains that the President's
Memorandum should be given effect as domestic law because ``this case
involves a valid Presidential action in the context of Congressional
‘acquiescence.' '' Brief for United States as \emph{Amicus Curiae}
11, n. 2. Under the \emph{Youngstown} tripartite framework, congressional
acquiescence is pertinent when the President's action falls within
the second category---that is, when he ``acts in absence of either
a congressional grant or denial of authority.'' 343 U. S., at 637
(Jackson, J., concurring). Here, however, as we have explained, the
President's effort to accord domestic effect to the \emph{Avena} judgment
does not meet that prerequisite.

  In any event, even if we were persuaded that congressional
acquiescence could support the President's asserted authority to
create domestic law pursuant to a non-selfexecuting treaty, such
acquiescence does not exist here. The United States first locates
congressional acquiescence in Congress's failure to act following the
President's resolution of prior ICJ controversies. A review of the
Executive's actions in those prior cases, however, cannot support
the claim that Congress acquiesced in this particular exercise of
Presidential authority, for none of them remotely involved transforming
an international obligation into domestic law and thereby displacing
state law.\footnotemark[14]


^14 Rather, in the \emph{Case Concerning Military and Paramilitary
Activities in and Against Nicaragua} (\\Nicar.} v. \emph{U. S.\\), 1986
I. C. J. 14 (Judgment of June 27), the President determined that
the United States would \emph{not} comply with the ICJ's conclusion
that the United States owed reparations to Nicaragua. In the \emph{Case
Concerning Delimitation of the Maritime} \newpage \emph{Boundary in the Gulf
of Maine Area} (\\Can.} v. \emph{U. S.\\), 1984 I. C. J. 246 (Judgment
of Oct. 12), a federal agency---the National Oceanic and Atmospheric
Administration---issued a final rule which complied with the ICJ's
boundary determination. The \emph{Case Concerning Rights of Nationals of
the United States of America in Morocco} (\\Fr.} v. \emph{U. S.\\), 1952
I. C. J. 176 (Judgment of Aug. 27), concerned the legal status of
United States citizens living in Morocco; it was not enforced in United
States courts.

^   The final two cases arose under the Vienna Convention. In the
\emph{LaGrand Case} (\\F. R. G.} v. \emph{U. S.\\), 2001 I. C. J. 466
(Judgment of June 27), the ICJ ordered the review and reconsideration
of convictions and sentences of German nationals denied consular
notification. In response, the State Department sent letters to the
States ``encouraging'' them to consider the Vienna Convention in
the clemency process. Brief for United States as \emph{Amicus Curiae}
20--21. Such encouragement did not give the ICJ judgment direct
effect as domestic law; thus, it cannot serve as precedent for doing
so in which Congress might be said to have acquiesced. In the \emph{Case
Concerning the Vienna Convention on Consular Relations} (\\Para.} v.
\emph{U. S.\\), 1998 I. C. J. 248 (Judgment of Apr. 9), the ICJ issued a
provisional order, directing the United States to ``\\take all measures
at its disposal} to ensure that [Breard] is not executed pending the
final decision in [the ICJ's] proceedings.'' \emph{Breard,} 523 U.
S., at 374 (internal quotation marks omitted; emphasis added). In
response, the Secretary of State sent a letter to the Governor of
Virginia requesting that he stay Breard's execution. \emph{Id.,}
at 378. When Paraguay sought a stay of execution from this Court,
the United States argued that it had taken every measure at its
disposal: because ``our federal system imposes limits on the federal
government's ability to interfere with the criminal justice systems
of the States,'' those measures included ``only persuasion,'' not
``legal compulsion.'' Brief for United States as \emph{Amicus Curiae,}
O. T. 1997, No. 97--8214 (A--732), p. 51. This of course is
precedent contrary to the proposition asserted by the Solicitor General
in this case.\newpage 

  The United States also directs us to the President's ``related''
statutory responsibilities and to his ``established role'' in
litigating foreign policy concerns as support for the President's
asserted authority to give the ICJ's decision in \emph{Avena} the
force of domestic law. Brief for United States as \emph{Amicus Curiae}
16--19. Congress has indeed authorized the President to represent
the United States before the United Nations, the ICJ, and the Security
Council, 22 U.~S.~C. \S~287, but the authority of the President
to represent the United \newpage  States before such bodies speaks to the
President's \emph{international} responsibilities, not any unilateral
authority to create domestic law. The authority expressly conferred by
Congress in the international realm cannot be said to ``invite'' the
Presidential action at issue here. See \emph{Youngstown, supra,} at 637
(Jackson, J., concurring). At bottom, none of the sources of authority
identified by the United States supports the President's claim that
Congress has acquiesced in his asserted power to establish on his own
federal law or to override state law.

  None of this is to say, however, that the combination of a
non-self-executing treaty and the lack of implementing legislation
precludes the President from acting to comply with an international
treaty obligation. It is only to say that the Executive cannot
unilaterally execute a non-self-executing treaty by giving it domestic
effect. That is, the non-selfexecuting character of a treaty constrains
the President's ability to comply with treaty commitments by
unilaterally making the treaty binding on domestic courts. The President
may comply with the treaty's obligations by some other means, so long
as they are consistent with the Constitution. But he may not rely upon a
non-self-executing treaty to ``establish binding rules of decision that
preempt contrary state law.'' Brief for United States as \emph{Amicus
Curiae} 5.

\subsubsection{2}

  We thus turn to the United States' claim that---independent of
the United States' treaty obligations---the Memorandum is a valid
exercise of the President's foreign affairs authority to resolve
claims disputes with foreign nations. \emph{Id.,} at 12--16. The
United States relies on a series of cases in which this Court has upheld
the authority of the President to settle foreign claims pursuant to an
executive agreement. See \emph{Garamendi,} 539 U. S., at 415; \emph{Dames
\& Moore,} 453 U. S., at 679--680; \emph{United States} v. \emph{Pink,}
315 U.~S. 203, 229 (1942); \newpage  \emph{United States} v. \emph{Belmont,}
301 U.~S. 324, 330 (1937). In these cases this Court has explained
that, if pervasive enough, a history of congressional acquiescence
can be treated as a ``gloss on ‘Executive Power' vested in the
President by \S~1 of Art. II.'' \emph{Dames \& Moore, supra,} at 686
(some internal quotation marks omitted).

  This argument is of a different nature than the one rejected above.
Rather than relying on the United States' treaty obligations, the
President relies on an independent source of authority in ordering
Texas to put aside its procedural bar to successive habeas petitions.
Nevertheless, we find that our claims-settlement cases do not support
the authority that the President asserts in this case.

  The claims-settlement cases involve a narrow set of circumstances:
the making of executive agreements to settle civil claims between
American citizens and foreign governments or foreign nationals. See,
\emph{e. g., Belmont, supra,} at 327. They are based on the view that
``a systematic, unbroken, executive practice, long pursued to the
knowledge of the Congress and never before questioned,'' can ``raise
a presumption that the [action] had been [taken] in pursuance of its
consent.'' \emph{Dames \& Moore, supra,} at 686 (internal quotation marks
omitted). As this Court explained in \emph{Garamendi}:

    ``Making executive agreements to settle claims of Amer ican
    nationals against foreign governments is a particu larly
    longstanding practice~.~.~.~. Given the fact that the
    practice goes back over 200 years, and has received congressional
    acquiescence throughout its history, the conclusion that the
    President's control of foreign rela tions includes the settlement
    of claims is indisputable.'' 539 U. S., at 415 (internal
    quotation marks and brack ets omitted).

\noindent Even still, the limitations on this source of executive power are
clearly set forth and the Court has been careful to note \newpage  that
``[p]ast practice does not, by itself, create power.'' \emph{Dames \&
Moore, supra,} at 686.

  The President's Memorandum is not supported by a ``particularly
longstanding practice'' of congressional acquiescence, see \emph{Garamendi,
supra,} at 415, but rather is what the United States itself has
described as ``unprecedented action,'' Brief for United States as
\emph{Amicus Curiae} in \emph{Sanchez-Llamas,} O. T. 2005, Nos. 05--51 and
04--10566, pp. 29--30. Indeed, the Government has not identified
a single instance in which the President has attempted (or Congress
has acquiesced in) a Presidential directive issued to state courts,
much less one that reaches deep into the heart of the State's police
powers and compels state courts to reopen final criminal judgments
and set aside neutrally applicable state laws. Cf. \emph{Brecht} v.
\emph{Abrahamson,} 507 U.~S. 619, 635 (1993) (``States possess primary
authority for defining and enforcing the criminal law'' (quoting
\emph{Engle} v. \emph{Isaac,} 456 U.~S. 107, 128 (1982); internal quotation
marks omitted)). The Executive's narrow and strictly limited
authority to settle international claims disputes pursuant to an
executive agreement cannot stretch so far as to support the current
Presidential Memorandum.

\subsubsection{3}

  Medellín argues that the President's Memorandum is a valid exercise
of his ``[T]ake Care'' power. Brief for Petitioner 28. The United
States, however, does not rely upon the President's responsibility to
``take Care that the Laws be faithfully executed.'' U. S. Const.,
Art. II, \S~3. We think this a wise concession. This authority allows
the President to execute the laws, not make them. For the reasons we
have stated, the \emph{Avena} judgment is not domestic law; accordingly,
the President cannot rely on his Take Care powers here.

  The judgment of the Texas Court of Criminal Appeals is affirmed.

\begin{flushright}\emph{It is so ordered.}\end{flushright}
