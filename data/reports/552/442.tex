% Court
% Thomas

\setcounter{page}{444}

  \textsc{Justice Thomas} delivered the opinion of the Court.

  In 2004, voters in the State of Washington passed an initiative
changing the State's primary election system. The People's Choice
Initiative of 2004, or Initiative 872 (I--872), provides that
candidates for office shall be identified on the ballot by their
self-designated ``party preference''; that voters may vote for any
candidate; and that the top two votegetters for each office, regardless
of party preference, advance to the general election. The Court of
Appeals for the Ninth Circuit held I--872 facially invalid as imposing
an unconstitutional burden on state political parties' First Amendment
rights. Because I--872 does not on its face impose a severe burden on
political parties' associational rights, and because respondents'
arguments to the contrary rest on factual assumptions about voter
confusion that can be evaluated only in the context of an as-applied
challenge, we reverse. \newpage 


\section{I}

  For most of the past century, Washington voters selected nominees for
state and local offices using a blanket primary.\footnotemark[1] From 1935 until
2003, the State used a blanket primary that placed candidates from all
parties on one ballot and allowed voters to select a candidate from any
party. See 1935 Wash. Laws \S\S~1--5, pp. 60--64. Under this
system, the candidate who won a plurality of votes within each major
party became that party's nominee in the general election. See 2003
Wash. Laws \S~919, p. 775.

  California used a nearly identical primary in its own elections until
our decision in \emph{California Democratic Party} v. \emph{Jones,} 530
U. S. 567 (2000). In \emph{Jones,} four political parties challenged
California's blanket primary, arguing that it unconstitutionally
burdened their associational rights by forcing them to associate with
voters who did not share their beliefs. We agreed and struck down
the blanket primary as inconsistent with the First Amendment. In so
doing, we emphasized the importance of the nomination process as
``‘the crucial juncture at which the appeal to common principles
may be translated into concerted action, and hence to political power
in the community.' '' \emph{Id.,} at 575 (quoting \emph{Tashjian} v.
\emph{Republican Party of Conn.,} 479 U.~S. 208, 216 (1986)). We
observed that a party's right to exclude is central to its freedom
of association, and is never ``more important than in the process
of selecting its nominee.'' 530 U. S., at 575. California's
blanket primary, we concluded, severely burdened the parties' freedom
of association because it \newpage  forced them to allow nonmembers to
participate in selecting the parties' nominees. That the parties
retained the right to endorse their preferred candidates did not render
the burden any less severe, as ``[t]here is simply no substitute for a
party's selecting its own candidates.'' \emph{Id.,} at 581.

^1 The term ``blanket primary'' refers to a system in which ``any
person, regardless of party affiliation, may vote for a party's
nominee.'' \emph{California Democratic Party} v. \emph{Jones,} 530
U.~S. 567, 576, n. 6 (2000). A blanket primary is distinct from
an ``open primary,'' in which a person may vote for any party's
nominees, but must choose among that party's nominees for all offices,
\emph{ibid.,} and the more traditional ``closed primary,'' in which
``only persons who are members of the political party\dots can vote
on its nominee,'' \emph{id.,} at 570.

  Because California's blanket primary severely burdened the
parties' associational rights, we subjected it to strict scrutiny,
carefully examining each of the state interests offered by California in
support of its primary system. We rejected as illegitimate three of the
asserted interests: ``producing elected officials who better represent
the electorate,'' ``expanding candidate debate beyond the scope of
partisan concerns,'' and ensuring ``the right to an effective vote''
by allowing nonmembers of a party to vote in the majority party's
primary in `` ‘safe' '' districts. \emph{Id.,} at 582--584.
We concluded that the remaining interests---promoting fairness,
affording voters greater choice, increasing voter participation, and
protecting privacy---were not compelling on the facts of the case.
Even if they were, the partisan California primary was not narrowly
tailored to further those interests because a nonpartisan blanket
primary, in which the top two votegetters advance to the general
election regardless of party affiliation, would accomplish each
of those interests without burdening the parties' associational
rights. \emph{Id.,} at 585--586. The nonpartisan blanket primary had
``all the characteristics of the partisan blanket primary, save the
constitutionally crucial one: Primary voters [were] not choosing a
party's nominee.'' \emph{Ibid.}

  After our decision in \emph{Jones,} the Court of Appeals for the
Ninth Circuit struck down Washington's primary as ``materially
indistinguishable from the California scheme.'' \emph{Democratic Party
of Washington State} v. \emph{Reed,} 343 F. 3d 1198, 1203 (2003). The
Washington State Grange\footnotemark[2] promptly pro\newpage posed I--872 as a
replacement.It passed with nearly 60% of the vote and became effective
in December 2004.

^2 The Washington State Grange is a fraternal, social, and civic
organization chartered by the National Grange in 1889. Although
originally formed to represent the interests of farmers, the
organization has advo \newpage  cated a variety of goals, including
women's suffrage, rural electrification, protection of water
resources, and universal telephone service. The State Grange also
supported the Washington constitutional amendment establishing
initiatives and referendums and sponsored the 1934 blanket primary
initiative.

  Under I--872, all elections for ``partisan offices''\footnotemark[4]
are conducted in two stages: a primary and a general election. To
participate in the primary, a candidate must file a ``declaration of
candidacy'' form, on which he declares his ``major or minor party
preference, or independent status.'' Wash. Rev. Code \S~29A.24.030
(Supp. 2005). Each candidate and his party preference (or independent
status) is in turn designated on the primary election ballot. A
political party cannot prevent a candidate who is unaffiliated with,
or even repugnant to, the party from designating it as his party
of preference. See App. 396--397, 595 (declaration of James K.
Pharris, Exhibit C: Ruling Order, May 18, 2005, Wash. Admin. Code
\S~434--215--015). In the primary election, voters may select
``any candidate listed on the ballot, regardless of the party
preference of the candidates or the voter.'' \emph{Id.,} at 606,
\S~434--262--012.

  The candidates with the highest and second-highest vote totals advance
to the general election, regardless of their \newpage  party preferences.
\emph{Ibid.} Thus, the general election may pit two candidates with
the same party preference against one another.\footnotemark[5] Each candidate's
party preference is listed on the general election ballot, and may not
be changed between the primary and general elections. See \emph{id.,} at
601, \S~434--230--040.


^3 Respondents make much of the fact that the promoters of I--872
presented it to Washington voters as a way to preserve the primary
system in place from 1935 to 2003. But our task is not to judge
I--872 based on its promoters' assertions about its similarity, or
lack thereof, to the unconstitutional primary; we must evaluate the
constitutionality of I--872 on its own terms. Whether the language of
I--872 was purposely drafted to survive a \emph{Jones}-type constitutional
challenge is irrelevant to whether it has successfully done so.

^4 `` ‘Partisan office' means a public office for which a candidate
may indicate a political party preference on his or her declaration of
candidacy and have that preference appear on the primary and general
election ballot in conjunction with his or her name.'' Wash. Rev.
Code \S~29A.04.110 (Supp. 2005).

  Immediately after the State enacted regulations to implement I--872,
the Washington State Republican Party filed suit against a number of
county auditors challenging the law on its face. The party contended
that the new system violates its associational rights by usurping its
right to nominate its own candidates and by forcing it to associate
with candidates it does not endorse. The Washington State Democratic
Central Committee and Libertarian Party of Washington State joined the
suit as plaintiffs. The Washington State Grange joined as a defendant,
and the State of Washington was substituted for the county auditors as
defendant. The United States District Court for the Western District of
Washington granted the political parties' motions for summary judgment
and enjoined the implementation of I--872. See \emph{Washington State
Republican Party} v. \emph{Logan,} 377 F. Supp. 2d 907, 932 (2005).

  The Court of Appeals affirmed. 460 F. 3d 1108, 1125 (CA9 2006). It
held that the I--872 primary severely burdens the political parties'
associational rights because the partypreference designation on the
ballot creates a risk that primary winners will be perceived as the
parties' nominees and produces an ``impression of associatio[n]''
between a candidate and his party of preference even when the party does
not associate, or wish to be associated, with the candidate. \emph{Id.,}
at 1119. The Court of Appeals noted a ``constitutionally \newpage 
significant distinction between ballots and other vehicles for political
expression,'' reasoning that the risk of perceived association is
particularly acute when ballots include party labels because such labels
are typically used to designate candidates' views on issues of public
concern. \emph{Id.,} at 1121. And it determined that the State's
interests underlying I--872 were not sufficiently compelling to justify
the severe burden on the parties' association. Concluding that the
provisions of I--872 providing for the party-preference designation
on the ballot were not severable, the court struck down I--872 in its
entirety.

^5 This is not a hypothetical outcome. The Court of Appeals observed
that, had the 1996 gubernatorial primary been conducted under the
I--872 system, two Democratic candidates and no Republican candidate
would have advanced from the primary to the general election. See 460
F. 3d 1108, 1114, n. 8 (CA9 2006).

  We granted certiorari, 549 U.~S. 1251 (2007), to determine
whether I--872, on its face, violates the political parties'
associational rights.

\section{II}

  Respondents object to I--872 not in the context of an actual
election, but in a facial challenge. Under \emph{United States} v.
\emph{Salerno,} 481 U.~S. 739 (1987), a plaintiff can only succeed in
a facial challenge by ``establish[ing] that no set of circumstances
exists under which the Act would be valid,'' \emph{i. e.,} that the law is
unconstitutional in all of its applications. \emph{Id.,} at 745. While
some Members of the Court have criticized the \emph{Salerno} formulation,
all agree that a facial challenge must fail where the statute has a ``
‘plainly legitimate sweep.' '' \emph{Washington} v. \emph{Glucksberg,}
521 U.~S. 702, 739--740, and n. 7 (1997) (\textsc{Stevens,} J., concurring
in judgments). Washington's primary system survives under either
standard, as we explain below.\footnotemark[6] In determining whether a law is
facially in\newpage valid, we must be careful not to go beyond the
statute's facial requirements and speculate about ``hypothetical''
or ``imaginary'' cases. See \emph{United States} v. \emph{Raines,} 362
U. S. 17, 22 (1960) (``The delicate power of pronouncing an Act
of Congress unconstitutional is not to be exercised with reference
to hypothetical cases thus imagined''). The State has had no
opportunity to implement I--872, and its courts have had no occasion
to construe the law in the context of actual disputes arising from the
electoral context, or to accord the law a limiting construction to avoid
constitutional questions. Cf. \emph{Yazoo \& Mississippi Valley R. Co.} v.
\emph{Jackson Vinegar Co.,} 226 U.~S. 217, 220 (1912) (``How the state
court may apply [a statute] to other cases, whether its general words
may be treated as more or less restrained, and how far parts of it may
be sustained if others fail are matters upon which we need not speculate
now''). Exercising judicial restraint in a facial challenge ``frees
the Court not only from unnecessary pronouncement on constitutional
issues, but also from premature interpretations of statutes in areas
where their constitutional application might be cloudy.'' \emph{Raines,
supra,} at 22.

^6 Our cases recognize a second type of facial challenge in the First
Amendment context under which a law may be overturned as impermissibly
overbroad because a ``substantial number'' of its applications are
unconstitutional, `` ‘judged in relation to the statute's plainly
legitimate sweep.' '' \emph{New York} v. \emph{Ferber,} 458 U.~S. 747,
769--771 (1982) (quoting \emph{Broadrick} v. \emph{Oklahoma,} 413 U.~S. 601,
615 (1973)). We generally do not apply the `` ‘strong medicine'
'' of overbreadth analysis where the parties fail to describe the
instances of arguable overbreadth of the contested \newpage  law. See
\emph{New York State Club Assn., Inc.} v. \emph{City of New York,} 487 U. S.
1, 14 (1988).

  Facial challenges are disfavored for several reasons. Claims of
facial invalidity often rest on speculation. As a consequence, they
raise the risk of ``premature interpretation of statutes on the
basis of factually barebones records.'' \emph{Sabri} v. \emph{United
States,} 541 U.~S. 600, 609 (2004) (internal quotation marks and
brackets omitted). Facial challenges also run contrary to the
fundamental principle of judicial restraint that courts should neither
`` ‘anticipate a question of constitutional law in advance of
the necessity of deciding it' '' nor `` ‘formulate a rule of
constitutional law broader than is required by the precise facts to
which it is to be applied.' '' \emph{Ashwander} v. \emph{TVA,} 297 U.~S.
288, 346--347 (1936) (Brandeis, J., concurring) (quoting \emph{Liverpool,
New York \& Philadelphia} \newpage  \emph{S. S. Co.} v. \emph{Commissioners of
Emigration,} 113 U.~S. 33, 39 (1885)). Finally, facial challenges
threaten to short circuit the democratic process by preventing laws
embodying the will of the people from being implemented in a manner
consistent with the Constitution. We must keep in mind that `` ‘[a]
ruling of unconstitutionality frustrates the intent of the elected
representatives of the people.' '' \emph{Ayotte} v. \emph{Planned
Parenthood of Northern New Eng.,} 546 U. S. 320, 329 (2006) (quoting
\emph{Regan} v. \emph{Time, Inc.,} 468 U. S. 641, 652 (1984) (plurality
opinion)). It is with these principles in view that we turn to the
merits of respondents' facial challenge to I--872.

\section{A}

  The States possess a `` ‘broad power to prescribe the
``Times, Places and Manner of holding Elections for Senators and
Representatives,'' Art. I, \S~4, cl. 1, which power is matched
by state control over the election process for state offices.' ''
\emph{Clingman} v. \emph{Beaver,} 544 U.~S. 581, 586 (2005) (quoting
\emph{Tashjian,} 479 U. S., at 217); \emph{Timmons} v. \emph{Twin Cities Area
New Party,} 520 U.~S. 351, 358 (1997) (same). This power is not
absolute, but is ``subject to the limitation that [it] may not be
exercised in a way that violates\dots specific provisions of the
Constitution.'' \emph{Williams} v. \emph{Rhodes,} 393 U.~S. 23, 29
(1968). In particular, the State has the `` ‘responsibility
to observe the limits established by the First Amendment rights of
the State's citizens,' '' including the freedom of political
association. \emph{Eu} v. \emph{San Francisco County Democratic Central
Comm.,} 489 U.~S. 214, 222 (1989) (quoting \emph{Tashjian, supra,} at
217).

  Election regulations that impose a severe burden on associational
rights are subject to strict scrutiny, and we uphold them only if they
are ``narrowly tailored to serve a compelling state interest.''
\emph{Clingman, supra,} at 586; see also \emph{Rhodes, supra,} at 31
(`` ‘[O]nly a compelling state interest in the regulation of a
subject within the State's constitutional power to regulate can
justify limiting First Amendment \newpage  freedoms' '' (quoting
\emph{NAACP} v. \emph{Button,} 371 U.~S. 415, 438 (1963))). If a statute
imposes only modest burdens, however, then ``the State's important
regulatory interests are generally sufficient to justify reasonable,
nondiscriminatory restrictions'' on election procedures. \emph{Anderson}
v. \emph{Celebrezze,} 460 U.~S. 780, 788 (1983). ``Accordingly, we
have repeatedly upheld reasonable, politically neutral regulations that
have the effect of channeling expressive activity at the polls.''
\emph{Burdick} v. \emph{Takushi,} 504 U.~S. 428, 438 (1992).

  The parties do not dispute these general principles; rather, they
disagree about whether I--872 severely burdens respondents'
associational rights. That disagreement begins with \emph{Jones.}
Petitioners argue that the I--872 primary is indistinguishable from the
alternative \emph{Jones} suggested would be constitutional. In \emph{Jones} we
noted that a nonpartisan blanket primary, where the top two votegetters
proceed to the general election regardless of their party, was a less
restrictive alternative to California's system because such a primary
does not nominate candidates. 530 U. S., at 585--586 (The nonpartisan
blanket primary ``has all the characteristics of the partisan blanket
primary, save the constitutionally crucial one: Primary voters are not
choosing a party's nominee''). Petitioners are correct that we
assumed that the nonpartisan primary we described in \emph{Jones} would
be constitutional. But that is not dispositive here because we had
no occasion in \emph{Jones} to determine whether a primary system that
indicates each candidate's party preference on the ballot, in effect,
chooses the parties' nominees.

  That question is now squarely before us. Respondents argue that
I--872 is unconstitutional under \emph{Jones} because it has the same
``constitutionally crucial'' infirmity that doomed California's
blanket primary: It allows primary voters who are unaffiliated with a
party to choose the party's nominee. Respondents claim that candidates
who progress to the general election under I--872 will become the
\emph{de facto} nominees \newpage  of the parties they prefer, thereby
violating the parties' right to choose their own standard bearers,
see \emph{Timmons, supra,} at 359, and altering their messages. They
rely on our statement in \emph{Jones} reaffirming ``the special place the
First Amendment reserves for, and the special protection it accords,
the process by which a political party ‘select[s] a standard bearer
who best represents the party's ideologies and preferences.' ''
\emph{Jones,} 530 U. S., at 575 (quoting \emph{Eu, supra,} at 224).

  The flaw in this argument is that, unlike the California primary, the
I--872 primary does not, by its terms, choose parties' nominees. The
essence of nomination---the choice of a party representative---does
not occur under I--872. The law never refers to the candidates as
nominees of any party, nor does it treat them as such. To the contrary,
the election regulations specifically provide that the primary ``does
not serve to determine the nominees of a political party but serves to
winnow the number of candidates to a final list of two for the general
election.'' App. 606, Wash. Admin. Code \S~434--262--012. The
top two candidates from the primary election proceed to the general
election regardless of their party preferences. Whether parties nominate
their own candidates outside the state-run primary is simply irrelevant.
In fact, parties may now nominate candidates by whatever mechanism
they choose because I--872 repealed Washington's prior regulations
governing party nominations.\footnotemark[7]

^7 It is true that parties may no longer indicate their nominees on the
ballot, but that is unexceptionable: The First Amendment does not give
political parties a right to have their nominees designated as such
on the ballot. See \emph{Timmons} v. \emph{Twin Cities Area New Party,}
520 U.~S. 351, 362--363 (1997) (``We are unpersuaded, however, by
the party's contention that it has a right to use the ballot itself
to send a particularized message, to its candidate and to the voters,
about the nature of its support for the candidate''). Parties do
not gain such a right simply because the State affords candidates the
opportunity to indicate their party preference on \newpage  the ballot.
``Ballots serve primarily to elect candidates, not as forums for
political expression.'' \emph{Id.,} at 363.\newpage 

  Respondents counter that, even if the I--872 primary does not
actually choose parties' nominees, it nevertheless burdens their
associational rights because voters will assume that candidates on the
general election ballot are the nominees of their preferred parties.
This brings us to the heart of respondents' case---and to the fatal
flaw in their argument. At bottom, respondents' objection to I--872
is that voters will be confused by candidates' party-preference
designations. Respondents' arguments are largely variations on
this theme. Thus, they argue that even if voters do not assume that
candidates on the general election ballot are the nominees of their
parties, they will at least assume that the parties associate with,
and approve of, them. This, they say, compels them to associate with
candidates they do not endorse, alters the messages they wish to convey,
and forces them to engage in counterspeech to disassociate themselves
from the candidates and their positions on the issues.

  We reject each of these contentions for the same reason: They
all depend, not on any facial requirement of I--872, but on the
possibility that voters will be confused as to the meaning of the
party-preference designation. But respondents' assertion that voters
will misinterpret the party-preference designation is sheer speculation.
It ``depends upon the belief that voters can be ‘misled' by
party labels. But ‘[o]ur cases reflect a greater faith in the
ability of individual voters to inform themselves about campaign
issues.' '' \emph{Tashjian,} 479 U. S., at 220 (quoting \emph{Anderson,
supra,} at 797). There is simply no basis to presume that a
well-informed electorate will interpret a candidate's party-preference
designation to mean that the candidate is the party's chosen nominee
or representative or that the party associates with or approves of the
candidate. See \emph{New York State Clubn Assn., Inc.} v. \emph{City of
New York,} 487 U.~S. 1, 13--14 (1988) \newpage  (rejecting a facial
challenge to a law regulating club membership and noting that ``[w]e
could hardly hold otherwise on the record before us, which contains no
specific evidence on the characteristics of \emph{any} club covered by
the [l]aw''). This strikes us as especially true here, given that
it was the voters of Washington themselves, rather than their elected
representatives, who enacted I--872.

  Of course, it is \emph{possible} that voters will misinterpret the
candidates' party-preference designations as reflecting endorsement by
the parties. But these cases involve a facial challenge, and we cannot
strike down I--872 on its face based on the mere possibility of voter
confusion. See \emph{Yazoo,} 226 U. S., at 219 (``[T]his court must deal
with the case in hand and not with imaginary ones''); \emph{Pullman Co.}
v. \emph{Knott,} 235 U.~S. 23, 26 (1914) (A statute ``is not to be upset
upon hypothetical and unreal possibilities, if it would be good upon the
facts as they are''). Because respondents brought their suit as a
facial challenge, we have no evidentiary record against which to assess
their assertions that voters will be confused. See \emph{Timmons,} 520
U. S., at 375--376 (\textsc{Stevens,} J., dissenting) (rejecting judgments
based on ``imaginative theoretical sources of voter confusion'' and
``entirely hypothetical'' outcomes). Indeed, because I--872 has
never been implemented, we do not even have ballots indicating how party
preference will be displayed. It stands to reason that whether voters
will be confused by the party-preference designations will depend in
significant part on the form of the ballot. The Court of Appeals assumed
that the ballot would not place abbreviations like `` ‘D' '' and
`` ‘R,' '' or `` ‘Dem.' '' and `` ‘Rep.' '' after
the names of candidates, but would instead ``clearly state that a
particular candidate ‘prefers' a particular party.'' 460 F. 3d,
at 1121, n. 20. It thought that even such a clear statement did too
little to eliminate the risk of voter confusion.

  But we see no reason to stop there. As long as we are speculating
about the form of the ballot---and we can do no \newpage  more than
speculate in this facial challenge---we must, in fairness to the voters
of the State of Washington who enacted I--872 and in deference to the
executive and judicial officials who are charged with implementing it,
ask whether the ballot could conceivably be printed in such a way as to
eliminate the possibility of widespread voter confusion and with it the
perceived threat to the First Amendment. See \emph{Ayotte,} 546 U. S.,
at 329 (noting that courts should not nullify more of a state law than
necessary so as to avoid frustrating the intent of the people and their
duly elected representatives); \emph{Ward} v. \emph{Rock Against Racism,} 491
U. S. 781, 795--796 (1989) (`` ‘[I]n evaluating a facial challenge
to a state law, a federal court must\dots consider any limiting
construction that a state court or enforcement agency has proffered'
'' (quoting \emph{Hoffman Estates} v. \emph{Flipside, Hoffman Estates, Inc.,}
455 U.~S. 489, 494, n. 5 (1982))).

  It is not difficult to conceive of such a ballot. For example,
petitioners propose that the actual I--872 ballot could include
prominent disclaimers explaining that party preference reflects only the
self-designation of the candidate and not an official endorsement by the
party. They also suggest that the ballots might note preference in the
form of a candidate statement that emphasizes the candidate's personal
determination rather than the party's acceptance of the candidate,
such as ``my party preference is the Republican Party.'' Additionally,
the State could decide to educate the public about the new primary
ballots through advertising or explanatory materials mailed to voters
along with their ballots.\footnotemark[8] We are satisfied that there are a variety
of ways in which the State could implement I--872 that would eliminate
any real threat of voter confusion. And without the spec\newpage ter
of widespread voter confusion, respondents' arguments about forced
association\footnotemark[9] and compelled speech\footnotemark[10] fall flat.

^8 Washington counties have broad authority to conduct elections
entirely by mail ballot rather than at in-person polling places. See
Wash. Rev. Code \S~29A.48.010. As a result, over 90% of Washington
voters now vote by mail. See Tr. of Oral Arg. 11.


  Our conclusion that these implementations of I--872 would be
consistent with the First Amendment is fatal to respondents'
facial challenge. See \emph{Schall} v. \emph{Martin,} 467 U.~S. 253,
264 (1984) (a facial challenge fails where ``at least some''
constitutional applications exist). Each of their arguments rests on
factual assumptions about voter confusion, and each fails for the same
reason: In the absence of evidence, we cannot assume that Washington's
voters will be misled. See \emph{Jones,} 530U.S., at600(\textsc{Stevens,}
J., dissenting) (``[A]n empirically debatable assumption\dots is
too thin a reed to support a credible First Amendment distinction''
between permissible and impermissible burdens on association). That
\newpage  factual determination must await an as-applied challenge. On
its face, I--872 does not impose any severe burden on respondents'
associational rights.

^9 Respondents rely on \emph{Hurley} v. \emph{Irish-American Gay, Lesbian and
Bisexual Group of Boston, Inc.,} 515 U.~S. 557 (1995) (holding that
a State may not require a parade to include a group if the parade's
organizer disagrees with the group's message), and \emph{Boy Scouts of
America} v. \emph{Dale,} 530 U.~S. 640 (2000) (holding that the Boy
Scouts' freedom of expressive association was violated by a state
law requiring the organization to admit a homosexual scoutmaster).
In those cases, \emph{actual} association threatened to distort the
groups' intended messages. We are aware of no case in which the mere
\emph{impression} of association was held to place a severe burden on a
group's First Amendment rights, but we need not decide that question
here.

^10 Relying on \emph{Pacific Gas \& Elec. Co.} v. \emph{Public Util. Comm'n
of Cal.,} 475 U.~S. 1 (1986) (holding that a state agency may not
require a utility company to include a third-party newsletter in
its billing envelope), respondents argue that the threat of voter
confusion will force them to speak to clarify their positions. Because
I--872 does not actually force the parties to speak, however, \emph{Pacific
Gas \& Elec.} is inapposite. I--872 does not require the parties to
reproduce another's speech against their will; nor does it co-opt the
parties' own conduits for speech. Rather, it simply provides a place
on the ballot for candidates to designate their party preferences.
Facilitation of speech to which a political party may choose to
respond does not amount to forcing the political party to speak. Cf.
\emph{Rumsfeld} v. \emph{Forum for Academic and Institutional Rights, Inc.\\,
547 U.~S. 47, 64--65 (2006).

\section{B}

  Because we have concluded that I--872 does not severely burden
respondents, the State need not assert a compelling interest. See
\emph{Clingman,} 544 U. S., at 593 (``When a state electoral provision
places no heavy burden on associational rights, ‘a State's important
regulatory interests will usually be enough to justify reasonable,
nondiscriminatory restrictions' '' (quoting \emph{Timmons,} 520 U. S.,
at 358)). The State's asserted interest in providing voters with
relevant information about the candidates on the ballot is easily
sufficient to sustain I--872. See \emph{Anderson,} 460 U. S., at 796
(``There can be no question about the legitimacy of the State's
interest in fostering informed and educated expressions of the popular
will in a general election'').\footnotemark[11]

\section{III}

  Respondents ask this Court to invalidate a popularly enacted election
process that has never been carried out. Immediately after implementing
regulations were enacted, respondents obtained a permanent injunction
against the enforcement of I--872. The First Amendment does not require
this extraordinary and precipitous nullification of the will of the
people. Because I--872 does not on its face provide for \newpage  the
nomination of candidates or compel political parties to associate with
or endorse candidates, and because there is no basis in this facial
challenge for presuming that candidates' party-preference designations
will confuse voters, I--872 does not on its face severely burden
respondents' associational rights. We accordingly hold that I--872
is facially constitutional. The judgment of the Court of Appeals is
reversed.

^11 Respondent Libertarian Party of Washington argues that I--872
is unconstitutional because of its implications for ballot access,
trademark protection of party names, and campaign finance. We do not
consider the ballot access and trademark arguments as they were not
addressed below and are not encompassed by the question on which we
granted certiorari: ``Does Washington's primary election system
.~.~. violate the associational rights of political parties because
candidates are permitted to identify their political party preference on
the ballot?'' Pet. for Cert. in No. 06--730, p. i. The campaign
finance issue also was not addressed below and is more suitable for
consideration on remand.

\begin{flushright}\emph{It is so ordered.}\end{flushright}
