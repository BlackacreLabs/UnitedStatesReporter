% Opinion of the Court
% Thomas

\setcounter{page}{380}

  \textsc{Justice Thomas} delivered the opinion of the Court.

  In this age discrimination case, the District Court excluded testimony
by nonparties alleging discrimination at the hands of supervisors of the
defendant company who played no role in the adverse employment decision
challenged by the plaintiff. The Court of Appeals, having concluded
that the District Court improperly applied a \emph{per se} rule excluding
the evidence, engaged in its own analysis of the relevant factors
under Federal Rules of Evidence 401 and \newpage  403, and remanded with
instructions to admit the challenged testimony. We granted certiorari on
the question whether the Federal Rules of Evidence required admission
of the testimony. We conclude that such evidence is neither \emph{per se}
admissible nor \emph{per se} inadmissible. Because it is not entirely clear
whether the District Court applied a \emph{per se} rule, we vacate the
judgment of the Court of Appeals and remand for the District Court to
conduct the relevant inquiry under the appropriate standard.

\section{I}

  Respondent Ellen Mendelsohn was employed in the Business Development
Strategy Group of petitioner Sprint/ United Management Company (Sprint)
from 1989 until 2002, when Sprint terminated her as a part of an
ongoing companywide reduction in force. She sued Sprint under the
Age Discrimination in Employment Act of 1967 (ADEA), 81 Stat. 602,
as amended, 29 U.~S.~C. \S~621 \emph{et seq.,} alleging disparate
treatment based on her age.

  In support of her claim, Mendelsohn sought to introduce testimony by
five other former Sprint employees who claimed that their supervisors
had discriminated against them because of age. Three of the witnesses
alleged that they heard one or more Sprint supervisors or managers make
remarks denigrating older workers. One claimed that Sprint's intern
program was a mechanism for age discrimination and that she had seen
a spreadsheet suggesting that a supervisor considered age in making
layoff decisions. Another witness was to testify that he had been given
an unwarranted negative evaluation and ``banned'' from working at
Sprint because of his age, and that he had witnessed another employee
being harassed because of her age. App. 17a. The final witness
alleged that Sprint had required him to get permission before hiring
anyone over age 40, that after his termination he had been replaced by a
younger employee, and that Sprint had rejected his subsequent employment
applications. \newpage 

  None of the five witnesses worked in the Business Development Strategy
Group with Mendelsohn, nor had any of them worked under the supervisors
in her chain of command, which included James Fee, Mendelsohn's direct
supervisor; Paul Reddick, Fee's direct manager and the decisionmaker
in Mendelsohn's termination; and Bill Blessing, Reddick's supervisor
and head of the Business Development Strategy Group. Neither did any of
the proffered witnesses report hearing discriminatory remarks by Fee,
Reddick, or Blessing.

  Sprint moved \emph{in limine} to exclude the testimony, arguing that
it was irrelevant to the central issue in the case: whether Reddick
terminated Mendelsohn because of her age. See Fed. Rules Evid. 401, 402.
Sprint claimed that the testimony would be relevant only if it came from
employees who were ``similarly situated'' to Mendelsohn in that they
had the same supervisors. App. 156a. Sprint also argued that, under
Rule 403, the probative value of the evidence would be substantially
outweighed by the danger of unfair prejudice, confusion of the issues,
misleading of the jury, and undue delay.

  In a minute order, the District Court granted the motion, excluding,
in relevant part, evidence of ``discrimination against employees
not similarly situated to plaintiff.'' App. to Pet. for Cert.
24a. In clarifying that Mendelsohn could only ``offer evidence of
discrimination against Sprint employees who are similarly situated to
her,'' the court defined `` ‘[s]imilarly situated employees,' for
the purpose of this ruling, [as] requir[ing] proof that (1) Paul Ruddick
\emph{[sic]} was the decision-maker in any adverse employment action; and
(2) temporal proximity.'' \emph{Ibid.} Beyond that, the District
Court provided no explanation of the basis for its ruling. As the trial
proceeded, the judge orally clarified that the minute order was meant to
exclude only testimony ``that Sprint treated other people unfairly on
the basis of age,'' and would not bar testimony going to the ``totally
different'' question \newpage  ``whether the [reduction in force], which
is [Sprint's] stated nondiscriminatory reason, is a pretext for age
discrimination.'' App. 295a--296a.

  The Court of Appeals for the Tenth Circuit treated the minute order
as the application of a \emph{per se} rule that evidence from employees
with other supervisors is irrelevant to proving discrimination in an
ADEA case. Specifically, it concluded that the District Court abused
its discretion by relying on \emph{Aramburu} v. \emph{Boeing Co.,} 112 F.
3d 1398 (CA10 1997). 466 F. 3d 1223, 1227--1228 (CA10 2006).
\emph{Aramburu} held that ``[s]imilarly situated employees,'' for the
purpose of showing disparate treatment in employee discipline, ``are
those who deal with the same supervisor and are subject to the same
standards governing performance evaluation and discipline.'' 112 F.
3d, at 1404 (internal quotation marks omitted). The Court of Appeals
viewed that case as inapposite because it addressed discriminatory
discipline, not a companywide policy of discrimination. The Court of
Appeals then determined that the evidence was relevant and not unduly
prejudicial, and reversed and remanded for a new trial. We granted
certiorari, 551 U.~S. 1113 (2007), to determine whether, in an
employment discrimination action, the Federal Rules of Evidence require
admission of testimony by nonparties alleging discrimination at the
hands of persons who played no role in the adverse employment decision
challenged by the plaintiff.

\section{II}

  The parties focus their dispute on whether the Court of Appeals
correctly held that the evidence was relevant and not unduly prejudicial
under Rules 401 and 403. We conclude, however, that the Court of Appeals
should not have engaged in that inquiry. Rather, as explained below, we
hold that the Court of Appeals erred in concluding that the District
Court applied a \emph{per se} rule. Given the circumstances of this case
and the unclear basis of the District \newpage  Court's decision, the
Court of Appeals should have remanded the case to the District Court for
clarification.

\subsection{A}

  In deference to a district court's familiarity with the details of
the case and its greater experience in evidentiary matters, courts of
appeals afford broad discretion to a district court's evidentiary
rulings. This Court has acknowledged:

      ``A district court is accorded a wide discretion in de termining
    the admissibility of evidence under the Federal Rules. Assessing
    the probative value of [the proffered evidence], and weighing
    any factors counseling against admissibility is a matter first
    for the district court's sound judgment under Rules 401 and
    403~.~.~.~.'' \emph{United States} v. \emph{Abel,} 469 U.~S. 45,
    54 (1984).

\noindent This is particularly true with respect to Rule 403 since it requires an
``on-the-spot balancing of probative value and prejudice, potentially
to exclude as unduly prejudicial some evidence that already has been
found to be factually relevant.'' 1 S. Childress \& M. Davis, Federal
Standards of Review \S~4.02, p. 4--16 (3d ed. 1999). Under this
deferential standard, courts of appeals uphold Rule 403 rulings unless
the district court has abused its discretion. See \emph{Old Chief} v.
\emph{United States,} 519 U.~S. 172, 183, n. 7 (1997).

  Here, however, the Court of Appeals did not accord the District
Court the deference we have described as the ``hallmark of
abuse-of-discretion review.'' \emph{General Elec. Co.} v. \emph{Joiner,}
522 U.~S. 136, 143 (1997). Instead, it reasoned that the District
Court had ``erroneous[ly] conclu[ded] that \emph{Aramburu} controlled the
fate of the evidence in this case.'' 466 F. 3d, at 1230, n. 4.

  To be sure, Sprint in its motion \emph{in limine} argued, with a
citation to \emph{Aramburu}'s categorical bar, that ``[e]mployees may
be similarly situated only if they had the same supervi\newpage sor,''
App. 163a, and the District Court's minute order mirrors that
blanket language.

  But the District Court's discussion of the evidence neither cited
\emph{Aramburu} nor gave any other indication that its decision relied on
that case\\.} The minute order included only two sentences discussing
the admissibility of the evidence:

    ``Plaintiff may offer evidence of discrimination against Sprint
    employees who are similarly situated to her. ‘Similarly situated
    employees,' for the purpose of this ruling, requires proof that
    (1) Paul Ruddick \emph{[sic]} was the decision-maker in any adverse
    employment action; and (2) temporal proximity.'' App. to Pet. for
    Cert. 24a.

\noindent Contrary to the Court of Appeals' conclusion, these sentences include
no analysis suggesting that the District Court applied a \emph{per se} rule
excluding this type of evidence.

  Mendelsohn argued on appeal\footnotemark[1] that the District Court must
have viewed \emph{Aramburu} as controlling because Sprint cited the
case in support of its \emph{in limine} motion. But neither party's
submissions to the District Court suggested that \emph{Aramburu} was
controlling. Sprint's memorandum in support of its motion mentioned
the case only in a string citation, and not for the proposition
that only ``similarly situated'' witnesses' testimony would be
admissible.\footnotemark[2] App. 163a. Mendelsohn did not cite the case in her
memorandum in opposition, see \emph{id.,} at 208a, and Sprint did not
address it in its reply brief, see \emph{id.,} at 221a. \newpage 

^1 Although, as noted above, the parties do not address in their filings
before this Court the grounds on which we base our decision, we shall
consider the relevant arguments they made before the Court of Appeals.

^2 Even if Sprint had argued that \emph{Aramburu} requires a \emph{per
se} rule excluding such evidence, it would be inappropriate for
the reviewing court to assume, absent indication in the District
Court's opinion, that the lower court adopted a party's incorrect
argument. Cf. \emph{Lawrence} v. \emph{Chater,} 516 U.~S. 163, 183 (1996)
(\textsc{Scalia,} J., dissenting) (``[W]e should not assume that a court of
appeals has adopted a legal position only because [a party] supported
it'').

  Mendelsohn further argued that the District Court's use of the
phrase ``similarly situated,'' also used in \emph{Aramburu,} evidenced
its reliance on that case. Although the District Court used the same
phrase, we decline to read the District Court's decision as relying
on a case that was not controlling. \emph{Aramburu} defined the phrase
``similarly situated'' in the entirely different context of a
plaintiff's allegation that nonminority employees were treated more
favorably than minority employees. 112 F. 3d, at 1403--1406. Absent
reason to do so, we should not assume the District Court adopted that
``similarly situated'' analysis when it addressed a very different
kind of evidence. An appellate court should not presume that a district
court intended an incorrect legal result when the order is equally
susceptible of a correct reading, particularly when the applicable
standard of review is deferential.

  Mendelsohn additionally argued that the District Court must have meant
to apply such a rule because that was the nature of the argument in
Sprint's \emph{in limine} motion. But the \emph{in limine} motion did not
suggest that the evidence is never admissible; it simply argued that
such evidence lacked sufficient probative value ``in this case'' to be
relevant or outweigh prejudice and delay. App. 156a.

  When a district court's language is ambiguous, as it was here, it
is improper for the court of appeals to presume that the lower court
reached an incorrect legal conclusion. A remand directing the district
court to clarify its order is generally permissible and would have been
the better approach in this case.

\subsection{B}

  In the Court of Appeals' view, the District Court excluded the
evidence as \emph{per se} irrelevant, and so had no occasion to reach the
question whether such evidence, if relevant, should be excluded under
Rule 403. The Court of Appeals, upon concluding that such evidence
was not \emph{per se} irrelevant, de\newpage cided that it was relevant
in the circumstances of this case and undertook its own balancing
under Rule 403. But questions of relevance and prejudice are for the
District Court to determine in the first instance. \emph{Abel,} 469
U. S., at 54 (``Assessing the probative value of [evidence], and
weighing any factors counseling against admissibility is a matter
first for the district court's sound judgment under Rules 401 and
403~.~.~.''). Rather than assess the relevance of the evidence
itself and conduct its own balancing of its probative value and
potential prejudicial effect, the Court of Appeals should have allowed
the District Court to make these determinations in the first instance,
explicitly and on the record.\footnotemark[3] See \emph{Pullman-Standard} v.
\emph{Swint,} 456 U.~S. 273, 291 (1982) (When a district court ``fail[s]
to make a finding because of an erroneous view of the law, the usual
rule is that there should be a remand for further proceedings to permit
the trial court to make the missing findings''). With respect
to evidentiary questions in general and Rule 403 in particular, a
district court virtually always is in the better position to assess the
admissibility of the evidence in the context of the particular case
before it.

  We note that, had the District Court applied a \emph{per se} rule
excluding the evidence, the Court of Appeals would have been correct
to conclude that it had abused its discretion. Relevance and prejudice
under Rules 401 and 403 are determined in the context of the facts and
arguments in a particular case, and thus are generally not amenable to
broad \emph{per se} rules. See Advisory Committee's Notes on Fed. Rule
Evid. 401, 28 U.~S.~C. App., p. 864 (``Relevancy is not an inherent
characteristic of any item of evidence but exists only as a relation
between an item of evidence and a matter properly \newpage  provable in
the case''). But, as we have discussed, there is no basis in the
record for concluding that the District Court applied a blanket rule.

^3 The only exception to this rule is when ``the record permits only
one resolution of the factual issue.'' \emph{Pullman-Standard} v.
\emph{Swint,} 456 U.~S. 273, 292 (1982). The evidence here, however, is
not of that dispositive character.

\section{III}

  The question whether evidence of discrimination by other supervisors
is relevant in an individual ADEA case is fact based and depends on
many factors, including how closely related the evidence is to the
plaintiff's circumstances and theory of the case. Applying Rule 403
to determine if evidence is prejudicial also requires a fact-intensive,
contextspecific inquiry. Because Rules 401 and 403 do not make such
evidence \emph{per se} admissible or \emph{per se} inadmissible, and because
the inquiry required by those Rules is within the province of the
District Court in the first instance, we vacate the judgment of the
Court of Appeals and remand the case with instructions to have the
District Court clarify the basis for its evidentiary ruling under the
applicable Rules.

\begin{flushright}\emph{It is so ordered.}\end{flushright}
