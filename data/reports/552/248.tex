% Opinion of the Court
% Stevens

\setcounter{page}{250}

  \textsc{Justice Stevens} delivered the opinion of the Court.

  In \emph{Massachusetts Mut. Life Ins. Co.} v. \emph{Russell,} 473 U. S. 134 (1985), we held that a participant in a disability plan that paid a fixed level of benefits could not bring suit under \S~502(a)(2) of the Employee Retirement Income Security Act of 1974 (ERISA), 88 Stat. 891, 29 U.~S.~C. \S~1132(a)(2), to recover consequential damages arising from delay in the processing of her claim. In this case we consider whether that statutory provision authorizes a participant in a defined contribution pension plan to sue a fiduciary whose alleged misconduct impaired the value of plan assets in the participant's individual account.\footnotemark[1]

      Relying on our decision in \emph{Russell,} the Court of Appeals for the Fourth Circuit held that \S~502(a)(2) ``provides remedies only for entire plans, not for individuals~.~.~.~. Recovery under this subsection must ‘inure[ ] to the benefit of the plan \emph{as a whole,\\' not to particular persons with rights under the plan.'' 450 F. 3d 570, 572--573 (2006) (quoting \emph{Russell,} 473 U. S., at 140). While language in our \emph{Russell} opinion is consistent with that conclusion, the rationale for \emph{Russell}'s holding supports the opposite result in this case.

\section{I}

  Petitioner filed this action in 2004 against his former employer, DeWolff, Boberg \& Associates, Inc. (DeWolff), and the ERISA-regulated 401(k) retirement savings plan administered by DeWolff (Plan). The Plan permits participants to direct the investment of their contributions in accordance \newpage  with specified procedures and requirements. Petitioner alleged that in 2001 and 2002 he directed DeWolff to make certain changes to the investments in his individual account, but DeWolff never carried out these directions. Petitioner claimed that this omission ``depleted'' his interest in the Plan by approximately \$150,000, and amounted to a breach of fiduciary duty under ERISA. The complaint sought `` ‘makewhole' or other equitable relief as allowed by [\S~502(a)(3)],'' as well as ``such other and further relief as the court deems just and proper.'' Civil Action No. 2:04--1747--18 (D. S. C.), p. 4, 2 Record, Doc. 1.

\footnotetext[1]{As its names imply, a ``defined contribution plan'' or ``individual account plan'' promises the participant the value of an individual account at retirement, which is largely a function of the amounts contributed to that account and the investment performance of those contributions. A ``defined benefit plan,'' by contrast, generally promises the participant a fixed level of retirement income, which is typically based on the employee's years of service and compensation. See \S\S~3(34)--(35), 88 Stat. 838, 29 U.~S.~C. \S\S~1002(34)--(35); P. Schneider \& B. Freedman, ERISA: A Comprehensive Guide \S~3.02 (2d ed. 2003).}

  Respondents filed a motion for judgment on the pleadings, arguing that the complaint was essentially a claim for monetary relief that is not recoverable under \S~502(a)(3). Petitioner countered that he ``d[id] not wish for the court to award him any money, but\dots simply want[ed] the plan to properly reflect that which would be his interest in the plan, but for the breach of fiduciary duty.'' Reply to Defendants Motion to Dismiss, p. 7, 3 \emph{id.,} Doc. 17. The District Court concluded, however, that since respondents did not possess any disputed funds that rightly belonged to petitioner, he was seeking damages rather than equitable relief available under \S~502(a)(3). Assuming, \emph{arguendo,} that respondents had breached a fiduciary duty, the District Court nonetheless granted their motion.

  On appeal petitioner argued that he had a cognizable claim for relief under \S\S~502(a)(2) and 502(a)(3) of ERISA. The Court of Appeals stated that petitioner had raised his \S~502(a)(2) argument for the first time on appeal, but nevertheless rejected it on the merits.

  Section 502(a)(2) provides for suits to enforce the liabilitycreating provisions of \S~409, concerning breaches of fiduciary duties that harm plans.\footnotemark[2] The Court of Appeals cited lan \newpage  guage from our opinion in \emph{Russell} suggesting that these provisions ``protect the entire plan, rather than the rights of an individual beneficiary.'' 473 U. S., at 142. It then characterized the remedy sought by petitioner as ``personal'' because he ``desires recovery to be paid into his plan account, an instrument that exists specifically for his benefit,'' and concluded:

\footnotetext[2]{Section 409(a) provides:

``Any person who is a fiduciary with respect to a plan who breaches any of the responsibilities, obligations, or duties imposed upon fiduciaries by \newpage  this title shall be personally liable to make good to such plan any losses to the plan resulting from each such breach, and to restore to such plan any profits of such fiduciary which have been made through use of assets of the plan by the fiduciary, and shall be subject to such other equitable or remedial relief as the court may deem appropriate, including removal of such fiduciary. A fiduciary may also be removed for a violation of section 411 of this Act.'' 88 Stat. 886, 29 U.~S.~C. \S~1109(a).}

      \begin{quote}

		  ``We are therefore skeptical that plaintiff's individual remedial interest can serve as a legitimate proxy for the plan in its entirety, as [\S~502(a)(2)] requires. To be sure, the recovery plaintiff seeks could be seen as accruing to the plan in the narrow sense that it would be paid into plaintiff's personal plan \emph{account,} which is part of the plan. But such a view finds no license in the statutory text, and threatens to undermine the careful limitations Congress has placed on the scope of ERISA relief.'' 450 F. 3d, at 574.

      \end{quote}

  The Court of Appeals also rejected petitioner's argument that the make-whole relief he sought was ``equitable'' within the meaning of \S~502(a)(3). Although our grant of certiorari, 551 U.~S. 1130 (2007), encompassed the \S~502(a)(3) issue, we do not address it because we conclude that the Court of Appeals misread \S~502(a)(2).

\section{II}

  As the case comes to us we must assume that respondents breached fiduciary obligations defined in \S~409(a), and that \newpage  those breaches had an adverse impact on the value of the Plan assets in petitioner's individual account. Whether petitioner can prove those allegations and whether respondents may have valid defenses to the claim are matters not before us.\footnotemark[3] Although the record does not reveal the relative size of petitioner's account, the legal issue under \S~502(a)(2) is the same whether his account includes 1% or 99% of the total assets in the Plan.

  As we explained in \emph{Russell,} and in more detail in our later opinion in \emph{Varity Corp.} v. \emph{Howe,} 516 U.~S. 489, 508--512 (1996), \S~502(a) of ERISA identifies six types of civil actions that may be brought by various parties. The second, which is at issue in this case, authorizes the Secretary of Labor as well as plan participants, beneficiaries, and fiduciaries, to bring actions on behalf of a plan to recover for violations of the obligations defined in \S~409(a). The principal statutory duties imposed on fiduciaries by that section ``relate to the proper management, administration, and investment of fund assets,'' with an eye toward ensuring that ``the benefits authorized by the plan'' are ultimately paid to participants and beneficiaries. \emph{Russell,} 473 U. S., at 142; see also \emph{Varity,} 516 U. S., at 511--512 (noting that \S~409's fiduciary obligations ``relat[e] to the plan's financial integrity'' and ``reflec[t] a special congressional concern about plan asset management''). The misconduct alleged by petitioner in this case falls squarely within that category.\footnotemark[4]\newpage

\footnotetext[3]{For example, we do not decide whether petitioner made the alleged investment directions in accordance with the requirements specified by the Plan, whether he was required to exhaust remedies set forth in the Plan before seeking relief in federal court pursuant to \S~502(a)(2), or whether he asserted his rights in a timely fashion.}

\footnotetext[4]{The record does not reveal whether the alleged \$150,000 injury represents a decline in the value of assets that DeWolff should have sold or an increase in the value of assets that DeWolff should have purchased. Contrary to respondents' argument, however, \S~502(a)(2) encompasses ap\newpage propriate claims for ``lost profits.'' See Brief for Respondents 12--13. Under the common law of trusts, which informs our interpretation of ERISA's fiduciary duties, see \emph{Varity,} 516 U. S., at 496--497, trustees are ``chargeable with .~.~. any profit which would have accrued to the trust estate if there had been no breach of trust,'' including profits forgone because the trustee ``fails to purchase specific property which it is his duty to purchase.'' 1 Restatement (Second) of Trusts \S205, and Comment \emph{i} (1957); \S~211; see also 3 A. Scott, Law on Trusts \S\S~205, 211 (3d ed. 1967).}

  The misconduct alleged in \emph{Russell,} by contrast, fell outside this category. The plaintiff in \emph{Russell} received all of the benefits to which she was contractually entitled, but sought consequential damages arising from a delay in the processing of her claim. 473 U. S., at 136--137. In holding that \S~502(a)(2) does not provide a remedy for this type of injury, we stressed that the text of \S~409(a) characterizes the relevant fiduciary relationship as one ``with respect to a plan,'' and repeatedly identifies the ``plan'' as the victim of any fiduciary breach and the recipient of any relief. See \emph{id.,} at 140. The legislative history likewise revealed that ``the crucible of congressional concern was misuse and mismanagement of plan assets by plan administrators.'' \emph{Id.,} at 141, n. 8. Finally, our review of ERISA as a whole confirmed that \S\S502(a)(2) and 409 protect ``the financial integrity of the plan,'' \emph{id.,} at 142, n. 9, whereas other provisions specifically address claims for benefits, see \emph{id.,} at 143--144 (discussing \S\S~502(a)(1)(B) and 503). We therefore concluded:

    \begin{quote}

		``A fair contextual reading of the statute makes it abundantly clear that its draftsmen were primarily concerned with the possible misuse of plan assets, and with remedies that would protect the entire plan, rather than with the rights of an individual beneficiary.'' \emph{Id.,} at 142.

    \end{quote}

  \emph{Russell}'s emphasis on protecting the ``entire plan'' from fiduciary misconduct reflects the former landscape of employee benefit plans. That landscape has changed. \newpage  Defined contribution plans dominate the retirement plan scene today.\footnotemark[5]

      \begin{quote}

		  In contrast, when ERISA was enacted, and when \emph{Russell} was decided, ``the [defined benefit] plan was the norm of American pension practice.'' J. Langbein, S. Stabile, \& B. Wolk, Pension and Employee Benefit Law 58 (4th ed. 2006); see also Zelinsky, The Defined Contribution Paradigm, 114 Yale L. J. 451, 471 (2004) (discussing the ``significant reversal of historic patterns under which the traditional defined benefit plan was the dominant paradigm for the provision of retirement income''). Unlike the defined contribution plan in this case, the disability plan at issue in \emph{Russell} did not have individual accounts; it paid a fixed benefit based on a percentage of the employee's salary. See \emph{Russell} v. \emph{Massachusetts Mut. Life Ins. Co.,} 722 F. 2d 482, 486 (CA9 1983).

      \end{quote}

  The ``entire plan'' language in \emph{Russell} speaks to the impact of \S~409 on plans that pay defined benefits. Misconduct by the administrators of a defined benefit plan will not affect an individual's entitlement to a defined benefit unless it creates or enhances the risk of default by the entire plan. It was that default risk that prompted Congress to require defined benefit plans (but not defined contribution plans) to satisfy complex minimum funding requirements, and to make premium payments to the Pension Benefit Guaranty Corporation for plan termination insurance. See Zelinsky, 114 Yale L. J., at 475--478.

  For defined contribution plans, however, fiduciary misconduct need not threaten the solvency of the entire plan to \newpage  reduce benefits below the amount that participants would otherwise receive. Whether a fiduciary breach diminishes plan assets payable to all participants and beneficiaries, or only to persons tied to particular individual accounts, it creates the kind of harms that concerned the draftsmen of \S~409. Consequently, our references to the ``entire plan'' in \emph{Russell,} which accurately reflect the operation of \S~409 in the defined benefit context, are beside the point in the defined contribution context.

\footnotetext[5]{See, \emph{e. g.,} D. Rajnes, An Evolving Pension System: Trends in Defined Benefit and Defined Contribution Plans, Employee Benefit Research Institute (EBRI) Issue Brief No. 249 (Sept. 2002), http://www.ebri.org/pdf/ briefspdf/0902ib.pdf (all Internet materials as visited Jan. 28, 2008, and available in Clerk of Court's case file); Facts from EBRI: Retirement Trends in the United States Over the Past Quarter-Century (June 2007), http://www.ebri.org/pdf/publications/facts/0607fact.pdf.}

  Other sections of ERISA confirm that the ``entire plan'' language from \emph{Russell,} which appears nowhere in \S~409 or \S~502(a)(2), does not apply to defined contribution plans. Most significant is \S~404(c), which exempts fiduciaries from liability for losses caused by participants' exercise of control over assets in their individual accounts. See also 29 CFR \S~2550.404c--1 (2007). This provision would serve no real purpose if, as respondents argue, fiduciaries never had any liability for losses in an individual account.

  We therefore hold that although \S~502(a)(2) does not provide a remedy for individual injuries distinct from plan injuries, that provision does authorize recovery for fiduciary breaches that impair the value of plan assets in a participant's individual account. Accordingly, the judgment of the Court of Appeals is vacated, and the case is remanded for further proceedings consistent with this opinion.\footnotemark[6]

\begin{flushright}\emph{It is so ordered.}\end{flushright}

\footnotetext[6]{After our grant of certiorari respondents filed a motion to dismiss the writ, contending that the case is moot because petitioner is no longer a participant in the Plan. While his withdrawal of funds from the Plan may have relevance to the proceedings on remand, we denied their motion because the case is not moot. A plan ``participant,'' as defined by \S~3(7) of ERISA, 29 U.~S.~C. \S~1002(7), may include a former employee with a colorable claim for benefits. See, \emph{e. g., Harzewski} v. \emph{Guidant Corp.,} 489 F. 3d 799 (CA7 2007).}
