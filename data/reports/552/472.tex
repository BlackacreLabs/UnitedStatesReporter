% Court
% Alito

\setcounter{page}{474}

  \textsc{Justice Alito} delivered the opinion of the Court.

  Petitioner Allen Snyder was convicted of first-degree murder in a
Louisiana court and was sentenced to death. He asks us to review a
decision of the Louisiana Supreme Court rejecting his claim that the
prosecution exercised some of its peremptory jury challenges based
on race, in violation of \emph{Batson} v. \emph{Kentucky,} 476 U.~S. 79
(1986). We hold that the trial court committed clear error in its
ruling on a \emph{Batson} objection, and we therefore reverse.

\section{I}

  The crime for which petitioner was convicted occurred in August 1995.
At that time, petitioner and his wife, Mary, had separated. On August
15, they discussed the possibility of reconciliation, and Mary agreed
to meet with petitioner the next day. That night, Mary went on a date
with Howard Wilson. During the evening, petitioner repeatedly attempted
to page Mary, but she did not respond. At approximately 1:30 a.m. on
August 16, Wilson drove up to the home of Mary's mother to drop Mary
off. Petitioner was waiting at the scene armed with a knife. He opened
the driver's side door of Wilson's car and repeatedly stabbed the
occu\newpage  pants, killing Wilson and wounding Mary. The State charged
petitioner with first-degree murder and sought the death penalty based
on the aggravating circumstance that petitioner had knowingly created a
risk of death or great bodily harm to more than one person. See La.
Code Crim. Proc. Ann., Art. 905.4(A)(4) (West Supp. 2008).

  \emph{Voir dire} began on Tuesday, August 27, 1996, and proceeded as
follows. During the first phase, the trial court screened the panel
to identify jurors who did not meet Louisiana's requirements for
jury service or claimed that service on the jury or sequestration for
the duration of the trial would result in extreme hardship. More than
50 prospective jurors reported that they had work, family, or other
commitments that would interfere with jury service. In each of those
instances, the nature of the conflicting commitments was explored, and
some of these jurors were dismissed. App. 58--164.

  In the next phase, the court randomly selected panels of 13 potential
jurors for further questioning. \emph{Id.,} at 166--167. The defense
and prosecution addressed each panel and questioned the jurors both as a
group and individually. At the conclusion of this questioning, the court
ruled on challenges for cause. Then, the prosecution and the defense
were given the opportunity to use peremptory challenges (each side had
12) to remove remaining jurors. The court continued this process of
calling 13-person panels until the jury was filled. In accordance with
Louisiana law, the parties were permitted to exercise ``backstrikes.''
That is, they were allowed to use their peremptories up until the time
when the final jury was sworn and thus were permitted to strike jurors
whom they had initially accepted when the jurors' panels were called.
See La. Code Crim. Proc. Ann., Art. 795(B)(1) (West 1998); \emph{State}
v. \emph{Taylor,} 93--2201, pp. 22--23 (La. 2/28/96), 669 So. 2d 364,
376.

  Eighty-five prospective jurors were questioned as members of a panel.
Thirty-six of these survived challenges for \newpage  cause; 5 of the
36 were black (as is petitioner); and all 5 of the prospective black
jurors were eliminated by the prosecution through the use of peremptory
strikes. The jury found petitioner guilty of first-degree murder and
determined that he should receive the death penalty.

  On direct appeal, the Louisiana Supreme Court conditionally affirmed
petitioner's conviction. The court rejected petitioner's \emph{Batson}
claim but remanded the case for a \emph{nunc pro tunc} determination of
petitioner's competency to stand trial. \emph{State} v. \emph{Snyder,}
98--1078 (La. 4/14/99), 750 So. 2d 832. Two justices dissented and
would have found a \emph{Batson} violation. See \emph{id.,} at 866 (Johnson,
J., dissenting), 863 (Lemmon, J., concurring in part and dissenting in
part).

  On remand, the trial court found that petitioner had been competent
to stand trial, and the Louisiana Supreme Court affirmed that
determination. \emph{State} v. \emph{Snyder,} 1998--1078 (La. 4/14/04), 874
So. 2d 739. Petitioner petitioned this Court for a writ of certiorari,
and while his petition was pending, this Court decided \emph{Miller-El}
v. \emph{Dretke,} 545 U.~S. 231 (2005). We then granted the petition,
vacated the judgment, and remanded the case to the Louisiana Supreme
Court for further consideration in light of \emph{Miller-El. Snyder}
v. \emph{Louisiana,} 545 U.~S. 1137 (2005). On remand, the Louisiana
Supreme Court again rejected Snyder's \emph{Batson} claim, this time by
a vote of 4 to 3. See 1998--1078 (La. 9/6/ 06), 942 So. 2d 484. We
again granted certiorari, 551 U.~S. 1144 (2007), and now reverse.

\section{II}

  \emph{Batson} provides a three-step process for a trial court to use in
adjudicating a claim that a peremptory challenge was based on race:

    `` ‘First, a defendant must make a prima facie showing that a
    peremptory challenge has been exercised on the basis of race[;
    s]econd, if that showing has been made, \newpage  the prosecution must
    offer a race-neutral basis for striking the juror in question[;
    and t]hird, in light of the parties' submissions, the trial
    court must determine whether the defendant has shown purposeful
    discrimination.' '' \emph{Miller-El} v. \emph{Dretke, supra,} at 277
    (\textsc{Thomas,} J., dissenting) (quoting \emph{Miller-El} v. \emph{Cockrell,}
    537 U.~S. 322, 328--329 (2003)).

  On appeal, a trial court's ruling on the issue of discriminatory
intent must be sustained unless it is clearly erroneous. See
\emph{Hernandez} v. \emph{New York,} 500 U.~S. 352, 369 (1991) (plurality
opinion); \emph{id.,} at 372 (O'Connor, J., joined by \textsc{Scalia,} J.,
concurring in judgment). The trial court has a pivotal role in
evaluating \emph{Batson} claims. Step three of the \emph{Batson} inquiry
involves an evaluation of the prosecutor's credibility, see 476
U. S., at 98, n. 21, and ``the best evidence [of discriminatory
intent] often will be the demeanor of the attorney who exercises the
challenge,'' \emph{Hernandez,} 500 U. S., at 365 (plurality opinion).
In addition, race-neutral reasons for peremptory challenges often invoke
a juror's demeanor (\\e. g.,} nervousness, inattention), making the
trial court's firsthand observations of even greater importance. In
this situation, the trial court must evaluate not only whether the
prosecutor's demeanor belies a discriminatory intent, but also whether
the juror's demeanor can credibly be said to have exhibited the basis
for the strike attributed to the juror by the prosecutor. We have
recognized that these determinations of credibility and demeanor lie
`` ‘peculiarly within a trial judge's province,' '' \emph{ibid.}
(quoting \emph{Wainwright} v. \emph{Witt,} 469 U.~S. 412, 428 (1985)), and
we have stated that ``in the absence of exceptional circumstances, we
would defer to [the trial court],'' 500 U. S., at 366 (plurality
opinion).

\section{III}

  Petitioner centers his \emph{Batson} claim on the prosecution's strikes
of two black jurors, Jeffrey Brooks and Elaine Scott. \newpage  Because
we find that the trial court committed clear error in overruling
petitioner's \emph{Batson} objection with respect to Mr. Brooks, we have
no need to consider petitioner's claim regarding Ms. Scott. See,
\emph{e. g., United States} v. \emph{Vasquez-Lopez,} 22 F. 3d 900, 902 (CA9
1994) (``[T]he Constitution forbids striking even a single prospective
juror for a discriminatory purpose''); \emph{United States} v. \emph{Lane,}
866 F. 2d 103, 105 (CA4 1989); \emph{United States} v. \emph{Clemons,} 843 F.
2d 741, 747 (CA3 1988); \emph{United States} v. \emph{Battle,} 836 F. 2d 1084,
1086 (CA8 1987); \emph{United States} v. \emph{David,} 803 F. 2d 1567, 1571
(CA11 1986).

  In \emph{Miller-El} v. \emph{Dretke,} the Court made it clear that in
considering a \emph{Batson} objection, or in reviewing a ruling claimed to
be \emph{Batson} error, all of the circumstances that bear upon the issue
of racial animosity must be consulted. 545 U. S., at 239. Here, as just
one example, if there were persisting doubts as to the outcome, a court
would be required to consider the strike of Ms. Scott for the bearing
it might have upon the strike of Mr. Brooks. In this case, however, the
explanation given for the strike of Mr. Brooks is by itself unconvincing
and suffices for the determination that there was \emph{Batson} error.

  When defense counsel made a \emph{Batson} objection concerning the strike
of Mr. Brooks, a college senior who was attempting to fulfill his
student-teaching obligation, the prosecution offered two race-neutral
reasons for the strike. The prosecutor explained:

    ``I thought about it last night. Number 1, the main reason is that
    he looked very nervous to me throughout the questioning. Number
    2, he's one of the fellows that came up at the beginning [of
    \emph{voir dire\\] and said he was going to miss class. He's a student
    teacher. My main concern is for that reason, that being that he
    might, to go home quickly, come back with guilty of a lesser verdict
    so there wouldn't be a penalty phase. Those are my two reasons.''
    App. 444.

\newpage  Defense counsel disputed both explanations, \emph{id.,} at
444--445, and the trial judge ruled as follows: ``All right. I'm
going [to] allow the challenge. I'm going to allow the challenge,''
\emph{id.,} at 445. We discuss the prosecution's two proffered
grounds for striking Mr. Brooks in turn.

\section{A}

  With respect to the first reason, the Louisiana Supreme Court was
correct that ``nervousness cannot be shown from a cold transcript,
which is why\dots the [trial] judge's evaluation must be given
much deference.'' 942 So. 2d, at 496. As noted above, deference
is especially appropriate where a trial judge has made a finding that
an attorney credibly relied on demeanor in exercising a strike. Here,
however, the record does not show that the trial judge actually made a
determination concerning Mr. Brooks' demeanor. The trial judge was
given two explanations for the strike. Rather than making a specific
finding on the record concerning Mr. Brooks' demeanor, the trial judge
simply allowed the challenge without explanation. It is possible that
the judge did not have any impression one way or the other concerning
Mr. Brooks' demeanor. Mr. Brooks was not challenged until the day
after he was questioned, and by that time dozens of other jurors had
been questioned. Thus, the trial judge may not have recalled Mr.
Brooks' demeanor. Or, the trial judge may have found it unnecessary to
consider Mr. Brooks' demeanor, instead basing his ruling completely on
the second proffered justification for the strike. For these reasons,
we cannot presume that the trial judge credited the prosecutor's
assertion that Mr. Brooks was nervous.

\section{B}

  The second reason proffered for the strike of Mr. Brooks---his
student-teaching obligation---fails even under the highly deferential
standard of review that is applicable here. At the beginning of \emph{voir
dire,} when the trial court asked the \newpage  members of the venire
whether jury service or sequestration would pose an extreme hardship,
Mr. Brooks was 1 of more than 50 members of the venire who expressed
concern that jury service or sequestration would interfere with work,
school, family, or other obligations.

  When Mr. Brooks came forward, the following exchange took place:

      ``MR. JEFFREY BROOKS:\dots I'm a student at Southern
    University, New Orleans. This is my last semester. My major requires
    me to student teach, and today I've already missed a half a
    day. That is part of my---it's required for me to graduate this
    semester.

      ``[DEFENSE COUNSEL]: Mr. Brooks, if you---how many days would you
    miss if you were sequestered on this jury? Do you teach every day?

      ``MR. JEFFREY BROOKS: Five days a week.

      ``[DEFENSE COUNSEL]: Five days a week.

      ``MR. JEFFREY BROOKS: And it's 8:30 through 3:00.

      ``[DEFENSE COUNSEL]: If you missed this week, is there any way
    that you could make it up this semester?

      ``MR. JEFFREY BROOKS: Well, the first two weeks I observe, the
    remaining I begin teaching, so there is something I'm missing
    right now that will better me towards my teaching career.

      ``[DEFENSE COUNSEL]: Is there any way that you could make up
    the observed observation \emph{[sic]} that you're missing today, at
    another time?

      ``MR. JEFFREY BROOKS: It may be possible, I'm not sure.

      ``[DEFENSE COUNSEL]: Okay. So that---

      ``THE COURT: Is there anyone we could call, like a Dean or
    anything, that we could speak to?

      ``MR. JEFFREY BROOKS: Actually I spoke to my Dean, Doctor
    Tillman, who's at the university probably right now.

      ``THE COURT: All right.

      \newpage  ``MR. JEFFREY BROOKS: Would you like to speak to him?

      ``THE COURT: Yeah.

      ``MR. JEFFREY BROOKS: I don't have his card on me.

      ``THE COURT: Why don't you give [a law clerk] his number, give
    [a law clerk] his name and we'll call him and we'll see what we
    can do.

      ``(MR. JEFFREY BROOKS LEFT THE BENCH).''

    App. 102--104.

  Shortly thereafter, the court again spoke with Mr. Brooks:

      ``THE LAW CLERK: Jeffrey Brooks, the requirement for his teaching
    is a three hundred clock hour observation. Doctor Tillman at
    Southern University said that as long as it's just this week,
    he doesn't see that it would cause a problem with Mr. Brooks
    completing his observation time within this semester.

      ``(MR. BROOKS APPROACHED THE BENCH)

      ``THE COURT: We talked to Doctor Tillman and he says he doesn't
    see a problem as long as it's just this week, you know, he'll
    work with you on it. Okay?

      ``MR. JEFFREY BROOKS: Okay.

      ``(MR. JEFFREY BROOKS LEFT THE BENCH).''

    \emph{Id.,} at 116.

  Once Mr. Brooks heard the law clerk's report about the conversation
with Doctor Tillman, Mr. Brooks did not express any further concern
about serving on the jury, and the prosecution did not choose to
question him more deeply about this matter.

  The colloquy with Mr. Brooks and the law clerk's report took place
on Tuesday, August 27; the prosecution struck Mr. Brooks the following
day, Wednesday, August 28; the guilt phase of petitioner's trial ended
the next day, Thursday, August 29; and the penalty phase was completed
by the end of the week, on Friday, August 30.\newpage 

  The prosecutor's second proffered reason for striking Mr. Brooks
must be evaluated in light of these circumstances. The prosecutor
claimed to be apprehensive that Mr. Brooks, in order to minimize the
student-teaching hours missed during jury service, might have been
motivated to find petitioner guilty, not of first-degree murder, but
of a lesser included offense because this would obviate the need for
a penalty phase proceeding. But this scenario was highly speculative.
Even if Mr. Brooks had favored a quick resolution, that would not have
necessarily led him to reject a finding of first-degree murder. If the
majority of jurors had initially favored a finding of first-degree
murder, Mr. Brooks' purported inclination might have led him to agree
in order to speed the deliberations. Only if all or most of the other
jurors had favored the lesser verdict would Mr. Brooks have been in a
position to shorten the trial by favoring such a verdict.

  Perhaps most telling, the brevity of petitioner's trial---
something that the prosecutor anticipated on the record during \emph{voir
dire\\\footnotemark[1]---meant that serving on the jury would not have seriously
interfered with Mr. Brooks' ability to complete his required student
teaching. As noted, petitioner's trial was completed by Friday,
August 30. If Mr. Brooks, who reported to court and was peremptorily
challenged on Wednesday, August 28, had been permitted to serve,
he would have missed only two additional days of student teaching,
Thursday, August 29, and Friday, August 30. Mr. Brooks' dean promised
to ``work with'' Mr. Brooks to see that he was able to make up any
student-teaching time that he missed due to jury service; the dean
stated that he did not think that this would be a problem; and the
record contains no suggestion that Mr. Brooks remained troubled after
hearing the report of the dean's remarks. In addition, although the
record does not include the academic calendar of \newpage  Mr. Brooks'
university, it is apparent that the trial occurred relatively early in
the fall semester. With many weeks remaining in the term, Mr. Brooks
would have needed to make up no more than an hour or two per week in
order to compensate for the time that he would have lost due to jury
service. When all of these considerations are taken into account, the
prosecutor's second proffered justification for striking Mr. Brooks is
suspicious.

^1 See, \emph{e. g.,} App. 98, 105, 111, 121, 130, 204.


  The implausibility of this explanation is reinforced by the
prosecutor's acceptance of white jurors who disclosed conflicting
obligations that appear to have been at least as serious as Mr.
Brooks'. We recognize that a retrospective comparison of jurors
based on a cold appellate record may be very misleading when alleged
similarities were not raised at trial. In that situation, an appellate
court must be mindful that an exploration of the alleged similarities
at the time of trial might have shown that the jurors in question were
not really comparable. In this case, however, the shared characteristic,
\emph{i. e.,} concern about serving on the jury due to conflicting
obligations, was thoroughly explored by the trial court when the
relevant jurors asked to be excused for cause.\footnotemark[2]

  A comparison between Mr. Brooks and Roland Laws, a white juror, is
particularly striking. During the initial stage of \emph{voir dire,} Mr.
Laws approached the court and offered strong reasons why serving on the
sequestered jury would cause him hardship. Mr. Laws stated that he was
``a self-employed general contractor,'' with ``two houses that are
nearing completion, one [with the occupants]\dots moving in this
weekend.'' \emph{Id.,} at 129. He explained that, if he served on the
jury, ``the people won't [be able to] move

^2 The Louisiana Supreme Court did not hold that petitioner had
procedurally defaulted reliance on a comparison of the African-American
jurors whom the prosecution struck with white jurors whom the
prosecution accepted. On the contrary, the State Supreme Court itself
made such a comparison. See 942 So. 2d 484, 495--496 (2006).
\newpage  in.'' \emph{Id.,} at 130. Mr. Laws also had demanding family
obligations:

    ``[M]y wife just had a hysterectomy, so I'm running the kids back
    and forth to school, and we're not originally from here, so I have
    no family in the area, so between the two things, it's kind of bad
    timing for me.'' \emph{Ibid.}

\noindent Although these obligations seem substantially more pressing than Mr.
Brooks', the prosecution questioned Mr. Laws and attempted to elicit
assurances that he would be able to serve despite his work and family
obligations. See \emph{ibid.} (prosecutor asking Mr. Laws ``[i]f you
got stuck on jury duty anyway\dots would you try to make other
arrangements as best you could?''). And the prosecution declined the
opportunity to use a peremptory strike on Mr. Laws. \emph{Id.,} at 549.
If the prosecution had been sincerely concerned that Mr. Brooks would
favor a lesser verdict than first-degree murder in order to shorten the
trial, it is hard to see why the prosecution would not have had at least
as much concern regarding Mr. Laws.

  The situation regarding another white juror, John Donnes, although
less fully developed, is also significant. At the end of the first
day of \emph{voir dire,} Mr. Donnes approached the court and raised the
possibility that he would have an important work commitment later that
week. \emph{Id.,} at 349. Because Mr. Donnes stated that he would
know the next morning whether he would actually have a problem, the
court suggested that Mr. Donnes raise the matter again at that time.
\emph{Ibid.} The next day, Mr. Donnes again expressed concern about
serving, stating that, in order to serve, ``I'd have to cancel too
many things,'' including an urgent appointment at which his presence
was essential. \emph{Id.,} at 467--468. Despite Mr. Donnes' concern,
the prosecution did not strike him. \emph{Id.,} at 490.

  As previously noted, the question presented at the third stage of
the \emph{Batson} inquiry is `` ‘whether the defendant has \newpage 
shown purposeful discrimination.' '' \emph{Miller-El} v. \emph{Dretke,}
545 U.S., at 277 (\textsc{Thomas,} J., dissenting). The prosecution's
proffer of this pretextual explanation naturally gives rise to an
inference of discriminatory intent. See \emph{id.,} at 252 (opinion of
the Court) (noting the ``pretextual significance'' of a ``stated
reason [that] does not hold up''); \emph{Purkett} v. \emph{Elem,} 514 U. S.
765, 768 (1995) \emph{(per curiam)} (``At [the third] stage, implausible
or fantastic justifications may (and probably will) be found to be
pretexts for purposeful discrimination''); \emph{Hernandez,} 500 U. S.,
at 365 (plurality opinion) (``In the typical peremptory challenge
inquiry, the decisive question will be whether counsel's race-neutral
explanation for a peremptory challenge should be believed''). Cf.
\emph{St. Mary's Honor Center} v. \emph{Hicks,} 509 U.~S. 502, 511 (1993)
(``[R]ejection of the defendant's proffered [nondiscriminatory]
reasons will \emph{permit} the trier of fact to infer the ultimate fact of
intentional discrimination'').

  In other circumstances, we have held that, once it is shown that a
discriminatory intent was a substantial or motivating factor in an
action taken by a state actor, the burden shifts to the party defending
the action to show that this factor was not determinative. See
\emph{Hunter} v. \emph{Underwood,} 471 U.~S. 222, 228 (1985). We have not
previously applied this rule in a \emph{Batson} case, and we need not
decide here whether that standard governs in this context. For present
purposes, it is enough to recognize that a peremptory strike shown to
have been motivated in substantial part by discriminatory intent could
not be sustained based on any lesser showing by the prosecution. And in
light of the circumstances here---including absence of anything in the
record showing that the trial judge credited the claim that Mr. Brooks
was nervous, the prosecution's description of both of its proffered
explanations as ``main concern[s],'' App. 444, and the adverse
inference noted above---the record does not show that the prosecution
would have pre-emptively challenged Mr. Brooks based on his nervousness
alone. See \emph{Hunter, supra,} at 228. \newpage  Nor is there any
realistic possibility that this subtle question of causation could be
profitably explored further on remand at this late date, more than a
decade after petitioner's trial.

\hrule

  We therefore reverse the judgment of the Louisiana Supreme Court and
remand the case for further proceedings not inconsistent with this
opinion.

\begin{flushright}\emph{It is so ordered.}\end{flushright}
