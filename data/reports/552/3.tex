% Court
% Per Curiam

\setcounter{page}{4}

  \textsc{Per Curiam.}

  Daniel Siebert was convicted and sentenced to death in the State of
Alabama for the murder of Linda Jarman. Siebert's conviction and
sentence were affirmed on direct appeal, and the certificate of judgment
issued on May 22, 1990. This Court denied certiorari on November 5,
1990. \emph{Siebert} v. \emph{Alabama,} 498 U.~S. 963. On August 25,
1992, Siebert filed a petition for postconviction relief in Alabama
state court. The state courts denied the petition as untimely, however,
because it was filed approximately three months after the expiration of
the then-applicable 2-year statute of limitations, Ala. Rule Crim.
Proc. 32.2(c) (2000--2001), which began to run from the date the
certificate of judgment issued.[[*]] The Alabama Supreme Court denied
certiorari on September 15, 2000. Siebert did not seek review in this
Court. On September 14, 2001, Siebert filed a petition for a federal
writ of habeas corpus, see 28 U.~S.~C. \S~2254, in the District
Court for the Northern District of Alabama.

  The Antiterrorism and Effective Death Penalty Act of 1996 (AEDPA)
established a 1-year statute of limitations for filing a federal habeas
petition. \S~2244(d)(1). The limitations period is tolled, however,
while ``a properly filed application for State post-conviction or other
collateral review with respect to the pertinent judgment or claim is
pending.'' \S~2244(d)(2). Because Siebert's direct appeal became
final before AEDPA became effective, the 1-year limitations period began
to run from April 24, 1996, AEDPA's effective date. See \emph{Carey} v.
\emph{Saffold,} 536 U.~S. 214, 217 (2002). Thus, \newpage  absent tolling,
Siebert's federal habeas petition would be untimely by over four
years.

^* At the time Siebert's petition was before the Alabama courts, Rule
32.2(c) provided that ``the court shall not entertain any petition,''
with certain exceptions not applicable here, ``unless the petition is
filed\dots within two (2) years after the issuance of the certificate
of judgment by the Court of Criminal Appeals.'' The Rule has since been
amended to provide for a 1-year limitations period, but is otherwise
unchanged. See Ala. Rule Crim. Proc. 32.2(c) (2007--2008).

  The District Court dismissed Siebert's habeas petition as untimely,
reasoning that an application for state postconviction relief is
not ``properly filed'' if it was rejected by the state court on
statute-of-limitations grounds. The Court of Appeals reversed, however,
holding that Siebert's state postconviction petition was ``properly
filed'' within the meaning of \S~2244(d)(2), because the state
time bar was not jurisdictional and the Alabama courts therefore had
discretion in enforcing it. See \emph{Siebert} v. \emph{Campbell,} 334 F.
3d 1018, 1030 (CA11 2003) \emph{(per curiam).} The Court of Appeals
accordingly remanded to the District Court to consider the merits of
Siebert's petition.

  While Siebert's habeas petition was pending on remand in the
District Court, we decided \emph{Pace} v. \emph{DiGuglielmo,} 544 U.~S.
408 (2005). In \emph{Pace,} we held that a state postconviction petition
rejected by the state court as untimely is not ``properly filed''
within the meaning of \S~2244(d)(2). \emph{Id.,} at 414, 417.
Relying on \emph{Pace,} the District Court again found that Siebert's
state postconviction petition was not ``properly filed,'' and
dismissed his federal habeas petition as untimely. The Court of Appeals,
however, reversed and remanded. In a one-paragraph opinion, the court
distinguished \emph{Pace} on the ground that Rule 32.2(c), unlike the
statute of limitations at issue in \emph{Pace,} ``operate[s] as an
affirmative defense.'' 480 F. 3d 1089, 1090 (CA11 2007). Thus, the
court found its prior holding---that Siebert's state postconviction
petition was ``properly filed'' because the state court rejected it on
a nonjurisdictional ground---stood as the law of the case.\\Ibid.}

  The Court of Appeals' carveout of time limits that operate as
affirmative defenses is inconsistent with our holding in \emph{Pace.}
Although the Pennsylvania statute of limitations at issue in \emph{Pace}
happens to have been a jurisdictional time bar under state law, see
\emph{Commonwealth} v. \emph{Banks,} 556 Pa. 1, 5--6, \newpage  726 A. 2d 374,
376 (1999), the jurisdictional nature of the time limit was not
the basis for our decision. Rather, we built upon a distinction that
we had earlier articulated in \emph{Artuz} v. \emph{Bennett,} 531 U.~S.
4 (2000), between postconviction petitions rejected on the basis of
``‘filing' conditions,'' which are not ``properly filed'' under
\S2244(d)(2), and those rejected on the basis of ``procedural bars
[that] go to the ability to obtain relief,'' which are. \emph{Pace,
supra,} at 417 (citing \emph{Artuz, supra,} at 10--11). We found that
statutes of limitations are ``filing'' conditions because they ``go
to the very initiation of a petition and a court's ability to consider
that petition.'' \emph{Pace,} 544 U. S., at 417. Thus, we held ``that
time limits, \emph{no matter their form,} are ‘filing' conditions,''
and that a state postconviction petition is therefore not ``properly
filed'' if it was rejected by the state court as untimely. \emph{Ibid.}
(emphasis added).

  In short, our holding in \emph{Pace} turned not on the nature of the
particular time limit relied upon by the state court, but rather on the
fact that time limits generally establish ``conditions to filing''
a petition for state postconviction relief. Whether a time limit is
jurisdictional, an affirmative defense, or something in between, it is
a ``condition to filing,'' \emph{Artuz, supra,} at 9---it places a
limit on how long a prisoner can wait before filing a postconviction
petition. The fact that Alabama's Rule 32.2(c) is an affirmative
defense that can be waived (or is subject to equitable tolling) renders
it no less a ``filing'' requirement than a jurisdictional time bar
would be; it only makes it a less stringent one. Indeed, in \emph{Pace}
we cited the very statute at issue in this case as an example of such
a ``filing'' requirement. See 544 U. S., at 417, n. 7 (citing Ala.
Rule Crim. Proc. 32.2(c) (2004--2005)).

  Excluding from \emph{Pace}'s scope those time limits that operate
as affirmative defenses would leave a gaping hole in what we plainly
meant to be a general rule, as statutes of limitations are often
affirmative defenses. See, \emph{e. g.,} Fed. Rule Civ. \newpage  Proc.
8(c); \emph{Kirkland} v. \emph{State,} 143 Idaho 544, 546, 149 P. 3d 819, 821
(2006) (``The statute of limitations for petitions for post-conviction
relief is not jurisdictional. It ‘is an affirmative defense that may
be waived if it is not pleaded by the defendant' '' (quoting \emph{Cole}
v. \emph{State,} 135 Idaho 107, 110, 15 P. 3d 820, 823 (2000); citation
omitted)); \emph{People} v. \emph{Boclair,} 202 Ill. 2d 89, 101, 789 N. E.
2d 734, 742 (2002) (holding that time bar for filing postconviction
petition is ``an affirmative defense and can be raised, waived, or
forfeited, by the State''). What is more, whether a time limit is
jurisdictional or an affirmative defense is often a disputed question,
as the interpretive history of Rule 32.2(c) itself illustrates, see
\emph{Ex parte Ward,} 46 So. 3d 888, 894 (2007) (noting confusion in the
Alabama lower courts over whether Rule 32.2(c) is jurisdictional).
Under the Court of Appeals' approach, federal habeas courts would
have to delve into the intricacies of state procedural law in deciding
whether a postconviction petition rejected by the state courts as
untimely was nonetheless ``properly filed'' under \S~2244(d)(2). Our
decision in \emph{Pace} precludes such an approach.

  We therefore reiterate now what we held in \emph{Pace:} ``When a
postconviction petition is untimely under state law, ‘that [is] the
end of the matter' for purposes of \S~2244(d)(2).'' 544 U. S., at
414 (quoting \emph{Carey,} 536 U. S., at 226; alteration in original).
Because Siebert's petition for state postconviction relief was
rejected as untimely by the Alabama courts, it was not ``properly
filed'' under \S~2244(d)(2). Accordingly, he was not entitled to
tolling of AEDPA's 1-year statute of limitations.

  The petition for certiorari is granted. The judgment of the Court of
Appeals is reversed, and the case is remanded for further proceedings
consistent with this opinion.

\begin{flushright}\emph{It is so ordered.}\end{flushright}
