Type: Dissenting
Author: Ginsburg

\setcounter{page}{144}

  \textsc{Justice Ginsburg,} dissenting.

  I agree that adhering to \emph{Kendall, Finn,} and \emph{Soriano} is irreconcilable with the reasoning and result in \emph{Irwin,} and therefore join \textsc{Justice Stevens}' dissent. I write separately to explain why I would regard this case as an appropriate occasion to revisit those precedents even if we had not already ``directly overrule[d]'' them. Cf. \emph{Irwin} v. \emph{Department of Veterans Affairs,} 498 U.~S. 89, 98 (1990) (White, J., concurring in part and concurring in judgment).

  \emph{Stare decisis} is an important, but not an inflexible, doctrine in our law. See \emph{Burnet} v. \emph{Coronado Oil \& Gas Co.,} 285 U. S. 393, 405 (1932) (Brandeis, J., dissenting) (``\emph{Stare decisis} is not\dots a universal, inexorable command.''). The policies underlying the doctrine---stability and predictability---are at their strongest when the Court is asked to change its mind, though nothing else of significance has changed. See Powell, Stare Decisis and Judicial Restraint, 47 Wash. \& Lee L. Rev. 281, 286--287 (1990). As to the matter before us, our perception of the office of a time limit on suits against the Government has changed significantly since the decisions relied upon by the Court. We have recognized that ``the same rebuttable presumption of equitable tolling applicable to suits against private defendants should also apply to suits against the United States,'' \emph{Irwin,} 498 U. S., at 95--96, and that ``limitations principles should generally apply to the Government in the same way that they apply to private parties,'' \emph{Franconia Associates} v. \emph{United States,} 536 U.~S. 129, 145 (2002) (internal quotation marks omitted). See also \emph{Scarborough} v. \emph{Principi,} 541 U.~S. 401, 420--422 (2004). It damages the coherence of the law if we cling to outworn precedent at odds with later, more enlightened decisions.

  I surely do not suggest that overruling is routinely in order whenever a majority disagrees with a past decision, and I acknowledge that ``[c]onsiderations of \emph{stare decisis} have special force in the area of statutory interpretation,''\newpage  \emph{Patterson} v. \emph{McLean Credit Union,} 491 U.~S. 164, 172 (1989). But concerns we have previously found sufficiently weighty to justify revisiting a statutory precedent counsel strongly in favor of doing so here. First, overruling \emph{Kendall} v. \emph{United States,} 107 U.~S. 123 (1883), \emph{Finn} v. \emph{United States,} 123 U.~S. 227 (1887), and \emph{Soriano} v. \emph{United States,} 352 U.~S. 270 (1957), would, as the Court concedes, see \emph{ante,} at 138--139, ``achieve a uniform interpretation of similar statutory language,'' \emph{Rodriguez de Quijas} v. \emph{Shearson/American Express, Inc.,} 490 U.~S. 477, 484 (1989). Second, we have recognized the propriety of revisiting a decision when ``intervening development of the law'' has ``removed or weakened [its] conceptual underpinnings.''\emph{Patterson,} 491 U. S., at 173. \emph{Irwin} and \emph{Franconia}---not to mention our recent efforts to apply the term ``jurisdictional'' with greater precision, see, \emph{e. g., Arbaugh} v. \emph{Y\&H Corp.,} 546 U.~S. 500, 515--516 (2006)---have left no tenable basis for \emph{Kendall} and its progeny.

  Third, it is altogether appropriate to overrule a precedent that has become ``a positive detriment to coherence and consistency in the law.'' \emph{Patterson,} 491 U. S., at 173. The inconsistency between the \emph{Kendall} line and \emph{Irwin} is a source of both theoretical incoherence and practical confusion. For example, 28 U. S. C. \S~2401(a) contains a time limit materially identical to the one in \S~2501. Courts of Appeals have divided on the question whether \S~2401(a)'s limit is ``jurisdictional.'' Compare \emph{Center for Biological Diversity} v. \emph{Hamilton,} 453 F. 3d 1331, 1334 (CA11 2006) \emph{(per curiam),} with \emph{Cedars-Sinai Medical Center} v. \emph{Shalala,} 125 F. 3d 765, 770 (CA9 1997). See also \emph{Harris} v. \emph{Federal Aviation Admin.,} 353 F. 3d 1006, 1013, n. 7 (CADC 2004) (recognizing that \emph{Irwin} may have undermined Circuit precedent holding that \S~2401(a) is ``jurisdictional''). Today's decision hardly assists lower courts endeavoring to answer this question. While holding that the language in \S~2501 is ``jurisdictional,'' the Court also implies that \emph{Irwin} governs the interpretation of\newpage  all statutes we have not yet construed---including, presumably, the identically worded \S~2401.See \emph{ante,} at 137--138.

  Moreover, as the Court implicitly concedes, see \emph{ante,} at 139, the strongest reason to adhere to precedent provides no support for the \emph{Kendall-Finn-Soriano} line. ``\emph{Stare decisis} has added force when the legislature, in the public sphere, and citizens, in the private realm, have acted in reliance on a previous decision, for in this instance overruling the decision would dislodge settled rights and expectations or require an extensive legislative response.'' \emph{Hilton} v. \emph{South Carolina Public Railways Comm'n,} 502 U. S. 197, 202 (1991). The Government, however, makes no claim that either private citizens or Congress have relied upon the ``jurisdictional'' status of \S~2501. There are thus strong reasons to abandon---and notably slim reasons to adhere to---the anachronistic interpretation of \S~2501 adopted in \emph{Kendall.}

  Several times, in recent Terms, the Court has discarded statutory decisions rendered infirm by what a majority considered to be better informed opinion. See, \emph{e. g., Leegin Creative Leather Products, Inc.} v. \emph{PSKS, Inc.,} 551 U.~S. 877, 907 (2007) (overruling \emph{Dr. Miles Medical Co.} v. \emph{John D. Park \& Sons Co.,} 220 U. S. 373 (1911)); \emph{Bowles} v. \emph{Russell,} 551 U.~S. 205, 214 (2007) (overruling \emph{Thompson} v. \emph{INS,} 375 U.~S. 384 (1964) \emph{(per curiam),} and \emph{Harris Truck Lines, Inc.} v. \emph{Cherry Meat Packers, Inc.,} 371 U.~S. 215 (1962) \emph{(per curiam)}); \emph{Illinois Tool Works Inc.} v. \emph{Independent Ink, Inc.,} 547 U.~S. 28, 42--43 (2006) (overruling, \emph{inter alia, Morton Salt Co.} v. \emph{G. S. Suppiger Co.,} 314 U.~S. 488 (1942)); \emph{Hohn} v. \emph{United States,} 524 U.~S. 236, 253 (1998) (overruling \emph{House} v. \emph{Mayo,} 324 U.~S. 42 (1945) \emph{(per curiam)}). In light of these overrulings, the Court's decision to adhere to \emph{Kendall, Finn,} and \emph{Soriano}---while offering nothing to justify their reasoning or results---is, to say the least, perplexing. After today's decision, one will need a crystal ball to predict when this Court will reject, and when it will cling to, its prior decisions interpreting legislative texts.\newpage  I would reverse the judgment rendered by the Federal Circuit majority. In accord with dissenting Judge Newman, I would hold that the Court of Appeals had no warrant to declare the petitioner's action time barred.
