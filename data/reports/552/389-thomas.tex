% Dissenting
% Thomas

\setcounter{page}{408}

  \textsc{Justice Thomas,} with whom \textsc{Justice Scalia} joins, dissenting.

  Today the Court decides that a ``charge'' of age discrimination
under the Age Discrimination in Employment Act of 1967 (ADEA) is
whatever the Equal Employment Opportunity Commission (EEOC) says it is.
The filing at issue in this case did not state that it was a charge
and did not include a charge form; to the contrary, it included a
form that expressly stated it was for the purpose of ``pre-charge''
counseling. What is more, the EEOC did not assign it a charge number,
notify the employer of the complainant's\footnotemark[1] allegations, or commence
enforcement proceedings. Notwithstanding these facts, the Court
concludes, counterintuitively, that respondent's filing is a charge
because it manifests an intent for the EEOC to take ``some action.''
\emph{Ante,} at 400. Because the standard the Court applies is broader
than the ordinary meaning of the term ``charge,'' and because it is
so malleable that it effectively absolves the EEOC of its obligation to
administer the ADEA according to discernible standards, I respectfully
dissent.

\section{I}

  As the Court notes, the ADEA directs the agency to take certain
actions upon receipt of a ``charge'' but does not define that word.
\emph{Ante,} at 395. Because there is nothing to suggest that Congress used
``charge'' as a term of art, we must construe it ``in accordance
with its ordinary or natural meaning.'' See \emph{FDIC} v. \emph{Meyer,}
510 U.~S. 471, 476 (1994). Dictionaries define a ``charge'' as an
accusation or indictment. \newpage  See, \emph{e. g.,} American Heritage
Dictionary 312 (4th ed. 2000); Webster's Third New International
Dictionary 377 (1993). In legal parlance, a ``charge'' is generally
a formal allegation of wrongdoing that initiates legal proceedings
against an alleged wrongdoer. In criminal law, for example, a charge
is defined as ``[a] formal accusation of an offense as a preliminary
step to prosecution.''Black's Law Dictionary 248 (8th ed. 2004).
Similarly, in this context, a ``charge'' is a formal accusation
of discrimination that objectively manifests an intent to initiate
enforcement proceedings against the employer. Just as a complaint
or police report that describes the commission of a crime is not a
``charge'' under the criminal law, so too here, a document that
merely describes the alleged discrimination and requests the EEOC's
assistance, but does not objectively manifest an intent to initiate
enforcement proceedings, is not a ``charge'' within the meaning of the
ADEA.

^1 This opinion will refer to potentially charging parties who contact
the EEOC about discrimination as ``complainants.'' I use this term for
simplicity and do not intend to invoke the distinction in the EEOC's
regulations between complainants and charging parties. See 29 CFR
\S~1626.3 (2007). Similarly, I use ``respondent'' not as it appears
in the EEOC's regulations---referring to the ``prospective defendant
in a charge or complaint,'' \emph{ibid.\\---but as a reference to the
responding parties in this case.

  This understanding of a ``charge'' is common in administrative
law. The regulations governing allegations of unlawful employment
practices at the Government Accountability Office, for example,
define ``charge'' as ``any request filed\dots to investigate
any matter'' within the jurisdiction of the agency. 4 CFR \S~28.3
(2007). In actions alleging unfair labor practices, the ``purpose of
the charge is\dots to set in motion the [National Labor Relations]
Board's investigative machinery.''\\Flex Plastics, Inc.,} 262 N.
L. R. B. 651, 652 (1982). In accordance with the charge's purpose
of triggering an investigation that involves notice to the employer,
agencies often indicate that the charge will not be kept confidential.
For example, the EEOC anticipates that a charge usually will be
released to the employer.See, \emph{e. g.,} 1 EEOC Compliance Manual
\S~2.2(b), p. 2:0001 (Aug. 2002) (hereinafter EEOC Manual) (providing
that correspondence may be processed as a charge if, \emph{inter alia,}
it ``does not express concerns about confidentiality''); \emph{id.,}
\S~3.6, at 3:0001 (June 2001) (noting that ``it is EEOC policy to
.~.~. serve the [employer] with a copy \newpage  of ADEA charges unless
this will impede EEOC's law enforcement functions'').

  The ordinary understanding of the term ``charge'' applies equally
in the employment discrimination context, where a charge is a formal
accusation that an employer has violated, or will violate, employment
discrimination laws. See 29 CFR \S~1626.3 (2007) (describing a
charge as an allegation that an employer ``has engaged in or is
about to engage in actions in violation of the Act''). The charge
is presented to the agency with jurisdiction over such matters---the
EEOC---to trigger enforcement proceedings that are intended to eliminate
violations of the ADEA. See 29 U.~S.~C. \S~626(d) (directing the
agency, upon receipt of a charge, to notify the employer and take steps
to eliminate the allegedly unlawful practice). I therefore agree
with the EEOC that the statutory term ``charge'' must mean, at a
minimum,\footnotemark[2] a writing that objectively indicates an intent to initiate
the agency's enforcement processes. See Brief for United States as
\emph{Amicus Curiae} 15 (noting that a charge must ``objectively manifest
an intent to make a formal accusation'' of an ADEA violation).\footnotemark[3]
In any event, respondent's documents do not \newpage  objectively
indicate an intent to initiate the EEOC's processes; any test that
construes them otherwise is, in my opinion, an unreasonable construction
of the statutory term ``charge,'' and unworthy of deference.See
\emph{Chevron U. S. A. Inc.} v. \emph{Natural Resources Defense Council,
Inc.,} 467 U.~S. 837, 843--845 (1984).


^2 I do not mean to foreclose the possibility that the EEOC may include
additional elements in its definition, as long as they are reasonable
constructions of the statutory term ``charge.''See \emph{Chevron U. S.
A. Inc.} v. \emph{Natural Resources Defense Council, Inc.,} 467 U.~S.
837, 843--845 (1984).

^3 As the EEOC acknowledges, its position on whether intent is required
has varied over the years. See Brief for United States as \emph{Amicus
Curiae} 8, 16--17, n. 8. In 1983, the agency issued its regulations,
which contain no intent requirement. Final Procedural Regulations;
Age Discrimination in Employment Act, 48 Fed. Reg. 138. Five years
later, it argued against an intent requirement as \emph{amicus curiae} in
\emph{Steffen} v. \emph{Meridian Life Ins. Co.,} 859 F. 2d 534, 544 (CA7
1988) (``The EEOC, which has appeared as amicus curiae on Steffen's
behalf, has supported Steffen's contention that a completed Intake
Questionnaire, in and of itself, constitutes a charge''). In
2002, the agency issued an internal memorandum and internal guidance
documents including an intent requirement.See Memorandum from
Elizabeth M. Thornton, Director, Office of Field Programs, EEOC, to
All District, Area, and Local Office Directors et al. (Feb. 21, 2002),
online at http://\newpage www.eeoc.gov/charge/memo-2-21-02.html (all
Internet materials as visited Feb. 22, 2008, and available in Clerk
of Court's case file); 1 EEOC Manual \S~2.2(b), at 2:0001 (noting
that correspondence must, \emph{inter alia,} ``constitut[e] a clear and
timely request for EEOC to act'' before it can be construed as a
charge). The EEOC contradicted itself four years later, when it again
took the position that there was no intent requirement in \emph{Gordon}
v. \emph{Shafer Contracting Co.,} 469 F. 3d 1191, 1194 (CA8 2006) (``In
an amicus brief, the EEOC urges us to accept such a verified Intake
Questionnaire as satisfying the charge requirement''); see also
Brief for United States as \emph{Amicus Curiae} 16--17, n. 8. The
following year, the EEOC issued another internal memorandum and updated
the Frequently Asked Questions section of its Web site, including
the intent requirement in each.Memorandum from Nicholas M. Inzeo,
Director, Office of Field Programs, EEOC, to All District, Field, Area,
and Local Office Directors et al. (Aug. 13, 2007) (hereinafter Inzeo
Memorandum), online at http://www.eeoc.gov/charge/memo-8-13-07.html;
EEOC Frequently Asked Questions (hereinafter EEOC FAQ), Answer to
``How do I file a charge of employment discrimination?'' online at
https://eeoc.custhelp.com.

\section{II}

  The cumulative effect of two aspects of respondent's documents, the
Court holds, illustrates that she filed a charge of discrimination:
first, her request in her affidavit that the agency take action, and
second, her marking of a box on the questionnaire form consenting to the
release of her identity to her employer, Federal Express Corporation
(FedEx). \emph{Ante,} at 405--406. In my view, neither of these
factors, separately or together, objectively indicates that respondent
intended to initiate the EEOC's processes.

  The last substantive paragraph of respondent's affidavit
said: ``Please force Federal Express to end their age
discrimination~.~.~.~.'' App. 273. But the issue here is not
whether respondent wanted the EEOC to cause the com\newpage pany's
compliance by \emph{any} means; it is whether she wanted the EEOC
immediately to employ the particular method of enforcement that consists
of filing a charge. Her request to ``force Federal Express to end
their age discrimination'' could have been met by the agency's
\emph{beginning} the interviewing and counseling process that would
ultimately lead to a charge. Or the agency could have proceeded to
enforcement without a charge. See \emph{infra,} at 417, n. 5 (discussing
the EEOC's authority to investigate age discrimination in the absence
of any charge). Alternatively, after receiving indications of repeated
violations by a particular company on many intake questionnaires, the
agency could have approached the company informally, effectively forcing
compliance by the threat of agency litigation. See B. Lindemann \& D.
Kadue, Age Discrimination in Employment Law 470 (2003) (``The EEOC
may commence litigation under the ADEA without having to first file a
charge, so long as it has attempted conciliation''). That sort of
action would also have satisfied respondent's request. Respondent's
statement to the EEOC no more constitutes expression of a present intent
to file a charge than her request to a lawyer that he put an end to
her employer's discrimination would constitute expression of present
intent to file a complaint. The Court is simply wrong to say that a
charge must merely request that the agency take ``some action,''
\emph{ante,} at 400, or ``whatever action is necessary to vindicate
her rights,'' \emph{ante,} at 398, or unspecified ``remedial action
to protect the employee's rights,'' \emph{ante,} at 402. To the
contrary, a charge must request that the agency take \emph{the particular
form} of remedial action that results from filing a charge.

  Aside from revealing the ambiguity in its definition of a
``charge,'' the Court's constructions stretch the term far beyond
what it can bear. A mere request for help from a complainant---who,
the Court acknowledges, may ``have no detailed knowledge of the
relevant statutory mechanisms and agency processes,'' \emph{ante,} at
403---cannot be equated with \newpage  an intent to file a charge. The
Court's test permits no principled basis for distinguishing a request
for the agency to take what might be described as ``pre-charge''
actions, such as interviewing and counseling, from a request for the
agency to commence enforcement proceedings. All are properly considered
``agency action,'' all presumably would be part of the agency's
remedial processes, and all are designed to protect the employee's
rights. But a complainant's intent to trigger actions unrelated to
charge processing plainly cannot form the basis for distinguishing
charges from other inquiries because it lacks any grounding in the
meaning of the statutory term.

  Even if respondent's statement, viewed in isolation, could
reasonably be understood as reflecting the requisite intent, it must
be viewed in context. It is clear that respondent's filing, taken
as a whole, did not amount to a request for the EEOC to commence
enforcement proceedings. In fact, respondent's affidavit is replete
with indications of an intent \emph{not} to commence formal agency action.
The entire first paragraph is an extensive statement that respondent
had been assured her affidavit would be kept confidential, App. 266,
suggesting that she did not intend the document to initiate enforcement
proceedings, which would require the EEOC to notify FedEx of her
allegations. See 1 EEOC Manual \S~2.2(b), at 2:0001 (stating that
correspondence expressing concerns about confidentiality should not be
treated as a charge). She identified the document as a ``complaint.''
App. 266. And although the document was notarized and respondent
attested to its truthfulness, nowhere did she state that she authorized
the EEOC to attempt to resolve the dispute.\\Id.,} at 266--274.
Finally, the affidavit was attached to the intake questionnaire, which
also gave no objective indication of any intent to activate the EEOC's
enforcement proceedings.

  As the Court concedes, the agency would not consider respondent's
intake questionnaire a charge. \emph{Ante,} at 405. In\newpage deed,
we are in agreement that the form contains numerous indicators that
it will \emph{not} be considered a charge. \emph{Ibid.} (stating that the
``design of the form\dots does not give rise to the inference that
the employee requests action against the employer,'' and ``[i]n fact
the wording of the questionnaire suggests the opposite''). The title
of the form, ``Intake Questionnaire,''\footnotemark[4] suggests that its purpose
is preliminary information gathering, not the filing of a formal charge.
Likewise, the statement at the top of the form indicates that further
steps are anticipated: ``Please answer the following questions, telling
us briefly why you have been discriminated against in employment. An
officer of the EEOC will talk with you after you complete this form.''
App. 265. The form gives the complainant the opportunity to keep her
identity confidential. \emph{Ibid.} And it contains a Privacy Act
statement on the back, prominently referenced on the front of the form,
which states that the information provided on the questionnaire ``will
be used by Commission employees to determine the existence of facts
relevant to a decision as to whether the Commission has jurisdiction
over \emph{potential charges,} complaints or allegations of employment
discrimination and to provide such \emph{pre-charge filing} counseling as
is appropriate.''\\Ibid.} (emphasis added).

  The Court apparently believes that these objective indicators are
trumped by the fact that respondent marked the \newpage  box authorizing
the agency to disclose her identity to her employer. That portion of
the form states: ``Normally, your identity will be disclosed to the
organization which allegedly discriminated against you. Do you .~.~.
Consent or\dots not consent to such disclosure?''\\Ibid.}
Since the form states it is for a narrow purpose and that identities
of complainants are normally disclosed, there is no reason to view
respondent's checking of the box as converting the form's stated
narrow purpose to a broader one.

^4 An apparently more recent version of Form 283 is entitled ``Charge
Questionnaire,'' and states that, ``[w]hen this form constitutes the
only timely written statement of alleg[ed]\dots discrimination,
the Commission will, consistent with 29 CFR 1601.12(b) and 29 CFR
1626.8(b), consider it to be a sufficient charge of discrimination under
the relevant statute(s).'' 1 EEOC Manual, Exh. 1--B, at 1:0006
(June 2001); see also B. Lindemann \& D. Kadue, Age Discrimination
in Employment Law 477, n. 14 (2003). Although the ``Charge
Questionnaire'' form is dated ``Test 10/94,'' and is the only
questionnaire form included in the EEOC Manual, it was not the form
respondent used. Her intake questionnaire form was dated 1987.App.
265.

  In comparison to the intake questionnaire, the ``Charge of
Discrimination'' form contains a number of objective indications that
it will trigger the agency's enforcement processes. Indeed, its very
title clearly indicates that it is a charge, and it contains a space for
a charge number. 1 EEOC Manual Exh. 2--C, at 2:0009. Although both
forms require the complainant to sign and attest that the information
is correct, only the charge of discrimination requests an attestation
that the complainant intends to initiate the agency's procedures.
Just above the space for the complainant's signature, the form states
``I want this charge filed with both the EEOC and the State or local
Agency, if any. I will advise the agencies if I change my address or
telephone number and I will cooperate fully with them in the processing
of my charge in accordance with their procedures.'' \emph{Ibid.} The
form notes ``Charging Party'' at the bottom of the space for the
signature. \emph{Ibid.} And it states on the back that ``[t]he purpose
of the charge, whether recorded initially on this form or in some other
way reduced to writing and later recorded on this form, is to invoke
the jurisdiction of the Commission.'' \emph{Id.,} at 2:0010. Also on
the back, under ``ROUTINE USES,'' the charge of discrimination states
that ``[i]nformation provided on this form will be used by Commission
employees to guide the Commission's investigatory activities.''
\emph{Ibid.} Although the EEOC prefers to receive a completed charge
form, see Brief for United States as \emph{Amicus Curiae} 18, n. 9
\newpage  (noting that ``EEOC's preferred practice is indeed to receive
a completed Form 5 whenever possible''), another writing could
indicate a complainant's intent to commence the EEOC's enforcement
processes. But the form chosen by the complainant must be viewed as
strong evidence of the complainant's intent, and that evidence should
be deemed overcome only if the document, viewed as a whole, compels that
conclusion.

  For the reasons I have described, respondent's intake questionnaire
and attached affidavit do not objectively indicate that she intended to
initiate the EEOC's enforcement processes. The Court's conclusion
that the two factors ``were enough to bring the entire filing
within the definition of charge,'' \emph{ante,} at 406, is not
supported by the facts and, in my view, reveals that the Court's
standard is sufficiently vacuous to permit the agency's \emph{post hoc}
interpretation of a document to control. But we cannot, under the guise
of deference, sanction an agency's use of a standard that the agency
has not adequately explained.Cf., \emph{e. g., Pearson} v. \emph{Shalala,}
164 F. 3d 650, 660--661 (CADC 1999) (equating an agency's denial of a
party's request based on the application of a vague term with simply
saying ``no'' without explanation).

  The malleability of the Court's test is further revealed by its
statement that ``[t]here might be instances where the indicated
discrimination is so clear or pervasive that the agency could
infer from the allegations themselves that action is requested and
required.'' \emph{Ante,} at 405. The clarity or pervasiveness of
alleged discrimination is irrelevant to the employee's intent
to file a charge. Although the Court states that the ``agency
is not required to treat every completed Intake Questionnaire as
a charge,'' \emph{ibid.,} it apparently would permit the EEOC
to do so, because under the Court's test the EEOC can infer
intent from circumstances---such as ``clear\newpage  or pervasive''
discrimination---that have no grounding in the ``intent to act''
requirement.\footnotemark[5]

\section{III}

  Yet another indication that respondent's documents did not
objectively manifest an intent to initiate the EEOC's enforcement
processes is that the agency did not treat them as a charge. It did
not assign a charge number, and it did not notify FedEx or commence
its enforcement proceedings. This is not surprising: The EEOC accepts
charges via a thorough intake process\footnotemark[6] in which completed intake
questionnaires are not typically viewed as charges, but are used to
assist the EEOC in developing the charge. A complainant visiting an EEOC
office may be asked to complete an intake questionnaire. See EEOC FAQ,
Answer to ``How do I file a charge of employment discrimination?''
online at https:// eeoc.custhelp.com. An EEOC investigator then
conducts a precharge interview, 1 EEOC Manual \S~2.4, at 2:0001; 2 B.
Lindemann \& P. Grossman, Employment Discrimination \newpage  Law 1685 (4th
ed. 2007), covering a range of topics, including applicable laws, the
complainant's allegations and other possibly discriminatory practices,
confidentiality, time limits, notice requirements, and private suit
rights. See 1 EEOC Manual \S\S~2.4(a)--(g), at 2:0001--2:0003.
Using the information contained in the intake questionnaire and gathered
during the interview, the investigator drafts the charge on a Form 5
Charge of Discrimination according to specific agency instructions, and
also drafts an affidavit containing background data. See \emph{id.,}
\S~2.5, at 2:0003--2:0005. The investigator assigns a charge
number and begins the process of serving notice on the employer and
investigating the allegations.See 2 Lindemann \& Grossman, \emph{supra,}
at 1685--1690.

^5 Perhaps the Court's statement is intended to address the EEOC's
authority to investigate alleged discrimination even in the absence
of a charge. Under Title VII, these are called ``Commissioner
Charges.'' See, \emph{e. g.,} 29 CFR \S~1601.11(a). While the
ADEA does not provide for such charges, the EEOC has independent
authority to investigate age discrimination in the absence of any
charge.See 29 U.~S.~C. \S~626(a); 29 CFR \S~1626.4; \emph{Gilmer}
v. \emph{Interstate/Johnson Lane Corp.,} 500 U.~S. 20, 28 (1991); 1 EEOC
Manual \S~8.1, at 8:0001 (June 2001). If this is what the Court
means by its statement that allegations of ``clear or pervasive''
discrimination may indicate to the agency that action is ``required,''
\emph{ante,} at 405, then it is not clear how it is relevant to
the standards at issue in this case for evaluating an individual
complainant's filing.

^6 This process, in all respects relevant to this case, has
been consistently used by the agency since shortly after it
assumed jurisdiction over ADEA actions in 1979.See 1 EEOC Manual
\S\S~2.1--2.7, at 2:0001--2:0006 (2002); 2 B. Lindemann \& P.
Grossman, Employment Discrimination Law 1220 (3d ed. 1996); B. Schlei \&
P. Grossman, Employment Discrimination Law 939--940, 942, 948 (2d ed.
1983).

  Charges are thus typically completed and filed by the agency, not the
complainant. See \emph{Edelman} v. \emph{Lynchburg College,} 535 U.~S. 106,
115, n. 9 (2002) (``The general practice of EEOC staff members is to
prepare a formal charge of discrimination for the complainant to review
and to verify'' (citing Brief for United States et al. as \emph{Amici
Curiae} 24)); EEOC FAQ, Answers to ``Where can I obtain copies of the
forms to file a charge?'' (stating that the agency's policy is not
to provide blank charge forms); ``How do I file a charge of employment
discrimination?'' (``When the field office has all the information it
needs, you will be counseled regarding the strengths and weaknesses of a
potential charge and/or you will receive a completed charge form (Form
5) for your signature''), online at https://eeoc.custhelp.com. Once
the charge is complete, the EEOC notifies the employer of the charge,
usually attaching a copy of the completed charge form.1 EEOC Manual
\S~3.6, at 3:0001 (``While 29 CFR \S~1626.11 only requires notice to
the [employer] that an ADEA charge has been filed, it is EEOC policy to
also serve the [employer] with a copy of ADEA charges unless this will
impede EEOC's law enforcement functions''); \newpage  Inzeo Memorandum,
online at http://www.eeoc.gov/charge/ memo-8-13-07.html.

  To be sure, the EEOC is prepared to accept charges by other methods.
If the complainant cannot or will not visit an EEOC office, an
investigator may conduct the precharge interview and take the charge
by telephone, see 1 EEOC Manual \S\S~2.3, 2.4, at 2:0001, but
the agency must reduce the allegations to writing before they will
be considered a charge, see 29 CFR \S~1626.8(b) (``[A] charge is
sufficient when the Commission receives from the person making the
charge either a written statement or information reduced to writing by
the Commission that conforms to the requirements of \S~1626.6'').
When the EEOC receives correspondence that is a potential charge,
the investigator must contact the complainant and conduct an intake
interview. See 1 EEOC Manual \S~2.2(a), at 2:0001. Alternatively,
if the correspondence ``contains all information necessary to begin
investigating, constitutes a clear and timely request for EEOC to act,
and does not express concerns about confidentiality or retaliation,''
then the investigator may process it as a charge without conducting an
interview.See \emph{id.,} \S~2.2(b), at 2:0001.

  Thus, while the EEOC does not typically view an intake questionnaire
as a charge, I would not rule out the possibility that, in appropriate
circumstances, an intake questionnaire, like other correspondence,
could contain the elements necessary to constitute a charge. But an
intake questionnaire---even one accompanied by an affidavit---should
not be construed as a charge unless it objectively indicates an intent
to initiate the EEOC's enforcement processes. As I have explained,
respondent's intake questionnaire and attached affidavit fall short of
that standard. I would hold that the documents respondent filed with the
EEOC were not a charge and thus did not preserve her right to sue.

  The implications of the Court's decision will reach far beyond
respondent's case. Today's decision does nothing---ab\newpage solutely
nothing---to solve the problem that under the EEOC's current processes
no one can tell, \emph{ex ante,} whether a particular filing is or is not a
charge. Given the Court's utterly vague criteria, whatever the agency
later decides to regard as a charge is a charge---and the statutorily
required notice to the employer and conciliation process will be evaded
in the future as it has been in this case. The Court's failure to
apply a clear and sensible rule renders its decision of little use in
future cases to complainants, employers, or the agency.

  For these reasons, I would reverse the judgment below.
