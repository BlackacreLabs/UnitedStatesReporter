% Concurring
% Roberts

\setcounter{page}{459}

  \textsc{Chief Justice Roberts,} with whom \textsc{Justice Alito} joins,
concurring.

  I share \textsc{Justice Scalia}'s concern that permitting a candidate
to identify his political party preference on an official election
ballot---regardless of whether the candidate is endorsed by the party or
is even a member---may effectively force parties to accept candidates
they do not want, amounting to forced association in violation of the
First Amendment.

  I do think, however, that whether voters \emph{perceive} the candidate
and the party to be associated is relevant to the constitutional
inquiry. Our other forced-association cases indicate as much. In \emph{Boy
Scouts of America} v. \emph{Dale,} 530 U.~S. 640, 653 (2000), we said
that Dale's presence in the Boy Scouts would ``force the organization
to send a message\dots [to] the world'' that the Scouts approved of
homosexuality. In other words, accepting Dale would lead outsiders to
believe the Scouts endorsed homosexual conduct. Largely for that reason,
we held that the First Amendment entitled the Scouts to exclude Dale.
\emph{Id.,} at 659. Similarly, in \emph{Hurley} v. \emph{Irish-American Gay,
Lesbian and Bisexual Group of Boston, Inc.,} 515 U.~S. 557 (1995),
we allowed the organizers of Boston's St. Patrick's Day Parade to
exclude a pro-gay rights float because the float's presence in the
parade might create the impression that the organizers agreed with the
float sponsors' message. See \emph{id.,} at 575--577.\newpage 

  Voter perceptions matter, and if voters do not actually believe the
parties and the candidates are tied together, it is hard to see how
the parties' associational rights are adversely implicated. See
\emph{Rumsfeld} v. \emph{Forum for Academic and Institutional Rights, Inc.,}
547 U.~S. 47, 65 (2006) (rejecting law schools' First Amendment
objection to military recruiters on campus because no reasonable
person would believe the ``law schools agree[d] with any speech by
recruiters''). After all, individuals frequently claim to favor this
or that political party; these preferences, without more, do not create
an unconstitutional forced association.

  What makes these cases different, as \textsc{Justice Scalia} explains,
is the place where the candidates express their party preferences: on
the ballot. See \emph{post,} at 465 (dissenting opinion) (noting ``the
special role that a state-printed ballot plays in elections''). And
what makes the ballot ``special'' is precisely the effect it has on
voter impressions. See \emph{Cook} v. \emph{Gralike,} 531 U.~S. 510, 532
(2001) (Rehnquist, C. J., concurring in judgment) (``[T]he ballot
.~.~. is the last thing the voter sees before he makes his choice'');
\emph{Anderson} v. \emph{Martin,} 375 U.~S. 399, 402 (1964) (``[D]irecting
the citizen's attention to the single consideration of race .~.~.
may decisively influence the citizen to cast his ballot along racial
lines'').

  But because respondents brought this challenge before the State of
Washington had printed ballots for use under the new primary regime, we
have no idea what those ballots will look like. Petitioners themselves
emphasize that the content of the ballots in the pertinent respect is
yet to be determined. See Reply Brief for Washington State Grange
2--4, 7--13.

  If the ballot is designed in such a manner that no reasonable
voter would believe that the candidates listed there are nominees or
members of, or otherwise associated with, the parties the candidates
claimed to ``prefer,'' the I--872 primary system would likely pass
constitutional muster. I cannot say on the present record that it
would be impossible for \newpage  the State to design such a ballot.
Assuming the ballot is so designed, voters would not regard the listed
candidates as ``party'' candidates, any more than someone saying ``I
like Campbell's soup'' would be understood to be associated with
Campbell's. Voters would understand that the candidate does not speak
on the party's behalf or with the party's approval. On the other
hand, if the ballot merely lists the candidates' preferred parties
next to the candidates' names, or otherwise fails clearly to convey
that the parties and the candidates are not necessarily associated, the
I--872 system would not survive a First Amendment challenge.

  \textsc{Justice Scalia} complains that ``[i]t is hard to know how to
respond'' to such mistaken views, \emph{post,} at 467 (dissenting
opinion), but he soldiers on nonetheless. He would hold that a
party is burdened by a candidate's statement of preference even if
no reasonable voter believes from the ballot that the party and the
candidate are associated. I take his point to be that a particular
candidate's ``endorsement'' of a party might alter the party's
message, and this violates the party's freedom of association. See
\emph{post,} at 468 (dissenting opinion).

  But there is no general right to stop an individual from saying, ``I
prefer this party,'' even if the party would rather he not. Normally,
the party protects its message in such a case through responsive speech
of its own. What makes these cases different of course is that the State
controls the content of the ballot, which we have never considered a
public forum. See \emph{Timmons} v. \emph{Twin Cities Area New Party,}
520 U.~S. 351, 363 (1997) (ballots are not ``forums for political
expression''). Neither the candidate nor the party dictates the
message conveyed by the ballot. In such a case, it is important to know
what the ballot actually says---both about the candidate and about
the party's association with the candidate. It is possible that no
reasonable voter in Washington State will regard the listed candidates
as members of, or otherwise associated with, the political parties the
candidates claim to prefer. Nothing in my analysis requires the \newpage 
parties to produce studies regarding voter perceptions on this score,
but I would wait to see what the ballot says before deciding whether it
is unconstitutional.

  Still, I agree with \textsc{Justice Scalia} that the history of the
challenged law suggests the State is not particularly interested in
devising ballots that meet these constitutional requirements. See
\emph{post,} at 468 (dissenting opinion). But this record simply does
not allow us to say with certainty that the election system created by
I--872 is unconstitutional. Accordingly, I agree with the Court that
respondents' present challenge to the law must fail, and I join the
Court's opinion.
