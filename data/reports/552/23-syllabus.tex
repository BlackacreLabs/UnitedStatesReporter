% Syllabus
% Reporter of Decisions

\setcounter{page}{23}

  Under federal law, the maximum prison term for a felon convicted of possessing a firearm is ordinarily ten years. See 18 U.~S.~C. \S~924(a)(2). If the offender's prior criminal record includes at least three convictions for ``violent felon[ies,]'' however, the Armed Career Criminal Act (ACCA) mandates a minimum term of 15 years. See \S~924(e)(1). Congress defined the term ``violent felony'' to include specified crimes ``punishable by imprisonment for a term exceeding one year,'' \S~924(e)(2)(B), but also provided that a state-law misdemeanor may qualify as a ``violent felony'' if the offense is punishable by a term of more than two years, \S~921(a)(20)(B). Congress amended \S~921(a)(20) in 1986 to exclude from qualification for enhanced sentencing ``[a]ny conviction which has been expunged, or set aside or for which a person has been pardoned or has had civil rights [\emph{i. e.,} rights to vote, hold office, and serve on a jury] restored.''

  Petitioner Logan pleaded guilty to being a felon in possession of a firearm and received a 15-year sentence, the mandatory minimum under ACCA. In imposing this sentence, the court took account of three Wisconsin misdemeanor battery convictions, each of them punishable by a 3-year maximum sentence, and none of them revoking any of Logan's civil rights. Logan challenged his sentence on the ground that his state-court convictions fell within \S~921(a)(20)'s ``civil rights restored'' exemption from ACCA's reach. Rights retained, Logan argued, should be treated the same as rights revoked but later restored. The District Court disagreed, holding that the exemption applies only to defendants whose civil rights were both lost and restored, and the Seventh Circuit affirmed.

\emph{Held:}

  The exemption contained in \S~921(a)(20) does not cover the case of an offender who retained civil rights at all times, and whose legal status, postconviction, remained in all respects unaltered by any state dispensation.Pp. 30--37.

  (a) The ordinary meaning of the word ``restored''---giving back something that has been taken away---does not include retention of something never lost. Moreover, the context in which ``restored'' appears in \S~921(a)(20) counsels adherence to the word's ordinary meaning. In \S~921(a)(20), the words ``civil rights restored'' appear in the company of \newpage  ``expunged,'' ``set aside,'' and ``pardoned.'' Each of those terms describes a measure by which the government relieves an offender of some or all of the consequences of his conviction. In contrast, a defendant who retains rights is simply left alone. He receives no status-altering dispensation, no token of forgiveness from the government.Pp. 30--32.

  (b) Logan's dominant argument against a plain-meaning approach is not persuasive. He relies on the harsh result a literal reading could yield: Unless retention of rights is treated as legally equivalent to restoration of rights, he maintains, less serious offenders will be subject to ACCA's enhanced penalties while more serious offenders in the same State, who have had civil rights restored, may escape heightened punishment. Logan urges that this result is not merely anomalous; it is absurd, particularly in States where restoration of civil rights occurs automatically upon release from prison.P. 32.

  Logan's harsh or absurd consequences argument overlooks \S~921(a) (20)'s ``unless'' clause, under which an offender gains no exemption from ACCA's application through an expungement, set-aside, pardon, or restoration of civil rights if the dispensation ``expressly provides that the [offender] may not ship, transport, possess, or receive firearms.'' Many States that restore felons' civil rights (or accord another measure of forgiveness) nonetheless impose or retain firearms disabilities. Further, Wisconsin no longer punishes misdemeanors by more than two years' imprisonment, and thus no longer has any misdemeanors that qualify as ACCA predicates.Pp. 32--33.

  The resolution Logan proposes, in any event, would correct one potential anomaly while creating others. Under Logan's proposed construction, all crimes, including first-degree murder, would be treated as crimes for which ``civil rights [have been] restored'' in a State that does not revoke any offender's civil rights, while less serious crimes committed elsewhere would not. Accepting Logan's argument would also undercut \S~921(a)(20)(B), which subjects to ACCA state misdemeanor convictions punishable by more than two years' imprisonment. Because misdemeanors generally entail no revocation of civil rights, reading the word ``restored'' to include ``retained'' would yield this curiosity: An offender would fall within ACCA's reach if his three prior offenses carried potential prison terms of over two years, but would be released from ACCA's grip by virtue of his retention of civil rights. This Court is disinclined to say that what Congress imposed with one hand (exposure to ACCA) it withdrew with the other (exemption from ACCA). Even assuming that when Congress revised \S~921(a)(20) in 1986, it labored under the misapprehension that all misdemeanants and felons at least temporarily forfeit civil rights, and indulging the further assumption that courts may repair such a congressional oversight or mistake, \newpage  this Court is not equipped to say what statutory alteration, if any, Congress would have made had its attention trained on offenders who retained civil rights; nor can the Court recast \S~921(a)(20) in Congress' stead.Pp. 33--35.

  Section 922(g)(9)---which was adopted ten years after \S~921(a)(20) was given its current shape and which outlaws possession of a firearm by anyone ``convicted\dots ofa misdemeanor crime of domestic violence''---cautions against any assumption that Congress did not mean to deny the \S~921(a)(20) exemption to offenders who retained their civil rights. Tailored to \S~922(g)(9), Congress adopted a definitional provision, \S~921(a)(33)(B)(ii), corresponding to \S~921(a)(20), which specifies expungement, set-aside, pardon, or restoration of rights as dispensations that can cancel lingering effects of a conviction. That provision also demonstrates that the words ``civil rights restored'' do not cover a person whose civil rights were never taken away. It provides for restoration of civil rights as a qualifying dispensation only ``if the law of the applicable jurisdiction provides for the loss of civil rights'' in the first place. Section 921(a)(33)(B)(ii) also rebuts Logan's absurdity argument. Statutory terms may be interpreted against their literal meaning where the words could not conceivably have been intended to apply to the case at hand.See, \emph{e. g., Green} v. \emph{Bock Laundry Machine Co.,} 490 U.~S. 504, 511. In \S~921(a)(33)(B)(ii), however, Congress explicitly distinguished between ``restored'' and ``retained,'' thereby making it more than conceivable that the Legislature, albeit an earlier one, meant to do the same in \S~921(a)(20).Pp. 35--37.

453 F. 3d 804, affirmed.\newpage 

  \textsc{Ginsburg,} J., delivered the opinion for a unanimous Court.

  \emph{Richard A. Coad} argued the cause for petitioner. With him on the briefs were \emph{Brian T. Fahl} and \emph{Jeffrey T. Green.}

  \emph{Daryl Joseffer} argued the cause for the United States. With him on the brief were \emph{Solicitor General Clement, Assistant Attorney General Fisher, Deputy Solicitor General Dreeben,} and \emph{Joel M. Gershowitz.}[[*]]

\footnotetext[*]{\emph{Stephen P. Halbrook} filed a brief for the National Rifle Association of America, Inc., as \emph{amicus curiae} urging reversal.

\emph{Elliot H. Scherker, Julissa Rodriguez, Peter Goldberger, Mary Price,} and \emph{Barbara E. Bergman} filed a brief for the National Association of Criminal Defense Lawyers et al. as \emph{amici curiae.}}
