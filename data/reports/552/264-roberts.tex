% Dissenting
% Roberts

\setcounter{page}{291}

  \textsc{Chief Justice Roberts,} with whom \textsc{Justice Kennedy} joins, dissenting.

  Some of our new rulings on the meaning of the United States Constitution apply retroactively---to cases already concluded---and some do not. This Court has held that the question whether a particular ruling is retroactive is itself a question of federal law. It is basic that when it comes to any such question of federal law, it is ``the province and duty'' of this Court ``to say what the law is.'' \emph{Marbury} v. \emph{Madison,} 1 Cranch 137, 177 (1803). State courts are the final arbiters of their own state law; this Court is the final arbiter \newpage  of federal law. State courts are therefore bound by our rulings on whether our cases construing federal law are retroactive.

  The majority contravenes these bedrock propositions. The end result is startling: Of two criminal defendants, each of whom committed the same crime, at the same time, whose convictions became final on the same day, and each of whom raised an identical claim at the same time under the Federal Constitution, one may be executed while the other is set free---the first despite being correct on his claim, and the second because of it. That result is contrary to the Supremacy Clause and the Framers' decision to vest in ``one supreme Court'' the responsibility and authority to ensure the uniformity of federal law. Because the Constitution requires us to be more jealous of that responsibility and authority, I respectfully dissent.

\section{I}

  One year after \emph{Teague} v. \emph{Lane,} 489 U.~S. 288 (1989)---our leading modern precedent on retroactivity---\emph{Teague}'s author explained:

      \begin{quote}

		  ``The determination whether a constitutional decision of this Court is retroactive\dots is a matter of federal law. When questions of state law are at issue, state courts generally have the authority to determine the retroac tivity of their own decisions. The retroactive applica bility of a constitutional decision of this Court, however, ‘is every bit as much of a federal question as what particular federal constitutional provisions themselves mean, what they guarantee, and whether they have been denied.' '' \emph{American Trucking Assns., Inc.} v. \emph{Smith,} 496 U.~S. 167, 177--178 (1990) (plurality opinion of O'Con nor, J.) (quoting \emph{Chapman} v. \emph{California,} 386 U.~S. 18, 21 (1967); citation omitted).

      \end{quote}

\noindent For that reason, ``we have consistently required that state courts adhere to our retroactivity decisions.'' \emph{American} \newpage \emph{Trucking, supra,} at 178 (citing \emph{Michigan} v. \emph{Payne,} 412 U. S. 47 (1973), and \emph{Arsenault} v. \emph{Massachusetts,} 393 U.~S. 5 (1968) (\\per curiam\\)). Even more recently, we held that the ``Supremacy Clause does not allow federal retroactivity doctrine to be supplanted by the invocation of a contrary approach to retroactivity under state law.'' \emph{Harper} v. \emph{Virginia Dept. of Taxation,} 509 U.~S. 86, 100 (1993) (citation omitted).

  Indeed, about the only point on which our retroactivity jurisprudence has been consistent is that the retroactivity of new federal rules is a question of federal law binding on States. The Court's contrary holding is based on a misreading of our precedent and a misunderstanding of the nature of retroactivity generally.

\section{A}

  As the Court correctly points out, before 1965 we took for granted the proposition that all federal constitutional rights, including rights that represented a break from earlier precedent, would be given full retroactive effect on both direct and collateral review. That all changed with \emph{Linkletter} v. \emph{Walker,} 381 U.~S. 618 (1965). In that case, a Louisiana prisoner brought a federal habeas petition arguing that illegally seized evidence was introduced against him at trial in violation of \emph{Mapp} v. \emph{Ohio,} 367 U.~S. 643 (1961). \emph{Mapp,} however, had been decided after his conviction became final. We granted certiorari to decide whether the \emph{Mapp} rule ``operates retrospectively upon cases finally decided in the period prior to \emph{Mapp.}'' 381 U. S., at 619--620. In answering this question, we broke from our past practice of assuming full retroactivity, holding that ``we are neither required to apply, nor prohibited from applying, a decision retrospectively.'' \emph{Id.,} at 629. Our analysis turned entirely on the nature and scope of the particular constitutional right at issue: ``[W]e must\dots weigh the merits and demerits [of retroactive application] in each case by looking to the prior history of the rule in question, its purpose and effect, and whether retrospective operation will further or retard its operation.'' \emph{Ibid.} \newpage  Under this framework, we held that \emph{Mapp} would apply only prospectively. 381 U. S., at 639--640.

  The next year, we decided \emph{Johnson} v. \emph{New Jersey,} 384 U. S. 719 (1966). \emph{Johnson} was a direct appeal from the New Jersey Supreme Court's denial of \emph{state} collateral relief. The precise question in \emph{Johnson} was whether the rules announced in \emph{Escobedo} v. \emph{Illinois,} 378 U.~S. 478 (1964), and \emph{Miranda} v. \emph{Arizona,} 384 U.~S. 436 (1966), would apply to state prisoners whose convictions had become final before those cases were decided. In holding that \emph{Escobedo} and \emph{Miranda} should apply only prospectively, 384 U. S., at 732, we imported \emph{Linkletter}'s mode of retroactivity analysis into review of state postconviction proceedings, 384 U. S., at 726--727. Finally, in \emph{Stovall} v. \emph{Denno,} 388 U.~S. 293 (1967), we announced that, for purposes of retroactivity analysis, ``no distinction is justified between convictions now final, as in the instant case, and convictions at various stages of trial and direct review.'' \emph{Id.,} at 300.

  Thus, by 1967, the \emph{Linkletter} analysis was applied in review of criminal convictions, whether final or not. No matter at what stage of proceedings this Court considered a retroactivity question, the issue was decided with reference to the purposes and practical impact of the precise federal right in question: ``Each constitutional rule of criminal procedure has its own distinct functions, its own background of precedent, and its own impact on the administration of justice, and the way in which these factors combine [to decide the retroactivity issue] must inevitably vary with the [constitutional] dictate involved.'' \emph{Johnson, supra,} at 728.

  Because the question of retroactivity was so tied up with the nature and purpose of the underlying federal constitutional right, it would have been surprising if any of our cases had suggested that States were free to apply new rules of federal constitutional law retroactively even when we would not. As one of the more thoughtful legal scholars put it in discussing the effect of the \emph{Linkletter} analysis on state col\newpage lateral review, ``[i]f a state gave relief in such a case on the exclusive authority of \emph{Mapp,} under the rationale of the \emph{Linkletter} opinion it would presumably have to be reversed.'' Mishkin, Foreword: The High Court, The Great Writ, and the Due Process of Time and Law, 79 Harv. L. Rev. 56, 91, n. 132 (1965).

  Our precedents made clear that States could give greater substantive protection under their own laws than was available under federal law, and could give whatever retroactive effect to \emph{those} laws they wished. As the Court explained in \emph{Johnson,} ``[o]f course, States are still entirely free to effectuate under their own law stricter standards than those we have laid down and to apply those standards in a broader range of cases than is required by this decision.'' 384 U. S., at 733. The clear implication of this statement was that States could apply their own retroactivity rules only to new substantive rights ``under their own law,'' not to new federal rules announced by this Court.

  Thus, contrary to the Court's view, our early retroactivity cases nowhere suggested that the retroactivity of new federal constitutional rules of criminal procedure was anything other than ``a matter of federal law.'' \emph{Daniel} v. \emph{Louisiana,} 420 U.~S. 31, 32 (1975) (\\per curiam.\\) It is no surprise, then, that when we held that a particular right would not apply retroactively, the language in our opinions did not indicate that our decisions were optional. See, \emph{e. g., Fuller} v. \emph{Alaska,} 393 U.~S. 80, 81 (1968) (\\per curiam\\) (the rule announced in \emph{Lee} v. \emph{Florida,} 392 U.~S. 378 (1968), ``\\is to be applied} only to trials in which the evidence is sought to be introduced after the date of [that] decision'' (emphasis added)). And, of course, when we found that a state court erred in holding that a particular right should not apply retroactively, the state court was bound to comply. See, \emph{e. g., Kitchens} v. \emph{Smith,} 401 U.~S. 847 (1971) (\\per curiam\\); \emph{McConnell} v. \emph{Rhay,} 393 U.~S. 2, 3--4 (1968) (\\per curiam\\); \emph{Arsenault,} \emph{supra}, at 6.\newpage 

  Although nothing in our decisions suggested that state courts could determine the retroactivity of new federal rules according to their own lights, we had no opportunity to confront the issue head on until \emph{Payne,} 412 U.~S. 47.\footnotemark[1] In \emph{Payne,} the defendant had argued before the Michigan Supreme Court that his resentencing violated the rule we had announced in \emph{North Carolina} v. \emph{Pearce,} 395 U.~S. 711 (1969). In considering this question, the state court noted that this Court had ``not yet decided whether \emph{Pearce} is to be applied retroactively.'' \emph{People} v. \emph{Payne,} 386 Mich. 84, 90, n. 3, 191 N. W. 2d 375, 378, n. 2 (1971). Nevertheless, without so much as citing any federal retroactivity precedent, the court decided that it would ``apply \emph{Pearce} in the present case in order to instruct our trial courts as to the Michigan interpretation of an ambiguous portion of \emph{Pearce} .~.~.~, pending clarification by the United States Supreme Court.'' \emph{Id.,} at 91, n. 3, 191 N. W. 2d, at 378, n. 2.

  We granted certiorari in \emph{Payne} only on the question of retroactivity, and decided that \emph{Pearce} should not apply retroactively. In reversing the contrary decision of the state court, our language was not equivocal: ``Since the resentencing hearing in this case took place approximately two years before \emph{Pearce} was decided, we hold that the Michigan Supreme Court erred in applying its proscriptions here.'' 412 U. S., at 57.

  The majority argues that \emph{Payne} did not preclude States from applying retroactivity rules different from those we announced; rather, the argument goes, the Michigan Supreme Court simply elected to follow the federal retroactivity rule, ``pending clarification.'' See \emph{ante,} at 282--284. That is certainly a possible reading of \emph{Payne,} but not the most plausible one. The Michigan Supreme Court did not purport to rest its decision to apply \emph{Pearce} retroactively on the federal \newpage  \emph{Linkletter} analysis, and this Court's reversal is most reasonably read as \emph{requiring} state courts to apply our federal retroactivity decisions. Notably, this is not the first time Members of this Court have debated the meaning of \emph{Payne,} with \emph{Teague}'s author explaining that \emph{Payne} supports the proposition that ``we have consistently required that state courts adhere to our retroactivity decisions,'' \emph{American Trucking,} 496 U. S., at 178 (plurality opinion of O'Connor, J.), and the author of today's opinion disagreeing in dissent, see \emph{id.,} at 210, n. 4 (opinion of \textsc{Stevens,} J.). But whichever way \emph{Payne} is read, it either offers no support for the majority's position, because the state court simply applied federal retroactivity rules, or flatly rejects the majority's position, because the state court failed to apply federal retroactivity rules, and was told by this Court that it must.

\footnotetext[1]{\emph{Payne} came to us on direct appeal, but as noted, \emph{supra,} at 294, we did not at the time distinguish between direct appeal and collateral review for purposes of retroactivity.}

  Meanwhile, Justice Harlan had begun dissenting in our retroactivity cases, pressing the view that new rules announced by the Court should be applied in all cases not yet final, without regard to the analysis set forth in \emph{Linkletter.} See \emph{Desist} v. \emph{United States,} 394 U. S. 244, 256--269 (1969); \emph{Mackey} v. \emph{United States,} 401 U.~S. 667, 675--702 (1971) (opinion concurring in judgments in part and dissenting in part). In \emph{Griffith} v. \emph{Kentucky,} 479 U. S. 314 (1987), we abandoned \emph{Linkletter} as it applied to cases still on direct review and adopted Justice Harlan's view in such cases. Noting that nonretroactivity on direct appeal ``violates basic norms of constitutional adjudication'' and that ``selective application of new rules violates the principle of treating similarly situated defendants the same,'' 479 U. S., at 322, 323, we held that ``a new rule for the conduct of criminal prosecutions is to be applied retroactively to all cases, \emph{state or federal,} pending on direct review or not yet final,'' \emph{id.,} at 328 (emphasis added). Just as in previous cases, \emph{Griffith} by its terms bound state courts to apply our retroactivity decisions.

  Two months after \emph{Griffith} was decided, we granted certiorari in \emph{Yates} v. \emph{Aiken,} 484 U.~S. 211 (1988). In that case, a \newpage  South Carolina state habeas court had decided that our decision in \emph{Francis} v. \emph{Franklin,} 471 U.~S. 307 (1985), should not be applied retroactively. If the authority of state courts to apply their own retroactivity rules were well established under our precedents---as the majority would have it, see \emph{ante,} at 275--282---this case should have been easily decided on the ground that whatever the federal retroactivity rule, the State could adopt its own rule on the retroactivity of newly announced federal constitutional standards.

  Instead, the State argued to this Court ``that we should adopt Justice Harlan's theory that a newly announced constitutional rule should not be applied retroactively to cases pending on collateral review unless'' the rule meets certain criteria---the flip side of Justice Harlan's view about cases on direct review that we had accepted in \emph{Griffith.} 484 U. S., at 215. Under that approach, the State argued, \emph{Francis} would not be applied retroactively on collateral review. 484 U. S., at 215. In response, we discussed Justice Harlan's ``distinction between direct review and collateral review.'' \emph{Ibid.} We found, however, that it was ``not necessary to determine whether we should\dots adopt Justice Harlan's reasoning as to the retroactivity of cases announcing new constitutional rules to cases pending on collateral review,'' \emph{id.,} at 215--216, because \emph{Francis} did not announce a new rule.

  This Court went on, however, to address South Carolina's alternative argument---that it ``has the authority to establish the scope of its own habeas corpus proceedings,'' which would allow it in the case before the Court ``to refuse to apply a new rule of federal constitutional law retroactively in such a proceeding.'' 484 U. S., at 217. This argument should sound familiar---whatever the federal retroactivity rule, a State may establish its own retroactivity rule for its own collateral proceedings. This Court rejected that proposition, not only because it did not regard \emph{Francis} as a new rule, but also because the state court did not ``plac[e] any limit on the issues that it will entertain in collateral proceed\newpage ings.'' 484 U. S., at 218. As this Court explained, if the state court ``consider[s] the merits of the federal claim, it has a duty to grant the relief that \emph{federal law} requires.'' \emph{Ibid.} (emphasis added).

  Given all this, the present case should come out the way it does only if \emph{Teague} changed the nature of retroactivity as a creature of federal law binding on the States, and adopted the argument rejected in \emph{Yates}---that when it comes to retroactivity, a State ``has the authority to establish the scope of its own habeas corpus proceedings.'' \emph{Teague} did no such thing.

\section{B}

  In \emph{Teague,} we completed the project of conforming our view on the retroactivity of new rules of criminal procedure to those of Justice Harlan. Justice O'Connor's plurality opinion posed the problem by noting, with more than a bit of understatement, that the ``\emph{Linkletter} retroactivity standard has not led to consistent results.'' 489 U. S., at 302. In light of these concerns, and because of ``‘the important distinction between direct review and collateral review,' '' \emph{id.,} at 307 (quoting \emph{Yates, supra,} at 215), we generally adopted Justice Harlan's approach to retroactivity on collateral review, 489 U. S., at 310, just as we had previously adopted his approach on direct review in \emph{Griffith.}

  The \emph{Linkletter} approach to retroactivity was thus overruled in favor of the Harlan approach in two steps: \emph{Griffith} and \emph{Teague.} There is no dispute that \emph{Griffith} is fully binding on States; a new rule ``is to be applied retroactively to all cases, \emph{state or federal,} pending on direct review or not yet final.'' 479 U. S., at 328 (emphasis added). \emph{Teague} is simply the other side of the coin, and it too should be binding in ``all cases, state or federal.'' The fact that \emph{Linkletter} was overruled in two stages rather than one should not lead to a different result.

  Indeed, \emph{Teague} did not purport to distinguish between federal and state collateral review. Justice O'Connor's opinion \newpage noted that ``in \emph{Yates} v. \emph{Aiken,} we were asked to decide whether the rule announced in \emph{Francis} v. \emph{Franklin} should be applied to a defendant on collateral review at the time that case was decided,'' but that we were able to decide the case on alternative grounds. 489 U. S., at 307 (citations omitted). This citation of \emph{Yates}---a state habeas case---makes clear that \emph{Teague} contemplated no difference between retroactivity of new federal rules in state and federal collateral proceedings. Thus, our unqualified holding---that ``[u]nless they fall within an exception to the general rule, new constitutional rules of criminal procedure will not be applicable to those cases which have become final before the new rules are announced,'' 489 U. S., at 310 (plurality opinion)---is enough to decide this case.

  Moreover, the reasons the \emph{Teague} Court provided for adopting Justice Harlan's view apply to state as well as federal collateral review. The majority is quite right that \emph{Teague} invoked the interest in comity between the state and federal sovereigns. \emph{Id.,} at 308. But contrary to the impression conveyed by the majority, there was more to \emph{Teague} than that. \emph{Teague} also relied on the interest in finality: ``Application of constitutional rules not in existence at the time a conviction became final seriously undermines the principle of finality which is essential to the operation of our criminal justice system. Without finality, the criminal law is deprived of much of its deterrent effect.'' \emph{Id.,} at 309. The Court responds by flatly stating that ``finality of state convictions is a \emph{state} interest, not a federal one.'' \emph{Ante,} at 280. But while it is certainly true that finality of state convictions is a state interest, that does mean it is not also a federal one. After all, our decision in \emph{Griffith} made finality the touchstone for retroactivity of new federal rules, and bound States to that judgment. See 479 U. S., at 328 (new rules are ``to be applied retroactively to all cases, \emph{state or federal,} pending on direct review or not yet final'' (emphasis added)).\newpage 

  It is quite a radical proposition to assert that this Court has nothing to say about an interest ``essential to the operation of our criminal justice system,'' without which ``the criminal law is deprived of much of its deterrent effect,'' when the question is whether this interest is being undermined by the very rules of \emph{federal} constitutional procedure that we are charged with expounding. A State alone may ``evaluate, and weigh the importance of'' finality interests, \emph{ante,} at 280, when it decides which substantive rules of criminal procedure \emph{state law} affords; it is quite a leap to hold, as the Court does, that they alone can do so in the name of the Federal Constitution.

  \emph{Teague} was also based on the inequity of the \emph{Linkletter} approach to retroactivity. After noting that the disparate treatment of similarly situated defendants led us in \emph{Griffith} to adopt Justice Harlan's view for cases on direct appeal, the Court then explained that the ``\emph{Linkletter} standard also led to unfortunate disparity in the treatment of similarly situated defendants on collateral review.'' 489 U. S., at 305 (plurality opinion). See also \emph{id.,} at 316 (the Court's new approach to retroactivity ``avoids the inequity resulting from the uneven application of new rules to similarly situated defendants'').

  This interest in reducing the inequity of haphazard retroactivity standards and disuniformity in the application of federal law is quite plainly a predominantly federal interest. Indeed, it was one of the main reasons we cited in \emph{Griffith} for imposing a uniform rule of retroactivity upon \emph{state} courts for cases on direct appeal. And, more to the point, it is the very interest that animates the Supremacy Clause and our role as the ``one supreme Court'' charged with enforcing it.

  Justice Story, writing for the Court, noted nearly two centuries ago that the Constitution requires ``\emph{uniformity} of decisions throughout the whole United States, upon all subjects within [its] purview.'' \emph{Martin} v. \emph{Hunter's Lessee,} 1 Wheat. 304, 347--348 (1816). Indeed, the ``fundamental principle'' of \newpage  our Constitution, as Justice O'Connor once put it, is ``that a single sovereign's laws should be applied equally to all.'' Our Judicial Federalism, 35 Case W. Res. L. Rev. 1, 4 (1984--1985). States are free to announce their own state-law rules of criminal procedure, and to apply them retroactively in whatever manner they like. That is fully consistent with the principle that ``a single sovereign's laws should be applied equally to all.'' But the Court's opinion invites just the sort of disuniformity in federal law that the Supremacy Clause was meant to prevent. The same determination of a federal constitutional violation at the same stage in the criminal process can result in freedom in one State and loss of liberty or life in a neighboring State.\footnotemark[2] The Court's opinion allows ``a single sovereign's law''---the Federal Constitution, as interpreted by this Court---to be applied differently in every one of the several States.

  Finally, from \emph{Linkletter} through \emph{Johnson} to \emph{Teague,} we have always emphasized that determining whether a new federal right is retroactive turns on the nature of the substantive federal rule at issue. See \emph{Linkletter,} 381 U. S., at \newpage  629 (in deciding retroactivity, we ``loo[k] to the prior history of the rule in question, its purpose and effect, and whether retrospective operation will further or retard its operation''); \emph{Johnson,} 384 U. S., at 728 (``Each constitutional rule of criminal procedure has its own distinct functions, its own background of precedent, and its own impact on the administration of justice, and the way in which these factors combine [to decide the retroactivity issue] must inevitably vary with the dictate involved''); \emph{Teague, supra,} at 311--315 (plurality opinion) (deciding whether rule is applicable to cases on collateral review turns on whether the rule ``places ‘certain kinds of primary, private individual conduct beyond the power of the criminal law-making authority to proscribe,' '' and whether the rule is an ``absolute prerequisite to fundamental fairness that is ‘implicit in the concept of ordered liberty' ''). That is how we determine retroactivity---by carefully examining the underlying federal right. See, \emph{e. g., Whorton} v. \emph{Bockting,} 549 U.~S. 406, 418--421 (2007); \emph{Schriro} v. \emph{Summerlin,} 542 U.~S. 348, 353--354 (2004); \emph{Sawyer} v. \emph{Smith,} 497 U.~S. 227, 243--245 (1990); \emph{Penry} v. \emph{Lynaugh,} 492 U.~S. 302, 318--319 (1989).

\footnotetext[2]{The Court points out that the defendants in such a case are differently situated because they violated the laws of and were tried in different States. \emph{Ante,} at 290. But disparate treatment under substantively different state laws is something we expect in our federal system; disparate treatment under the same Federal Constitution is quite a different matter.}

  ^ The majority also points out that the rule announced in \emph{Griffith} v. \emph{Kentucky,} 479 U.~S. 314 (1987)---that full retroactive application ends with the conclusion of direct appeal---creates its own disuniformity, because finality turns on how quickly a State brings its direct appeals to a close. \emph{Ante,} at 291. The same point was raised by the \emph{Griffith} dissenters, 479 U. S., at 331--332 (opinion of White, J.), and rejected as pertinent by the majority in that case, \emph{id.,} at 327--328. The disuniformity that the majority emphasizes today and the dissenters emphasized in \emph{Griffith} is a necessary consequence of our having chosen a relatively clear rule---finality---to delineate the line between full retroactivity and presumptive nonretroactivity. The relevant point is that whatever inequity arises from the \emph{Griffith} rule, it is based on a balancing of costs and benefits that \emph{this} Court---not 50 different sovereigns---has performed.

  When this Court decides that a particular right shall not be applied retroactively, but a state court finds that it should, it is at least in part because of a different assessment by the state court of the nature of the underlying federal right---something on which the Constitution gives this Court the final say. The nature and scope of the new rules we announce directly determines whether they will be applied retroactively on collateral review. Today's opinion stands for the unfounded proposition that while we alone have the final say in expounding the former, we have no control over the latter.

\section{II}

  The Court's holding is not only based on a misreading of our retroactivity cases, but also on a misunderstanding of the nature of retroactivity generally. The majority's decision is \newpage  grounded on the erroneous view that retroactivity is a remedial question. See \emph{ante,} at 290--291 (``It is important to keep in mind that our jurisprudence concerning the ‘retroactivity' of ‘new rules' of constitutional law is primarily concerned, not with the question whether a constitutional violation occurred, but with the availability or nonavailability of remedies''). But as explained in the lead opinion in \emph{American Trucking\\---penned by the author of the lead opinion in \emph{Teague}---it is an ``error'' to ``equat[e] a decision not to apply a rule retroactively with the judicial choice of a remedy.'' 496 U. S., at 194 (plurality opinion of O'Connor, J.). As Justice O'Connor went on to emphasize, ``[n]or do this Court's retroactivity decisions, whether in the civil or criminal sphere, support the\dots assertion that our retroactivity doctrine is a remedial principle.'' \emph{Ibid.} ``While application of the principles of retroactivity may have remedial effects, they are not themselves remedial principles~.~.~.~. A decision defining the operative conduct or events that will be adjudicated under old law does not, in itself, specify an appropriate remedy.'' \emph{Id.,} at 195. See also \emph{Lemon} v. \emph{Kurtzman,} 411 U.~S. 192, 199 (1973) (plurality opinion) (describing the question of retroactivity as ``whether we will apply a new constitutional rule of criminal law in reviewing judgments of conviction obtained under a prior standard,'' and contrasting this with the question of the ``appropriate scope of federal equitable remedies'').

  In other words, when we ask whether and to what extent a rule will be retroactively applied, we are asking what law---new or old---will apply. As we have expressly noted, ``[t]he \emph{Teague} doctrine .~.~. does not involve a special ‘remedial' limitation on the principle of ‘retroactivity' as much as it reflects a limitation inherent in the principle itself.'' \emph{Reynoldsville Casket Co.} v. \emph{Hyde,} 514 U.~S. 749, 758 (1995).

  The foregoing prompts a lengthy rejoinder from the Court, to the effect that it is wrong to view retroactivity as a federal choice-of-law question rather than a remedial one. \newpage  That view, we are told, was rejected by five Justices in \emph{American Trucking} and then by the Court in \emph{Harper. Ante,} at 284--288. But the proposition on which five Members of the Court agreed in \emph{American Trucking,} and that the Court adopted in \emph{Harper,} was that the \emph{Griffith} rule of retroactivity---that is, that newly announced constitutional decisions should apply to all cases on direct review---should apply to civil cases as well as criminal. See \emph{American Trucking,} 496 U.S., at 201 (\textsc{Scalia,} J., concurring in judgment) (``I share \textsc{Justice Stevens}' perception that prospective decisionmaking is incompatible with the judicial role, which is to say what the law is, not to prescribe what it shall be''); \emph{id.,} at 212 (\textsc{Stevens,} J., dissenting) (``Fundamental notions of fairness and legal process dictate that the same rules should be applied to all similar cases on direct review''); \emph{Harper,} 509 U. S., at 97 (``When this Court applies a rule of federal law to the parties before it, that rule is the controlling interpretation of federal law and must be given full retroactive effect in all cases still open on direct review'').

  Neither \textsc{Justice Scalia}'s concurrence in \emph{American} \emph{Trucking} combined with the dissent, nor the Court's opinion in \emph{Harper,} resolved that retroactivity was a remedial question. That is why, the year after \emph{American Trucking} was decided, two of the Justices in today's majority could explain:

    \begin{quote}

		\noindent ``Since the question is whether the court should apply the old rule or the new one, retroactivity is properly seen in the first instance as a matter of choice of law, ‘a choice\dots between the principle of forward operation and that of relation backward.' \emph{Great Northern R. Co.} v. \emph{Sunburst Oil \& Refining Co.,} 287 U.~S. 358, 364 (1932). Once a rule is found to apply ‘backward,' there may then be a further issue of remedies, \emph{i. e.,} whether the party prevailing under a new rule should obtain the same re lief that would have been awarded if the rule had been an old one. Subject to possible constitutional thresh olds, the remedial inquiry is one governed by state law, \newpage  at least where the case originates in state court. See \emph{American Trucking Assns., Inc.} v. \emph{Smith,} 496 U.~S. 167, 210 (1990) (\textsc{Stevens,} J., dissenting). \emph{But the antecedent choice-of-law question is a federal one where the rule at issue itself derives from federal law, constitutional or otherwise.} See \emph{Smith, supra,} at 177--178 (plurality opinion).'' \emph{James B. Beam Distilling Co.} v. \emph{Georgia,} 501 U.~S. 529, 534--535 (1991) (opinion of \textsc{Souter,} J., joined by \textsc{Stevens,} J.) (citation omitted; em phasis added).

    \end{quote}

  And \emph{Harper} certainly did not view the retroactivity of federal rules as a remedial question for state courts. Quite the contrary: \emph{Harper} held that the ``Supremacy Clause does not allow federal retroactivity doctrine to be supplanted by the invocation of a contrary approach to retroactivity under state law,'' 509 U. S., at 100 (citation omitted), and expressly treated retroactivity and remedy as separate questions, \emph{id.,} at 100--102.

  The majority explains that when we announce a new rule of law, we are not ``‘\emph{creating} the law,' '' but rather `` ‘\emph{declaring} what the law already is.' '' \emph{Ante,} at 286 (quoting \emph{American Trucking, supra,} at 201 (\textsc{Scalia,} J., concurring in judgment)). But this has nothing to do with the question before us. The point may lead to the conclusion that nonretroactivity of our decisions is improper---the position the Court has adopted in both criminal and civil cases on direct review---but everyone agrees that full retroactivity is not required on collateral review. It necessarily follows that we must choose whether ``new'' or ``old'' law applies to a particular category of cases. Suppose, for example, that a defendant, whose conviction became final before we announced our decision in \emph{Crawford} v. \emph{Washington,} 541 U.~S. 36 (2004), argues (correctly) on collateral review that he was convicted in violation of both \emph{Crawford} and \emph{Ohio} v. \emph{Roberts,} 448 U.S. 56 (1980), the case that \emph{Crawford} overruled. Under our decision in \emph{Whorton,} 549 U.~S. 406, the ``new'' rule announced in \newpage \emph{Crawford} would not apply retroactively to the defendant. But I take it to be uncontroversial that the defendant would nevertheless get the benefit of the ``old'' rule of \emph{Roberts,} even under the view that the rule not only is but always has been an incorrect reading of the Constitution. See, \emph{e. g., Yates,} 484 U. S., at 218. Thus, the question whether a particular federal rule will apply retroactively is, in a very real way, a choice between new and old law. The issue in this case is who should decide.

  The proposition that the question of retroactivity---that is, the choice between new or old law in a particular case---is distinct from the question of remedies has several important implications for this case. To begin with, whatever intuitive appeal may lie in the majority's statement that ``the remedy a state court chooses to provide its citizens for violations of the Federal Constitution is primarily a question of state law,'' \emph{ante,} at 288, the statement misses the mark. The relevant inquiry is not about remedy; it is about choice of law---new or old. There is no reason to believe, either legally or intuitively, that States should have any authority over this question when it comes to which \emph{federal} constitutional rules of criminal procedure to apply.\footnotemark[3]

  Indeed, when the question is what federal rule of decision from this Court should apply to a particular case, no Court but this one---which has the ultimate authority ``to say what the law is,'' \emph{Marbury,} 1 Cranch, at 177---should have final say over the answer. See \emph{Harper, supra,} at 100 (``Supremacy Clause does not allow federal retroactivity doctrine to \newpage  be supplanted by the invocation of a contrary approach to retroactivity under state law'' (citation omitted)). This is enough to rebut the proposition that there is no ``source of [our] authority'' to bind state courts to follow our retroactivity decisions. \emph{Ante,} at 290. Retroactivity is a question of federal law, and our final authority to construe it cannot, at this point in the Nation's history, be reasonably doubted.

\footnotetext[3]{A federal court applying state law under \emph{Erie R. Co.} v. \emph{Tompkins,} 304 U.~S. 64 (1938), follows state choice-of-law rules as well, see \emph{Klaxon Co.} v. \emph{Stentor Elec. Mfg. Co.,} 313 U. S. 487, 496 (1941). It is not free to follow its own federal rule simply because the issue arises in federal court. By the same token, a state court considering a federal constitutional claim on collateral review should follow the federal rule on whether new or old law applies. It is not free to follow its own state-law view on the question simply because the issue arises in state court.}

  Principles of federalism protect the prerogative of States to extend greater rights under their own laws than are available under federal law. The question here, however, is the availability of protection under the Federal Constitution---specifically, the Confrontation Clause of the Sixth Amendment. It is no intrusion on the prerogatives of the States to recognize that it is for this Court to decide such a question of federal law, and that our decision is binding on the States under the Supremacy Clause.

  Consider the flip side of the question before us today: If a State interprets its own constitution to provide protection beyond that available under the Federal Constitution, and has ruled that this interpretation is not retroactive, no one would suppose that a federal court could hold otherwise, and grant relief under state law that a state court would refuse to grant. The result should be the same when a state court is asked to give retroactive effect to a right under the Federal Constitution that this Court has held is not retroactive.

  The distinction between retroactivity and available remedies highlights the fact that the majority's assertion ``that \emph{Teague}'s general rule of nonretroactivity was an exercise of this Court's power to interpret the federal habeas statute,'' \emph{ante,} at 278---even if correct---is neither here nor there.\footnotemark[4] \newpage While Congress has substantial control over federal courts' ability to grant relief for violations of the Federal Constitution, the Constitution gives us the responsibility to decide what its provisions mean. And with that responsibility necessarily comes the authority to determine the scope of those provisions---when they apply and when they do not.

\footnotetext[4]{The majority's assertion, however, is a bit of an overstatement. \emph{Teague} v. \emph{Lane,} 489 U.~S. 288 (1989), would be an odd form of statutory interpretation; 28 U.~S.~C. \S~2254 is cited once in passing, 489 U. S., at 298, and \S~2243---the statute that the Court believes \emph{Teague} was interpreting---is not cited at all. As support for its proposition, the Court cites several cases having nothing to do with retroactivity, and numerous concurring \newpage  and dissenting opinions that did not command a majority. See \emph{ante,} at 278, and n. 15.}

  This proposition---and the importance of the distinction between retroactivity and available remedies---were confirmed when we considered the availability of federal collateral review of state convictions under the Antiterrorism and Effective Death Penalty Act of 1996 (AEDPA). See 28 U.~S.~C. \S~2254(d)(1). Whatever control Congress has over federal courts' ability to grant postconviction remedies, the availability or scope of those remedies has no bearing on our decisions about whether new or old law should apply in a particular case. That is why, after AEDPA's passage, we view the \emph{Teague} inquiry as distinct from that under AEDPA. See \emph{Horn} v. \emph{Banks,} 536 U.~S. 266, 272 (2002) (\\per curiam\\) (``While it is of course a necessary prerequisite to federal habeas relief that a prisoner satisfy the AEDPA standard of review set forth in 28 U.~S.~C. \S~2254(d)~.~.~.~, none of our post-AEDPA cases have suggested that a writ of habeas corpus should automatically issue if a prisoner satisfies the AEDPA standard, or that AEDPA relieves courts from the responsibility of addressing properly raised \emph{Teague} arguments''). The majority today views the issue as simply one of what remedies a State chooses to apply; our cases confirm that the question whether a federal decision is retroactive is one of federal law distinct from the issue of available remedies.

  Lurking behind today's decision is of course the question of just how free state courts are to define the retroactivity of our decisions interpreting the Federal Constitution. I do not see any basis in the majority's logic for concluding that \newpage  States are free to hold our decisions retroactive when we have held they are not, but not free to hold that they are not when we have held they are. Under the majority's reasoning, in either case the availability of relief in state court is a question for those courts to evaluate independently. The majority carefully reserves that question, see \emph{ante,} at 269, n. 4, confirming that the majority regards it as open.

  Nor is there anything in today's decision suggesting that States could not adopt more nuanced approaches to retroactivity. For example, suppose we hold that the Sixth Amendment right to be represented by particular counsel of choice, recently announced in \emph{United States} v. \emph{Gonzalez-Lopez,} 548 U.~S. 140 (2006), is a new rule that does not apply retroactively. Under the majority's rationale, a state court could decide that it nonetheless will apply \emph{Gonzalez-Lopez} retroactively, but only if the defendant could prove prejudice, or some other criterion we had rejected as irrelevant in defining the substantive right. Under the majority's logic, that would not be a misapplication of our decision in \emph{GonzalezLopez}---which specifically rejected any required showing of prejudice, \emph{id.,} at 147--148---but simply a state decision on the scope of available remedies in state court. The possible permutations---from State to State, and federal right to federal right---are endless.

\hrule

  Perhaps all this will be dismissed as fine parsing of somewhat arcane precedents, over which reasonable judges may disagree. Fair enough; but I would hope that enough has been said at least to refute the majority's assertion that its conclusion is dictated by our prior cases. This dissent is compelled not simply by disagreement over how to read those cases, but by the fundamental issues at stake---our role under the Constitution as the final arbiter of federal law, both as to its meaning and its reach, and the accompanying duty to ensure the uniformity of that federal law.\newpage 

  Stephen Danforth's conviction became final before the new rule in \emph{Crawford} was announced. In \emph{Whorton} v. \emph{Bockting,} 549 U.~S. 406, we held that \emph{Crawford} shall not be applied retroactively on collateral review. That should be the end of the matter. I respectfully dissent.
