% Opinion of the Court
% Ginsburg

\setcounter{page}{349}

  \textsc{Justice Ginsburg} delivered the opinion of the Court.

  As this Court recognized in \emph{Southland Corp.} v. \emph{Keating,}
465 U. S. 1 (1984), the Federal Arbitration Act (FAA or Act), 9
U.S.C. \S1 \emph{et seq.} (2000 ed. and Supp. V), establishes a national
policy favoring arbitration when the parties contract for that mode
of dispute resolution. The Act, which rests on Congress' authority
under the Commerce Clause, supplies not simply a procedural framework
applicable in federal courts; it also calls for the application, in
state as well as federal courts, of federal substantive law regarding
arbitration. 465 U. S., at 16. More recently, in \emph{Buckeye Check
Cashing, Inc.} v. \emph{Cardegna,} 546 U.~S. 440 (2006), the Court
clarified that, when parties agree to arbitrate all disputes arising
under their contract, questions concerning the validity of the entire
contract are to be resolved by the arbitrator in the first instance, not
by a federal or state court.

  The instant petition presents the following question: Does the
FAA override not only state statutes that refer certain state-law
controversies initially to a judicial forum, but also state statutes
that refer certain disputes initially to an administrative agency? We
hold today that, when parties agree to arbitrate all questions arising
under a contract, state laws lodging primary jurisdiction in another
forum, \newpage  whether judicial or administrative, are superseded by the
FAA.

\section{I}

  This case concerns a contract between respondent Alex E. Ferrer, a
former Florida trial court judge who currently appears as ``Judge
Alex'' on a Fox television network program, and petitioner Arnold
M. Preston, a California attorney who renders services to persons
in the entertainment industry. Seeking fees allegedly due under the
contract, Preston invoked the parties' agreement to arbitrate ``any
dispute\dots relating to the terms of [the contract] or the breach,
validity, or legality thereof\dots in accordance with the rules [of
the American Arbitration Association].'' App. 18.

  Preston's demand for arbitration, made in June 2005, was countered a
month later by Ferrer's petition to the California Labor Commissioner
charging that the contract was invalid and unenforceable under the
California Talent Agencies Act (TAA), Cal. Lab. Code Ann. \S~1700
\emph{et seq.} (West 2003 and Supp. 2008). Ferrer asserted that Preston
acted as a talent agent without the license required by the TAA, and
that Preston's unlicensed status rendered the entire contract void.
\footnotemark[1]

  The Labor Commissioner's hearing officer, in November 2005,
determined that Ferrer had stated a ``colorable basis for exercise of
the Labor Commissioner's jurisdiction.'' App. 33. The officer
denied Ferrer's motion to stay the arbitration, however, on the ground
that the Labor Commissioner lacked authority to order such relief.
Ferrer then filed suit in the Los Angeles Superior Court, seeking a
declaration that the controversy between the parties ``arising from
the [c]ontract, including in particular the issue of the validity of
the [c]ontract, is not subject to arbitration.'' \emph{Id.,} \newpage  at
29. As interim relief, Ferrer sought an injunction restraining Preston
from proceeding before the arbitrator. Preston responded by moving to
compel arbitration.

^1 The TAA uses the term ``talent agency'' to describe both
corporations and individual talent agents. We use the terms ``talent
agent'' and ``talent agency'' interchangeably.

  In December 2005, the Superior Court denied Preston's motion to
compel arbitration and enjoined Preston from proceeding before the
arbitrator ``unless and until the Labor Commissioner determines that
.~.~. she is without jurisdiction over the disputes between Preston
and Ferrer.'' No. BC342454 (Dec. 7, 2005), App. C to Pet. for Cert.
18a, 26a--27a. During the pendency of Preston's appeal from the
Superior Court's decision, this Court reaffirmed, in \emph{Buckeye,}
that challenges to the validity of a contract providing for arbitration
ordinarily ``should\dots be considered by an arbitrator, not a
court.'' 546 U. S., at 446.

  In a 2-to-1 decision issued in November 2006, the California Court
of Appeal affirmed the Superior Court's judgment. The appeals
court held that the relevant provision of the TAA, Cal. Lab. Code
Ann. \S~1700.44(a) (West 2003), vests ``exclusive original
jurisdiction'' over the dispute in the Labor Commissioner. 145
Cal. App. 4th 440, 447, 51 Cal. Rptr. 3d 628, 634. \emph{Buckeye} is
``inapposite,'' the court said, because that case ``did not involve
an administrative agency with exclusive jurisdiction over a disputed
issue.'' 145 Cal. App. 4th, at 447, 51 Cal. Rptr. 3d, at 634. The
dissenting judge, in contrast, viewed \emph{Buckeye} as controlling; she
reasoned that the FAA called for immediate recognition and enforcement
of the parties' agreement to arbitrate and afforded no basis for
distinguishing prior resort to a state administrative agency from prior
resort to a state court. 145 Cal. App. 4th, at 450--451, 51 Cal.
Rptr. 3d, at 636--637 (Vogel, J., dissenting).

  The California Supreme Court denied Preston's petition for review.
No. S149190 (Feb. 14, 2007), 2007 Cal. LEXIS 1539, App. A to Pet.
for Cert. 1a. We granted certiorari to determine whether the FAA
overrides a state law vesting \newpage  initial adjudicatory authority in
an administrative agency. 551 U.~S. 1190 (2007).

\section{II}

  An easily stated question underlies this controversy. Ferrer claims
that Preston was a talent agent who operated without a license in
violation of the TAA. Accordingly, he urges, the contract between the
parties, purportedly for ``personal management,'' is void, and Preston
is entitled to no compensation for any services he rendered. Preston, on
the other hand, maintains that he acted as a personal manager, not as a
talent agent, hence his contract with Ferrer is not governed by the TAA
and is both lawful and fully binding on the parties.

  Because the contract between Ferrer and Preston provides that ``any
dispute\dots relating to the\dots validity, or legality,''
of the agreement ``shall be submitted to arbitration,'' App. 18,
Preston urges that Ferrer must litigate ``his TAA defense in the
arbitral forum,'' Reply Brief 31. Ferrer insists, however, that the
``personal manager'' or ``talent agent'' inquiry falls, under
California law, within the exclusive original jurisdiction of the Labor
Commissioner, and that the FAA does not displace the Commissioner's
primary jurisdiction. Brief for Respondent 14, 30, 40--44.

  The dispositive issue, then, contrary to Ferrer's suggestion, is not
whether the FAA preempts the TAA wholesale. See \emph{id.,} at 44--48.
The FAA plainly has no such destructive aim or effect. Instead, the
question is simply who decides whether Preston acted as personal manager
or as talent agent.

\section{III}

  Section 2 of the FAA states:

      ``A written provision in any\dots contract evidencing
    a transaction involving commerce to settle by arbitration a
    controversy thereafter arising out of such contract or transaction
   \dots shall be valid, irrevocable, and enforce\newpage able, save
    upon such grounds as exist at law or in equity for the revocation of
    any contract.'' 9 U.~S.~C. \S~2.

\noindent Section 2 ``declare[s] a national policy favoring arbitration'' of
claims that parties contract to settle in that manner. \emph{Southland
Corp.,} 465 U. S., at 10. That national policy, we held in
\emph{Southland,} ``appli[es] in state as well as federal courts''
and ``foreclose[s] state legislative attempts to undercut the
enforceability of arbitration agreements.'' \emph{Id.,} at 16.
The FAA's displacement of conflicting state law is ``now
well-established,'' \emph{Allied-Bruce Terminix Cos.} v. \emph{Dobson,} 513
U.~S. 265, 272 (1995), and has been repeatedly reaffirmed, see,
\emph{e. g., Buckeye,} 546 U. S., at 445--446; \emph{Doctor's Associates,}
\emph{Inc.} v. \emph{Casarotto,} 517 U.~S. 681, 684--685 (1996); \emph{Perry}
v. \emph{Thomas,} 482 U.~S. 483, 489 (1987).\footnotemark[2]

  A recurring question under \S~2 is who should decide whether
``grounds\dots exist at law or in equity'' to invalidate an
arbitration agreement. In \emph{Prima Paint Corp.} v. \emph{Flood \&}
\emph{Conklin Mfg. Co.,} 388 U.~S. 395, 403--404 (1967), we held that
attacks on the validity of an entire contract, as distinct from attacks
aimed at the arbitration clause, are within the arbitrator's ken.

  The litigation in \emph{Prima Paint} originated in federal court, but the
same rule, we held in \emph{Buckeye,} applies in state court. 546 U. S.,
at 447--448. The plaintiffs in \emph{Buckeye} alleged that the contracts
they signed, which contained arbitration clauses, were illegal under
state law and void \emph{ab initio. Id.,} at 443. Relying on \emph{Southland,}
we held that the plaintiffs' challenge was within the province of the
arbitrator to decide. See 546 U. S., at 446. \newpage 

^2 Although Ferrer urges us to overrule \emph{Southland,} he relies on the
same arguments we considered and rejected in \emph{Allied-Bruce Terminix
Cos.} v. \emph{Dobson,} 513 U.~S. 265 (1995). Compare Brief for
Respondent 55--59 with Brief for Attorney General of Alabama et al. as
\emph{Amici Curiae} in \emph{Allied-Bruce Terminix Cos.} v. \emph{Dobson,} O. T.
1994, No. 93--1001, pp. 11--19. Adhering to precedent, we do not
take up Ferrer's invitation to overrule \emph{Southland.}

  \emph{Buckeye} largely, if not entirely, resolves the dispute before
us. The contract between Preston and Ferrer clearly ``evidenc[ed] a
transaction involving commerce,'' 9 U.~S.~C. \S~2, and Ferrer
has never disputed that the written arbitration provision in the
contract falls within the purview of \S~2. Moreover, Ferrer sought
invalidation of the contract as a whole. In the proceedings below, he
made no discrete challenge to the validity of the arbitration clause.
See 145 Cal. App. 4th, at 449, 51 Cal. Rptr. 3d, at 635 (Vogel,
J., dissenting).\footnotemark[3] Ferrer thus urged the Labor Commissioner and
California courts to override the contract's arbitration clause on a
ground that \emph{Buckeye} requires the arbitrator to decide in the first
instance.

\section{IV}

  Ferrer attempts to distinguish \emph{Buckeye} by arguing that the TAA
merely requires exhaustion of administrative remedies before the parties
proceed to arbitration. We reject that argument.

\section{A}

  The TAA regulates talent agents and talent agency agreements.
``Talent agency'' is defined, with exceptions not relevant here, as
``a person or corporation who engages in the occupation of procuring,
offering, promising, or attempting to procure employment or engagements
for an artist or artists.'' Cal. Lab. Code Ann. \S~1700.4(a) (West
2003). The definition \newpage  ``does not cover other services for
which artists often contract, such as personal and career management
(i. e., advice, direction, coordination, and oversight with respect to
an artist's career or personal or financial affairs).'' \emph{Styne}
v. \emph{Stevens,} 26 Cal. 4th 42, 51, 26 P. 3d 343, 349 (2001) (emphasis
deleted). The TAA requires talent agents to procure a license from
the Labor Commissioner. \S~1700.5. ``In furtherance of the [TAA's]
protective aims, an unlicensed person's contract with an artist
to provide the services of a talent agency is illegal and void.''
\emph{Ibid.\\\footnotemark[4]

^3 Ferrer's petition to the Labor Commissioner sought a declaration
that the contract ``is void under the [TAA].'' App. 23. His
complaint in Superior Court seeking to enjoin arbitration asserted:
``[T]he [c]ontract is void by reason of [Preston's] attempt to
procure employment for [Ferrer] in violation of the [TAA],'' and
``the [c]ontract's arbitration clause does not vest authority in an
arbitrator to determine whether the contract is void.'' \emph{Id.,} at
27. His brief in the appeals court stated: ``Ferrer does not contend
that the arbitration clause in the [c]ontract was procured by fraud.
Ferrer contends that Preston unlawfully acted as an unlicensed talent
agent and hence cannot enforce the [c]ontract.'' Brief for Respondent
in No. B188997, p. 18.

  Section 1700.44(a) of the TAA states:

    ``In cases of controversy arising under this chapter, the
    parties involved shall refer the matters in dispute to the Labor
    Commissioner, who shall hear and determine the same, subject to an
    appeal within 10 days after determi nation, to the superior court
    where the same shall be heard de novo.''

\noindent Absent a notice of appeal filed within ten days, the Labor
Commissioner's determination becomes final and binding on the parties.
\emph{REO Broadcasting Consultants} v. \emph{Martin,} 69 Cal. App. 4th 489,
495, 81 Cal. Rptr. 2d 639, 642--643 (1999).\footnotemark[5]

  The TAA permits arbitration in lieu of proceeding before the Labor
Commissioner if an arbitration provision ``in a contract between a
talent agency and [an artist]'' both ``provides for reasonable notice
to the Labor Commissioner of the time and place of all arbitration
hearings'' and gives the Com\newpage  missioner ``the right to attend
all arbitration hearings.'' \S~1700.45. This prescription
demonstrates that there is no inherent conflict between the TAA and
arbitration as a dispute resolution mechanism. But \S~1700.45 was
of no utility to Preston. He has consistently maintained that he is
\emph{not} a talent agent as that term is defined in \S~1700.4(a), but
is, instead, a personal manager not subject to the TAA's regulatory
regime. 145 Cal. App. 4th, at 444, 51 Cal. Rptr. 3d, at 631. To
invoke \S~1700.45, Preston would have been required to concede a point
fatal to his claim for compensation---\\i. e.,} that he is a talent
agent, albeit an unlicensed one---and to have drafted his contract in
compliance with a statute that he maintains is inapplicable.

^4 Courts ``may void the entire contract'' where talent agency
services regulated by the TAA are ``inseparable from [unregulated]
managerial services.'' \emph{Marathon Entertainment, Inc.} v. \emph{Blasi,}
42 Cal. 4th 974, 998, 174 P. 3d 741, 744 (2008). If the contractual
terms are severable, however, ``an isolated instance'' of unlicensed
conduct ``does not automatically bar recovery for services that could
lawfully be provided without a license.'' \emph{Ibid.}

^5 To appeal the Labor Commissioner's decision, an aggrieved party
must post a bond of at least \$1,000 and up to twice the amount of any
judgment approved by the Commissioner. \S~1700.44(a).

  Procedural prescriptions of the TAA thus conflict with the FAA's
dispute resolution regime in two basic respects: First, the TAA, in
\S~1700.44(a), grants the Labor Commissioner exclusive jurisdiction
to decide an issue that the parties agreed to arbitrate, see
\emph{Buckeye,} 546 U. S., at 446; second, the TAA, in \S~1700.45,
imposes prerequisites to enforcement of an arbitration agreement
that are not applicable to contracts generally, see \emph{Doctor's
Associates, Inc.,} 517 U. S., at 687.

\section{B}

  Ferrer contends that the TAA is nevertheless compatible with the FAA
because \S~1700.44(a) merely postpones arbitration until after the
Labor Commissioner has exercised her primary jurisdiction. Brief
for Respondent 14, 40. The party that loses before the Labor
Commissioner may file for \emph{de novo} review in Superior Court. See
\S~1700.44(a). At that point, Ferrer asserts, either party could move
to compel arbitration under Cal. Civ. Proc. Code Ann. \S~1281.2 (West
2007), and thereby obtain an arbitrator's determination prior to
judicial review. See Brief for Respondent 13.

  That is not the position Ferrer took in the California courts. In his
complaint, he urged the Superior Court to \newpage  declare that ``the
[c]ontract, including in particular the issue of the validity of the
[c]ontract, \emph{is not subject to arbitration,}'' and he sought an
injunction stopping arbitration ``unless and until, \emph{if ever,} the
Labor Commissioner determines that he/ she has no jurisdiction over the
parties' dispute.'' App. 29 (emphasis added). Ferrer also told
the Superior Court: ``[I]f\dots the Commissioner rules that the
[c]ontract is void, Preston may appeal that ruling and have a hearing de
novo \emph{before this Court.}'' Appellant's App. in No. B188997 (Cal.
App.), p. 157, n. 1 (emphasis added).

  Nor does Ferrer's current argument---that \S~1700.44(a) merely
postpones arbitration---withstand examination. Section 1700.44(a)
provides for \emph{de novo} review in Superior Court, not elsewhere.\footnotemark[6]
Arbitration, if it ever occurred following the Labor Commissioner's
decision, would likely be long delayed, in contravention of Congress'
intent ``to move the parties to an arbitrable dispute out of court
and into arbitration as quickly and easily as possible.'' \emph{Moses
H. Cone Memorial Hospital} v. \emph{Mercury Constr. Corp.,} 460 U. S.
1,22 (1983). If Ferrer prevailed in the California courts, moreover,
he would no doubt argue that judicial findings of fact and conclusions
of law, made after a full and fair \emph{de novo} hearing in court, are
binding on the parties and preclude the arbitrator from making any
contrary rulings.

  A prime objective of an agreement to arbitrate is to achieve
``streamlined proceedings and expeditious results.'' \emph{Mitsubishi
Motors Corp.} v. \emph{Soler Chrysler-Plymouth, Inc.,} \newpage  473 U.~S.
614, 633 (1985). See also \emph{Allied-Bruce Terminix Cos.,} 513 U. S.,
at 278; \emph{Southland Corp.,} 465 U. S., at 7. That objective would
be frustrated even if Preston could compel arbitration in lieu of
\emph{de novo} Superior Court review. Requiring initial reference of the
parties' dispute to the Labor Commissioner would, at the least, hinder
speedy resolution of the controversy.

^6 From Superior Court an appeal lies in the Court of Appeal. Cal.
Civ. Proc. Code Ann. \S~904.1(a) (West 2007); Cal. Rule of Court
8.100(a) (Appellate Rules) (West 2007 rev. ed.). Thereafter, the
losing party may seek review in the California Supreme Court, Rule
8.500(a)(1) (Appellate Rules), perhaps followed by a petition for a
writ of certiorari in this Court, 28 U.~S.~C. \S~1257. Ferrer
has not identified a single case holding that California law permits
interruption of this chain of appeals to allow the arbitrator to review
the Labor Commissioner's decision. See Tr. of Oral Arg. 35.

  Ferrer asks us to overlook the apparent conflict between the
arbitration clause and \S~1700.44(a) because proceedings before the
Labor Commissioner are administrative rather than judicial. Brief
for Respondent 40--48. Allowing parties to proceed directly to
arbitration, Ferrer contends, would undermine the Labor Commissioner's
ability to stay informed of potentially illegal activity, \emph{id.,}
at 43, and would deprive artists protected by the TAA of the Labor
Commissioner's expertise, \emph{id.,} at 41--43.

  In \emph{Gilmer} v. \emph{Interstate/Johnson Lane Corp.,} 500 U.S. 20
(1991), we considered and rejected a similar argument, namely, that
arbitration of age discrimination claims would undermine the role of the
Equal Employment Opportunity Commission (EEOC) in enforcing federal law.
The ``mere involvement of an administrative agency in the enforcement
of a statute,'' we held, does not limit private parties' obligation
to comply with their arbitration agreements. \emph{Id.,} at 28--29.

  Ferrer points to our holding in \emph{EEOC} v. \emph{Waffle House,}
\emph{Inc.,} 534 U.~S. 279, 293--294 (2002), that an arbitration
agreement signed by an employee who becomes a discrimination complainant
does not bar the EEOC from filing an enforcement suit in its own name.
He further emphasizes our observation in \emph{Gilmer} that individuals who
agreed to arbitrate their discrimination claims would ``still be free
to file a charge with the EEOC.'' 500 U. S., at 28. Consistent with
these decisions, Ferrer argues, the arbitration clause in his contract
with Preston leaves undisturbed the Labor Com\newpage missioner's
independent authority to enforce the TAA. See Brief for Respondent
44--48. And so it may.\footnotemark[7] But in proceedings under \S~1700.44(a),
the Labor Commissioner functions not as an advocate advancing a cause
before a tribunal authorized to find the facts and apply the law;
instead, the Commissioner serves as impartial arbiter. That role is just
what the FAA-governed agreement between Ferrer and Preston reserves for
the arbitrator. In contrast, in \emph{Waffle House} and in the \emph{Gilmer}
aside Ferrer quotes, the Court addressed the role of an agency, not as
adjudicator but as prosecutor, pursuing an enforcement action in its
own name or reviewing a discrimination charge to determine whether to
initiate judicial proceedings.

  Finally, it bears repeating that Preston's petition presents
precisely and only a question concerning the forum in which the
parties' dispute will be heard. See \emph{supra,} at 352. ``By
agreeing to arbitrate a statutory claim, a party does not forgo
the substantive rights afforded by the statute; it only submits to
their resolution in an arbitral\dots forum.'' \emph{Mitsubishi
Motors Corp.,} 473 U. S., at 628. So here, Ferrer relinquishes no
substantive rights the TAA or other California law may accord him. But
under the contract he signed, he cannot escape resolution of those
rights in an arbitral forum.

  In sum, we disapprove the distinction between judicial and
administrative proceedings drawn by Ferrer and adopted by the appeals
court. When parties agree to arbitrate all questions arising under a
contract, the FAA supersedes state laws lodging primary jurisdiction in
another forum, whether judicial or administrative. \newpage 

^7 Enforcement of the parties' arbitration agreement in this case
does not displace any independent authority the Labor Commissioner may
have to investigate and rectify violations of the TAA. See Brief for
Respondent 47 (``[T]he Commissioner has independent investigatory
authority and may receive information concerning alleged violations of
the TAA from any source.'' (citation omitted)). See also Tr. of Oral
Arg. 13--14.

\section{V}

  Ferrer's final attempt to distinguish \emph{Buckeye} relies on \emph{Volt
Information Sciences, Inc.} v. \emph{Board of Trustees of Leland Stanford
Junior Univ.,} 489 U.~S. 468 (1989). \emph{Volt} involved a California
statute dealing with cases in which ``[a] party to [an] arbitration
agreement is also a party to a pending court action\dots [involving]
a third party [not bound by the arbitration agreement], arising out of
the same transaction or series of related transactions.'' Cal. Civ.
Proc. Code Ann. \S~1281.2(c) (West 2007). To avoid the ``possibility
of conflicting rulings on a common issue of law or fact,'' the statute
gives the Superior Court authority, \emph{inter alia,} to stay the court
proceeding ``pending the outcome of the arbitration'' or to stay the
arbitration ``pending the outcome of the court action.'' \emph{Ibid.}

  Volt Information Sciences and Stanford University were parties to
a construction contract containing an arbitration clause. When a
dispute arose and Volt demanded arbitration, Stanford sued Volt and
two other companies involved in the construction project. Those other
companies were not parties to the arbitration agreement; Stanford
sought indemnification from them in the event that Volt prevailed
against Stanford. At Stanford's request, the Superior Court stayed
the arbitration. The California Court of Appeal affirmed the stay
order. Volt and Stanford incorporated \S~1281.2(c) into their
agreement, the appeals court held. They did so by stipulating that
the contract---otherwise silent on the priority of suits drawing in
parties not subject to arbitration---would be governed by California
law. \emph{Board of Trustees of Leland Stanford Junior Univ.} v. \emph{Volt
Information Sciences, Inc.,} 240 Cal. Rptr. 558, 561 (1987) (officially
depublished). Relying on the Court of Appeal's interpretation of the
contract, we held that the FAA did not bar a stay of arbitration pending
the resolution of Stanford's Superior Court suit against Volt and the
two companies not bound by the arbitration agreement. \newpage 


  Preston and Ferrer's contract also contains a choice-of-law clause,
which states that the ``agreement shall be governed by the laws of
the state of California.'' App. 17. A separate saving clause
provides: ``If there is any conflict between this agreement and any
present or future law,'' the law prevails over the contract ``to the
extent necessary to bring [the contract] within the requirements of
said law.'' \emph{Id.,} at 18. Those contractual terms, according
to Ferrer, call for the application of California procedural law,
including \S~1700.44(a)'s grant of exclusive jurisdiction to the
Labor Commissioner.

  Ferrer's reliance on \emph{Volt} is misplaced for two discrete reasons.
First, arbitration was stayed in \emph{Volt} to accommodate litigation
involving third parties who were strangers to the arbitration agreement.
Nothing in the arbitration agreement addressed the order of proceedings
when pending litigation with third parties presented the prospect of
inconsistent rulings. We thought it proper, in those circumstances, to
recognize state law as the gap filler.

  Here, in contrast, the arbitration clause speaks to the matter
in controversy; it states that ``any dispute\dots relating to
.~.~. the breach, validity, or legality'' of the contract should be
arbitrated in accordance with the American Arbitration Association (AAA)
rules. App. 18. Both parties are bound by the arbitration agreement; the
question of Preston's status as a talent agent relates to the validity
or legality of the contract; there is no risk that related litigation
will yield conflicting rulings on common issues; and there is no other
procedural void for the choice-of-law clause to fill.

  Second, we are guided by our more recent decision in \emph{Mastrobuono}
v. \emph{Shearson Lehman Hutton, Inc.,} 514 U.S.52 (1995). Although the
contract in \emph{Volt} provided for ``arbitration in accordance with the
Construction Industry Arbitration Rules of the American Arbitration
Association,'' 489 U. S., at 470, n. 1 (internal quotation marks
omitted), Volt never argued that incorporation of those rules trumped
the choice-of-law clause contained in the contract, see Brief for
\newpage  Appellant, and Reply Brief, in \emph{Volt Information Sciences,
Inc.} v. \emph{Board of Trustees of Leland Stanford Junior Univ.,} O. T.
1988, No. 87--1318. Therefore, neither our decision in \emph{Volt} nor
the decision of the California appeals court in that case addressed
the import of the contract's incorporation by reference of privately
promulgated arbitration rules.

  In \emph{Mastrobuono,} we reached that open question while interpreting
a contract with both a New York choice-of-law clause and a clause
providing for arbitration in accordance with the rules of the
National Association of Securities Dealers (NASD). 514 U. S., at
58--59.\footnotemark[8] The ``best way to harmonize'' the two clauses, we
held, was to read the choice-of-law clause ``to encompass substantive
principles that New York courts would apply, but not to include [New
York's] special rules limiting the authority of arbitrators.''
\emph{Id.,} at 63--64.

  Preston and Ferrer's contract, as noted, provides for arbitration
in accordance with the AAA rules. App. 18. One of those rules
states that ``[t]he arbitrator shall have the power to determine the
existence or validity of a contract of which an arbitration clause forms
a part.'' AAA, Commercial Arbitration Rules ¶ R--7(b) (2007),
online at http://www.adr.org/sp.asp?id=22440 (as visited Feb. 15, 2008,
and in Clerk of Court's case file). The incorporation of the AAA
rules, and in particular Rule 7(b), weighs against inferring from the
choice-of-law clause an understanding shared by Ferrer and Preston that
their disputes would be heard, in \newpage  the first instance, by the
Labor Commissioner. Following the guide \emph{Mastrobuono} provides, the
``best way to harmonize'' the parties' adoption of the AAA rules and
their selection of California law is to read the latter to encompass
prescriptions governing the substantive rights and obligations of the
parties, but not the State's ``special rules limiting the authority
of arbitrators.'' 514 U. S., at 63--64.

^8 The question in \emph{Mastrobuono} was whether the arbitrator could
award punitive damages. See \emph{Mastrobuono} v. \emph{Shearson Lehman
Hutton, Inc.,} 514 U.~S. 52, 53--54 (1995). New York law prohibited
arbitrators, but not courts, from awarding such damages. \emph{Id.,}
at 55. The NASD rules, in contrast, authorized ``damages and other
relief,'' which, according to an NASD arbitration manual, included
punitive damages. \emph{Id.,} at 61 (internal quotation marks omitted).
Relying on \emph{Volt,} respondents argued that the choice-of-law clause
incorporated into the parties' arbitration agreement New York's
ban on arbitral awards of punitive damages. Opposing that argument,
petitioners successfully urged that the agreement to arbitrate in
accordance with the NASD rules controlled.

\hrule

  For the reasons stated, the judgment of the California Court of Appeal
is reversed, and the case is remanded for further proceedings not
inconsistent with this opinion.

\begin{flushright}\emph{It is so ordered.}\end{flushright}
