% Syllabus
% Reporter of Decisions

\setcounter{page}{120}

  Respondent Van Patten sought habeas relief under 28 U.~S.~C. \S~2254, arguing that his Sixth Amendment right to counsel was violated because his trial counsel, though linked to the courtroom by speakerphone, was physically absent from his state plea hearing. The District Court denied relief, but the Seventh Circuit reversed, holding that the case's novel claim should be resolved under \emph{United States} v. \emph{Cronic,} 466 U.~S. 648 (prejudice may be presumed), rather than \emph{Strickland} v. \emph{Washington,} 466 U.~S. 668 (requiring a showing of counsel's deficient performance and prejudice to defendant). This Court remanded in light of \emph{Carey} v. \emph{Musladin,} 549 U. S. 70, and its explanation of \S~2254's ``clearly established Federal law'' requirement. The Seventh Circuit reaffirmed, holding that this case, unlike \emph{Musladin,} does not involve an open constitutional question since this Court has recognized a defendant's right to relief if his counsel was actually or constructively absent at a critical stage of the proceeding.

\emph{Held:}

  Because this Court's precedents give no clear answer to the question presented in this case, it cannot be said that the state court unreasonably applied clearly established federal law, and therefore, \S~2254 relief is unauthorized. \emph{Strickland}'s two-pronged test typically applies to ineffective-assistance-of-counsel claims, but \emph{Cronic} held that test unnecessary when ``circumstances [exist] that are so likely to prejudice the accused that the cost of litigating their effect in a particular case is unjustified,'' 466 U. S., at 658, and found that one such circumstance is counsel's total absence during a critical stage of the proceeding. However, no decision of this Court squarely addresses the issue in this case, clearly establishes that \emph{Cronic} should replace \emph{Strickland} in this novel factual context, or clearly holds that counsel's participation by speakerphone should be treated on par with total absence. Certiorari granted; 489 F. 3d 827, reversed and remanded.
