% Dissenting
% Stevens

\setcounter{page}{167}

  \textsc{Justice Stevens,} with whom \textsc{Justice Souter} and \textsc{Justice Ginsburg} join, dissenting.

  Charter Communications, Inc., inflated its revenues by \$17 million in order to cover up a \$15 to \$20 million expected cashflow shortfall. It could not have done so absent the knowingly fraudulent actions of Scientific-Atlanta, Inc., and Motorola, Inc. Investors relied on Charter's revenue statements in deciding whether to invest in Charter and in doing so relied on respondents' fraud, which was itself a ``deceptive device'' prohibited by \S~10(b) of the Securities Exchange Act of 1934. 15 U.~S.~C. \S~78j(b). This is enough to satisfy the requirements of \S~10(b) and enough to distinguish this case from \emph{Central Bank of Denver, N. A.} v. \emph{First Interstate Bank of Denver, N. A.,} 511 U.~S. 164 (1994).

  The Court seems to assume that respondents' alleged conduct could subject them to liability in an enforcement proceeding initiated by the Government, \emph{ante,} at 166, but nevertheless concludes that they are not subject to liability in a private action brought by injured investors because they are, at most, guilty of aiding and abetting a violation of \S~10(b), \newpage  rather than an actual violation of the statute. While that conclusion results in an affirmance of the judgment of the Court of Appeals, it rests on a rejection of that court's reasoning. Furthermore, while the Court frequently refers to petitioner's attempt to ``expand'' the implied cause of action\footnotemark[1] ---a conclusion that begs the question of the contours of that cause of action---it is today's decision that results in a significant departure from \emph{Central Bank.}

  The Court's conclusion that no violation of \S~10(b) giving rise to a private right of action has been alleged in this case rests on two faulty premises: (1) the Court's overly broad reading of \emph{Central Bank,} and (2) the view that reliance requires a kind of super-causation---a view contrary to both the Securities and Exchange Commission's (SEC) position in a recent Ninth Circuit case\footnotemark[2] and our holding in \emph{Basic Inc.} v. \emph{Levinson,} 485 U.~S. 224 (1988). These two points merit separate discussion.

\section{I}

  The Court of Appeals incorrectly based its decision on the view that ``[a] device or contrivance is not ‘deceptive,' within the meaning of \S~10(b), absent some misstatement or a failure to disclose by one who has a duty to disclose.'' \emph{In re Charter Communications, Inc., Securities Litigation,} 443 F. 3d 987, 992 (CA8 2006). The Court correctly explains why the statute covers nonverbal as well as verbal deceptive conduct. \emph{Ante,} at 158. The allegations in this case---that respondents \newpage  produced documents falsely claiming costs had risen and signed contracts they knew to be backdated in order to disguise the connection between the increase in costs and the purchase of advertising---plainly describe ``deceptive devices'' under any standard reading of the phrase.

\footnotetext[1]{See \emph{ante,} at 161 (``[w]ere the implied cause of action to be extended to the practices described here~.~.~.~''); \emph{ante,} at 163 (``[t]he practical consequences of an expansion''); \emph{ante,} at 165 (``Concerns with the judicial creation of a private cause of action caution against its expansion. The decision to extend the cause of action is for the Congress, not for us'').}

\footnotetext[2]{See Brief for SEC as \emph{Amicus Curiae} in \emph{Simpson} v. \emph{AOL Time Warner Inc.,} No. 04--55665 (CA9), p. 21 (``The reliance requirement is satisfied where a plaintiff relies on a material deception flowing from a defendant's deceptive act, even though the conduct of other participants in the fraudulent scheme may have been a subsequent link in the causal chain leading to the plaintiff's securities transaction'').}

  What the Court fails to recognize is that this case is critically different from \emph{Central Bank} because the bank in that case did not engage in any deceptive act and, therefore, did not \emph{itself} violate \S~10(b). The Court sweeps aside any distinction, remarking that holding respondents liable would ``revive in substance the implied cause of action against all aiders and abettors except those who committed no deceptive act in the process of facilitating the fraud.'' \emph{Ante,} at 162--163. But the fact that Central Bank engaged in no deceptive conduct whatsoever---in other words, that it was at most an aider and abettor---sharply distinguishes \emph{Central Bank} from cases that do involve allegations of such conduct. 511 U. S., at 167 (stating that the question presented was ``whether private civil liability under \S~10(b) extends as well to those who do not engage in the manipulative or deceptive practice, but who aid and abet the violation'').

  The Central Bank of Denver was the indenture trustee for bonds issued by a public authority and secured by liens on property in Colorado Springs. After default, purchasers of \$2.1 million of those bonds sued the underwriters, alleging violations of \S~10(b); they also named Central Bank as a defendant, contending that the bank's delay in reviewing a suspicious appraisal of the value of the security made it liable as an aider and abettor. \emph{Id.,} at 167--168. The facts of this case would parallel those of \emph{Central Bank} if respondents had, for example, merely delayed sending invoices for set-top boxes to Charter. Conversely, the facts in \emph{Central Bank} would mirror those in the case before us today if the bank had knowingly purchased real estate in wash transactions at above-market prices in order to facilitate the appraiser's overvaluation of the security. \emph{Central Bank,} thus, poses no \newpage  obstacle to petitioner's argument that it has alleged a cause of action under \S~10(b).

\section{II}

  The Court's next faulty premise is that petitioner is required to allege that Scientific-Atlanta and Motorola made it ``necessary or inevitable for Charter to record the transactions as it did,'' \emph{ante,} at 161, in order to demonstrate reliance. Because the Court of Appeals did not base its holding on reliance grounds, see 443 F. 3d, at 992, the fairest course to petitioner would be for the majority to remand to the Court of Appeals to determine whether petitioner properly alleged reliance, under a correct view of what \S~10(b) covers.\footnotemark[3] Because the Court chooses to rest its holding on an absence of reliance, a response is required.

  In \emph{Basic Inc.,} 485 U. S., at 243, we stated that ``[r]eliance provides the requisite causal connection between a defendant's misrepresentation and a plaintiff's injury.'' The Court's view of the causation required to demonstrate reliance is unwarranted and without precedent.

  In \emph{Basic Inc.,} we held that the ``fraud-on-the-market'' theory provides adequate support for a presumption in private securities actions that shareholders (or former shareholders) in publicly traded companies rely on public material misstatements that affect the price of the company's stock. \emph{Id.,} at 248. The holding in \emph{Basic} is surely a sufficient response to the argument that a complaint alleging that deceptive acts \newpage  which had a material effect on the price of a listed stock should be dismissed because the plaintiffs were not subjectively aware of the deception at the time of the securities' purchase or sale. This Court has not held that investors must be aware of the specific deceptive act which violates \S~10b to demonstrate reliance.

\footnotetext[3]{Though respondents did argue to the Court of Appeals that reliance was lacking, see Brief for Appellee Motorola, Inc., in No. 05--1974 (CA8), p. 15, that argument was quite short and was based on an erroneously broad reading of \emph{Central Bank of Denver, N. A.} v. \emph{First Interstate Bank of Denver, N. A.,} 511 U.~S. 164 (1994), as discussed, \emph{supra,} at 169 and this page. The Court of Appeals mentioned reliance only once, stating that respondents ``did not issue any misstatement relied upon by the investing public.'' 443 F. 3d, at 992. Furthermore, that statement was made in the context of the Court of Appeals' holding that a deceptive act must be a misstatement or omission---a holding which the Court unanimously rejects.}

  The Court is right that a fraud-on-the-market presumption coupled with its view on causation would not support petitioner's view of reliance. The fraud-on-the-market presumption helps investors who cannot demonstrate that they, \emph{themselves,} relied on fraud that reached the market. But that presumption says nothing about causation from the other side: what an individual or corporation must do in order to have ``caused'' the misleading information that reached the market. The Court thus has it backwards when it first addresses the fraud-on-the-market presumption, rather than the causation required. See \emph{ante,} at 159. The argument is not that the fraud-on-the-market presumption is enough standing alone, but that a correct view of causation coupled with the presumption would allow petitioner to plead reliance.

  Lower courts have correctly stated that the causation necessary to demonstrate reliance is not a difficult hurdle to clear in a private right of action under \S~10(b). Reliance is often equated with ``‘transaction causation.''' \emph{Dura Pharmaceuticals, Inc.} v. \emph{Broudo,} 544 U.~S. 336, 341, 342 (2005). Transaction causation, in turn, is often defined as requiring an allegation that but for the deceptive act, the plaintiff would not have entered into the securities transaction. See, \emph{e. g., Lentell} v. \emph{Merrill Lynch \& Co.,} 396 F. 3d 161, 172 (CA2 2005); \emph{Binder} v. \emph{Gillespie,} 184 F. 3d 1059, 1065--1066 (CA9 1999).

  Even if but-for causation, standing alone, is too weak to establish reliance, petitioner has also alleged that respondents proximately caused Charter's misstatement of income; petitioner has alleged that respondents knew their deceptive \newpage  acts would be the basis for statements that would influence the market price of Charter stock on which shareholders would rely. Second Amended Consolidated Class Action Complaint ¶¶ 8, 98, 100, 109, App. 19a, 55a--56a, 59a. Thus, respondents' acts had the foreseeable effect of causing petitioner to engage in the relevant securities transactions. The Restatement (Second) of Torts \S~533, pp. 72--73 (1977), provides that ``[t]he maker of a fraudulent misrepresentation is subject to liability .~.~. if the misrepresentation, although not made directly to the other, is made to a third person and the maker intends or has reason to expect that its terms will be repeated or its substance communicated to the other.'' The sham transactions described in the complaint in this case had the same effect on Charter's profit and loss statement as a false entry directly on its books that included \$17 million of gross revenues that had not been received. And respondents are alleged to have known that the outcome of their fraudulent transactions would be communicated to investors.

  The Court's view of reliance is unduly stringent and unmoored from authority. The Court first says that if petitioner's concept of reliance is adopted the implied cause of action ``would reach the whole marketplace in which the issuing company does business.'' \emph{Ante,} at 160. The answer to that objection is, of course, that liability only attaches when the company doing business with the issuing company has \emph{itself} violated \S~10(b).\footnotemark[4] The Court next relies on what it views as a strict division between the ``realm of financing business'' and the ``ordinary business operations.'' \emph{Ante,} at 161. But petitioner's position does not merge the two: A corporation engaging in a business transaction with a partner who transmits false information to the market is only liable where the \newpage  corporation \emph{itself} violates \S~10(b). Such a rule does not invade the province of ``ordinary'' business transactions.

\footnotetext[4]{Because the kind of sham transactions alleged in this complaint are unquestionably isolated departures from the ordinary course of business in the American marketplace, it is hyperbolic for the Court to conclude that petitioner's concept of reliance would authorize actions ``against the entire marketplace in which the issuing company operates.'' \emph{Ante,} at 162.}

  The majority states that ``[s]ection 10(b) does not incorporate common-law fraud into federal law,'' citing \emph{SEC} v. \emph{Zandford,} 535 U.~S. 813 (2002). \emph{Ante,} at 162. Of course, not every common-law fraud action that happens to touch upon securities is an action under \S~10(b), but the Court's opinion in \emph{Zandford} did not purport to jettison all reference to common-law fraud doctrines from \S~10(b) cases. In fact, our prior cases explained that to the extent that ``the antifraud provisions of the securities laws are not coextensive with common-law doctrines of fraud,'' it is because commonlaw fraud doctrines might be too restrictive. \emph{Herman \& MacLean} v. \emph{Huddleston,} 459 U.~S. 375, 388--389 (1983). ``Indeed, an important purpose of the federal securities statutes was to rectify perceived deficiencies in the available common-law protections by establishing higher standards of conduct in the securities industry.'' \emph{Id.,} at 389. I, thus, see no reason to abandon common-law approaches to causation in \S~10(b) cases.

  Finally, the Court relies on the course of action Congress adopted after our decision in \emph{Central Bank} to argue that siding with petitioner on reliance would run contrary to congressional intent. Senate hearings on \emph{Central Bank} were held within one month of our decision.\footnotemark[5] Less than one year later, Senators Dodd and Domenici introduced S. 240, which became the Private Securities Litigation Reform Act of 1995 (PSLRA), 109 Stat. 737.\footnotemark[6] Congress stopped short of undoing \emph{Central Bank} entirely, instead adopting a compromise which restored the authority of the SEC to enforce aiding and abetting liability.\footnotemark[7] A private right of action based on \newpage  aiding and abetting violations of \S~10(b) was not, however, included in the PSLRA,\footnotemark[8] despite support from Senator Dodd and members of the Senate Subcommittee on Securities.\footnotemark[9] This compromise surely provides no support for extending \emph{Central Bank} in order to immunize an undefined class of actual violators of \S~10(b) from liability in private litigation. Indeed, as Members of Congress---including those who rejected restoring a private cause of action against aiders and abettors---made clear, private litigation under \S~10(b) continues to play a vital role in protecting the integrity of our securities markets.\footnotemark[10] That Congress chose not to restore \newpage  the aiding and abetting liability removed by \emph{Central Bank} does not mean that Congress wanted to exempt from liability the broader range of conduct that today's opinion excludes.

\footnotetext[5]{See S. Rep. No. 104--98, p. 2 (1995) (hereinafter S. Rep.).}

\footnotetext[6]{\emph{Id.,} at 1.}

\footnotetext[7]{The opinion in \emph{Central Bank} discussed only private remedies, but its rationale---that the text of \S~10(b) did not cover aiding and abetting---obviously limited the authority of public enforcement agencies. See 511 U. S., \newpage  at 199--200 (\textsc{Stevens,} J., dissenting); see also S. Rep., at 19 (``The Committee does, however, grant the SEC express authority to bring actions seeking injunctive relief or money damages against persons who knowingly aid and abet primary violators of the securities laws'').}

\footnotetext[8]{PSLRA, \S~104, 109 Stat. 757; see also S. Rep., at 19 (``The Committee believes that amending the 1934 Act to provide explicitly for private aiding and abetting liability actions under Section 10(b) would be contrary to S. 240's goal of reducing meritless securities litigation'').}

\footnotetext[9]{See \emph{id.,} at 51 (additional views of Sen. Dodd) (``I am pleased that the Committee bill grants the Securities and Exchange Commission explicit authority to bring actions against those who knowingly aid and abet primary violators. However, I remain concerned about liability in private actions and will continue work with other Committee members on this issue as we move to floor consideration''). Senators Sarbanes, Boxer, and Bryan also submitted additional views in which they stated that ``[w]hile the provision in the bill is of some help, the deterrent effect of the securities laws would be strengthened if aiding and abetting liability were restored in private actions as well.'' \emph{Id.,} at 49.}

\footnotetext[1]{ \emph{Id.,} at 8 (``The success of the U. S. securities markets is largely the result of a high level of investor confidence in the integrity and efficiency of our markets. The SEC enforcement program and the availability of private rights of action together provide a means for defrauded investors to recover damages and a powerful deterrent against violations of the securities laws''); see also \emph{Bateman Eichler, Hill Richards, Inc.} v. \emph{Berner,} 472 U.~S. 299, 310 (1985) (``Moreover, we repeatedly have emphasized that implied private actions provide ‘a most effective weapon in the enforcement' of the securities laws and are ‘a necessary supplement to Commission action' ''); Brief for Former SEC Commissioners as \emph{Amici Curiae} 4 (``[L]iability [of the kind at issue here] neither results in undue liability \newpage exposure for non-issuers, nor an undue burden upon capital formation. Holding liable wrongdoers who actively engage in fraudulent conduct that lacks a legitimate business purpose does not hinder, but rather enhances, the integrity of our markets and our economy. We believe that the integrity of our securities markets is their strength. Investors, both domestic and foreign, trust that fraud is not tolerated in our nation's securities markets and that strong remedies exist to deter and protect against fraud and to recompense investors when it occurs'').}

  The Court is concerned that such liability would deter overseas firms from doing business in the United States or ``shift securities offerings away from domestic capital markets.'' \emph{Ante,} at 164. But liability for those who violate \S~10(b) ``will not harm American competitiveness; in fact, investor faith in the safety and integrity of our markets \emph{is} their strength. The fact that our markets are the safest in the world has helped make them the strongest in the world.'' Brief for Former SEC Commissioners as \emph{Amici Curiae} 9.

  Accordingly, while I recognize that the \emph{Central Bank} opinion provides a precedent for judicial policymaking decisions in this area of the law, I respectfully dissent from the Court's continuing campaign to render the private cause of action under \S~10(b) toothless. I would reverse the decision of the Court of Appeals.

\section{III}

  While I would reverse for the reasons stated above, I must also comment on the importance of the private cause of action that Congress implicitly authorized when it enacted the Securities Exchange Act of 1934. A theme that underlies the Court's analysis is its mistaken hostility toward the \S~10(b) private cause of action.\footnotemark[11] \emph{Ante,} at 164--165. The Court's current view of implied causes of action is that they \newpage  are merely a ``relic'' of our prior ``heady days.'' \emph{Correctional Services Corp.} v. \emph{Malesko,} 534 U.~S. 61, 75 (2001) (\textsc{Scalia,} J., concurring). Those ``heady days'' persisted for 200 years.

\footnotetext[1]{ The Court does concede that Congress has now ratified the private cause of action in the PSLRA. See \emph{ante,} at 165.}

  During the first two centuries of this Nation's history much of our law was developed by judges in the common-law tradition. A basic principle animating our jurisprudence was enshrined in state constitution provisions guaranteeing, in substance, that ``every wrong shall have a remedy.''\footnotemark[12] \newpage  Fashioning appropriate remedies for the violation of rules of law designed to protect a class of citizens was the routine business of judges. See \emph{Marbury} v. \emph{Madison,} 1 Cranch 137, 166 (1803). While it is true that in the early days state law was the source of most of those rules, throughout our history---until 1975---the same practice prevailed in federal courts with regard to federal statutes that left questions of remedy open for judges to answer. In \emph{Texas \& Pacific R. Co.} v. \emph{Rigsby,} 241 U.~S. 33, 39 (1916), this Court stated the following:

\footnotetext[1]{ Today, the guarantee of a remedy for every injury appears in nearly three-quarters of state constitutions. Ala. Const., Art. I, \S~13; Ark. Const., Art. 2, \S~13; Colo. Const., Art. II, \S~6; Conn. Const., Art. I, \S~10; Del. Const., Art. I, \S~9; Fla. Const., Art. I, \S~21; Idaho Const., Art. I, \S~18; Ill. Const., Art. I, \S~12; Ind. Const., Art. I, \S~12; Kan. Const., Bill of Rights, \S~18; Ky. Const., \S~14; La. Const., Art. I, \S~22; Me. Const., Art. I, \S~19; Md. Const., Declaration of Rights, Art. 19; Mass. Const., pt. I, Art. 11; Minn. Const., Art. 1, \S~8; Miss. Const., Art. III, \S~24; Mo. Const., Art. I, \S~14; Mont. Const., Art. II, \S~16; Neb. Const., Art. I, \S~13; N. H. Const., pt. I, Art. 14; N. C. Const., Art. I, \S~18; N. D. Const., Art. I, \S~9; Ohio Const., Art. I, \S~16; Okla. Const., Art. 2, \S~6; Ore. Const., Art. I, \S~10; Pa. Const., Art. I, \S~11; R. I. Const., Art. I, \S~5; S. C. Const., Art. I, \S~9; S. D. Const., Art. VI, \S~20; Tenn. Const., Art. I, \S~17; Tex. Const., Art. I, \S~13; Utah Const., Art. I, \S~11; Vt. Const., ch. I, Art. 4; W. Va. Const., Art. III, \S~17; Wis. Const., Art. I, \S~9; Wyo. Const., Art. I, \S~8; see also Phillips, The Constitutional Right to a Remedy, 78 N. Y. U. L. Rev. 1309, 1310, n. 6 (2003) (hereinafter Phillips).}

  The concept of a remedy for every wrong most clearly emerged from Sir Edward Coke's scholarship on Magna Carta. See 1 Second Part of the Institutes of the Laws of England (1797). At the time of the ratification of the United States Constitution, Delaware, Massachusetts, Maryland, New Hampshire, and North Carolina had all adopted constitutional provisions reflecting the provision in Coke's scholarship. Del. Declaration of Rights and Fundamental Rules \S~12 (1776), reprinted in 2 W. Swindler, Sources and Documents of United States Constitutions 198 (1973) (hereinafter Swindler); Mass. Const., pt. I, Art. XI (1780), reprinted in 3 Federal and State Constitutions, Colonial Charters, and Other Organic Laws 1891 (F. Thorpe ed. 1909) (reprinted 1993) (hereinafter Thorpe); Md. Const., Declaration of Rights, Art. XVII (1776), in \emph{id.,} at 1688; N. H. Const., Art. XIV (1784), in 4 \emph{id.,} at 2455; N. C. Const., Declaration of Rights, Art. XIII (1776), in 5 \emph{id.,} at 2787, 2788; see also Phillips 1323--1324. Pennsylvania's Constitution of 1790 contains a guarantee. Pa. Const., Art. I, \S~11, in 5 \newpage  Thorpe 3101. Connecticut's 1818 Constitution, Art. I, \S~12, contained such a provision. Reprinted in 2 Swindler 145.

    \begin{quote}
		\newpage  ``A disregard of the command of the statute is a wrongful act, and where it results in damage to one of the class for whose especial benefit the statute was enacted, the right to recover the damages from the party in default is implied, according to a doctrine of the common law expressed in 1 Com. Dig., \emph{tit.} Action upon Statute (F), in these words: ‘So, in every case, where a statute enacts, or prohibits a thing for the benefit of a person, he shall have a remedy upon the same statute for the thing enacted for his advantage, or for the recompense of a wrong done to him contrary to the said law.' (\emph{Per} Holt, C. J., \emph{Anon.,} 6 Mod. 26, 27.)''
	\end{quote}

  Judge Friendly succinctly described the post-\emph{Rigsby,} pre-1975 practice in his opinion in \emph{Leist} v. \emph{Simplot,} 638 F. 2d 283, 298--299 (CA2 1980):

\begin{quote}

	``Following \emph{Rigsby} the Supreme Court recognized implied causes of action on numerous occasions, see, e. g., \emph{Wyandotte Transportation Co. v. United States,} 389 U.~S. 191 .~.~. (1967) (sustaining implied cause of action by United States for damages under Rivers and Harbors Act for removing negligently sunk vessel despite express remedies of \emph{in rem} action and criminal penalties); \emph{United States v. Republic Steel Corp.,} 362 U.S. 482\dots (1960) (sustaining implied cause of action by United \newpage  States for an injunction under the Rivers and Harbors Act); \emph{Tunstall v. Locomotive Firemen \& Enginemen,} 323 U.~S. 210 \dots (1944) (sustaining implied cause of action by union member against union for discrimination among members despite existence of Board of Mediation); \emph{Sullivan v. Little Hunting Park, Inc.,} 396 U. S. 229\dots (1969) (sustaining implied private cause of action under 42 U.~S.~C. \S~1982); \emph{Allen v. State Board of Elections,} 393 U.~S. 544\dots (1969) (sustaining implied private cause of action under \S~5 of the Voting Rights Act despite the existence of a complex regulatory scheme and explicit rights of action in the Attorney General); and, of course, the aforementioned decisions under the securities laws. As the Supreme Court itself has recognized, the period of the 1960's and early 1970's was one in which the ‘Court had consistently found implied remedies.' \emph{Cannon v. University of Chicago,} 441 U.~S. 677, 698\dots (1979).''

\end{quote}

  In a law-changing opinion written by Justice Brennan in 1975, the Court decided to modify its approach to private causes of action. \emph{Cort} v. \emph{Ash,} 422 U.~S. 66 (constraining courts to use a strict four-factor test to determine whether Congress intended a private cause of action). A few years later, in \emph{Cannon} v. \emph{University of Chicago,} 441 U.~S. 677 (1979), we adhered to the strict approach mandated by \emph{Cort} v. \emph{Ash} in 1975, but made it clear that ``our evaluation of congressional action in 1972 must take into account its contemporary legal context.'' 441 U. S., at 698--699. That context persuaded the majority that Congress had intended the courts to authorize a private remedy for members of the protected class.

  Until \emph{Central Bank,} the federal courts continued to enforce a broad implied cause of action for the violation of statutes enacted in 1933 and 1934 for the protection of investors. As Judge Friendly explained:

      \begin{quote}

		  \newpage  ``During the late 1940's, the 1950's, the 1960's and the early 1970's there was widespread, indeed almost general, recognition of implied causes of action for damages under many provisions of the Securities Exchange Act, including not only the antifraud provisions, \S\S~10 and 15(c)(1), see \emph{Kardon v. National Gypsum Co.,} 69 F. Supp. 512, 513--14 (E.D.Pa.1946); \emph{Fischman v. Raytheon Mfg. Co.,} 188 F. 2d 783, 787 (2 Cir. 1951) (Frank, J.); \emph{Fratt v. Robinson,} 203 F. 2d 627, 631--33 (9 Cir. 1953), but many others. These included the provision, \S~6(a)(1), requiring securities exchanges to enforce compliance with the Act and any rule or regulation made thereunder, see \emph{Baird v. Franklin,} 141 F. 2d 238, 239, 240, 244--45 (2 Cir.), \emph{cert. denied,} 323 U.~S. 737\dots (1944), and provisions governing the solicitation of proxies, see \emph{J. I. Case Co. v. Borak,} 377 U.~S. 426, 431--35\dots (1964)~.~.~.~. Writing in 1961, Professor Loss remarked with respect to violations of the antifraud provisions that with one exception ‘not a single judge has expressed himself to the contrary.' 3 Securities Regulation 1763--64. See also Bromberg \& Lowenfels, [Securities Fraud \& Commodities Fraud] \S~2.2 (462) [(1979)] (describing 1946--1974 as the ‘expansion era' in implied causes of action under the securities laws). When damage actions for violation of \S~10(b) and Rule 10b--5 reached the Supreme Court, the existence of an implied cause of action was not deemed worthy of extended discussion. \emph{Superintendent of Insurance v. Bankers Life \& Casualty Co.,} 404 U.S. 6\dots (1971).'' \emph{Leist,} 638 F. 2d, at 296--297 (footnote omitted).

      \end{quote}

  In light of the history of court-created remedies and specifically the history of implied causes of action under \S~10(b), the Court is simply wrong when it states that Congress did not impliedly authorize this private cause of action ``when it first enacted the statute.'' \emph{Ante,} at 167. Courts near in \newpage  time to the enactment of the securities laws recognized that the principle in \emph{Rigsby} applied to the securities laws.\footnotemark[13] Congress enacted \S~10(b) with the understanding that federal courts respected the principle that every wrong would have a remedy. Today's decision simply cuts back further on Congress' intended remedy. I respectfully dissent.

\footnotetext[1]{ See, \emph{e. g., Slavin} v. \emph{Germantown Fire Ins. Co.,} 174 F. 2d 799 (CA3 1949); \emph{Baird} v. \emph{Franklin,} 141 F. 2d 238, 244--245 (CA2) (``The fact that the statute provides no machinery or procedure by which the individual right of action can proceed is immaterial. It is well established that members of a class for whose protection a statutory duty is created may sue for injuries resulting from its breach and that the common law will supply a remedy if the statute gives none''), cert. denied, 323 U.~S. 737 (1944); \emph{Kardon} v. \emph{National Gypsum Co.,} 69 F. Supp. 512, 514 (ED Pa. 1946) (``[T]he right to recover damages arising by reason of violation of a statute\dots is so fundamental and so deeply ingrained in the law that where it is not expressly denied the intention to withhold it should appear very clearly and plainly'').}
