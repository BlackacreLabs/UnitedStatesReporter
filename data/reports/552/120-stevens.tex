Author: Stevens
\setcounter{page}{126}
Type: Concurring

  \textsc{Justice Stevens,} concurring in the judgment.

  An unfortunate drafting error in the Court's opinion in \emph{United States} v. \emph{Cronic,} 466 U.~S. 648 (1984), makes it necessary to join the Court's judgment in this case.

  In \emph{Cronic,} this Court explained that some violations of the right to counsel arise in ``circumstances that are so likely to prejudice the accused that the cost of litigating their effect in a particular case is unjustified.'' \emph{Id.,} at 658. One such circumstance exists when the accused is ``denied the presence of counsel at a critical stage of the prosecution.'' \emph{Id.,} at 662. We noted that the ``presence'' of lawyers ``is essential\newpage because they are the means through which the other rights of the person on trial are secured.'' \emph{Id.,} at 653. Regrettably, \emph{Cronic} did not ``clearly establis[h]'' the full scope of the defendant's right to the presence of an attorney. See 28 U.~S.~C. \S~2254(d)(1).

  The Court of Appeals apparently read ``the presence of counsel'' in \emph{Cronic} to mean ``the presence of counsel \emph{in open court.}'' Initially, all three judges on the panel assumed that the constitutional right at stake was the right to have counsel by one's side at all critical stages of the proceeding.[[*]]

\footnotetext[*]{In his opinion for a unanimous panel, Judge Evans explained at length why respondent had not had the assistance of counsel at a critical stage of the proceeding---the plea hearing---which resulted in a sentence of imprisonment for 25 years. He wrote, in part:

``The Sixth Amendment's right-to-counsel guarantee recognizes ‘the obvious truth that the average defendant does not have the professional legal skill to protect himself when brought before a tribunal with power to take his life or liberty.' \emph{Johnson v. Zerbst,} 304 U.~S. 458, 462--63\dots (1938). ‘Of all the rights that an accused person has, the right to be represented by counsel is by far the most pervasive for it affects his ability to assert any other rights he may have.' \emph{Cronic,} 466 U. S. at 654\dots (citation omitted). Thus, a defendant requires an attorney's ‘guiding hand' through every stage of the proceedings against him. \emph{Powell v. Alabama,} 287 U.~S. 45, 53\dots (1932); \emph{Cronic,} 466 U. S. at 658~.~.~.~. It is well-settled that a court proceeding in which a defendant enters a plea (a guilty plea or, as here, a plea of no contest) is a ‘critical stage' where an attorney's presence is crucial because ‘defenses may be .~.~. irretrievably lost, if not then and there asserted.' \emph{Hamilton v. Alabama,} 368 U.~S. 52, 54\dots (1961). \emph{See also White v. Maryland,} 373 U.~S. 59, 60\dots (1963); \emph{United States ex rel. Thomas v. O'Leary,} 856 F. 2d 1011, 1014 (7th Cir. 1988). Indeed, with plea bargaining the norm and trial the exception, for most criminal defendants a change of plea hearing is \emph{the} critical stage of their prosecution.

``In deciding whether to dispense with the two-part \emph{Strickland} [v. \emph{Washington,} 466 U.~S. 668 (1984),] inquiry, a court must evaluate whether the ‘surrounding circumstances make it unlikely that the defendant could have received the effective assistance of counsel,' \emph{Cronic,} 466 U.S. at 666,\dots and thus ‘justify a presumption that [the] conviction was insufficiently reliable to satisfy the Constitution,' \emph{id.} at 662~.~.~.~. In this case, although the transcript shows that the state trial judge did his best to conduct the plea colloquy with care, the arrangements made it impossible for Van Pat\newpage ten to have the ‘assistance of counsel' in anything but the most perfunctory sense. Van Patten stood alone before judge and prosecutor. Unlike the usual defendant in a criminal case, he could not turn to his lawyer for private legal advice, to clear up misunderstandings, to seek reassurance, or to discuss any last-minute misgivings. Listening over an audio connection, counsel could not detect and respond to cues from his client's demeanor that might have indicated he did not understand certain aspects of the proceeding, or that he was changing his mind. If Van Patten wished to converse with his attorney, anyone else in the courtroom could effectively eavesdrop. (We assume the district attorney would balk if he were expected to conduct last-minute consultations with his staff via speakerphone in open court, ‘on the record,' with the defendant taking in every word.) No advance arrangements had been made for a private line in a private place, and even if one could ‘perhaps' have been provided, it would have required a special request by Van Patten and, apparently, a break in the proceedings. In short, this was not an auspicious setting for someone about to waive very valuable constitutional rights.'' \emph{Van Patten} v. \emph{Deppisch,} 434 F. 3d 1038, 1042--1043 (CA7 2006).\newpage See also \emph{Van Patten} v. \emph{Deppisch,} No. 04--1276, 2006 U. S. App. LEXIS 5147 (CA7, Feb. 27, 2006) (noting that no member of the Seventh Circuit requested a vote on the warden's petition for rehearing en banc). In my view, this interpretation is correct. The fact that in 1984, when \emph{Cronic} was decided, neither the parties nor the Court contemplated representation by attorneys who were not present in the flesh explains the author's failure to add the words ``in open court'' after the word ``present.''}

  As the Court explains today, however, the question is not the reasonableness of the federal court's interpretation of \emph{Cronic,} but rather whether the Wisconsin court's narrower reading of that opinion was ``objectively unreasonable.'' \emph{Williams} v. \emph{Taylor,} 529 U.~S. 362, 409 (2000). In light of \emph{Cronic}'s references to the ``complete denial of counsel'' and ``totally absent'' counsel, 466 U. S., at 659, and n. 25, and the opinion's failure to state more explicitly that the defendant is entitled to ``the presence of counsel [in open court],'' \emph{id.,} at 662, I acquiesce in this Court's conclusion that the statecourt decision was not an unreasonable application of clearly\newpage established federal law. In doing so, however, I emphasize that today's opinion does not say that the state courts' interpretation of \emph{Cronic} was correct, or that we would have accepted that reading if the case had come to us on direct review rather than by way of 28 U.~S.~C. \S~2254. See \emph{ante,} at 126; see also \emph{Williams,} 529 U. S., at 410 (``[A]n \emph{unreasonable} application of federal law is different from an \emph{incorrect} application of federal law'').
