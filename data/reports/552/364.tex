% Opinion of the Court
% Breyer

\setcounter{page}{367}

  \textsc{Justice Breyer} delivered the opinion of the Court.

  We here consider whether a federal statute that prohibits States from
enacting any law ``related to'' a motor carrier ``price, route,
or service'' pre-empts two provisions of a Maine tobacco law, which
regulate the delivery of tobacco to customers within the State. 49
U.~S.~C. \S\S~14501(c)(1), 41713(b)(4)(A); see Me. Rev. Stat. Ann.,
Tit. 22, \S\S~1555-- C(3)(C), 1555--D (second sentence) (2004). We
hold that the federal law pre-empts both provisions.

\section{I}

\subsection{A}

  In 1978, Congress ``determin[ed] that ‘maximum reliance on
competitive market forces' '' would favor lower airline fares
and better airline service, and it enacted the Airline \newpage 
Deregulation Act. \emph{Morales} v. \emph{Trans World Airlines, Inc.,} 504
U. S. 374, 378 (1992) (quoting 49 U.~S.~C. App. \S~1302(a)(4) (1988
ed.)); see 92 Stat. 1705. In order to ``ensure that the States would
not undo federal deregulation with regulation of their own,'' that Act
``included a pre-emption provision'' that said ``no State .~.~.
shall enact or enforce any law\dots relating to rates, routes, or
services of any air carrier.'' \emph{Morales, supra,} at 378; 49 U. S.
C. App. \S~1305(a)(1) (1988 ed.).

  In 1980, Congress deregulated trucking. See Motor Carrier Act
of 1980, 94 Stat. 793. And a little over a decade later, in 1994,
Congress similarly sought to pre-empt state trucking regulation. See
Federal Aviation Administration Authorization Act of 1994, 108 Stat.
1605--1606; see also ICC Termination Act of 1995, 109 Stat. 899.
In doing so, it borrowed language from the Airline Deregulation Act
of 1978 and wrote into its 1994 law language that says: ``[A] State
.~.~. may not enact or enforce a law\dots related to a price,
route, or service of any motor carrier\dots with respect to the
transportation of property.'' 49 U.~S.~C. \S~14501(c)(1); see
also \S~41713(b)(4)(A) (similar provision for combined motor-air
carriers).

  The State of Maine subsequently adopted An Act To Regulate the
Delivery and Sales of Tobacco Products and To Prevent the Sale of
Tobacco Products to Minors, 2003 Me. Acts p. 1089, two sections of
which are relevant here. The first section forbids anyone other than
a Maine-licensed tobacco retailer to accept an order for delivery of
tobacco. Me. Rev. Stat. Ann., Tit. 22, \S~1555--C(1). It then adds
that, when a licensed retailer accepts an order and ships tobacco, the
retailer must ``utilize a delivery service'' that provides a special
kind of \emph{recipient-verification} service. \S~1555--C(3)(C).
The delivery service must make certain that (1) the person who bought
the tobacco is the person to whom the package is addressed; (2) the
person to whom the package is addressed is of legal age to purchase
tobacco; (3) the person to whom \newpage  the package is addressed has
himself or herself signed for the package; and (4) the person to whom
the package is addressed, if under the age of 27, has produced a valid
government-issued photo identification with proof of age. \emph{Ibid.}
Violations are punishable by civil penalties. See \S\S~1555--C(3)(E)
to C(3)(F) (first offense up to \$1,500; subsequent offenses up to
\emph{$5,000).

  The second section forbids any person ``knowingly'' to
``transport'' a ``tobacco product'' to ``a person'' in Maine
unless either the sender or the receiver has a Maine license.
\S~1555--D. It then adds that a ``person is \emph{deemed to know} that
a package contains a tobacco product'' (1) if the package is marked
as containing tobacco and displays the name and license number of a
Maine-licensed tobacco retailer; or (2) if the person receives the
package from someone whose name appears on a list of \emph{un}-licensed
tobacco retailers that Maine's attorney general distributes to various
package-delivery companies. \emph{Ibid.} (emphasis added); see also
\S\S~1555-- C(3)(B), 1555--D(1). Violations are again punishable
by civil penalties. \S~1555--D(2) (up to \$1,500 per violation against
violator and/or violator's employer).

\subsection{B}

  Respondents, several transport carrier associations, brought this
lawsuit in federal court, claiming that federal law pre-empts several
sections of Maine's statute. The District Court held (among other
things) that federal law preempts the portions of the two sections we
have described, namely, the ``recipient-verification'' provision
(\S~1555-- C(3)(C)) and the ``deemed to know'' provision (the second
sentence of \S~1555--D). See 377 F. Supp. 2d 197, 220 (Me. 2005).
On appeal, the Court of Appeals for the First Circuit agreed that
federal law pre-empted the two provisions. 448 F. 3d 66, 82 (2006).
We granted certiorari to review these determinations. 551 U.~S. 1144
(2007). \newpage 

\section{II}

\subsection{A}

  In \emph{Morales,} this Court interpreted the pre-emption provision in
the Airline Deregulation Act of 1978. See 504 U. S., at 378. And
we follow \emph{Morales} in interpreting similar language in the 1994 Act
before us here. We have said that ``when judicial interpretations have
settled the meaning of an existing statutory provision, repetition of
the same language in a new statute indicates, as a general matter,
the intent to incorporate its judicial interpretations as well.''
\emph{Merrill Lynch, Pierce, Fenner \& Smith Inc.} v. \emph{Dabit,}
547 U.~S. 71, 85 (2006) (internal quotation marks and alteration
omitted). Here, the Congress that wrote the language before us copied
the language of the air-carrier pre-emption provision of the Airline
Deregulation Act of 1978. Compare 49 U.~S.~C. \S\S~14501(c)(1),
41713(b)(4)(A), with 49 U.~S.~C. App. \S~1305(a)(1) (1988 ed.); see
also H. R. Conf. Rep. No. 103--677, pp. 82--83, 85 (1994) (hereinafter
H. R. Conf. Rep.). And it did so fully aware of this Court's
interpretation of that language as set forth in \emph{Morales.} See H. R.
Conf. Rep., at 83 (motor carriers will enjoy ``the identical intrastate
preemption of prices, routes and services as that originally contained
in'' the Airline Deregulation Act); \emph{ibid.} (expressing agreement
with ``the broad preemption interpretation adopted by the United States
Supreme Court in \emph{Morales}''); \emph{id.,} at 85.

  In \emph{Morales,} the Court determined: (1) that ``[s]tate enforcement
actions \emph{having a connection with, or reference to\\,'' carrier
`` ‘rates, routes, or services' are pre-empted,'' 504 U. S.,
at 384 (emphasis added); (2) that such pre-emption may occur even
if a state law's effect on rates, routes, or services ``is only
indirect,'' \emph{id.,} at 386 (internal quotation marks omitted); (3)
that, in respect to pre-emption, it makes no difference whether a state
law is ``consistent'' or ``inconsistent'' with federal regulation,
\emph{id.,} at 386--387 (emphasis deleted); \newpage  and (4) that
pre-emption occurs at least where state laws have a ``significant
impact'' related to Congress' deregulatory and pre-emption-related
objectives, \emph{id.,} at 390. The Court described Congress'
overarching goal as helping ensure transportation rates, routes,
and services that reflect ``maximum reliance on competitive market
forces,'' thereby stimulating ``efficiency, innovation, and low
prices,'' as well as ``variety'' and ``quality.'' \emph{Id.,} at
378 (internal quotation marks omitted). \emph{Morales} held that, given
these principles, federal law pre-empts States from enforcing their
consumer-fraud statutes against deceptive airline-fare advertisements.
\emph{Id.,} at 391. See \emph{American Airlines, Inc.} v. \emph{Wolens,} 513
U. S. 219, 226--228 (1995) (federal law pre-empts application of a
State's general consumer-protection statute to an airline's frequent
flyer program).

  Finally, \emph{Morales} said that federal law might not pre-empt state
laws that affect fares in only a ``tenuous, remote, or peripheral
.~.~. manner,'' such as state laws forbidding gambling. 504 U.
S., at 390 (internal quotation marks omitted). But the Court did not
say where, or how, ``it would be appropriate to draw the line,'' for
the state law before it did not ``present a borderline question.''
\emph{Ibid.} (internal quotation marks omitted); see also \emph{Wolens,
supra,} at 226.

\subsection{B}

  In light of \emph{Morales,} we find that federal law pre-empts the Maine
laws at issue here. Section 1555--C(3)(C) of the Maine statute forbids
licensed tobacco retailers to employ a ``delivery service'' unless
that service follows particular delivery procedures. Me. Rev. Stat.
Ann., Tit. 22, \S~1555--C(3)(C). In doing so, it focuses on trucking
and other motor-carrier services (which make up a substantial portion of
all ``delivery services,'' \S~1551(1--C)), thereby creating a
direct ``connection with'' motor-carrier services. See \emph{Morales,}
504 U. S., at 384.

  At the same time, the provision has a ``significant'' and adverse
``impact'' in respect to the federal Act's ability to \newpage 
achieve its pre-emption-related objectives. \emph{Id.,} at 390. The
Solicitor General and the carrier associations claim (and Maine does not
deny) that the law will require carriers to offer a system of services
that the market does not now provide (and which the carriers would
prefer not to offer). And even were that not so, the law would freeze
into place services that carriers might prefer to discontinue in the
future. The Maine law thereby produces the very effect that the federal
law sought to avoid, namely, a State's direct substitution of its own
governmental commands for ``competitive market forces'' in determining
(to a significant degree) the services that motor carriers will provide.
\emph{Id.,} at 378 (internal quotation marks omitted).

  We concede that the regulation here is less ``direct'' than
it might be, for it tells \emph{shippers} what to choose rather than
\emph{carriers} what to do. Nonetheless, the effect of the regulation is
that carriers will have to offer tobacco delivery services that differ
significantly from those that, in the absence of the regulation, the
market might dictate. And that being so, ``treating sales restrictions
and purchase restrictions differently for pre-emption purposes would
make no sense.'' \emph{Engine Mfrs. Assn.} v. \emph{South Coast Air
Quality Management Dist.,} 541 U.~S. 246, 255 (2004). If federal
law preempts state efforts to regulate, and consequently to affect,
the advertising \emph{about} carrier rates and services at issue in
\emph{Morales,} it must pre-empt Maine's efforts to regulate carrier
delivery services themselves.

  Section 1555--D's ``deemed to know'' provision applies yet more
directly to motor-carrier services. The provision creates a conclusive
presumption of carrier knowledge that a shipment contains tobacco when
it is marked as originating from a Maine-licensed tobacco retailer or
is sent by anyone Maine has specifically identified as an unlicensed
tobacco retailer. That presumption means that the Maine law imposes
civil liability upon the carrier, not simply for its knowing \newpage 
transport of (unlicensed) tobacco, but for the carrier's \emph{failure}
\emph{sufficiently to examine every package.} The provision thus requires
the carrier to check each shipment for certain markings and to compare
it against the Maine attorney general's list of proscribed shippers.
And it thereby directly regulates a significant aspect of the motor
carrier's package pickup and delivery service. In this way it creates
the kind of state-mandated regulation that the federal Act pre-empts.

  Maine replies that the regulation will impose no significant
additional costs upon carriers. But even were that so (and the
carriers deny it), Maine's reply is off the mark. As with the
recipient-verification provision, the ``deemed to know'' provision
would freeze in place and immunize from competition a service-related
system that carriers do not (or in the future might not) wish to
provide. \emph{Supra,} at 371--372. To allow Maine to insist that the
carriers provide a special checking system would allow other States
to do the same. And to interpret the federal law to permit these, and
similar, state requirements could easily lead to a patchwork of state
service-determining laws, rules, and regulations. That state regulatory
patchwork is inconsistent with Congress' major legislative effort to
leave such decisions, where federally unregulated, to the competitive
marketplace. See H. R. Conf. Rep., at 87. If federal law pre-empts
state regulation of the details of an air carrier's frequent flyer
program, a program that primarily \emph{promotes} carriage, see \emph{Wolens,
supra,} at 226--228, it must pre-empt state regulation of the
essential details of a motor carrier's system for picking up, sorting,
and carrying goods---essential details of the carriage itself.

\subsection{C}

  Maine's primary arguments focus upon the \emph{reason why} it has
enacted the provisions in question. Maine argues for an exception from
pre-emption on the ground that its laws help it prevent minors from
obtaining cigarettes. In Maine's \newpage  view, federal law does not
pre-empt a State's efforts to protect its citizens' public health,
particularly when those laws regulate so dangerous an activity as
underage smoking.

  Despite the importance of the public health objective, we cannot
agree with Maine that the federal law creates an exception on that
basis, exempting state laws that it would otherwise pre-empt. The
Act says nothing about a public health exception. To the contrary,
it explicitly lists a set of exceptions (governing motor vehicle
safety, certain local route controls, and the like), but the list says
nothing about public health. See 49 U.~S.~C. \S\S~14501(c)(2)
to (c)(3); see also \S~41713(b)(4)(B). Maine suggests that the
provision's history indicates that Congress' primary concern
was not with the sort of law it has enacted, but instead with state
``economic'' regulation. See, \emph{e. g.,} H. R. Conf. Rep., at 88;
see also \emph{Columbus} v. \emph{Ours Garage \& Wrecker Service, Inc.,} 536
U. S. 424, 440 (2002). But it is frequently difficult to distinguish
between a State's ``economic''-related and ``health''-related
motivations, see \emph{infra,} at 375, and, indeed, the parties
vigorously dispute Maine's actual motivation for the laws at issue
here. Consequently, it is not surprising that Congress declined to
insert the term ``economic'' into the operative language now before
us, despite having at one time considered doing so. See S. Rep. No.
95--631, p. 171 (1978) (reprinting Senate bill).

  Maine's argument for an implied ``public health'' or ``tobacco''
exception to federal pre-emption rests largely upon (1) legislative
history containing a list of nine States, with laws resembling
Maine's, that Congress thought did not regulate ``intrastate prices,
routes and services of motor carriers,'' see H. R. Conf. Rep., at
86; and (2) the Synar Amendment, a law that denies States federal
funds unless they forbid sales of tobacco to minors, see 42 U.~S.~C.
\S\S~300x-- 26(a)(1), (b)(1). The legislative history, however,
does not suggest Congress made a firm judgment about, or even focused
upon, the issue now before us. And the Synar \newpage  Amendment nowhere
mentions the particular state enforcement method here at issue; indeed,
it does not mention specific state enforcement methods at all.

  Maine's inability to find significant support for some kind of
``public health'' exception is not surprising. ``Public health''
does not define itself. Many products create ``public health'' risks
of differing kind and degree. To accept Maine's justification in
respect to a rule regulating services would legitimate rules regulating
routes or rates for similar public health reasons. And to allow Maine
directly to regulate carrier services would permit other States
to do the same. Given the number of States through which carriers
travel, the number of products, the variety of potential adverse
public health effects, the many different kinds of regulatory rules
potentially available, and the difficulty of finding a legal criterion
for separating permissible from impermissible public-health-oriented
regulations, Congress is unlikely to have intended an implicit general
``public health'' exception broad enough to cover even the shipments
at issue here.

  This is not to say that this federal law generally pre-empts state
public health regulation: for instance, state regulation that broadly
prohibits certain forms of conduct and affects, say, truckdrivers, only
in their capacity as members of the public (\\e. g.,} a prohibition on
smoking in certain public places). We have said that federal law does
not pre-empt state laws that affect rates, routes, or services in ``too
tenuous, remote, or peripheral a manner.'' \emph{Morales,} 504 U. S., at
390 (internal quotation marks omitted). And we have written that the
state laws whose ``effect'' is ``forbidden'' under federal law are
those with a ``\emph{significant} impact'' on carrier rates, routes, or
services. \emph{Id.,} at 388, 390 (emphasis added).

  In this case, the state law is not general, it does not affect
truckers solely in their capacity as members of the general public, the
impact is significant, and the connection with trucking is not tenuous,
remote, or peripheral. The state \newpage  statutes aim directly at the
carriage of goods, a commercial field where carriage by commercial motor
vehicles plays a major role. The state statutes require motor-carrier
operators to perform certain services, thereby limiting their ability
to provide incompatible alternative services; and they do so simply
because the State seeks to enlist the motor-carrier operators as
allies in its enforcement efforts. Given these circumstances, from the
perspective of pre-emption, this case is no more ``borderline'' than
was \emph{Morales.} \emph{Id.,} at 390 (internal quotation marks omitted);
see also \emph{Wolens,} 513 U. S., at 226.

  Maine adds that it possesses legal authority to prevent \emph{any}
tobacco shipments from entering into or moving within the State, and
that the broader authority must encompass the narrower authority to
regulate the \emph{manner} of tobacco shipments. But even assuming purely
for argument's sake that Maine possesses the broader authority, its
conclusion does not follow. To accept that conclusion would permit Maine
to regulate carrier routes, carrier rates, and carrier services, all on
the ground that such regulation would not restrict carriage of the goods
as seriously as would a total ban on shipments. And it consequently
would severely undermine the effectiveness of Congress' pre-emptive
provision. Indeed, it would create the very exception that we have just
rejected, extending that exception to all other products a State might
ban. We have explained why we do not believe Congress intended that
result. \emph{Supra,} at 373--375 and this page.

  Finally, Maine says that to set aside its regulations will seriously
harm its efforts to prevent cigarettes from falling into the hands
of minors. The Solicitor General denies that this is so. He suggests
that Maine, like other States, can prohibit all persons from providing
tobacco products to minors (as it already has, see Me. Rev. Stat.
Ann., Tit. 22, \S~1555--B(2) (Supp. 2007)); that it can ban all
non-face-to-face sales of tobacco; that it might pass other laws
of general \newpage  (non-carrier-specific) applicability; and that
it can, if necessary, seek appropriate federal regulation (see,
\emph{e. g.,} H. R. 4081, 110th Cong., 1st Sess. (2007) (proposed bill
regulating tobacco shipment); H. R. 4128, 110th Cong., 1st Sess.,
\S\S~1411--1416, pp. 577--583 (2007) (proposed bill providing
criminal penalties for trafficking in contraband tobacco)).
Regardless, given \emph{Morales,} where the Court held that federal law
pre-empts state consumer-protection laws, we find that federal law must
also pre-empt Maine's efforts directly to regulate carrier services.

  For these reasons, the judgment of the Court of Appeals is affirmed.

\begin{flushright}\emph{It is so ordered.}\end{flushright}
