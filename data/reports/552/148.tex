% Court
% Kennedy

\setcounter{page}{152}

  \textsc{Justice Kennedy} delivered the opinion of the Court.

  We consider the reach of the private right of action the Court has found implied in \S~10(b) of the Securities Exchange Act of 1934, 48 Stat. 891, as amended, 15 U.~S.~C. \S~78j(b), and SEC Rule 10b--5, 17 CFR \S~240.10b--5 (2007). In this suit investors alleged losses after purchasing common stock. They sought to impose liability on entities who, acting both as customers and suppliers, agreed to arrangements that allowed the investors' company to mislead its auditor and issue \newpage  a misleading financial statement affecting the stock price. We conclude the implied right of action does not reach the customer/supplier companies because the investors did not rely upon their statements or representations. We affirm the judgment of the Court of Appeals.

\section{I}

  This class-action suit by investors was filed against Charter Communications, Inc., in the United States District Court for the Eastern District of Missouri. Stoneridge Investment Partners, LLC, a limited liability company organized under the laws of Delaware, was the lead plaintiff and is petitioner here.

  Charter issued the financial statements and the securities in question. It was a named defendant along with some of its executives and Arthur Andersen LLP, Charter's independent auditor during the period in question. We are concerned, though, with two other defendants, respondents here. Respondents are Scientific-Atlanta, Inc., and Motorola, Inc. They were suppliers, and later customers, of Charter.

  For purposes of this proceeding, we take these facts, alleged by petitioner, to be true. Charter, a cable operator, engaged in a variety of fraudulent practices so its quarterly reports would meet Wall Street expectations for cable subscriber growth and operating cashflow. The fraud included misclassification of its customer base; delayed reporting of terminated customers; improper capitalization of costs that should have been shown as expenses; and manipulation of the company's billing cutoff dates to inflate reported revenues. In late 2000, Charter executives realized that, despite these efforts, the company would miss projected operating cashflow numbers by \$15 to \$20 million. To help meet the shortfall, Charter decided to alter its existing arrangements with respondents, Scientific-Atlanta and Motorola. Peti\newpage tioner's theory as to whether Arthur Andersen was altogether misled or, on the other hand, knew the structure of the contract arrangements and was complicit to some degree, is not clear at this stage of the case. The point, however, is neither controlling nor significant for our present disposition, and in our decision we assume it was misled.

  Respondents supplied Charter with the digital cable converter (set-top) boxes that Charter furnished to its customers. Charter arranged to overpay respondents \$20 for each set-top box it purchased until the end of the year, with the understanding that respondents would return the overpayment by purchasing advertising from Charter. The transactions, it is alleged, had no economic substance; but, because Charter would then record the advertising purchases as revenue and capitalize its purchase of the set top boxes, in violation of generally accepted accounting principles, the transactions would enable Charter to fool its auditor into approving a financial statement showing it met projected revenue and operating cashflow numbers. Respondents agreed to the arrangement.

  So that Arthur Andersen would not discover the link between Charter's increased payments for the boxes and the advertising purchases, the companies drafted documents to make it appear the transactions were unrelated and conducted in the ordinary course of business. Following a request from Charter, Scientific-Atlanta sent documents to Charter stating---falsely---that it had increased production costs. It raised the price for set-top boxes for the rest of 2000 by \$20 per box. As for Motorola, in a written contract Charter agreed to purchase from Motorola a specific number of set-top boxes and pay liquidated damages of \$20 for each unit it did not take. The contract was made with the expectation Charter would fail to purchase all the units and pay Motorola the liquidated damages.

  To return the additional money from the set top box sales, Scientific-Atlanta and Motorola signed contracts with Char\newpage ter to purchase advertising time for a price higher than fair value. The new set-top box agreements were backdated to make it appear that they were negotiated a month before the advertising agreements. The backdating was important to convey the impression that the negotiations were unconnected, a point Arthur Andersen considered necessary for separate treatment of the transactions. Charter recorded the advertising payments to inflate revenue and operating cashflow by approximately \$17 million. The inflated number was shown on financial statements filed with the Securities and Exchange Commission (SEC) and reported to the public.

  Respondents had no role in preparing or disseminating Charter's financial statements. And their own financial statements booked the transactions as a wash, under generally accepted accounting principles. It is alleged respondents knew or were in reckless disregard of Charter's intention to use the transactions to inflate its revenues and knew the resulting financial statements issued by Charter would be relied upon by research analysts and investors.

  Petitioner filed a securities fraud class action on behalf of purchasers of Charter stock alleging that, by participating in the transactions, respondents violated \S~10(b) of the Securities Exchange Act of 1934 and SEC Rule 10b--5.

  The District Court granted respondents' motion to dismiss for failure to state a claim on which relief can be granted. The United States Court of Appeals for the Eighth Circuit affirmed. \emph{In re Charter Communications, Inc., Securities Litigation,} 443 F. 3d 987 (2006). In its view the allegations did not show that respondents made misstatements relied upon by the public or that they violated a duty to disclose; and on this premise it found no violation of \S~10(b) by respondents. \emph{Id.,} at 992. At most, the court observed, respondents had aided and abetted Charter's misstatement of its financial results; but, it noted, there is no private right of action for aiding and abetting a \S~10(b) violation. See \emph{Central Bank of Denver, N. A.} v. \emph{First Interstate Bank of Den\newpage ver, N. A.,} 511 U.~S. 164, 191 (1994). The court also affirmed the District Court's denial of petitioner's motion to amend the complaint, as the revised pleading would not change the court's conclusion on the merits. 443 F. 3d, at 993.

  Decisions of the Courts of Appeals are in conflict respecting when, if ever, an injured investor may rely upon \S~10(b) to recover from a party that neither makes a public misstatement nor violates a duty to disclose but does participate in a scheme to violate \S~10(b). Compare \emph{Simpson} v. \emph{AOL Time Warner Inc.,} 452 F. 3d 1040 (CA9 2006), with \emph{Regents of Univ.ofCal.} v. \emph{Credit Suisse First Boston (USA), Inc.,} 482 F. 3d 372 (CA5 2007). We granted certiorari. 549 U. S. 1304 (2007).

\section{II}

  Section 10(b) of the Securities Exchange Act makes it

    ``unlawful for any person, directly or indirectly, by the us of any means or instrumentality of interstate commerce or of th mails, or of any facility of any national securities exchang

    \hrule

    ``[t]o use or employ, in connection with the purchase or sale of any security\dots any manipulative or deceptive device or contrivance in contravention of such rules and regulations as the Commission may prescribe as necessary or appropriate in the public interest or for the protection of investors.'' 15 U.~S.~C. \S~78j.

\noindent The SEC, pursuant to this section, promulgated Rule 10b--5, which makes it unlawful

      ``(a) To employ any device, scheme, or artifice to defraud,

      ``(b) To make any untrue statement of a material fact or to omit to state a material fact necessary in order to make the statements made, in the light of the circumstances under which they were made, not misleading, or

      \newpage  ``(c) To engage in any act, practice, or course of business which operates or would operate as a fraud or deceit upon any person, ``in connection with the purchase or sale of any security.'' 17 CFR \S~240.10b--5.

\noindent Rule 10b--5 encompasses only conduct already prohibited by \S~10(b). \emph{United States} v. \emph{O'Hagan,} 521 U.~S. 642, 651 (1997). Though the text of the Securities Exchange Act does not provide for a private cause of action for \S~10(b) violations, the Court has found a right of action implied in the words of the statute and its implementing regulation. \emph{Superintendent of Ins. of N.Y.} v. \emph{Bankers Life \& Casualty Co.,} 404 U.~S. 6, 13, n. 9 (1971). In a typical \S~10(b) private action a plaintiff must prove (1) a material misrepresentation or omission by the defendant; (2) scienter; (3) a connection between the misrepresentation or omission and the purchase or sale of a security; (4) reliance upon the misrepresentation or omission; (5) economic loss; and (6) loss causation. See \emph{Dura Pharmaceuticals, Inc.} v. \emph{Broudo,} 544 U.~S. 336, 341--342 (2005).

  In \emph{Central Bank,} the Court determined that \S~10(b) liability did not extend to aiders and abettors. The Court found the scope of \S~10(b) to be delimited by the text, which makes no mention of aiding and abetting liability. 511 U. S., at 177. The Court doubted the implied \S~10(b) action should extend to aiders and abettors when none of the express causes of action in the securities Acts included that liability. \emph{Id.,} at 180. It added the following:

    \begin{quote}

		``Were we to allow the aiding and abetting action proposed in this case, the defendant could be liable without any showing that the plaintiff relied upon the aider and abettor's statements or actions. See also \emph{Chiarella} [v. \emph{United States,} 445 U. S. 222, 228 (1980)]. Allowing plaintiffs to circumvent the reliance requirement would disregard the careful limits on 10b--5 recovery mandated by our earlier cases.'' \emph{Ibid.}\newpage 

    \end{quote}

  The decision in \emph{Central Bank} led to calls for Congress to create an express cause of action for aiding and abetting within the Securities Exchange Act. Then-SEC Chairman Arthur Levitt, testifying before the Senate Securities Subcommittee, cited \emph{Central Bank} and recommended that aiding and abetting liability in private claims be established. S. Hearing No. 103--759, pp. 13--14 (1994). Congress did not follow this course. Instead, in \S~104 of the Private Securities Litigation Reform Act of 1995 (PSLRA), 109 Stat. 757, it directed prosecution of aiders and abettors by the SEC. 15 U.~S.~C. \S~78t(e).

  The \S~10(b) implied private right of action does not extend to aiders and abettors. The conduct of a secondary actor must satisfy each of the elements or preconditions for liability; and we consider whether the allegations here are sufficient to do so.

\section{III}

  The Court of Appeals concluded petitioner had not alleged that respondents engaged in a deceptive act within the reach of the \S~10(b) private right of action, noting that only misstatements, omissions by one who has a duty to disclose, and manipulative trading practices (where ``manipulative'' is a term of art, see, \emph{e. g., Santa Fe Industries, Inc.} v. \emph{Green,} 430 U.~S. 462, 476--477 (1977)) are deceptive within the meaning of the Rule. 443 F. 3d, at 992. If this conclusion were read to suggest there must be a specific oral or written statement before there could be liability under \S~10(b) or Rule 10b--5, it would be erroneous. Conduct itself can be deceptive, as respondents concede. In this case, moreover, respondents' course of conduct included both oral and written statements, such as the backdated contracts agreed to by Charter and respondents.

  A different interpretation of the holding from the Court of Appeals opinion is that the court was stating only that any deceptive statement or act respondents made was not actionable because it did not have the requisite proximate relation \newpage  to the investors' harm. That conclusion is consistent with our own determination that respondents' acts or statements were not relied upon by the investors and that, as a result, liability cannot be imposed upon respondents.

\section{A}

  Reliance by the plaintiff upon the defendant's deceptive acts is an essential element of the \S~10(b) private cause of action. It ensures that, for liability to arise, the ``requisite causal connection between a defendant's misrepresentation and a plaintiff's injury'' exists as a predicate for liability. \emph{Basic Inc.} v. \emph{Levinson,} 485 U.~S. 224, 243 (1988); see also \emph{Affiliated Ute Citizens of Utah} v. \emph{United States,} 406 U.~S. 128, 154 (1972) (requiring ``causation in fact''). We have found a rebuttable presumption of reliance in two different circumstances. First, if there is an omission of a material fact by one with a duty to disclose, the investor to whom the duty was owed need not provide specific proof of reliance. \emph{Id.,} at 153--154. Second, under the fraud-on-the-market doctrine, reliance is presumed when the statements at issue become public. The public information is reflected in the market price of the security. Then it can be assumed that an investor who buys or sells stock at the market price relies upon the statement. \emph{Basic, supra,} at 247.

  Neither presumption applies here. Respondents had no duty to disclose; and their deceptive acts were not communicated to the public. No member of the investing public had knowledge, either actual or presumed, of respondents' deceptive acts during the relevant times. Petitioner, as a result, cannot show reliance upon any of respondents' actions except in an indirect chain that we find too remote for liability.

\section{B}

  Invoking what some courts call ``scheme liability,'' see, \emph{e. g., In re Enron Corp. Securities, Derivative, \& ``ERISA''} \emph{Litigation,} 439 F. Supp. 2d 692, 723 (SD Tex. 2006), peti\newpage tioner nonetheless seeks to impose liability on respondents even absent a public statement. In our view this approach does not answer the objection that petitioner did not in fact rely upon respondents' own deceptive conduct.

  Liability is appropriate, petitioner contends, because respondents engaged in conduct with the purpose and effect of creating a false appearance of material fact to further a scheme to misrepresent Charter's revenue. The argument is that the financial statement Charter released to the public was a natural and expected consequence of respondents' deceptive acts; had respondents not assisted Charter, Charter's auditor would not have been fooled, and the financial statement would have been a more accurate reflection of Charter's financial condition. That causal link is sufficient, petitioner argues, to apply \emph{Basic}'s presumption of reliance to respondents' acts. See, \emph{e. g., Simpson,} 452 F. 3d, at 1051--1052; \emph{In re Parmalat Securities Litigation,} 376 F. Supp. 2d 472, 509 (SDNY 2005).

  In effect petitioner contends that in an efficient market investors rely not only upon the public statements relating to a security but also upon the transactions those statements reflect. Were this concept of reliance to be adopted, the implied cause of action would reach the whole marketplace in which the issuing company does business; and there is no authority for this rule.

  As stated above, reliance is tied to causation, leading to the inquiry whether respondents' acts were immediate or remote to the injury. In considering petitioner's arguments, we note \S~10(b) provides that the deceptive act must be ``in connection with the purchase or sale of any security.'' 15 U.~S.~C. \S~78j(b). Though this phrase in part defines the statute's coverage rather than causation (and so we do not evaluate the ``in connection with'' requirement of \S~10(b) in this case), the emphasis on a purchase or sale of securities does provide some insight into the deceptive acts that concerned the enacting Congress. See Black, Securities Commentary: \newpage  The Second Circuit's Approach to the ‘In Connection With' Requirement of Rule 10b--5, 53 Brooklyn L. Rev. 539, 541 (1987) (``[W]hile the ‘in connection with' and causation requirements are analytically distinct, they are related to each other, and discussion of the first requirement may merge with discussion of the second''). In all events we conclude respondents' deceptive acts, which were not disclosed to the investing public, are too remote to satisfy the requirement of reliance. It was Charter, not respondents, that misled its auditor and filed fraudulent financial statements; nothing respondents did made it necessary or inevitable for Charter to record the transactions as it did.

  Petitioner invokes the private cause of action under \S~10(b) and seeks to apply it beyond the securities markets---the realm of financing business---to purchase and supply contracts---the realm of ordinary business operations. The latter realm is governed, for the most part, by state law. It is true that if business operations are used, as alleged here, to affect securities markets, the SEC enforcement power may reach the culpable actors. It is true as well that a dynamic, free economy presupposes a high degree of integrity in all of its parts, an integrity that must be underwritten by rules enforceable in fair, independent, accessible courts. Were the implied cause of action to be extended to the practices described here, however, there would be a risk that the federal power would be used to invite litigation beyond the immediate sphere of securities litigation and in areas already governed by functioning and effective state-law guarantees. Our precedents counsel against this extension. See \emph{Marine Bank} v. \emph{Weaver,} 455 U. S. 551, 556 (1982) (``Congress, in enacting the securities laws, did not intend to provide a broad federal remedy for all fraud''); \emph{Santa Fe,} 430 U. S., at 479--480 (``There may well be a need for uniform federal fiduciary standards~.~.~.~. But those standards should not be supplied by judicial extension of \S~10(b) and Rule 10b--5 to ‘cover the corporate universe' '' (quoting Cary, Federalism \newpage and Corporate Law: Reflections Upon Delaware, 83 Yale L. J. 663, 700 (1974))). Though \S~10(b) is ``not ‘limited to preserving the integrity of the securities markets,' '' \emph{Bankers Life,} 404 U. S., at 12, it does not reach all commercial transactions that are fraudulent and affect the price of a security in some attenuated way.

  These considerations answer as well the argument that if this were a common-law action for fraud there could be a finding of reliance. Even if the assumption is correct, it is not controlling. Section 10(b) does not incorporate common-law fraud into federal law. See, \emph{e. g., SEC} v. \emph{Zandford,} 535 U.~S. 813, 820 (2002) (``[Section 10(b)] must not be construed so broadly as to convert every common-law fraud that happens to involve securities into a violation''); \emph{Central} \emph{Bank,} 511 U. S., at 184 (``Even assuming\dots a deeply rooted background of aiding and abetting tort liability, it does not follow that Congress intended to apply that kind of liability to the private causes of action in the securities Acts''); see also \emph{Dura,} 544 U. S., at 341. Just as \S~10(b) ``is surely badly strained when construed to provide a cause of action\dots to the world at large,'' \emph{Blue Chip Stamps} v. \emph{Manor Drug Stores,} 421 U.~S. 723, 733, n. 5 (1975), it should not be interpreted to provide a private cause of action against the entire marketplace in which the issuing company operates.

  Petitioner's theory, moreover, would put an unsupportable interpretation on Congress' specific response to \emph{Central Bank} in \S~104 of the PSLRA. Congress amended the securities laws to provide for limited coverage of aiders and abettors. Aiding and abetting liability is authorized in actions brought by the SEC but not by private parties. See 15 U.~S.~C. \S~78t(e). Petitioner's view of primary liability makes any aider and abettor liable under \S~10(b) if he or she committed a deceptive act in the process of providing assistance. Reply Brief for Petitioner 6, n. 2; Tr. of Oral Arg. 24. Were we to adopt this construction of \S~10(b), it would revive in substance the implied cause of action against all aiders \newpage  and abettors except those who committed no deceptive act in the process of facilitating the fraud; and we would undermine Congress' determination that this class of defendants should be pursued by the SEC and not by private litigants. See \emph{Alexander} v. \emph{Sandoval,} 532 U.~S. 275, 290 (2001) (``The express provision of one method of enforcing a substantive rule suggests that Congress intended to preclude others''); \emph{FDA} v. \emph{Brown \& Williamson Tobacco Corp.,} 529 U.~S. 120, 143 (2000) (``At the time a statute is enacted, it may have a range of plausible meanings. Over time, however, subsequent acts can shape or focus those meanings''); see also \emph{Seatrain Shipbuilding Corp.} v. \emph{Shell Oil Co.,} 444 U. S. 572, 596 (1980) (``[W]hile the views of subsequent Congresses cannot override the unmistakable intent of the enacting one, such views are entitled to significant weight, and particularly so when the precise intent of the enacting Congress is obscure'' (citations omitted)).

  This is not a case in which Congress has enacted a regulatory statute and then has accepted, over a long period of time, broad judicial authority to define substantive standards of conduct and liability. Cf. \emph{Leegin Creative Leather Products, Inc.} v. \emph{PSKS, Inc.,} 551 U. S. 877, 899 (2007). And in accord with the nature of the cause of action at issue here, we give weight to Congress' amendment to the Act restoring aiding and abetting liability in certain cases but not others. The amendment, in our view, supports the conclusion that there is no liability.

  The practical consequences of an expansion, which the Court has considered appropriate to examine in circumstances like these, see \emph{Virginia Bankshares, Inc.} v. \emph{Sandberg,} 501 U.~S. 1083, 1104--1105 (1991); \emph{Blue Chip,} 421 U. S., at 737, provide a further reason to reject petitioner's approach. In \emph{Blue Chip,} the Court noted that extensive discovery and the potential for uncertainty and disruption in a lawsuit allow plaintiffs with weak claims to extort settlements from innocent companies. \emph{Id.,} at 740--741. Adoption \newpage  of petitioner's approach would expose a new class of defendants to these risks. As noted in \emph{Central Bank,} contracting parties might find it necessary to protect against these threats, raising the costs of doing business. See 511 U. S., at 189. Overseas firms with no other exposure to our securities laws could be deterred from doing business here. See Brief for Organization for International Investment et al. as \emph{Amici Curiae} 17--20. This, in turn, may raise the cost of being a publicly traded company under our law and shift securities offerings away from domestic capital markets. Brief for NASDAQ Stock Market, Inc., et al. as \emph{Amici Curiae} 12--14.

\section{C}

  The history of the \S~10(b) private right and the careful approach the Court has taken before proceeding without congressional direction provide further reasons to find no liability here. The \S~10(b) private cause of action is a judicial construct that Congress did not enact in the text of the relevant statutes. See \emph{Lampf, Pleva, Lipkind, Prupis \& Petigrow} v. \emph{Gilbertson,} 501 U.~S. 350, 358--359 (1991); \emph{Blue Chip, supra,} at 729. Though the rule once may have been otherwise, see \emph{J. I. Case Co.} v. \emph{Borak,} 377 U.~S. 426, 432--433 (1964), it is settled that there is an implied cause of action only if the underlying statute can be interpreted to disclose the intent to create one, see, \emph{e. g., Alexander, supra,} at 286--287; \emph{Virginia Bankshares, supra,} at 1102; \emph{Touche Ross \& Co.} v. \emph{Redington,} 442 U.~S. 560, 575 (1979). This is for good reason. In the absence of congressional intent the Judiciary's recognition of an implied private right of action

    \begin{quote}

		``necessarily extends its authority to embrace a dispute Congress has not assigned it to resolve. This runs contrary to the established principle that ‘[t]he jurisdiction of the federal courts is carefully guarded against expansion by judicial interpretation\dots ,' \emph{American Fire \& Cas[ualty] Co.} v. \emph{Finn,} 341 U.~S. 6, 17 (1951), and con\newpage flicts with the authority of Congress under Art. III to set the limits of federal jurisdiction.'' \emph{Cannon} v. \emph{University of Chicago,} 441 U.~S. 677, 746--747 (1979) (Powell, J., dissenting) (citations and footnote omitted).

    \end{quote}

\noindent The determination of who can seek a remedy has significant consequences for the reach of federal power. See \emph{Wilder} v. \emph{Virginia Hospital Assn.,} 496 U.~S. 498, 509, n. 9 (1990) (requirement of congressional intent ``reflects a concern, grounded in separation of powers, that Congress rather than the courts controls the availability of remedies for violations of statutes'').

  Concerns with the judicial creation of a private cause of action caution against its expansion. The decision to extend the cause of action is for Congress, not for us. Though it remains the law, the \S~10(b) private right should not be extended beyond its present boundaries. See \emph{Virginia Bankshares, supra,} at 1102 (``[T]he breadth of the [private right of action] once recognized should not, as a general matter, grow beyond the scope congressionally intended''); see also \emph{Central Bank, supra,} at 173 (determining that the scope of conduct prohibited is limited by the text of \S~10(b)).

  This restraint is appropriate in light of the PSLRA, which imposed heightened pleading requirements and a loss causation requirement upon ``any private action'' arising from the Securities Exchange Act. See 15 U.~S.~C. \S~78u--4(b). It is clear these requirements touch upon the implied right of action, which is now a prominent feature of federal securities regulation. See \emph{Merrill Lynch, Pierce, Fenner \& Smith Inc.} v. \emph{Dabit,} 547 U.~S. 71, 81--82 (2006); \emph{Dura,} 544 U. S., at 345--346; see also S. Rep. No. 104--98, pp. 4--5 (1995) (recognizing the \S~10(b) implied cause of action, and indicating the PSLRA was intended to have ``Congress\dots reassert its authority in this area''); \emph{id.,} at 26 (indicating the pleading standards covered \S~10(b) actions). Congress thus ratified the implied right of action after the Court moved away from a broad willingness to imply private rights of action. See \newpage  \emph{Merrill Lynch, Pierce, Fenner \& Smith, Inc.} v. \emph{Curran,} 456 U.~S. 353, 381--382, and n. 66 (1982); cf. \emph{Borak, supra,} at 433. It is appropriate for us to assume that when \S~78u--4 was enacted, Congress accepted the \S~10(b) private cause of action as then defined but chose to extend it no further.

\section{IV}

  Secondary actors are subject to criminal penalties, see, \emph{e. g.,} 15 U.~S.~C. \S~78ff, and civil enforcement by the SEC, see, \emph{e. g.,} \S~78t(e). The enforcement power is not toothless. Since September 30, 2002, SEC enforcement actions have collected over \$10 billion in disgorgement and penalties, much of it for distribution to injured investors. See SEC, 2007 Performance and Accountability Report, p. 26, http://www.sec.gov/about/secpar2007.shtml (as visited Jan. 2, 2008, and available in Clerk of Court's case file). And in this case both parties agree that criminal penalties are a strong deterrent. See Brief for Respondents 48; Reply Brief for Petitioner 17. In addition some state securities laws permit state authorities to seek fines and restitution from aiders and abettors. See, \emph{e. g.,} Del. Code Ann., Tit. 6, \S~7325 (2005). All secondary actors, furthermore, are not necessarily immune from private suit. The securities statutes provide an express private right of action against accountants and underwriters in certain circumstances, see 15 U.~S.~C. \S~77k, and the implied right of action in \S~10(b) continues to cover secondary actors who commit primary violations, \emph{Central Bank,} 511 U. S., at 191.

  Here respondents were acting in concert with Charter in the ordinary course as suppliers and, as matters then evolved in the not so ordinary course, as customers. Unconventional as the arrangement was, it took place in the marketplace for goods and services, not in the investment sphere. Charter was free to do as it chose in preparing its books, conferring with its auditor, and preparing and then issuing its financial statements. In these circumstances the investors cannot be \newpage  said to have relied upon any of respondents' deceptive acts in the decision to purchase or sell securities; and as the requisite reliance cannot be shown, respondents have no liability to petitioner under the implied right of action. This conclusion is consistent with the narrow dimensions we must give to a right of action Congress did not authorize when it first enacted the statute and did not expand when it revisited the law.

  The judgment of the Court of Appeals is affirmed, and the case is remanded for further proceedings consistent with this opinion.

\begin{flushright}\emph{It is so ordered.}\end{flushright}
