% Opinion of the Court
% Scalia

\setcounter{page}{198}

  \textsc{Justice Scalia} delivered the opinion of the Court.

  The State of New York requires that political parties select their nominees for Supreme Court Justice at a convention of delegates chosen by party members in a primary election. We consider whether this electoral system violates the First Amendment rights of prospective party candidates.

\section{I}

\subsection{A}

  The Supreme Court of New York is the State's trial court of general jurisdiction, with an Appellate Division that hears appeals from certain lower courts. See N. Y. Const., Art. \newpage  VI, \S\S~7, 8. Under New York's current Constitution, the State is divided into 12 judicial districts, see Art. VI, \S~6(a); N. Y. Jud. Law Ann. \S~140 (West 2005), and Supreme Court Justices are elected to 14-year terms in each such district, see N. Y. Const., Art. VI, \S~6(c). The New York Legislature has provided for the election of a total of 328 Supreme Court Justices in this fashion. See N. Y. Jud. Law Ann. \S~140--a (West Supp. 2007).

  Over the years, New York has changed the method by which Supreme Court Justices are selected several times. Under the New York Constitution of 1821, Art. IV, \S~7, all judicial officers, except Justices of the Peace, were appointed by the Governor with the consent of the Senate. See 7 Sources and Documents of the U. S. Constitutions 181, 184--185 (W. Swindler ed. 1978). In 1846, New York amended its Constitution to require popular election of the Justices of the Supreme Court (and also the Judges of the New York Court of Appeals). \emph{Id.,} at 192, 200 (N. Y. Const. of 1846, Art. VI, \S~12). In the early years under that regime, the State allowed political parties to choose their own method of selecting the judicial candidates who would bear their endorsements on the general-election ballot. See, \emph{e. g.,} Report of Joint Committee of Senate and Assembly of New York, Appointed to Investigate Primary and Election Laws of This and Other States, S. Doc. No. 26, pp. 195--219 (1910). The major parties opted for party conventions, the same method then employed to nominate candidates for other state offices. \emph{Ibid.;} see also P. Ray, An Introduction to Political Parties and Practical Politics 94 (1913).

  In 1911, the New York Legislature enacted a law requiring political parties to select Supreme Court nominees (and most other nominees who did not run statewide) through direct primary elections. Act of Oct. 18, 1911, ch. 891, \S~45(4), 1911 N. Y. Laws pp. 2657, 2682. The primary system came to be criticized as a ``device capable of astute and successful manipulation by professionals,'' Editorial, The State Conven\newpage tion, N. Y. Times, May 1, 1917, p. 12, and the Republican candidate for Governor in 1920 campaigned against it as ``a fraud'' that `` ‘offered the opportunity for two things, for the demagogue and the man with money,' '' Miller Declares Primary a Fraud, N. Y. Times, Oct. 23, 1920, p. 4. A law enacted in 1921 required parties to select their candidates for the Supreme Court by a convention composed of delegates elected by party members. Act of May 2, 1921, ch. 479, \S\S~45(1), 110, 1921 N. Y. Laws pp. 1451, 1454, 1471.

  New York retains this system of choosing party nominees for Supreme Court Justice to this day. Section 6--106 of New York's election law sets forth its basic operation: ``Party nominations for the office of justice of the supreme court shall be made by the judicial district convention.'' N. Y. Elec. Law Ann. \S~6--106 (West 2007). A ``party'' is any political organization whose candidate for Governor received 50,000 or more votes in the most recent election. \S~1--104(3). In a September ``delegate primary,'' party members elect delegates from each of New York's 150 assembly districts to attend the party's judicial convention for the judicial district in which the assembly district is located. See N. Y. State Law Ann. \S~121 (West 2003); N. Y. Elec. Law Ann. \S\S~6--124, 8--100(1)(a) (West 2007). An individual may run for delegate by submitting to the Board of Elections a designating petition signed by 500 enrolled party members residing in the assembly district, or by five percent of such enrolled members, whichever is less. \S\S~6--136(2)(i), (3). These signatures must be gathered within a 37-day period preceding the filing deadline, which is approximately two months before the delegate primary. \S\S~6--134(4), 6--158(1). The delegates elected in these primaries are uncommitted; the primary ballot does not specify the judicial nominee whom they will support. \S~7--114.

  The nominating conventions take place one to two weeks after the delegate primary. \S\S~6--126, 6--158(5). Each of the 12 judicial districts has its own convention to nominate the \newpage  party's Supreme Court candidate or candidates who will run at large in that district in the general election. \S\S~6--124, 6--156. The general election takes place in November. \S~8--100(1)(c). The nominees from the party conventions appear automatically on the general-election ballot. \S~7--104(5). They may be joined on the general-election ballot by independent candidates and candidates of political organizations that fail to meet the 50,000 vote threshold for ``party'' status; these candidates gain access to the ballot by submitting timely nominating petitions with (depending on the judicial district) 3,500 or 4,000 signatures from voters in that district or signatures from five percent of the number of votes cast for Governor in that district in the prior election, whichever is less. \S\S~6--138, 6--142(2).

\subsection{B}

  Respondent López Torres was elected in 1992 to the civil court for Kings County---a court with more limited jurisdiction than the Supreme Court---having gained the nomination of the Democratic Party through a primary election. She claims that soon after her election, party leaders began to demand that she make patronage hires, and that her consistent refusal to do so caused the local party to oppose her unsuccessful candidacy at the Supreme Court nominating conventions in 1997, 2002, and 2003. The following year, López Torres---together with other candidates who had failed to secure the nominations of their parties, voters who claimed to have supported those candidates, and the New York branch of a public-interest organization called Common Cause---brought suit in federal court against the New York Board of Elections, which is responsible for administering and enforcing the New York election law. See \S\S~3--102, 3--104. They contended that New York's election law burdened the rights of challengers seeking to run against candidates favored by the party leadership, and deprived voters and candidates of their rights to gain access to the ballot and to associate in choosing their party's candidates. As rele\newpage vant here, they sought a declaration that New York's convention system for selecting Supreme Court Justices violates their First Amendment rights, and an injunction mandating the establishment of a direct primary election to select party nominees for Supreme Court Justice.

  The District Court issued a preliminary injunction granting the relief requested, pending the New York Legislature's enactment of a new statutory scheme. 411 F. Supp. 2d 212, 256 (EDNY 2006). A unanimous panel of the United States Court of Appeals for the Second Circuit affirmed. 462 F. 3d 161 (2006). It held that voters and candidates possess a First Amendment right to a ``realistic opportunity to participate in [a political party's] nominating process, and to do so free from burdens that are both severe and unnecessary.'' \emph{Id.,} at 187. New York's electoral law violated that right because of the quantity of signatures and delegate recruits required to obtain a Supreme Court nomination at a judicial convention, see \emph{id.,} at 197, and because of the apparent reality that party leaders can control delegates, see \emph{id.,} at 198--200. In the court's view, because ``one-party rule'' prevailed within New York's judicial districts, a candidate had a constitutional right to gain access to the party's convention, notwithstanding her ability to get on the general-election ballot by petition signatures. \emph{Id.,} at 193--195, 200. The Second Circuit's holding effectively returned New York to the system of electing Supreme Court Justices that existed before the 1921 amendments to the election law. We granted certiorari. 549 U. S. 1204 (2007).

\section{II}

\subsection{A}

  A political party has a First Amendment right to limit its membership as it wishes, and to choose a candidate-selection process that will in its view produce the nominee who best represents its political platform. \emph{Democratic Party of United States} v. \emph{Wisconsin ex rel. La Follette,} 450 U.~S. 107, \newpage  122 (1981); \emph{California Democratic Party} v. \emph{Jones,} 530 U.~S. 567, 574--575 (2000). These rights are circumscribed, however, when the State gives the party a role in the election process---as New York has done here by giving certain parties the right to have their candidates appear with party endorsement on the general-election ballot. Then, for example, the party's racially discriminatory action may become state action that violates the Fifteenth Amendment. See \emph{id.,} at 573. And then also the State acquires a legitimate governmental interest in ensuring the fairness of the party's nominating process, enabling it to prescribe what that process must be. \emph{Id.,} at 572--573. We have, for example, considered it to be ``too plain for argument'' that a State may prescribe party use of primaries or conventions to select nominees who appear on the general-election ballot. \emph{American Party of Tex.} v. \emph{White,} 415 U.~S. 767, 781 (1974). That prescriptive power is not without limits. In \emph{Jones,} for example, we invalidated on First Amendment grounds California's blanket primary, reasoning that it permitted nonparty-members to determine the candidate bearing the party's standard in the general election. 530 U. S., at 577. See also \emph{Eu} v. \emph{San Francisco County Democratic Central Comm.,} 489 U.~S. 214, 224 (1989); \emph{Tashjian} v. \emph{Republican Party of Conn.,} 479 U. S. 208, 214--217 (1986).

  In the present case, however, the party's associational rights are at issue (if at all) only as a shield and not as a sword. Respondents are in no position to rely on the right that the First Amendment confers on political parties to structure their internal party processes and to select the candidate of the party's choosing. Indeed, both the Republican and Democratic state parties have intervened from the very early stages of this litigation to defend New York's electoral law. The weapon wielded by these plaintiffs is their \emph{own} claimed associational right not only to join, but to have a certain degree of influence in, the party. They contend that New York's electoral system does not go far enough---\newpage  does not go as far as the Constitution demands---in ensuring that they will have a fair chance of prevailing in their parties' candidate-selection process.

  This contention finds no support in our precedents. We have indeed acknowledged an individual's associational right to vote in a party primary without undue state-imposed impediment. In \emph{Kusper} v. \emph{Pontikes,} 414 U.~S. 51, 57 (1973), we invalidated an Illinois law that required a voter wishing to change his party registration so as to vote in the primary of a different party to do so almost two full years before the primary date. But \emph{Kusper} does not cast doubt on all stateimposed limitations upon primary voting. In \emph{Rosario} v. \emph{Rockefeller,} 410 U.~S. 752 (1973), we upheld a New York State requirement that a voter have enrolled in the party of his choice at least 30 days before the previous general election in order to vote in the next party primary. In any event, respondents do not claim that they have been excluded from voting in the primary. Moreover, even if we extended \emph{Kusper} to cover not only the right to vote in the party primary but also the right to run, the requirements of the New York law (a 500-signature petition collected during a 37-day window in advance of the primary) are entirely reasonable. Just as States may require persons to demonstrate ``a significant modicum of support'' before allowing them access to the general-election ballot, lest it become unmanageable, \emph{Jenness} v. \emph{Fortson,} 403 U.~S. 431, 442 (1971), they may similarly demand a minimum degree of support for candidate access to a primary ballot. The signature requirement here is far from excessive. See, \emph{e. g., Norman} v. \emph{Reed,} 502 U.~S. 279, 295 (1992) (approving requirement of 25,000 signatures, or approximately two percent of the electorate); \emph{White, supra,} at 783 (approving requirement of one percent of the vote cast for Governor in the preceding general election, which was about 22,000 signatures).

  Respondents' real complaint is not that they cannot vote in the election for delegates, nor even that they cannot run \newpage  in that election, but that the convention process that follows the delegate election does not give them a realistic chance to secure the party's nomination. The party leadership, they say, inevitably garners more votes for its slate of delegates (delegates uncommitted to any judicial nominee) than the unsupported candidate can amass for himself. And thus the leadership effectively determines the nominees. But this says nothing more than that the party leadership has more widespread support than a candidate not supported by the leadership. No New York law compels election of the leadership's slate---or, for that matter, compels the delegates elected on the leadership's slate to vote the way the leadership desires. And no state law prohibits an unsupported candidate from attending the convention and seeking to persuade the delegates to support her. Our cases invalidating ballot-access requirements have focused on the requirements themselves, and not on the manner in which political actors function under those requirements. See, \emph{e. g., Bullock} v. \emph{Carter,} 405 U.~S. 134 (1972) (Texas statute required exorbitant filing fees); \emph{Williams} v. \emph{Rhodes,} 393 U.~S. 23 (1968) (Ohio statute required, \emph{inter alia,} excessive number of petition signatures); \emph{Anderson} v. \emph{Celebrezze,} 460 U. S. 780 (1983) (Ohio statute established unreasonably early filing deadline). Here respondents complain not of the state law, but of the voters' (and their elected delegates') preference for the choices of the party leadership.

  To be sure, we have, as described above, permitted States to set their faces against ``party bosses'' by requiring partycandidate selection through processes more favorable to insurgents, such as primaries. But to say that the State can require this is a far cry from saying that the Constitution demands it. None of our cases establishes an individual's constitutional right to have a ``fair shot'' at winning the party's nomination. And with good reason. What constitutes a ``fair shot'' is a reasonable enough question for legislative judgment, which we will accept so long as it does not too \newpage  much infringe upon the party's associational rights. But it is hardly a manageable constitutional question for judges---especially for judges in our legal system, where traditional electoral practice gives no hint of even the existence, much less the content, of a constitutional requirement for a ``fair shot'' at party nomination. Party conventions, with their attendant ``smoke-filled rooms'' and domination by party leaders, have long been an accepted manner of selecting party candidates. ``National party conventions prior to 1972 were generally under the control of state party leaders'' who determined the votes of state delegates. American Presidential Elections: Process, Policy, and Political Change 14 (H. Schantz ed. 1996). Selection by convention has never been thought unconstitutional, even when the delegates were not selected by primary but by party caucuses. See \emph{ibid.}

  The Second Circuit's judgment finesses the difficulty of saying how much of a shot is a ``fair shot'' by simply mandating a primary until the New York Legislature acts. This was, according to the Second Circuit, the New York election law's default manner of party-candidate selection for offices whose manner of selection is not otherwise prescribed. Petitioners question the propriety of this mandate, but we need not pass upon that here. Even conceding its propriety, there is good reason to believe that the elected members of the New York Legislature remain opposed to the primary, for the same reasons their predecessors abolished it 86 years ago: because it leaves judicial selection to voters uninformed about judicial qualifications, and places a high premium upon the ability to raise money. Should the New York Legislature persist in that view, and adopt something different from a primary and closer to the system that the Second Circuit invalidated, the question whether \emph{that} provides enough of a ``fair shot'' would be presented. We are not inclined to open up this new and excitingly unpredictable theater of election jurisprudence. Selection by convention has been a traditional means of choosing party nominees. While a State \newpage  may determine it is not desirable and replace it, it is not unconstitutional.

\subsection{B}

  Respondents put forward, as a special factor which gives them a First Amendment right to revision of party processes in the present case, the assertion that party loyalty in New York's judicial districts renders the general-election ballot ``uncompetitive.'' They argue that the existence of entrenched ``one-party rule'' demands that the First Amendment be used to impose additional competition in the nominee-selection process of the parties. (The asserted ``one-party rule,'' we may observe, is that of the Democrats in some judicial districts, and of the Republicans in others. See 411 F. Supp. 2d, at 230.) This is a novel and implausible reading of the First Amendment.

  To begin with, it is hard to understand how the competitiveness of the general election has anything to do with respondents' associational rights in the party's selection process. It makes no difference to the person who associates with a party and seeks its nomination whether the party is a contender in the general election, an underdog, or the favorite. Competitiveness may be of interest to the voters in the general election, and to the candidates who choose to run \emph{against} the dominant party. But we have held that those interests are well enough protected so long as all candidates have an adequate opportunity to appear on the generalelection ballot. In \emph{Jenness} we upheld a petition-signature requirement for inclusion on the general-election ballot of five percent of the eligible voters, see 403 U. S., at 442, and in \emph{Munro} v. \emph{Socialist Workers Party,} 479 U.~S. 189, 199 (1986), we upheld a petition-signature requirement of one percent of the vote in the State's primary. New York's general-election balloting procedures for Supreme Court Justice easily pass muster under this standard. Candidates who fail to obtain a major party's nomination via convention can still get on the general-election ballot for the judicial \newpage  district by providing the requisite number of signatures of voters resident in the district. N. Y. Elec. Law Ann. \S~6--142(2). To our knowledge, outside of the Fourteenth and Fifteenth Amendment contexts, see \emph{Jones,} 530 U. S., at 573, no court has ever made ``one-party entrenchment'' a basis for interfering with the candidate-selection processes of a party. (Of course, the \emph{lack} of one-party entrenchment will not cause free access to the general-election ballot to validate an otherwise unconstitutional restriction upon participation in a party's nominating process. See \emph{Bullock,} 405 U. S., at 146--147.)

  The reason one-party rule is entrenched may be (and usually is) that voters approve of the positions and candidates that the party regularly puts forward. It is no function of the First Amendment to require revision of those positions or candidates. The States can, within limits (that is, short of violating the parties' freedom of association), discourage party monopoly---for example, by refusing to show party endorsement on the election ballot. But the Constitution provides no authority for federal courts to prescribe such a course. The First Amendment creates an open marketplace where ideas, most especially political ideas, may compete without government interference. See \emph{Abrams} v. \emph{United States,} 250 U.~S. 616, 630 (1919) (Holmes, J., dissenting). It does not call on the federal courts to manage the market by preventing too many buyers from settling upon a single product.

  Limiting respondents' court-mandated ``fair shot at party endorsement'' to situations of one-party entrenchment merely multiplies the impracticable lines courts would be called upon to draw. It would add to those alluded to earlier the line at which mere party popularity turns into ``oneparty dominance.'' In the case of New York's election system for Supreme Court Justices, that line would have to be drawn separately for each of the 12 judicial districts---and in those districts that are ``competitive'' the current system \newpage  would presumably remain valid. But why limit the remedy to \emph{one}-party dominance? Does not the dominance of two parties similarly stifle competing opinions? Once again, we decline to enter the morass.

\hrule

  New York State has thrice (in 1846, 1911, and 1921) displayed a willingness to reconsider its method of selecting Supreme Court Justices. If it wishes to return to the primary system that it discarded in 1921, it is free to do so; but the First Amendment does not compel that. We reverse the Second Circuit's contrary judgment.

\begin{flushright}\emph{It is so ordered.}\end{flushright}
