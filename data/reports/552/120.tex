% Court
% Per Curiam

\setcounter{page}{120}

  \textsc{Per Curiam.}

  The Court of Appeals for the Seventh Circuit held that respondent Joseph Van Patten was entitled to relief under\newpage 28 U.~S.~C. \S~2254, reasoning that his lawyer's assistance was presumptively ineffective owing to his participation in a plea hearing by speakerphone. \emph{Van Patten} v. \emph{Deppisch,} 434 F. 3d 1038 (2006). We granted certiorari, vacated the judgment, and remanded the case for further consideration in light of \emph{Carey} v. \emph{Musladin,} 549 U.~S. 70 (2006). On remand, the Seventh Circuit adhered to its original decision, concluding that ``[n]othing in \emph{Musladin} requires that our 2006 opinion be changed.'' \emph{Van Patten} v. \emph{Endicott,} 489 F. 3d 827, 828 (2007) \emph{(per curiam)}. We grant the petition for certiorari now before us and this time reverse the judgment of the Seventh Circuit.

\section{I}

  Van Patten was charged with first-degree intentional homicide and pleaded no contest to a reduced charge of firstdegree reckless homicide. His counsel was not physically present at the plea hearing but was linked to the courtroom by speakerphone. After the state trial court imposed the maximum term of 25 years in prison, Van Patten retained different counsel and moved in the Wisconsin Court of Appeals to withdraw his no-contest plea. The thrust of the motion was that Van Patten's Sixth Amendment right to counsel had been violated by his trial counsel's physical absence from the plea hearing. The Wisconsin Court of Appeals noted that, under state law, a postconviction motion to withdraw a no-contest plea will be granted only if a defendant establishes ``manifest injustice'' by clear and convincing evidence. See \emph{State} v. \emph{Van Pattten,} No. 96--3036--CR (May 28, 1997), App. to Pet. for Cert. A47--A48. While the court acknowledged that ``the violation of the defendant's Sixth Amendment right to counsel may constitute a manifest injustice,'' \emph{id.,} at A48, it found that the absence of Van Patten's lawyer from the plea hearing did not violate his right to counsel: ``The plea hearing transcript neither indicates any de ficiency in the plea colloquy, nor suggests that Van Pat\newpage ten's attorney's participation by telephone interfered in any way with [Van Patten's] ability to communicate with his attorney about his plea. Van Patten confirmed that he had thoroughly discussed his case and plea decision with his attorney and was satisfied with the legal repre sentation he had received. The court gave Van Patten the opportunity to speak privately with his attorney over the phone if he had questions about the plea, but Van Patten declined. Further, when Van Patten exer cised his right to allocution at sentencing, in the per sonal presence of his attorney, he raised no objection to his plea.'' \emph{Id.,} at A49--A50. Applying \emph{Strickland} v. \emph{Washington,} 466 U.~S. 668 (1984), the court concluded that ``[t]he record does not support, nor does Van Patten's appellate brief include, any argument that counsel's performance was deficient or prejudicial,'' No. 96--3036--CR, App. to Pet. for Cert. A51, and denied Van Patten's motion.

  After the Wisconsin Supreme Court declined further review, Van Patten petitioned for a writ of habeas corpus under 28 U.~S.~C. \S~2254 in Federal District Court. The District Court denied relief, but the Court of Appeals for the Seventh Circuit reversed. It held that Van Patten's Sixth Amendment claim should have been resolved, not under \emph{Strickland}'s two-pronged test (which requires a showing of deficient performance and prejudice to the defendant), but under the standard discussed in \emph{United States} v. \emph{Cronic,} 466 U.~S. 648 (1984) (under which prejudice may be presumed). Although the Seventh Circuit recognized that this case ``presents [a] novel .~.~. question,'' \emph{Deppisch,} 434 F. 3d, at 1040, and conceded that ``[u]nder \emph{Strickland,} it seems clear [that] Van Patten would have no viable claim,'' \emph{id.,} at 1042, the court concluded that ``it is clear to us that Van Patten's case must be resolved under \emph{Cronic,}'' \emph{id.,} at 1043. The resolution was in Van Patten's favor.

  \newpage While the prison warden's petition for certiorari was pending, this Court decided \emph{Musladin, supra.} Musladin had invoked this Court's cases recognizing ``that certain courtroom practices are so inherently prejudicial that they deprive the defendant of a fair trial,'' \emph{id.,} at 72. The issue was the significance of these precedents in a case under \S~2254, which bars relief on any claim ``adjudicated on the merits'' in state court, unless the state court's decision ``was contrary to, or involved an unreasonable application of, clearly established Federal law, as determined by the Supreme Court of the United States.'' 28 U.~S.~C. \S~2254(d)(1).

  The prejudicial conduct involved in \emph{Musladin} was courtroom conduct of private actors. We held that the ``inheren[t] prejudic[e]'' test, which we thus far have applied only in cases involving government-sponsored conduct, see\emph{,e. g., Estelle} v. \emph{Williams,} 425 U.~S. 501 (1976); \emph{Holbrook} v. \emph{Flynn,} 475 U.~S. 560 (1986), did not clearly extend to the conduct of independently acting courtroom spectators. See \emph{Musladin, supra,} at 76 (``[A]lthough the Court articulated the test for inherent prejudice that applies to state conduct in \emph{Williams} and \emph{Flynn,} we have never applied that test to spectators' conduct. Indeed, part of the legal test of \emph{Williams} and \emph{Flynn}---asking whether the practices furthered an essential \emph{state} interest---suggests that those cases apply only to state-sponsored practices''). For that reason, we reversed the Court of Appeals' grant of habeas relief.

  \emph{Musladin}'s explanation of the ``clearly established Federal law'' requirement prompted us to remand Van Patten's case to the Seventh Circuit for further consideration. A majority of the panel reaffirmed its original judgment, however, on the ground that ``[u]nlike \emph{Musladin,} this case does not concern an open constitutional question,'' because ``[t]he Supreme Court has long recognized a defendant's right to relief if his defense counsel was actually or constructively absent at a critical stage of the proceedings.'' 489 F. 3d, at 828.\newpage Judge Coffey disagreed, observing that ``the United States Supreme Court has never held that an attorney is presumed to be ineffective if he participates in a plea hearing by speaker phone rather than by physical appearance.'' \emph{Ibid.} (emphasis deleted). He found that ``[t]he Majority Opinion does not comport with \emph{Musladin,}'' \emph{ibid}.\emph{,} and dissented from ``the court's erroneous decision to allow'' its original opinion ``to stand as written,'' \emph{id.,} at 829. We reach the same conclusion.

\subsection{II}

  \emph{Strickland, supra,} ordinarily applies to claims of ineffective assistance of counsel at the plea hearing stage. See \emph{Hill} v. \emph{Lockhart,} 474 U.~S. 52, 58 (1985) (``[T]he two-part \emph{Strickland} v. \emph{Washington} test applies to challenges to guilty pleas based on ineffective assistance of counsel''). And it was in a different context that \emph{Cronic} ``recognized a narrow exception to \emph{Strickland}'s holding that a defendant who asserts ineffective assistance of counsel must demonstrate not only that his attorney's performance was deficient, but also that the deficiency prejudiced the defense.'' \emph{Florida} v. \emph{Nixon,} 543 U. S. 175, 190 (2004) (discussing \emph{Cronic}). \emph{Cronic} held that a Sixth Amendment violation may be found ``without inquiring into counsel's actual performance or requiring the defendant to show the effect it had on the trial,'' \emph{Bell} v. \emph{Cone,} 535 U. S. 685, 695 (2002), when ``circumstances [exist] that are so likely to prejudice the accused that the cost of litigating their effect in a particular case is unjustified,'' \emph{Cronic, supra,} at 658. \emph{Cronic,} not \emph{Strickland,} applies ``when\dots the likelihood that any lawyer, even a fully competent one, could provide effective assistance is so small that a presumption of prejudice is appropriate without inquiry into the actual conduct of the trial,'' 466 U. S., at 659--660,[[*]] and \newpage  one circumstance warranting the presumption is the ``complete denial of counsel,'' that is, when ``counsel [is] either totally absent, or prevented from assisting the accused during a critical stage of the proceeding,'' \emph{id.,} at 659, and n. 25.

^* \emph{Cronic} also applies when ``there [is] a breakdown in the adversarial process,'' 466 U. S., at 662, such that ``counsel entirely fails to subject the prosecution's case to meaningful adversarial testing,'' \emph{id.,} at 659. We have made clear that ``[w]hen we spoke in \emph{Cronic} of the possibility of\newpage presuming prejudice based on an attorney's failure to test the prosecutor's case, we indicated that the attorney's failure must be complete.'' \emph{Bell} v. \emph{Cone,} 535 U.~S. 685, 696--697 (2002). It is undisputed that this standard has not been met here.

  No decision of this Court, however, squarely addresses the issue in this case, see \emph{Deppisch,} 434 F. 3d, at 1040 (noting that this case ``presents [a] novel\dots question''), or clearly establishes that \emph{Cronic} should replace \emph{Strickland} in this novel factual context. Our precedents do not clearly hold that counsel's participation by speakerphone should be treated as a ``complete denial of counsel,'' on par with total absence. Even if we agree with Van Patten that a lawyer physically present will tend to perform better than one on the phone, it does not necessarily follow that mere telephone contact amounted to total absence or ``prevented [counsel] from assisting the accused,'' so as to entail application of \emph{Cronic.} The question is not whether counsel in those circumstances will perform less well than he otherwise would, but whether the circumstances are likely to result in such poor performance that an inquiry into its effects would not be worth the time. Cf. \emph{United States} v. \emph{Gonzalez-Lopez,} 548 U.~S. 140, 147 (2006) (Sixth Amendment ensures ``\emph{effective} (not mistake-free) representation'' (emphasis in original)). Our cases provide no categorical answer to this question, and for that matter the several proceedings in this case hardly point toward one. The Wisconsin Court of Appeals held counsel's performance by speakerphone to be constitutionally effective; neither the Magistrate Judge, the District Court, nor the Seventh Circuit disputed this conclusion; and the Seventh Circuit itself stated that ``[u]nder \emph{Strickland,} it seems clear Van Patten would have no viable claim,'' \emph{Deppisch, supra,} at 1042.

  \newpage Because our cases give no clear answer to the question presented, let alone one in Van Patten's favor, ``it cannot be said that the state court ‘unreasonabl[y] appli[ed] clearly established Federal law.' '' \emph{Musladin,} 549 U. S., at 77 (quoting 28 U.~S.~C. \S~2254(d)(1)). Under the explicit terms of \S~2254(d)(1), therefore, relief is unauthorized.

\hrule

  Petitioner tells us that ``[i]n urging review, [the State] does not condone, recommend, or encourage the practice of defense counsel assisting clients by telephone rather than in person at court proceedings, even in nonadversarial hearings such as the plea hearing in this case,'' Pet. for Cert. 5, and he acknowledges that ``[p]erhaps, under similar facts in a direct federal appeal, the Seventh Circuit could have properly reached the same result it reached here,'' \emph{ibid}\emph{.} Our own consideration of the merits of telephone practice, however, is for another day, and this case turns on the recognition that no clearly established law contrary to the state court's conclusion justifies collateral relief.

  The judgment is reversed, and the case is remanded for further proceedings consistent with this opinion.

\begin{flushright}\emph{It is so ordered.}\end{flushright}
