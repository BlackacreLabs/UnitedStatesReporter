% Court
% Ginsburg

\setcounter{page}{90}

  \textsc{Justice Ginsburg} delivered the opinion of the Court.

  This Court's remedial opinion in \emph{United States} v. \emph{Booker,}
543 U.~S. 220, 244 (2005), instructed district courts to read the
United States Sentencing Guidelines as ``effectively advisory,''
\emph{id.,} at 245. In accord with 18 U.~S.~C. \S~3553(a), the
Guidelines, formerly mandatory, now serve as one factor among several
courts must consider in determining an appropriate sentence. \emph{Booker}
further instructed that ``reason\newpage ableness'' is the standard
controlling appellate review of the sentences district courts impose.

  Under the statute criminalizing the manufacture and distribution of
crack cocaine, 21 U.~S.~C. \S~841, and the relevant Guidelines
prescription, \S~2D1.1, a drug trafficker dealing in crack cocaine is
subject to the same sentence as one dealing in 100 times more powder
cocaine. The question here presented is whether, as the Court of
Appeals held in this case, ``a sentence\dots outside the guidelines
range is per se unreasonable when it is based on a disagreement with
the sentencing disparity for crack and powder cocaine offenses.''
174 Fed. Appx. 798, 799 (CA4 2006) \emph{(per curiam).} We hold that,
under \emph{Booker,} the cocaine Guidelines, like all other Guidelines,
are advisory only, and that the Court of Appeals erred in holding
the crack/powder disparity effectively mandatory. A district judge
must include the Guidelines range in the array of factors warranting
consideration. The judge may determine, however, that, in the particular
case, a within-Guidelines sentence is ``greater than necessary'' to
serve the objectives of sentencing. 18 U.~S.~C. \S~3553(a) (2000
ed. and Supp. V). In making that determination, the judge may consider
the disparity between the Guidelines' treatment of crack and powder
cocaine offenses.

\section{I}

  In September 2004, petitioner Derrick Kimbrough was indicted in the
United States District Court for the Eastern District of Virginia and
charged with four offenses: conspiracy to distribute crack and powder
cocaine; possession with intent to distribute more than 50 grams of
crack cocaine; possession with intent to distribute powder cocaine; and
possession of a firearm in furtherance of a drug-trafficking offense.
Kimbrough pleaded guilty to all four charges.

  Under the relevant statutes, Kimbrough's plea subjected him to
an aggregate sentence of 15 years to life in prison: 10 years to
life for the three drug offenses, plus a consecutive \newpage  term of
5 years to life for the firearm offense.In order to determine the
appropriate sentence within this statutory range, the District Court
first calculated Kimbrough's sentence under the advisory Sentencing
Guidelines.\footnotemark[2] Kimbrough's guilty plea acknowledged that he was
accountable for 56 grams of crack cocaine and 92.1 grams of powder
cocaine. This quantity of drugs yielded a base offense level of 32 for
the three drug charges. See United States Sentencing Commission,
Guidelines Manual \S~2D1.1(c) (Nov. 2004) (USSG). Finding that
Kimbrough, by asserting sole culpability for the crime, had testified
falsely at his codefendant's trial, the District Court increased his
offense level to 34. See \S~3C1.1. In accord with the presentence
report, the court determined that Kimbrough's criminal history
category was II. An offense level of 34 and a criminal history category
of II yielded a Guidelines range of 168 to 210 months for the three drug
charges. See \emph{id.,} ch. 5, pt. A, Sentencing Table. The Guidelines
sentence for the firearm offense was the statutory minimum, 60 months.
See \S2K2.4(b). Kimbrough's final advisory Guidelines range was
thus 228 to 270 months, or 19 to 22.5 years.

  A sentence in this range, in the District Court's judgment, would
have been ``greater than necessary'' to accomplish the \newpage  purposes
of sentencing set forth in 18 U.~S.~C. \S~3553(a). App. 72. As
required by \S~3553(a), the court took into account the ``nature
and circumstances'' of the offense and Kimbrough's ``history and
characteristics.'' \emph{Id.,} at 72--73. The court also commented
that the case exemplified the ``disproportionate and unjust effect
that crack cocaine guidelines have in sentencing.'' \emph{Id.,} at
72. In this regard, the court contrasted Kimbrough's Guidelines
range of 228 to 270 months with the range that would have applied had
he been accountable for an equivalent amount of powder cocaine: 97 to
106 months, inclusive of the 5-year mandatory minimum for the firearm
charge, see USSG \S~2D1.1(c); \emph{id.,} ch. 5, pt. A, Sentencing
Table. Concluding that the statutory minimum sentence was ``clearly
long enough'' to accomplish the objectives listed in \S~3553(a), the
court sentenced Kimbrough to 15 years, or 180 months, in prison plus 5
years of supervised release. App. 74--75.\footnotemark[3]


^1 The statutory range for possession with intent to distribute more
than 50 grams of crack is ten years to life. See 21 U.~S.~C.
\S~841(b)(1)(A)(iii) (2000 ed. and Supp. V). The same range applies
to the conspiracy offense. See \S~846 (2000 ed.). The statutory
range for possession with intent to distribute powder cocaine is 0 to 20
years. See \S~841(b)(1)(C) (Supp. V). Finally, the statutory range
for possession of a firearm in furtherance of a drug-trafficking offense
is five years to life. See 18 U.~S.~C. \S~924(c)(1)(A)(i).
The sentences for the three drug crimes may run concurrently, see
\S~3584(a), but the sentence for the firearm offense must be
consecutive, see \S~924(c)(1)(A).

^2 Kimbrough was sentenced in April 2005, three months after our
decision in \emph{United States} v. \emph{Booker,} 543 U.~S. 220 (2005),
rendered the Guidelines advisory. The District Court employed the
version of the Guidelines effective November 1, 2004.

  In an unpublished \emph{per curiam} opinion, the Fourth Circuit vacated
the sentence. Under Circuit precedent, the Court of Appeals observed, a
sentence ``outside the guidelines range is per se unreasonable when it
is based on a disagreement with the sentencing disparity for crack and
powder cocaine offenses.'' 174 Fed. Appx., at 799 (citing \emph{United
States} v. \emph{Eura,} 440 F. 3d 625, 633--634 (CA4 2006)).

  We granted certiorari, 551 U.~S. 1113 (2007), to determine
whether the crack/powder disparity adopted in the United States
Sentencing Guidelines has been rendered ``advisory'' by our decision
in \emph{Booker.\\\footnotemark[4] \newpage 

^3 The prison sentence consisted of 120 months on each of the three drug
counts, to be served concurrently, plus 60 months on the firearm count,
to be served consecutively.

^4 This question has divided the Courts of Appeals. Compare \emph{United}
\emph{States} v. \emph{Pickett,} 475 F. 3d 1347, 1355--1356 (CADC 2007)
(District Court erred when it concluded that it had no discretion to
consider the crack/ powder disparity in imposing a sentence), and
\emph{United States} v. \emph{Gunter,} 462 F. 3d 237, 248--249 (CA3 2006)
(same), with \emph{United States} v. \emph{Leatch,} 482 F. 3d 790, 791 (CA5
2007) \emph{(per curiam)} (sentencing court may not \newpage  impose a
sentence outside the Guidelines range based on its disagreement with the
crack/powder disparity), \emph{United States} v. \emph{Johnson,} 474 F. 3d
515, 522 (CA8 2007) (same), \emph{United States} v. \emph{Castillo,} 460 F. 3d
337, 361 (CA2 2006) (same), \emph{United States} v. \emph{Williams,} 456 F.
3d 1353, 1369 (CA11 2006) (same), \emph{United States} v. \emph{Miller,} 450
F. 3d 270, 275--276 (CA7 2006) (same), \emph{United States} v. \emph{Eura,}
440 F. 3d 625, 633--634 (CA4 2006) (same), and \emph{United States} v.
\emph{Pho,} 433 F. 3d 53, 62--63 (CA1 2006) (same).

\section{II}

  We begin with some background on the different treatment of
crack and powder cocaine under the federal sentencing laws.
Crack and powder cocaine are two forms of the same drug. Powder
cocaine, or cocaine hydrochloride, is generally inhaled through the
nose; it may also be mixed with water and injected. See United
States Sentencing Commission, Special Report to Congress: Cocaine
and Federal Sentencing Policy 5, 12 (Feb. 1995), available at
http://www.ussc.gov/crack/exec.htm (hereinafter 1995 Report). (All
Internet materials as visited Dec. 7, 2007, and included in Clerk
of Court's case file.) Crack cocaine, a type of cocaine base, is
formed by dissolving powder cocaine and baking soda in boiling water.
\emph{Id.,} at 14. The resulting solid is divided into single-dose
``rocks'' that users smoke. \emph{Ibid.} The active ingredient in
powder and crack cocaine is the same. \emph{Id.,} at 9. The two forms
of the drug also have the same physiological and psychotropic effects,
but smoking crack cocaine allows the body to absorb the drug much faster
than inhaling powder cocaine, and thus produces a shorter, more intense
high. \emph{Id.,} at 15--19.\footnotemark[5]

  Although chemically similar, crack and powder cocaine are handled
very differently for sentencing purposes. The 100to-1 ratio yields
sentences for crack offenses three to six times longer than those
for powder offenses involving equal amounts of drugs. See United
States Sentencing Commission, Report to Congress: Cocaine and
Federal Sentencing \newpage  Policy iv (May 2002), available at
http://www.ussc.gov/r_congress/02crack/2002crackrpt.pdf (hereinafter
2002 Report).\footnotemark[6] This disparity means that a major supplier of
powder cocaine may receive a shorter sentence than a low-level dealer
who buys powder from the supplier but then converts it to crack. See
1995 Report 193--194.

^5 Injecting powder cocaine produces effects similar to smoking crack
cocaine, but very few powder users inject the drug. See 1995 Report
18.

\section{A}

  The crack/powder disparity originated in the Anti-Drug Abuse Act of
1986 (1986 Act), 100 Stat. 3207. The 1986 Act created a two-tiered
scheme of five-and ten-year mandatory minimum sentences for drug
manufacturing and distribution offenses. Congress sought ``to link the
ten-year mandatory minimum trafficking prison term to major drug dealers
and to link the five-year minimum term to serious traffickers.'' 1995
Report 119. The 1986 Act uses the weight of the drugs involved in the
offense as the sole proxy to identify ``major'' and ``serious''
dealers. For example, any defendant responsible for 100 grams of
heroin is subject to the five-year mandatory minimum, see 21 U. S.
C. \S~841(b)(1)(B)(i) (2000 ed. and Supp. V), and any defendant
responsible for 1,000 grams of heroin is subject to the ten-year
mandatory minimum, see \S~841(b)(1)(A)(i).

  Crack cocaine was a relatively new drug when the 1986 Act was signed
into law, but it was already a matter of great public concern: ``Drug
abuse in general, and crack cocaine in particular, had become in public
opinion and in members' minds a problem of overwhelming dimensions.''
1995 Report 121. Congress apparently believed that crack was
significantly more dangerous than powder cocaine in that: (1) crack was
highly addictive; (2) crack users and dealers were more likely to be
violent than users and dealers of \newpage  other drugs; (3) crack was more
harmful to users than powder, particularly for children who had been
exposed by their mothers' drug use during pregnancy; (4) crack use was
especially prevalent among teenagers; and (5) crack's potency and low
cost were making it increasingly popular. See 2002 Report 90.

^6 As explained in Part II--C, \emph{infra,} the Sentencing Commission
amended the Guidelines and reduced sentences for crack offenses
effective November 1, 2007. Except as noted, this opinion refers to the
2004 Guidelines in effect at the time of Kimbrough's sentencing.

  Based on these assumptions, the 1986 Act adopted a ``100-to-1
ratio'' that treated every gram of crack cocaine as the equivalent
of 100 grams of powder cocaine. The 1986 Act's five-year mandatory
minimum applies to any defendant accountable for 5 grams of crack or
500 grams of powder, 21 U.~S.~C. \S~841(b)(1)(B)(ii), (iii); its
ten-year mandatory minimum applies to any defendant accountable for
50 grams of crack or 5,000 grams of powder, \S~841(b)(1)(A)(ii),
(iii).

  While Congress was considering adoption of the 1986 Act, the
Sentencing Commission was engaged in formulating the Sentencing
Guidelines.\footnotemark[7] In the main, the Commission developed Guidelines
sentences using an empirical approach based on data about past
sentencing practices, including 10,000 presentence investigation
reports. See USSG \S~1A.1, intro. comment., pt. A, ¶3. The
Commission ``modif[ied] and adjust[ed] past practice in the interests
of greater rationality, avoiding inconsistency, complying with
congressional instructions, and the like.'' \emph{Rita} v. \emph{United
States,} 551 U.~S. 338, 349 (2007).

  The Commission did not use this empirical approach in developing the
Guidelines sentences for drug-trafficking offenses. Instead, it employed
the 1986 Act's weight-driven scheme. The Guidelines use a drug
quantity table based on drug type and weight to set base offense levels
for drugtrafficking offenses. See USSG \S~2D1.1(c). In setting
of\newpage  fense levels for crack and powder cocaine, the Commission, in
line with the 1986 Act, adopted the 100-to-1 ratio. The statute itself
specifies only two quantities of each drug, but the Guidelines ``go
further and set sentences for the full range of possible drug quantities
using the same 100-to-1 quantity ratio.'' 1995 Report 1. The
Guidelines' drug quantity table sets base offense levels ranging from
12, for offenses involving less than 250 milligrams of crack (or 25
grams of powder), to 38, for offenses involving more than 1.5 kilograms
of crack (or 150 kilograms of powder). USSG \S~2D1.1(c).\footnotemark[8]

^7 Congress created the Sentencing Commission and charged it with
promulgating the Guidelines in the Sentencing Reform Act of 1984, 98
Stat. 1987, 18 U.~S.~C. \S~3551 \emph{et seq.} (2000 ed. and Supp. V),
but the first version of the Guidelines did not become operative until
November 1987, see 1995 Report iii--iv.

\section{B}

  Although the Commission immediately used the 100-to-1 ratio to
define base offense levels for all crack and powder offenses, it later
determined that the crack/powder sentencing disparity is generally
unwarranted. Based on additional research and experience with the
100-to-1 ratio, the Commission concluded that the disparity ``fails
to meet the sentencing objectives set forth by Congress in both the
Sentencing Reform Act and the 1986 Act.'' 2002 Report 91. In a
series of reports, the Commission identified three problems with the
crack/powder disparity.

  First, the Commission reported, the 100-to-1 ratio rested on
assumptions about ``the relative harmfulness of the two drugs and the
relative prevalence of certain harmful conduct associated with their
use and distribution that more recent research and data no longer
support.'' \emph{Ibid.;} see United States Sentencing Commission,
Report to Congress: Cocaine and Federal Sentencing Policy 8 (May
2007), available at http://www.ussc.gov/r_congress/cocaine2007.pdf
(hereinafter 2007 Report) (ratio Congress embedded in the statute
far ``overstate[s]'' both ``the relative harmfulness'' of
crack co\newpage caine, and the ``seriousness of most crack cocaine
offenses''). For example, the Commission found that crack is
associated with ``significantly less trafficking-related violence
.~.~. than previously assumed.'' 2002 Report 100. It also observed
that ``the negative effects of prenatal crack cocaine exposure
are identical to the negative effects of prenatal powder cocaine
exposure.'' \emph{Id.,} at 94. The Commission furthermore noted that
``the epidemic of crack cocaine use by youth never materialized to the
extent feared.'' \emph{Id.,} at 96.

^8 An offense level of 12 results in a Guidelines range of 10 to 16
months for a first-time offender; an offense level of 38 results in a
range of 235 to 293 months for the same offender. See USSG ch. 5, pt.
A, Sentencing Table.

  Second, the Commission concluded that the crack/powder disparity
is inconsistent with the 1986 Act's goal of punishing major drug
traffickers more severely than low-level dealers. Drug importers
and major traffickers generally deal in powder cocaine, which is
then converted into crack by streetlevel sellers. See 1995 Report
66--67. But the 100-to-1 ratio can lead to the ``anomalous'' result
that ``retail crack dealers get longer sentences than the wholesale
drug distributors who supply them the powder cocaine from which their
crack is produced.'' \emph{Id.,} at 174.

  Finally, the Commission stated that the crack/powder sentencing
differential ``fosters disrespect for and lack of confidence in the
criminal justice system'' because of a ``widelyheld perception''
that it ``promotes unwarranted disparity based on race.'' 2002 Report
103. Approximately 85 percent of defendants convicted of crack offenses
in federal court are black; thus the severe sentences required by
the 100-to-1 ratio are imposed ``primarily upon black offenders.''
\emph{Ibid.}

  Despite these observations, the Commission's most recent reports
do not urge identical treatment of crack and powder cocaine. In the
Commission's view, ``some differential in the quantity-based
penalties'' for the two drugs is warranted, \emph{id.,} at 102,
because crack is more addictive than powder, crack offenses are more
likely to involve weapons or bodily injury, and crack distribution is
associated with higher levels of crime, see \emph{id.,} at 93--94,
101--102. But the 100-to-1 crack/\newpage powder ratio, the Commission
concluded, significantly overstates the differences between the two
forms of the drug. Accordingly, the Commission recommended that the
ratio be ``substantially'' reduced. \emph{Id.,} at viii.

\section{C}

  The Commission has several times sought to achieve a reduction in the
crack/powder ratio. In 1995, it proposed amendments to the Guidelines
that would have replaced the 100-to-1 ratio with a 1-to-1 ratio.
Complementing that change, the Commission would have installed special
enhancements for trafficking offenses involving weapons or bodily
injury. See Amendments to the Sentencing Guidelines for United States
Courts, 60 Fed. Reg. 25075--25077 (1995). Congress, acting pursuant
to 28 U.~S.~C. \S~994(p),\footnotemark[9] rejected the amendments. See Pub.
L. 104--38, \S1, 109 Stat. 334. Simultaneously, however, Congress
directed the Commission to ``propose revision of the drug quantity
ratio of crack cocaine to powder cocaine under the relevant statutes and
guidelines.'' \S~2(a)(2), \emph{id.,} at 335.

  In response to this directive, the Commission issued reports
in 1997 and 2002 recommending that Congress change the 100-to-1
ratio prescribed in the 1986 Act. The 1997 Report proposed a 5-to-1
ratio. See United States Sentencing Commission, Special Report
to Congress: Cocaine and Federal Sentencing Policy 2 (Apr. 1997),
http://www.ussc.gov/r_congress/newcrack.pdf. The 2002 Report
recommended lowering the ratio ``at least'' to 20 to 1. 2002 Report
viii. Neither proposal prompted congressional action.

  The Commission's most recent report, issued in 2007, again urged
Congress to amend the 1986 Act to reduce the 100-to-1 ratio. This time,
however, the Commission did not simply await congressional action.
Instead, the Commission \newpage  adopted an ameliorating change in the
Guidelines. See 2007 Report 9. The alteration, which became effective
on November 1, 2007, reduces the base offense level associated with
each quantity of crack by two levels. See Amendments to the Sentencing
Guidelines for United States Courts, 72 Fed. Reg. 28571--28572
(2007).\footnotemark[10] This modest amendment yields sentences for crack
offenses between two and five times longer than sentences for equal
amounts of powder. See \emph{ibid.\\\footnotemark[11] Describing the amendment as
``only\dots apartial remedy'' for the problems generated by the
crack/powder disparity, the Commission noted that ``[a]ny comprehensive
solution requires appropriate legislative action by Congress.'' 2007
Report 10.

^9 Subsection 994(p) requires the Commission to submit Guidelines
amendments to Congress and provides that such amendments become
effective unless ``modified or disapproved by Act of Congress.''

\section{III}

  With this history of the crack/powder sentencing ratio in mind, we
next consider the status of the Guidelines tied to the ratio after our
decision in \emph{United States} v. \emph{Booker,} 543 U.~S. 220 (2005).
In \emph{Booker,} the Court held that the mandatory Sentencing Guidelines
system violated the Sixth Amendment. See \emph{id.,} at 226--227. The
\emph{Booker} remedial opinion determined that the appropriate cure was to
sever and excise the provision of the statute that rendered the \newpage 
Guidelines mandatory, 18 U.~S.~C. \S~3553(b)(1) (2000 ed., Supp.
IV).\footnotemark[12] This modification of the federal sentencing statute, we
explained, ``makes the Guidelines effectively advisory.'' 543 U. S.,
at 245.

^10 The amended Guidelines still produce sentencing ranges keyed to the
mandatory minimums in the 1986 Act. Under the pre-2007 Guidelines, the
5-and 50-gram quantities that trigger the statutory minimums produced
sentencing ranges that slightly \emph{exceeded} those statutory minimums.
Under the amended Guidelines, in contrast, the 5-and 50gram quantities
produce ``base offense levels corresponding to guideline ranges that
\emph{include} the statutory mandatory minimum penalties.'' 2007 Report
9.

^11 The Commission has not yet determined whether the amendment will
be retroactive to cover defendants like Kimbrough. Even under the
amendment, however, Kimbrough's Guidelines range would be 195 to 218
months---well above the 180-month sentence imposed by the District
Court. See Amendments to the Sentencing Guidelines for United States
Courts, 72 Fed. Reg. 28571--28572 (2007); USSG ch. 5, pt. A, Sentencing
Table.

  The statute, as modified by \emph{Booker,} contains an overarching
provision instructing district courts to ``impose a sentence
sufficient, but not greater than necessary,'' to accomplish the
goals of sentencing, including ``to reflect the seriousness of the
offense,'' ``to promote respect for the law,'' ``to provide just
punishment for the offense,'' ``to afford adequate deterrence to
criminal conduct,'' and ``to protect the public from further crimes
of the defendant.'' 18 U.~S.~C. \S~3553(a) (2000 ed. and Supp.
V). The statute further provides that, in determining the appropriate
sentence, the court should consider a number of factors, including
``the nature and circumstances of the offense,'' ``the history
and characteristics of the defendant,'' ``the sentencing range
established'' by the Guidelines, ``any pertinent policy statement''
issued by the Sentencing Commission pursuant to its statutory authority,
and ``the need to avoid unwarranted sentence disparities among
defendants with similar records who have been found guilty of similar
conduct.'' \emph{Ibid.} In sum, while the statute still requires a
court to give respectful consideration to the Guidelines, see \emph{Gall}
v. \emph{United States, ante,} at 46, 49, \emph{Booker} ``permits the court
to tailor the sentence in light of other statutory concerns as well,''
543 U. S., at 245--246.

  The Government acknowledges that the Guidelines ``are now advisory''
and that, as a general matter, ``courts may vary [from Guidelines
ranges] based solely on policy considerations, including disagreements
with the Guidelines.'' Brief for United States 16; cf. \emph{Rita,}
551 U. S., at 351 (a dis\newpage trict court may consider arguments that
``the Guidelines sentence itself fails properly to reflect \S~3553(a)
considerations''). But the Government contends that the Guidelines
adopting the 100-to-1 ratio are an exception to the ``general freedom
that sentencing courts have to apply the [\S~3553(a)] factors.''
Brief for United States 16. That is so, according to the Government,
because the ratio is a ``specific policy determinatio[n] that Congress
has directed sentencing courts to observe.'' \emph{Id.,} at 25. The
Government offers three arguments in support of this position. We
consider each in turn.


^12 The remedial opinion also severed and excised the provision of the
statute requiring \emph{de novo} review of departures from the Guidelines,
18 U.~S.~C. \S~3742(e), because that provision depended on the
Guidelines' mandatory status. \emph{Booker,} 543 U. S., at 245.

\section{A}

  As its first and most heavily pressed argument, the Government urges
that the 1986 Act itself prohibits the Sentencing Commission and
sentencing courts from disagreeing with the 100-to-1 ratio.\footnotemark[13] The
Government acknowledges that the ``Congress did not \emph{expressly}
direct the Sentencing Commission to incorporate the 100:1 ratio in the
Guidelines.'' Brief for United States 33 (brackets and internal
quotation marks omitted). Nevertheless, it asserts that the Act
``[i]mplicit[ly]'' requires the Commission and sentencing courts
to apply the 100-to-1 ratio. \emph{Id.,} at 32. Any deviation, the
Government urges, would be ``logically incoherent'' when combined with
mandatory minimum sentences based on the 100-to-1 ratio. \emph{Id.,} at
33.

  This argument encounters a formidable obstacle: It lacks grounding
in the text of the 1986 Act. The statute, by its terms, mandates only
maximum and minimum sentences: A person convicted of possession with
intent to distribute five grams or more of crack cocaine must be
sentenced to a mini\newpage mum of 5 years and the maximum term is 40
years. A person with 50 grams or more of crack cocaine must be sentenced
to a minimum of ten years and the maximum term is life. The statute
says nothing about the appropriate sentences within these brackets,
and we decline to read any implicit directive into that congressional
silence. See \emph{Jama} v. \emph{Immigration and Customs Enforcement,}
543 U.~S. 335, 341 (2005) (``We do not lightly assume that Congress
has omitted from its adopted text requirements that it nonetheless
intends to apply\dots .''). Drawing meaning from silence is
particularly inappropriate here, for Congress has shown that it knows
how to direct sentencing practices in express terms. For example,
Congress has specifically required the Sentencing Commission to set
Guidelines sentences for serious recidivist offenders ``at or near''
the statutory maximum. 28 U.~S.~C. \S~994(h). See also \S~994(i)
(``The Commission shall assure that the guidelines specify a sentence
to a substantial term of imprisonment'' for specified categories of
offenders.).

^13 The Government concedes that a district court may vary from
the 100-to-1 ratio if it does so ``based on the individualized
circumstance[s]'' of a particular case. Brief for United States 45. But
the Government maintains that the 100-to-1 ratio is binding in the sense
that a court may not give any weight to its own view that the ratio
itself is inconsistent with the \S~3553(a) factors.

  Our cautious reading of the 1986 Act draws force from \emph{Neal}
v. \emph{United States,} 516 U.~S. 284 (1996). That case involved
different methods of calculating lysergic acid diethylamide (LSD)
weights, one applicable in determining statutory minimum sentences, the
other controlling the calculation of Guidelines ranges. The 1986 Act
sets mandatory minimum sentences based on the weight of ``a mixture
or substance containing a detectable amount'' of LSD. 21 U.~S.~C.
\S~841(b)(1)(A)(v), (B)(v). Prior to \emph{Neal,} we had interpreted
that language to include the weight of the carrier medium (usually
blotter paper) on which LSD is absorbed even though the carrier is
usually far heavier than the LSD itself. See \emph{Chapman} v. \emph{United
States,} 500 U.~S. 453, 468 (1991). Until 1993, the Sentencing
Commission had interpreted the relevant Guidelines in the same way. That
year, however, the Commission changed its approach and ``instructed
courts to give each dose of LSD on a carrier medium a constructive or
presumed weight of 0.4 milligrams.'' \emph{Neal,} 516 U. S., \newpage  at
287 (citing USSG \S~2D1.1(c), n. (H) (Nov. 1995)). The Commission's
change significantly lowered the Guidelines range applicable to most
LSD offenses, but defendants remained subject to higher statutory
minimum sentences based on the combined weight of the pure drug and
its carrier medium. The defendant in \emph{Neal} argued that the revised
Guidelines and the statute should be interpreted consistently and that
the ``presumptive-weight method of the Guidelines should also control
the mandatory minimum calculation.'' 516 U. S., at 287. We rejected
that argument, emphasizing that the Commission had not purported to
interpret the statute and could not in any event overrule our decision
in \emph{Chapman.} See 516 U. S., at 293--295.

  If the Government's current position were correct, then the
Guidelines involved in \emph{Neal} would be in serious jeopardy. We have
just recounted the reasons alleged to justify reading into the 1986 Act
an implicit command to the Commission and sentencing courts to apply the
100-to-1 ratio to all quantities of crack cocaine. Those same reasons
could be urged in support of an argument that the 1986 Act requires
the Commission to include the full weight of the carrier medium in
calculating the weight of LSD for Guidelines purposes. Yet our opinion
in \emph{Neal} never questioned the validity of the altered Guidelines.
To the contrary, we stated: ``Entrusted within its sphere to make
policy judgments, the Commission may abandon its old methods in favor
of what it has deemed a more desirable ‘approach' to calculating
LSD quantities.'' \emph{Id.,} at 295.\footnotemark[14] If the 1986 Act does
not require the Commis\newpage sion to adhere to the Act's method for
determining LSD weights, it does not require the Commission---or, after
\emph{Booker,} sentencing courts---to adhere to the 100-to-1 ratio for
crack cocaine quantities other than those that trigger the statutory
mandatory minimum sentences.


^14 At oral argument, the Government sought to distinguish \emph{Neal}
v. \emph{United States,} 516 U.~S. 284 (1996), on the ground that the
validity of the amended Guidelines was not before us in that case. See
Tr. of Oral Arg. 25. That is true, but only because the Government did
not challenge the amendment. In fact, the Government's brief appeared
to acknowledge that the Commission may legitimately deviate from the
policies and methods embodied in the 1986 Act, even if the deviation
produces some inconsistency. See Brief for United States in \emph{Neal}
v. \emph{United States,} O. T. 1995, No. 94--9088, p. 26 (``When the
Commission's views about sentencing \newpage  policy depart from those
of Congress, it may become difficult to achieve entirely consistent
sentencing, but that is a matter for Congress, not the courts, to
address.''). Moreover, our opinion in \emph{Neal} assumed that the
amendment was a legitimate exercise of the Commission's authority.
See 516 U. S., at 294 (noting with apparent approval the Commission's
position that ``the Guidelines calculation is independent of the
statutory calculation'').

\section{B}

  In addition to the 1986 Act, the Government relies on Congress'
disapproval of the Guidelines amendment that the Sentencing Commission
proposed in 1995. Congress ``not only disapproved of the 1:1 ratio,''
the Government urges; it also made clear ``that the 1986 Act required
the Commission (and sentencing courts) to take drug quantities into
account, and to do so in a manner that respects the 100:1 ratio.''
Brief for United States 35.

  It is true that Congress rejected the Commission's 1995 proposal
to place a 1-to-1 ratio in the Guidelines, and that Congress also
expressed the view that ``the sentence imposed for trafficking in a
quantity of crack cocaine should generally exceed the sentence imposed
for trafficking in a like quantity of powder cocaine.'' Pub. L.
104--38, \S~2(a)(1)(A), 109 Stat. 334. But nothing in Congress'
1995 reaction to the Commission-proposed 1-to-1 ratio suggested that
crack sentences must exceed powder sentences by a ratio of 100 to 1.
To the contrary, Congress' 1995 action required the Commission to
recommend a ``revision of the drug quantity ratio of crack cocaine to
powder cocaine.'' \S~2(a)(2), \emph{id.,} at 335.

  The Government emphasizes that Congress required the Commission to
propose changes to the 100-to-1 ratio in \emph{both} \newpage  the 1986
Act and the Guidelines. This requirement, the Government contends,
implicitly foreclosed any deviation from the 100-to-1 ratio in the
Guidelines (or by sentencing courts) in the absence of a corresponding
change in the statute. See Brief for United States 35--36. But
it does not follow as the night follows the day that, by calling for
recommendations to change the statute, Congress meant to bar any
Guidelines alteration in advance of congressional action. The more
likely reading is that Congress sought proposals to amend both the
statute and the Guidelines because the Commission's criticisms of
the 100-to-1 ratio, see Part II--B, \emph{supra,} concerned the
exorbitance of the crack/powder disparity in both contexts.

  Moreover, as a result of the 2007 amendment, see \emph{supra,} at
99--100, the Guidelines now advance a crack/powder ratio that varies
(at different offense levels) between 25 to 1 and 80 to 1. See
Amendments to the Sentencing Guidelines for United States Courts, 72
Fed. Reg. 28571--28572. Adopting the Government's analysis, the
amended Guidelines would conflict with Congress' 1995 action, and
with the 1986 Act, because the current Guidelines ratios deviate from
the 100to-1 statutory ratio. Congress, however, did not disapprove or
modify the Commission-initiated 2007 amendment. Ordinarily, we resist
reading congressional intent into congressional inaction. See \emph{Bob
Jones Univ.} v. \emph{United States,} 461 U.~S. 574, 600 (1983). But
in this case, Congress failed to act on a proposed amendment to the
Guidelines in a high-profile area in which it had previously exercised
its disapproval authority under 28 U.~S.~C. \S~994(p). If nothing
else, this tacit acceptance of the 2007 amendment undermines the
Government's position, which is itself based on implications drawn
from congressional silence.

\section{C}

  Finally, the Government argues that if district courts are free
to deviate from the Guidelines based on disagreements \newpage  with
the crack/powder ratio, unwarranted disparities of two kinds will
ensue. See 18 U.~S.~C. \S~3553(a)(6) (sentencing courts shall
consider ``the need to avoid unwarranted sentence disparities'').
First, because sentencing courts remain bound by the mandatory minimum
sentences prescribed in the 1986 Act, deviations from the 100-to-1
ratio could result in sentencing ``cliffs'' around quantities that
trigger the mandatory minimums. Brief for United States 33 (internal
quotation marks omitted). For example, a district court could grant a
sizable downward variance to a defendant convicted of distributing 49
grams of crack but would be required by the statutory minimum to impose
a much higher sentence on a defendant responsible for only 1 additional
gram. Second, the Government maintains that, if district courts are
permitted to vary from the Guidelines based on their disagreement with
the crack/powder disparity, ``defendants with identical real conduct
will receive markedly different sentences, depending on nothing more
than the particular judge drawn for sentencing.'' \emph{Id.,} at 40.

  Neither of these arguments persuades us to hold the crack/ powder
ratio untouchable by sentencing courts. As to the first, the LSD
Guidelines we approved in \emph{Neal} create a similar risk of sentencing
``cliffs.'' An offender who possesses LSD on a carrier medium weighing
ten grams is subject to the ten-year mandatory minimum, see 21 U. S.
C. \S~841(b)(1)(A)(v), but an offender whose carrier medium weighs
slightly less may receive a considerably lower sentence based on the
Guidelines' presumptive-weight methodology. Concerning the second
disparity, it is unquestioned that uniformity remains an important
goal of sentencing. As we explained in \emph{Booker,} however, advisory
Guidelines combined with appellate review for reasonableness and
ongoing revision of the Guidelines in response to sentencing practices
will help to ``avoid excessive sentencing disparities.'' 543 U.
S., at 264. These measures will not eliminate variations between
district courts, but our opinion in \emph{Booker} rec\newpage ognized
that some departures from uniformity were a necessary cost of the
remedy we adopted. See \emph{id.,} at 263 (``We cannot and do not
claim that use of a ‘reasonableness' standard will provide the
uniformity that Congress originally sought to secure [through mandatory
Guidelines].''). And as to crack cocaine sentences in particular, we
note a congressional control on disparities: possible variations among
district courts are constrained by the mandatory minimums Congress
prescribed in the 1986 Act.\footnotemark[15]

  Moreover, to the extent that the Government correctly identifies risks
of ``unwarranted sentence disparities'' within the meaning of 18 U. S.
C. \S~3553(a)(6), the proper solution is not to treat the crack/powder
ratio as mandatory. Section 3553(a)(6) directs \emph{district courts} to
consider the need to avoid unwarranted disparities---along with other
\S~3553(a) factors---when imposing sentences. See \emph{Gall, ante,}
at 50, n. 6, 54. Under this instruction, district courts must take
account of sentencing practices in other courts and the ``cliffs''
resulting from the statutory mandatory minimum sentences. To reach an
appropriate sentence, these disparities must be weighed against the
other \S~3553(a) factors and any unwarranted disparity created by the
crack/powder ratio itself.

\section{IV}

  While rendering the Sentencing Guidelines advisory, \emph{Booker,}
543 U. S., at 245, we have nevertheless preserved a key role for the
Sentencing Commission. As explained in \emph{Rita} and \emph{Gall,} district
courts must treat the Guidelines as the ``starting point and the
initial benchmark,'' \emph{Gall, ante,} at 49. Congress established
the Commission to formulate and constantly refine national sentencing
standards. See \emph{Rita,} 551 U. S., at 347--350. Carrying out its
charge, the \newpage  Commission fills an important institutional role: It
has the capacity courts lack to ``base its determinations on empirical
data and national experience, guided by a professional staff with
appropriate expertise.'' \emph{United States} v. \emph{Pruitt,} 502 F. 3d
1154, 1171 (CA10 2007) (McConnell, J., concurring); see \emph{supra,} at
96. 

^15 The Sentencing Commission reports that roughly 70 percent of
crack offenders are responsible for drug quantities that yield base
offense levels at or only two levels above those that correspond to the
statutory minimums. See 2007 Report 25.

  We have accordingly recognized that, in the ordinary case, the
Commission's recommendation of a sentencing range will ``reflect a
rough approximation of sentences that might achieve \S~3553(a)'s
objectives.'' \emph{Rita,} 551 U. S., at 350. The sentencing judge, on
the other hand, has ``greater familiarity with\dots the individual
case and the individual defendant before him than the Commission or
the appeals court.'' \emph{Id.,} at 357--358. He is therefore
``in a superior position to find facts and judge their import under
\S~3553(a)'' in each particular case. \emph{Gall, ante,} at 51
(internal quotation marks omitted). In light of these discrete
institutional strengths, a district court's decision to vary from the
advisory Guidelines may attract greatest respect when the sentencing
judge finds a particular case ``outside the ‘heartland' to which
the Commission intends individual Guidelines to apply.'' \emph{Rita,}
551 U. S., at 351. On the other hand, while the Guidelines are no
longer binding, closer review may be in order when the sentencing
judge varies from the Guidelines based solely on the judge's view
that the Guidelines range ``fails properly to reflect \S~3553(a)
considerations'' even in a mine-run case. \emph{Ibid.} Cf. Tr. of Oral
Arg. in \emph{Gall} v. \emph{United States,} O. T. 2007, No. 06--7949, pp.
38--39.

  The crack cocaine Guidelines, however, present no occasion for
elaborative discussion of this matter because those Guidelines do
not exemplify the Commission's exercise of its characteristic
institutional role. In formulating Guidelines ranges for crack cocaine
offenses, as we earlier noted, the Commission looked to the mandatory
minimum sentences set in the 1986 Act, and did not take account of
``empirical data and national experience.'' See \emph{Pruitt,} 502
F. 3d, at 1171 \newpage  (McConnell, J., concurring). Indeed, the
Commission itself has reported that the crack/powder disparity produces
disproportionately harsh sanctions, \emph{i. e.,} sentences for crack
cocaine offenses ``greater than necessary'' in light of the purposes
of sentencing set forth in \S~3553(a). See \emph{supra,} at 97--98.
Given all this, it would not be an abuse of discretion for a district
court to conclude when sentencing a particular defendant that the
crack/powder disparity yields a sentence ``greater than necessary'' to
achieve \S~3553(a)'s purposes, even in a mine-run case.

\section{V}

  Taking account of the foregoing discussion in appraising the District
Court's disposition in this case, we conclude that the 180-month
sentence imposed on Kimbrough should survive appellate inspection.
The District Court began by properly calculating and considering the
advisory Guidelines range. It then addressed the relevant \S~3553(a)
factors. First, the court considered ``the nature and circumstances''
of the crime, see \S~3553(a)(1), which was an unremarkable
drug-trafficking offense. App. 72--73 (``[T]his defendant and another
defendant were caught sitting in a car with some crack cocaine and
powder by two police officers---that's the sum and substance of
it---[and they also had] a firearm.''). Second, the court considered
Kimbrough's ``history and characteristics.'' \S~3553(a)(1). The
court noted that Kimbrough had no prior felony convictions, that he had
served in combat during Operation Desert Storm and received an honorable
discharge from the Marine Corps, and that he had a steady history of
employment.

  Furthermore, the court alluded to the Sentencing Commission's
reports criticizing the 100-to-1 ratio, cf. \S~3553(a)(5) (2000
ed., Supp. V), noting that the Commission ``recognizes that crack
cocaine has not caused the damage that the Justice Department alleges
it has.'' App. 72. Comparing the Guidelines range to the range
that would have applied if Kimbrough had possessed an equal amount of
powder, the \newpage  court suggested that the 100-to-1 ratio itself
created an unwarranted disparity within the meaning of \S~3553(a).
Finally, the court did not purport to establish a ratio of its own.
Rather, it appropriately framed its final determination in line with
\S~3553(a)'s overarching instruction to ``impose a sentence
sufficient, but not greater than necessary,'' to accomplish the
sentencing goals advanced in \S~3553(a)(2). See \emph{supra,} at
101. Concluding that ``the crack cocaine guidelines [drove] the
offense level to a point higher than is necessary to do justice in this
case,'' App. 72, the District Court thus rested its sentence on
the appropriate considerations and ``committed no procedural error,''
\emph{Gall, ante,} at 56.

  The ultimate question in Kimbrough's case is ``whether the sentence
was reasonable---\\i. e.,} whether the District Judge abused his
discretion in determining that the \S~3553(a) factors supported a
sentence of [15 years] and justified a substantial deviation from the
Guidelines range.'' \emph{Ibid.} The sentence the District Court
imposed on Kimbrough was 4.5 years below the bottom of the Guidelines
range. But in determining that 15 years was the appropriate prison term,
the District Court properly homed in on the particular circumstances of
Kimbrough's case and accorded weight to the Sentencing Commission's
consistent and emphatic position that the crack/powder disparity is at
odds with \S~3553(a). See Part II--B, \emph{supra.} Indeed, aside
from its claim that the 100-to-1 ratio is mandatory, the Government
did not attack the District Court's downward variance as unsupported
by \S~3553(a). Giving due respect to the District Court's reasoned
appraisal, a reviewing court could not rationally conclude that the
4.5-year sentence reduction Kimbrough received qualified as an abuse of
discretion. See \emph{Gall, ante,} at 58--60; \emph{Rita,} 551 U. S., at
358--360.

\hrule

  For the reasons stated, the judgment of the United States Court of
Appeals for the Fourth Circuit is reversed, and the \newpage  case is
remanded for further proceedings consistent with this opinion.

\begin{flushright}\emph{It is so ordered.}\end{flushright}
