% Dissenting
% Scalia

\setcounter{page}{628}

  \textsc{Justice Scalia,} with whom \textsc{Justice Alito} joins, dissenting.

  With all due respect, I find the Court's opinion difficult to
accept. The New Jersey-Delaware Compact of 1905 (Compact or 1905
Compact), Art. VII, 34 Stat. 860, addressed the ``exercise [of]
riparian jurisdiction,'' and the power to ``make grants\dots of
riparian\dots rights.'' The particular riparian right at issue here
is the right of wharfing out. All are agreed that jurisdiction and power
over that right were given to New Jersey on its side of the Delaware
River. The Court says, however, that that jurisdiction and power was
not exclusive. I find that difficult to accept, because if Delaware
could forbid the wharfing out that Article VII allowed New Jersey to
permit, Article VII was a ridiculous nullity. That could not be what
was meant. The Court seeks to avoid that obstacle to credibility by
saying that Delaware's jurisdiction and power is limited to forbidding
``activities that go beyond the exercise of ordinary and usual riparian
rights.'' \emph{Ante,} at 615. It is only ``riparian structures and
operations of \emph{extraordinary character}'' over which Delaware retains
``overlapping authority to regulate.'' \emph{Ante,} at 603 (emphasis
added). But that also is difficult to accept, because the Court
explains neither the meaning nor the provenance of its ``extraordinary
character'' test. The exception (whatever it means) has absolutely
no basis in prior law, which regards as beyond the ``ordinary and
usual'' (and hence beyond the legitimate) only that wharfing out which
interferes with navigation. So unheard of is the exception that its
first appearance in this case is in the Court's opinion.

  I would sustain New Jersey's objections to the Special Master's
Report. \newpage 

\section{I}

  I must begin by clearing some underbrush. One of Delaware's
principal arguments---an argument accepted by the Master and implicitly
accepted by the Court---is that the 1905 Compact must not be construed
to limit Delaware's pre-Compact (albeit at the time unrecognized)
sovereign control over the Delaware River, because of the ``strong
presumption against defeat of a State's title'' in interpreting
agreements. See Report of Special Master 42--43 (Report) (quoting
\emph{United States} v. \emph{Alaska,} 521 U.~S. 1, 34 (1997); internal
quotation marks omitted). According to Delaware, this presumption
establishes that the 1905 Compact gave New Jersey the authority to
\emph{allocate} riparian rights, but left with Delaware the power to
\emph{veto} exercises of those rights under its general police-power
authority.

  I have written of this presumption elsewhere that it ``has little
if any independent legal force beyond what would be dictated by normal
principles of contract interpretation. It is simply a rule of presumed
(or implied-in-fact) intent.'' \emph{United States} v. \emph{Winstar
Corp.,} 518 U.~S. 839, 920 (1996) (opinion concurring in judgment).
It is a manifestation of the commonsense intuition that a State will
rarely contract away its sovereign power. That intuition is sound
enough in almost all state dealings with private citizens, and in some
state dealings with other States. It has no application here, however,
because the whole purpose of the 1905 Compact was precisely to come to
a compromise agreement on the exercise of the two States' sovereign
powers. Entered into at a time when Delaware and New Jersey disputed
the location of their boundary, the Compact demarcated the authority
between the two States with respect to service of civil and criminal
process on vessels, rights of fishery, and riparian rights on either
side of the Delaware River within the circle of a 12-mile radius
centered on the town of New Castle, Delaware. See Compact, 34 Stat.
858; \emph{New Jersey} v. \emph{Delaware,} 291 U.~S. 361, 377--378 (1934)
\emph{(New Jersey} v. \emph{Delaware II\\). There is no way the Compact can
be interpreted \emph{other} \newpage  \emph{than} as a yielding by both States
of what they claimed to be their sovereign powers. The only issue is
\emph{what} sovereign powers were yielded, and that is best determined from
the language of the Compact, with no thumb on the scales.

  Besides relying on the presumption, the Special Master believed
(and the Court believes) that New Jersey's claims must be viewed
askance because it is implausible that Delaware would have ``given
up all governing authority over the disputed area while receiving
nothing in return.'' \emph{Ante,} at 613. But Delaware received
plenty in return. First of all, it ensured access of its citizens to
fisheries on the side of the river claimed by New Jersey---something
it evidently cared more about than the power to control wharfing out
from the Jersey shore, which it had never theretofore exercised. And it
obtained (as the Compact observed) ``the amicable termination'' of
New Jersey's then-pending original action in the Supreme Court, which
had ``been pending for twenty-seven years and upwards.'' 34 Stat.
858--859. How plausible it was that Delaware would give up anything
to get rid of that suit surely depends upon how confident Delaware was
that it would prevail. And to tell the truth, the case appeared to be
going badly. As the Compact observed, the Supreme Court had issued a
preliminary injunction against Delaware ``restraining the execution
of certain statutes of the State of Delaware relating to fisheries.''
\emph{Id.,} at 859. The order issuing that injunction had remarked that
Delaware had now ``interfered with and claimed to control the right of
fishing'' which New Jerseyans had ``heretofore been accustomed'' to
exercise without Delaware's interference for over 70 years. Order
in \emph{New Jersey} v. \emph{Delaware,} No. 1, Orig. (filed 1877), Lodging
for Brief of State of Delaware in Opposition to State of New Jersey's
Motion to Reopen (Tab 1, pp. 52--54). By providing for dismissal
of New Jersey's suit, the Compact assured Delaware that the Supreme
Court's rather ominous sounding preliminary order would not become the
Court's \newpage  holding, perhaps the consequence of a rationale that
gave New Jersey jurisdiction in the river.

\section{II}

  Article VII of the 1905 Compact between New Jersey and Delaware reads
as follows:

    ``Each State may, on its own side of the river, continue to
    exercise riparian jurisdiction of every kind and na ture, and to
    make grants, leases, and conveyances of riparian lands and rights
    under the laws of the respec tive States.''34 Stat. 860.

\noindent As the Court recognizes, this provision allocates to each State
jurisdiction over a bundle of rights that, at the time of the Compact,
riparian landowners, or ``owners of land abutting on bodies of
water,'' possessed under the common law ``by reason of their
adjacency.'' 1 H. Farnham, Law of Waters and Water Rights \S~62,
p. 278 (1904) (Farnham). Those riparian rights included the right to
``fill in and to build wharves and other structures in the shallow
water in front of [the upland] and below low-water mark.'' \emph{Id.,}
\S~113b, at 534. A wharf, the type of structure at issue here,
``imports a place built or constructed for the purpose of loading
or unloading goods.'' \emph{Id.,} \S~111, at 520, n. 1. It was
considered ``a necessary incident of the right to construct [wharves
and piers] that they shall project to a distance from the shore
necessary to reach water which shall float vessels, the largest as
well as the smallest, that are engaged in commerce upon the water into
which they project.'' \emph{Id.,} at 522. Thus, wharves could be built
up to ``the point of navigability,'' J. Gould, Treatise on the Law
of Waters, including Riparian Rights \S~181, p. 352 (2d ed. 1891)
(Gould), so long as they did not ``interfere needlessly with the
right of navigation'' possessed by members of the general public upon
navigable waters, 1 Farnham \S~111, at 521.

  The two States would have been acquainted with this common law. New
Jersey case law comported with the horn\newpage book rules. According
to the State's Court of Errors and Appeals, it was ``undoubted''
and the ``common understanding'' that ``the owners of land bounding
on navigable waters had an absolute right to wharf out and otherwise
reclaim the land down to and even below low water, provided that they
did not thereby impede the paramount right of navigation.'' \emph{Bell}
v. \emph{Gough,} 23 N. J. L. 624, 658 (1852) (opinion of Elmer, J.); see
also J. Angell, Treatise on the Right of Property in Tide Waters and in
the Soil and Shores Thereof 234 (1847) (``[T]he right of a riparian
proprietor to ‘wharf out' into a public river, is a local custom in
New Jersey''); Gould \S~171, at 342 (``[T]he common understanding in
[New Jersey] carries the right [to wharf out] even below low-water mark,
provided there is no obstruction to the navigation''). Case authority
in Delaware seems to be lacking, but in \emph{New Jersey} v. \emph{Delaware
II} the State assured the Special Master at oral argument that ``it
is undoubtedly true in the State of Delaware\dots that the upland
owner had the right to wharf out\dots subject only that you must not
.~.~. obstruct navigation.'' 1 App. of New Jersey on Motion for
Summary Judgment 126a--1 (hereinafter NJ App.).

  Thus, under the plain terms of the 1905 Compact, each State had
``jurisdiction''---the ``authority of a sovereign power to govern
or legislate,'' Webster's International Dictionary of the English
Language 806 (1898)---over wharfing out on ``its own side of the
river.'' To emphasize that this jurisdiction was plenary---that it
included, for example, not merely the power to prohibit wharfing out
but also the power to permit it---Article VII specified that the
jurisdiction it conferred would be ``of every kind and nature.''

  And finally, the jurisdictional grant was not framed as though it was
conferring on either State some hitherto unexercised power. Rather, the
Compact provided that each State would \emph{``continue to''} exercise
the allocated ``riparian jurisdiction,'' clearly envisioning that each
State would wield in the future the same authority over riparian rights
it had \newpage  wielded in the past. 34 Stat. 860 (emphasis added). This
is significant because, before adoption of the Compact in 1905, New
Jersey alone had regulated the construction of riparian improvements
on New Jersey's side of the Delaware River. It had repeatedly
authorized the construction of piers and wharves that extended beyond
the low-water line. App. to Report C--4 to C--5 (listing New Jersey
Acts authorizing riparian landowners to construct wharves); 7 NJ App.
1196a-- 1199a. Delaware, by contrast, had never regulated riparian
rights on the New Jersey side, and indeed, at the time of the Compact
even on its own side there was ``little evidence of [the State's]
active involvement in shoreland development~.~.~.~.'' Report
69.

  I would think all of this quite conclusive of the fact that New
Jersey was given full and exclusive control over riparian rights
on the New Jersey side. The Court concludes that this was not so,
however, in part because of the alleged implausibility of Delaware's
``giv[ing] up all governing authority\dots while receiving nothing
in return,'' \emph{ante,} at 613 (a mistaken contention that I have
already addressed), and in part because ``riparian jurisdiction''
is different from ``exclusive jurisdiction,'' the term used in an
1834 Compact between New Jersey and New York, which referred to ``the
exclusive jurisdiction of and over the wharves, docks, and improvements,
made and to be made on the shore~.~.~.~.'' Act of June 28, 1834,
ch. 126, Art. Third, 4 Stat. 710.

  I willingly concede that exclusive riparian jurisdiction is not the
same as ``exclusive jurisdiction'' \emph{simpliciter.} It includes
only exclusive jurisdiction over \emph{riparian rights} which, as I have
described, include the right to erect wharves \emph{for the loading and
unloading of goods.} That jurisdiction does not necessarily include,
for example, the power to permit or forbid the construction of a casino
on the wharf, or even the power to serve legal process on the wharf.
Jurisdiction to control such matters---which were not established as
part of riparian rights by the common-law and \newpage  hornbook sources
that the parties relied on in framing the Compact---may well fall
outside the scope of the ``riparian jurisdiction'' that the Compact
grants. See, \emph{e. g., Tewksbury} v. \emph{Deerfield Beach,} 763 So.
2d 1071 (Fla. App. 1999) (operation of a restaurant on a dock is not
included within riparian rights). Such powers---which may well have
been conveyed by a grant of ``exclusive jurisdiction'' such as that
contained in the New York-New Jersey Compact---are not at issue in this
case. What is at issue is jurisdiction over the core riparian right of
building a wharf to be used for the loading and unloading of cargo.
And that \emph{that} jurisdiction was given exclusively to New Jersey is
made perfectly clear by the Compact's recognition of each State's
riparian jurisdiction only \emph{``onits ownside ofthe river.''} 34
Stat. 860 (emphasis added). It does not take vast experience in
textual interpretation to conclude that this implicitly excludes each
State's riparian jurisdiction \emph{on the other State's side of the
river. (Inclusio unius est exclusio alterius.)} There was no need,
therefore, to specify \emph{exclusive} riparian jurisdiction.

  The Court's position gains no support from the fact that the rights
of a private riparian owner `` ‘are always subordinate to the public
rights, and the state may regulate their exercise in the interest of
the public.' ''\\Ante,} at 612 (quoting 1 Farnham \S~63, at
284). The Compact did not purport to convey mere private rights, but
rather ``riparian \emph{jurisdiction} of every kind and nature.'' If that
means anything at all, it means that \emph{New Jersey} is the State that
``may regulate [the] exercise [of the rights of a private riparian
owner] in the interest of the public.'' Delaware's contention that
it retains the authority to prohibit under its police power even those
activities that are specifically allowed to New Jersey under the Compact
renders not just Article VII but most of the Compact a virtual nullity.
Article III, for example, gives the States ``common right of fishery
throughout, in, and over the waters'' of the Delaware. 34 Stat. 859.
But under its police powers a sovereign State could regulate fishing
\newpage  within its public navigable waters. See Gould \S~189, at 362.
Thus, under Delaware's view, just as its ownership of the riverbed
would allow it to trump New Jersey's authority to permit wharfing
out, so also its ownership of the riverbed would allow it to prevent
fishing. That would be an extraordinary result, since the litigation the
1905 Compact was designed to resolve arose over fishing rights, after
Delaware enacted a law in 1871 requiring New Jersey fishermen to obtain
a Delaware license. See Report 3--6.

\section{III}

  The Court, following the Special Master's analysis, see \emph{id.\\,
at 68--84, asserts that today's judgment is supported by the
parties' course of conduct after conclusion of the Compact. I frankly
think post-Compact conduct irrelevant to this case, since it can
properly be used only to clarify an ambiguous agreement, and there is no
ambiguity here. The Court, moreover, overstates the post-Compact conduct
favoring Delaware's position and understates the post-Compact conduct
favoring New Jersey. But even if post-Compact conduct is consulted, no
such conduct---none whatever---supports the Court's ``extraordinary
character'' test, whereas several instances of such conduct strongly
support the resolution I have suggested in this dissent.

  The Court relies upon four instances of Delaware's exercise of
jurisdiction over wharfing out from the Jersey shore, and two instances
of New Jersey's acquiescence in such an exercise---all postdating
1969. As to the former, the three structures extending from New Jersey
into Delaware built between 1969 and 2006 were permitted by Delaware,
\emph{ante,} at 621; and another application for a permit was denied,
\emph{ante,} at 619--620. The Court never establishes, however, that
these instances of Delaware's assertion of jurisdiction related to
wharves of ``extraordinary character,'' which is the only jurisdiction
that the Court's decree confers upon Delaware. At best, these
assertions of jurisdiction support not the \newpage  Court's opinion,
but rather Delaware's assertion that it may regulate all wharves on
the river---an assertion that the Court rejects. The same mismatch is
present with both instances of New Jersey's asserted acquiescence. One
of them was New Jersey's application for Delaware's permission to
refurbish the stone pier at Fort Mott State Park, described \emph{ante,}
at 621. That construction could not conceivably be characterized as
of ``extraordinary character,'' and thus New Jersey did not need to
ask Delaware for permission under the Court's theory. In the other
instance, described \emph{ante,} at 620, New Jersey's Coastal
Management Agency assured the Secretary of Commerce that `` ‘\emph{any}
New Jersey project extending beyond mean low water' '' (emphasis
added) had to be approved by Delaware's Coastal Management Agency as
well as New Jersey's. This again supports Delaware's theory of this
case, but not the Court's.[[*]]

  While post-Compact conduct provides no---absolutely \emph{zero}---support
for the Court's interpretation, it provides substantial support for
the one I have suggested. In \emph{New Jersey} v. \emph{Delaware II,} a case
before this Court involving precisely the meaning of the Compact, the
attorney general of \newpage  Delaware (obviously authorized to present
the State's position on the point) conceded to the Special Master
that ``Article VII of the Compact is obviously merely a recognition
of the rights of the riparian owners of New Jersey and a \emph{cession}
to the State of New Jersey by the State of Delaware of jurisdiction
to regulate those rights.'' 1 NJ App. 123a (emphasis added).
And at oral argument before the Special Master, Delaware's special
counsel---Clarence A. Southerland, a former state attorney general
and future Chief Justice of the Supreme Court of Delaware, see
Delaware Bar in the Twentieth Century 375 (H. Winslow, A. Bookout, \& P.
Hannigan eds. 1994)---explained that ``the Compact of 1905 expressly
acknowledged the rights of the citizens of New Jersey, at least, by
implication to wharf out'' and that New Jersey possessed ``all the
right \emph{to control the erection of those wharves and to say who shall
erect them.}'' 1 NJ App. 126a--1 (emphasis added). And in its
Supreme Court brief in that litigation, Delaware assured the Court,
without conditions, that ``Delaware has never questioned the right of
citizens of New Jersey to wharf out \emph{to navigable water} nor can such
a right be questioned now because it is clearly protected by the Compact
of 1905 between the States.'' \emph{Id.,} at 139a (emphasis added).
Delaware's Supreme Court brief rejected New Jersey's argument
that, if the Court found the boundary line to be the low-water mark
on the New Jersey shore, ``the interests of the riparian owners will
be either destroyed or seriously prejudiced.'' \emph{Id.,} at 140a.
That concern, Delaware said, was misguided because the 1905 Compact
``recognized the rights of riparian owners in the river to wharf
out.'' \emph{Ibid.} ``The effect of Article VII of the Compact,''
the brief explained, ``was that the State of Delaware recognized
the rights of the inhabitants on the east side of the river to wharf
out to navigable water. This right had never been questioned and was
undoubtedly inserted to put beyond question the \emph{riparian rights}
(as distinguished from \emph{title}) of land owners in New Jersey.''
\emph{Id.,} at 141a. These concessions are \newpage  powerful indication that
Delaware's understanding of the Compact was the same as the one I
assert.

^* The post-Compact-conduct argument is not the only portion of the
Court's reasoning that is a mismatch with its conclusion. So is its
reliance upon Article VIII of the Compact, \emph{ante,} at 611--612,
622---an argument so weak that it deserves only a footnote response.
Article VIII provides that nothing in the Compact ``shall affect the
territorial limits, rights, or jurisdiction of either State .~.~.
\emph{except as herein expressly set forth.}'' 34 Stat. 860 (emphasis
added). But New Jersey's riparian rights \emph{are} expressly set
forth, so the only question---the one I have addressed above---is what
those rights consist of. But accepting the Court's overreading of
Article VIII (which presumably requires each of the riparian rights to
be named one by one), it is utterly impossible to see why Article VII
is any more ``expres[s]'' in setting forth New Jersey's authority
over wharves that lack ``extraordinary character'' than it is in
setting forth her authority over wharves that possess it. Once again,
the argument supports not the Court's holding, but rather Delaware's
more expansive theory that it may regulate any and all wharves built
from the Jersey shoreline. There is, to tell the truth, nothing whatever
to support the Court's holding.


\section{IV}

  Our opinion in \emph{Virginia} v. \emph{Maryland,} 540 U.~S. 56 (2003),
effectively decided this case. It rejected the very same assertion of
a riverbed-owning State's supervening policepower authority over
constructions into the river from a State that had been conceded
riparian rights. That case involved two governing documents rather than
(as here) only one. The first, a 1785 compact, provided:

    `` ‘The citizens of each state respectively shall have full
    property in the shores of Potowmack river adjoining their lands,
    with all emoluments and advantages there unto belonging, and
    the privilege of making and carry ing out wharves and other
    improvements, so as not to obstruct or injure the navigation of the
    river.' '' \emph{Id.,} at 62.

\noindent The second, an arbitration award of 1877 that interpreted the earlier
compact, read as follows:

    `` ‘Virginia is entitled not only to full dominion over the soil
    to low-water mark on the south shore of the Potomac, but has a right
    to such use of the river beyond the line of low-water mark as may be
    necessary to the full enjoyment of her riparian ownership, without
    impeding the navigation or otherwise interfering with the proper use
    of it by Maryland, agreeably to the compact of seventeen hundred and
    eighty-five.' '' \emph{Id.,} at 62--63.

  We rejected Maryland's police-power authority to forbid Virginia's
construction of a water intake structure that extended into Maryland
territory, and held that ``Virginia's right ‘to erect .~.~.
structures connected with the shore' is inseparable from, and
‘necessary to,' the ‘full enjoyment of her riparian ownership'
of the soil to low-water mark.'' \emph{Id.,} \newpage  at 72. Maryland,
we observed, was ``doubtless correct that if her sovereignty over the
River was well settled as of 1785, we would apply a strong presumption
against reading the Compact as stripping her authority to regulate
activities on the River.'' \emph{Id.,} at 67. But because the ``scope
of Maryland's sovereignty over the River was in dispute both before
and after the 1785 Compact,'' no such presumption existed. \emph{Id.,}
at 68.

  Today's opinion, quoting the Special Master, claims that the result
in \emph{Virginia} v. \emph{Maryland} turned on ``‘the unique language
of the compact and arbitration award involved in that case.' ''
\emph{Ante,} at 617 (quoting Report 64, n. 118). But the case did not
say that. And of course virtually every written agreement or award
has ``unique language,'' so if we could only extend to other cases
legal principles pertaining to identical language our interpretive
jurisprudence would be limited indeed. The documents in \emph{Virginia}
v. \emph{Maryland} said in other words precisely what the Compact here
said: that one of the States (there, Virginia, here, New Jersey) was
given riparian rights, including the right to construct wharves and
improvements. And the holding of the case was that those rights could be
exercised free of police power or other interference by the State owning
the riverbed.

  The Court contends that in \emph{Virginia} v. \emph{Maryland} the
arbitration award, rather than the compact, ``was definitive,''
because it recognized the right of Virginia `` ‘\emph{qua} sovereign,'
'' and nowhere made the right `` ‘subject to Maryland's regulatory
authority.' '' \emph{Ante,} at 618 (quoting 540 U. S., at 72).
But Article VII of the Compact here at issue likewise spoke of the
rights of New Jersey ``\emph{qua} sovereign'' (what else does the
``exercise [of] riparian \emph{jurisdiction}'' mean?) and similarly did
not make those rights subject to Delaware's regulatory authority.
We stressed in \emph{Virginia} v. \emph{Maryland} that the salient factor
in the interpretation of the compact (and hence in the arbitration
award's interpretation of the compact) was that it was entered into
(like the Compact here) by way of \newpage  settlement of a continuing
boundary dispute. ``If any inference at all is to be drawn from [the
compact's] silence on the subject of regulatory authority,'' we
said, ``it is that each State was left to regulate the activities of
her own citizens.'' \emph{Id.,} at 67. \emph{Virginia} v. \emph{Maryland}
effectively decided this case.

\section{V}

  Finally, I must remark at greater length upon the Court's peculiar
limitation upon New Jersey's wharfing-out rights---that it excludes
wharves of ``extraordinary character.'' But for that limitation, the
Court's conclusion is precisely the same as my own: ``Given the
authority over riparian rights that the 1905 Compact preserves for New
Jersey, Delaware may not impede ordinary and usual exercises of the
right of riparian owners to wharf out from New Jersey's shore.''
\emph{Ante,} at 622. The Court inexplicably concludes, however, that
the liquefied natural gas (LNG) unloading wharf at stake in this
litigation ``goes well beyond the ordinary or usual.'' \emph{Ibid.}
Why? Because it possesses ``extraordinary character.''

  To our knowledge (and apparently to the Court's, judging by its
failure to cite any authority) the phrase has never been mentioned
before in any case involving limitations on wharfing out. What in the
world does it mean? Would a pink wharf or a zig-zagged wharf qualify?
Today's opinion itself gives the phrase no content other than to
say that ``Delaware's classification of the proposed LNG unloading
terminal as a ‘[h]eavy industry use' and a ‘bulk product transfer
facilit[y]'\dots has not been, and hardly could be, challenged as
inaccurate.'' \emph{Ante,} at 622--623. This rationale is bizarre.
There is no reason why \emph{any} designation by the Delaware Department of
Natural Resources and Environmental Control would be relevant to, let
alone controlling on, the meaning of the 1905 Compact; and no reason
why New Jersey's authority under the 1905 Compact should turn on
the statelaw question whether Delaware ``rationally categorize[s]''
a \newpage  wharf under its own statutes, \emph{ante,} at 623, n. 21.
Wharves were commonly used for ``heavy industry use'' when the 1905
Compact was adopted, and their primary commercial use was to transfer
bulk cargoes. One roughly contemporaneous book on the design and
building of wharves in America included information on appropriate
pavement material to enable use of trucks on wharves, the proper method
of laying down railroad tracks, and the construction of hatch cranes
for unloading cargo. See C. Greene, Wharves and Piers: Their Design,
Construction, and Equipment 191--194, 206--215 (1917). The Court
gives no reason why the terminal's character as a ``[h]eavy industry
use'' and a ``bulk product transfer facilit[y]'' matters in the
slightest. Indeed, the Court does not take its state-law reason for
``extraordinary character'' seriously, conceding that Delaware could
not regulate an \emph{identical} wharf for the ``bulk product transfer''
of ``tofu and bean sprouts,'' \emph{ante,} at 623, n. 21.

  Apart from the Delaware Department's ``[h]eavy industry use'' and
``bulk product transfer'' designations, the Court cites, as support
for its conclusion that this wharf is of ``extraordinary character,''
its own factual background section describing the wharf. See \emph{ante,}
at 622--623 (citing \emph{ante,} at 606--607). It is not clear which,
if any, of the facts discussed there the Court claims to be relevant,
and I am forced to speculate on what they might be.

  Could it be the size of the wharf, which is 2,000 feet long, see
\emph{ante,} at 606, and extends some 1,455 feet into Delaware territory,
see Brief for BP America Inc. et al. as \emph{Amici Curiae} 1--2?
But the Court cites \emph{not a single source} for this length limitation
upon wharfing out. We did not intimate, in holding in \emph{Virginia} v.
\emph{Maryland} that Virginia could authorize construction of a water
intake pipe extending 725 feet from its shoreline into Maryland, see
540 U. S., at 63, that the result turned on the length of the pipe.
As I have discussed, the common law \emph{did} establish a size limitation
for wharves: the wharf could not be extended so far as to interfere
need\newpage lessly with the public's ``right of navigation'' in
navigable waters. 1 Farnham \S~111, at 521. Wharves constructed to
access the water could ``project to a distance from the shore necessary
to reach water which shall float vessels, \emph{the largest as well as the
smallest.}'' \emph{Id.,} at 522 (emphasis added). Delaware has not
claimed that the wharf in this case will interfere with navigation
of the river, which is approximately one mile wide at this location,
see Brief for BP America Inc. et al. as \emph{Amici Curiae} 2. And the
record reveals that New Jersey, at least, anticipated that wharves on
its side of the river could extend as far as the wharf in this case by
establishing pierhead lines in 1877 and 1916 that extended ``\emph{below}
low water mark at distances varying from 378 to 3,550 feet.'' 1 NJ
App. 135a; see also 3 \emph{id.,} at 369a, 376a (affidavit of Richard G.
Castagna). (Pierhead lines mark the permissible ``outshore limit of
structures of any kind.'' Greene, \emph{supra,} at 27.)

  Could the fact rendering this a wharf of ``extraordinary character''
be that its construction would require the dredging of 1.24 million
cubic yards of soil within Delaware's territory? \emph{Ante,} at
606--607. This is suggested, perhaps, by the portion of the decree
which says that ``Delaware acted within the scope of its governing
authority to prohibit unreasonable uses of the\dots soil within
the twelve-mile circle.'' \emph{Ante,} at 624; see also \emph{ante,} at
607, n. 8. But no again. Although the record contains no evidence
of the dredge volumes required to construct the wharves on the river
at the time of the Compact's adoption, it does show that an 1896
navigational improvement required the dredging of 35 million cubic
yards from the Delaware River, and a 1907 dredging at Cape May Harbor,
New Jersey, removed 19.7 million cubic yards. 7 NJ App. 1224a, 1234a
(affidavit of J. Richard Weggel). At the very least, the dredging of
1.24 million cubic yards ``would have been familiar to or ascertainable
by individuals interested in riparian uses or structures at the time
the Compact was signed or ratified.'' \emph{Id.,} at 1227a. I do not
\newpage  know what to make of the Court's response that the instances
of dredging that I have cited involved ``public works.'' \emph{Ante,}
at 607, n. 8. Is that a limitation upon the Court's holding---only
\emph{private} wharves of ``extraordinary character'' can be regulated
by Delaware? But in fact dredging seems to have nothing to do with the
issue, since (once again) the Court acknowledges that the same wharf for
tofu and bean sprouts would be OK.

  Could the determinative fact be that the wharf would service
``[s]upertankers with capacities of up to 200,000 cubic meters (more
than 40 percent larger than any ship then carrying natural gas),''
\emph{ante,} at 606; that these ships ``would pass densely populated
areas'' and require establishment of ``a moving safety zone [that]
would restrict other vessels 3,000 feet ahead and behind, and 1,500
feet on all sides,'' \emph{ante,} at 606, n. 7? This is suggested,
perhaps, by the portion of the decree which says that ``Delaware acted
within the scope of its governing authority to prohibit unreasonable
uses of the river\dots within the twelve-mile circle.'' \emph{Ante,}
at 624. But surely not. Whatever power Delaware has to restrict
traffic on the waters of the United States (a question not presented by
this case, though one that seems not to inhibit the decree's blithe
positing of state ``authority to prohibit unreasonable uses of the
river,'' \emph{ibid.\\), it has no bearing on whether New Jersey can
build the \emph{wharf} without Delaware's interference.

  Could the determinative fact be that the wharf will be used to
transport liquefied natural gas, which is dangerous? No again. The Court
cites no support, and I am aware of none, for the proposition that the
common law forbade a wharf owner to load or unload hazardous goods. At
the time of the Compact's adoption, congressional sources reported
that the Delaware River was used to transport, among other items, coal
tar and pitch, sulfur, gunpowder, and explosives. Annual Report of the
Chief of Engineers, United States Army, H. R. Doc. No. 22, 59th Cong.,
2d Sess., 1031--1033 \newpage  (App. H) (1906) (tabulating commerce on
the Delaware River by item in 1904 and 1905). Books published some
time after the adoption of the Compact discuss the proper handling of
seaborne ``dangerous goods,'' including liquids such as benzene,
petroleum, and turpentine. See J. Aeby, Dangerous Goods (2d ed. 1922);
R. MacElwee \& T. Taylor, Wharf Management: Stevedoring and Storage 41,
221 (1921). There is not a shred of evidence that the parties to the
Compact understood that New Jersey and Delaware would not be authorized
to grant riparian rights for the loading and unloading of goods that
are---under some amorphous and unexplained criteria---dangerous.

  I say that none of these factors has any bearing upon whether, at
law, the wharfing out at issue here is anything more than the usual and
ordinary exercise of a riparian right. I am not so rash as to suggest,
however, that these factors had nothing to do with the Court's
decision. After all, our environmentally sensitive Court concedes
that if New Jersey had approved a wharf of equivalent dimensions, to
accommodate tankers of equivalent size, carrying tofu and bean sprouts,
Delaware could not have interfered. See \emph{ante,} at 623, n. 21.

\hrule

  According to one study, construction activities on the LNG facility
in this case would have created more than 1,300 new jobs, added \$277
million to New Jersey's gross state product, and produced \$13 million
in state and local tax revenues.J. Seneca et al., Economic Impacts of
BP's Proposed Crown Landing LNG Terminal 65 (Apr. 2007), online at
http://www.policy.rutgers.edu/news/reports/other/BPCrownLanding.pdf (as
visited Mar. 28, 2008, and available in Clerk of Court's case file).
Operation of the facility was projected to generate 231 permanent jobs,
and more than \$88 million in state and local tax revenues over a 30-year
period. \emph{Ibid.} Its delivery capacity would represent 15 percent
of the current consumption of natural gas in the region.\\Id.,} at
66. In \newpage  holding that Delaware may veto the project, the Court
owes New Jersey---not to mention an energy-starved Nation---something
more than its casual and unsupported statements that the wharf possesses
``extraordinary character'' and ``goes well beyond the ordinary or
usual.''

  Today's decision does not even have the excuse of achieving a
desirable result. If one were to design, \emph{ex ante,} the socially
optimal allocation of the power to permit and forbid wharfing out,
surely that power would be lodged with the sovereign that stands
most to gain from the benefits of a wharf, and most to lose from its
environmental and other costs. Unquestionably, that is the sovereign
with jurisdiction over the land from which the wharf is extended.
Delaware and New Jersey doubtless realized this when they agreed in
1905 that each of them would have jurisdiction over riparian rights on
its own side of the river. The genius of today's decision is that it
creates irrationality where sweet reason once prevailed---straining
mightily, against all odds, to ensure that the power to permit or forbid
``heavy industry use'' wharves in New Jersey shall rest with Delaware,
which has no interest whatever in facilitating the delivery of goods
to New Jersey, which has relatively little to lose from the dangerous
nature of those goods or the frequency and manner of their delivery,
and which may well have an interest in forcing the inefficient location
of employment-and tax-producing wharves on its own shore. It makes no
sense.

  Under its decree, ``[t]he Court retains jurisdiction to entertain
such further proceedings, enter such orders, and issue such writs as it
may from time to time deem necessary or desirable to give proper force
and effect to this Decree or to effectuate the rights of the parties.''
\emph{Ante,} at 624. This could mean, I suppose, that we can anticipate
a whole category of original actions in this Court that will clarify,
wharf by wharf, what is a wharf of ``extraordinary character.'' (Who
would have thought that such utterly indefinable and unpredictable
complexity lay hidden within the words of the \newpage  Compact?) More
likely, however, prospective builders of ``heavy industry use''
wharves from the New Jersey shore---of whatever size---will apply to
Delaware and simply go elsewhere if rejected.

  The wharf at issue in this litigation would have been viewed as an
ordinary and usual riparian use at the time the two States entered
into the 1905 Compact. Delaware accordingly may not prohibit its
construction. I respectfully dissent from the Court's judgment to the
contrary.
