% Opinion of the Court
% Scalia

\setcounter{page}{315}

  \textsc{Justice Scalia} delivered the opinion of the Court.

  We consider whether the pre-emption clause enacted in the Medical
Device Amendments of 1976, 21 U.~S.~C. \S~360k, bars common-law
claims challenging the safety and effectiveness of a medical device
given premarket approval by the Food and Drug Administration (FDA).

\section{I}

\section{A}

  The Federal Food, Drug, and Cosmetic Act (FDCA), 52 Stat. 1040, as
amended, 21 U.~S.~C. \S~301 \emph{et seq.,} has long required FDA
approval for the introduction of new drugs into the market. Until the
statutory enactment at issue here, however, the introduction of new
medical devices was left largely for the States to supervise as they saw
fit. See \emph{Medtronic, Inc.} v. \emph{Lohr,} 518 U.~S. 470, 475--476
(1996).

  The regulatory landscape changed in the 1960's and 1970's,
as complex devices proliferated and some failed. Most notably, the
Dalkon Shield intrauterine device, introduced in 1970, was linked to
serious infections and several deaths, not to mention a large number
of pregnancies. Thousands of tort claims followed. R. Bacigal,
The Limits of Litigation: The Dalkon Shield Controversy 3 (1990).
In the view of many, the Dalkon Shield failure and its aftermath
demonstrated the inability of the common-law tort system to manage the
risks associated with dangerous devices. See, \emph{e. g.,} S. Foote,
Managing the Medical Arms Race 151--152 (1992). Several States
adopted regulatory measures, including California, which in 1970
enacted a law requiring premarket approval of medical devices. 1970
Cal. Stats. ch. 1573, \newpage  \S\S~26670--26693; see also Leflar \&
Adler, The Preemption Pentad: Federal Preemption of Products Liability
Claims After \emph{Medtronic,} 64 Tenn. L. Rev. 691, 703, n. 66 (1997)
(identifying 13 state statutes governing medical devices as of 1976).

  Congress stepped in with passage of the Medical Device Amendments
of 1976 (MDA), 21 U.~S.~C. \S~360c \emph{et seq.,\\\footnotemark[1] which swept
back some state obligations and imposed a regime of detailed federal
oversight. The MDA includes an express pre-emption provision that
states:

      ``Except as provided in subsection (b) of this section, no State
    or political subdivision of a State may establish or continue
    in effect with respect to a device intended for human use any
    requirement---

      ``(1) which is different from, or in addition to, any re
quirement applicable under this chapter to the device, and

      ``(2) which relates to the safety or effectiveness of the device
    or to any other matter included in a requirement applicable to the
    device under this chapter.'' \S~360k(a).

\noindent The exception contained in subsection (b) permits the FDA to exempt some
state and local requirements from pre-emption.

  The new regulatory regime established various levels of oversight for
medical devices, depending on the risks they present. Class I, which
includes such devices as elastic bandages and examination gloves, is
subject to the lowest level of oversight: ``general controls,'' such
as labeling requirements. \S~360c(a)(1)(A); FDA, Device Advice:
Device Classes, http://www.fda.gov/cdrh/devadvice/3132.html (all
Internet materials as visited Feb. 14, 2008, and available in Clerk of
Court's case file). Class II, which includes such devices as powered
wheelchairs and surgical drapes, \emph{ibid.,} \newpage  is subject in
addition to ``special controls'' such as performance standards and
postmarket surveillance measures, \S~360c(a)(1)(B).

^1 Unqualified \S~360 \emph{et seq.} numbers hereinafter refer to sections
of 21 U.~S.~C.

  The devices receiving the most federal oversight are those in Class
III, which include replacement heart valves, implanted cerebella
stimulators, and pacemaker pulse generators, FDA, Device Advice: Device
Classes, \emph{supra.} In general, a device is assigned to Class III
if it cannot be established that a less stringent classification would
provide reasonable assurance of safety and effectiveness, and the
device is ``purported or represented to be for a use in supporting or
sustaining human life or for a use which is of substantial importance
in preventing impairment of human health,'' or ``presents a potential
unreasonable risk of illness or injury.'' \S~360c(a)(1)(C)(ii).

  Although the MDA established a rigorous regime of premarket approval
for new Class III devices, it grandfathered many that were already
on the market. Devices sold before the MDA's effective date may
remain on the market until the FDA promulgates, after notice and
comment, a regulation requiring premarket approval. \S\S~360c(f)(1),
360e(b)(1). A related provision seeks to limit the competitive
advantage grandfathered devices receive. A new device need not undergo
premarket approval if the FDA finds it is ``substantially equivalent''
to another device exempt from premarket approval. \S~360c(f)(1)(A).
The agency's review of devices for substantial equivalence is known as
the \S~510(k) process, named after the statutory provision describing
the review. Most new Class III devices enter the market through
\S~510(k). In 2005, for example, the FDA authorized the marketing of
3,148 devices under \S~510(k) and granted premarket approval to just 32
devices. P. Hutt, R. Merrill, \& L. Grossman, Food and Drug Law 992 (3d
ed. 2007).

  Premarket approval is a ``rigorous'' process. \emph{Lohr, supra,}
at 477. A manufacturer must submit what is typically a multivolume
application. FDA, Device Advice---Premar\newpage ket Approval (PMA)
18, http://www.fda.gov/cdrh/devadvice/pma/printer.html. It includes,
among other things, full reports of all studies and investigations of
the device's safety and effectiveness that have been published or
should reasonably be known to the applicant; a ``full statement''
of the device's ``components, ingredients, and properties and of
the principle or principles of operation''; ``a full description of
the methods used in, and the facilities and controls used for, the
manufacture, processing, and, when relevant, packing and installation
of, such device''; samples or device components required by the FDA;
and a specimen of the proposed labeling. \S~360e(c)(1). Before deciding
whether to approve the application, the agency may refer it to a panel
of outside experts, 21 CFR \S~814.44(a) (2007), and may request
additional data from the manufacturer, \S~360e(c)(1)(G).

  The FDA spends an average of 1,200 hours reviewing each application,
\emph{Lohr,} 518 U. S., at 477, and grants premarket approval
only if it finds there is a ``reasonable assurance'' of the
device's ``safety and effectiveness,'' \S~360e(d). The agency
must ``weig[h] any probable benefit to health from the use of the
device against any probable risk of injury or illness from such use.''
\S~360c(a)(2)(C). It may thus approve devices that present great
risks if they nonetheless offer great benefits in light of available
alternatives. It approved, for example, under its Humanitarian Device
Exemption procedures, a ventricular assist device for children with
failing hearts, even though the survival rate of children using
the device was less than 50 percent. FDA, Center for Devices and
Radiological Health, Debakey VAD Child Left Ventricular Assist
System-H030003, Summary of Safety and Probable Benefit 20 (2004),
http://www.fda.gov/cdrh/pdf3/H030003b.pdf.

  The premarket approval process includes review of the device's
proposed labeling. The FDA evaluates safety and effectiveness under
the conditions of use set forth on the label, \S~360c(a)(2)(B),
and must determine that the proposed labeling is neither false nor
misleading, \S~360e(d)(1)(A). \newpage 


  After completing its review, the FDA may grant or deny premarket
approval. \S~360e(d). It may also condition approval on adherence
to performance standards, 21 CFR \S~861.1(b)(3), restrictions
upon sale or distribution, or compliance with other requirements,
\S~814.82. The agency is also free to impose device-specific
restrictions by regulation. \S~360j(e)(1).

  If the FDA is unable to approve a new device in its proposed form,
it may send an ``approvable letter'' indicating that the device
could be approved if the applicant submitted specified information or
agreed to certain conditions or restrictions. 21 CFR \S~814.44(e).
Alternatively, the agency may send a ``not approvable'' letter,
listing the grounds that justify denial and, where practical, measures
that the applicant could undertake to make the device approvable.
\S~814.44(f).

  Once a device has received premarket approval, the MDA forbids
the manufacturer to make, without FDA permission, changes in
design specifications, manufacturing processes, labeling, or
any other attribute, that would affect safety or effectiveness.
\S~360e(d)(6)(A)(i). If the applicant wishes to make such a
change, it must submit, and the FDA must approve, an application for
supplemental premarket approval, to be evaluated under largely the
same criteria as an initial application. \S~360e(d)(6); 21 CFR
\S~814.39(c).

  After premarket approval, the devices are subject to reporting
requirements. \S~360i. These include the obligation to inform the
FDA of new clinical investigations or scientific studies concerning the
device which the applicant knows of or reasonably should know of, 21
CFR \S~814.84(b)(2), and to report incidents in which the device may
have caused or contributed to death or serious injury, or malfunctioned
in a manner that would likely cause or contribute to death or serious
injury if it recurred, \S~803.50(a). The FDA has the power to
withdraw premarket approval based on newly reported data or existing
information and must withdraw ap\newpage proval if it determines that a
device is unsafe or ineffective under the conditions in its labeling.
\S~360e(e)(1); see also \S~360h(e) (recall authority).

\section{B}

  Except as otherwise indicated, the facts set forth in this section
appear in the opinion of the Court of Appeals. The device at issue is an
Evergreen Balloon Catheter marketed by defendant-respondent Medtronic,
Inc. It is a Class III device that received premarket approval from the
FDA in 1994; changes to its label received supplemental approvals in
1995 and 1996.

  Charles Riegel underwent coronary angioplasty in 1996, shortly after
suffering a myocardial infarction. App. to Pet. for Cert. 56a. His
right coronary artery was diffusely diseased and heavily calcified.
Riegel's doctor inserted the Evergreen Balloon Catheter into his
patient's coronary artery in an attempt to dilate the artery, although
the device's labeling stated that use was contraindicated for patients
with diffuse or calcified stenoses. The label also warned that the
catheter should not be inflated beyond its rated burst pressure of
eight atmospheres. Riegel's doctor inflated the catheter five times,
to a pressure of 10 atmospheres; on its fifth inflation, the catheter
ruptured. Complaint 3. Riegel developed a heart block, was placed on
life support, and underwent emergency coronary bypass surgery.

  Riegel and his wife Donna brought this lawsuit in April 1999, in the
United States District Court for the Northern District of New York.
Their complaint alleged that Medtronic's catheter was designed,
labeled, and manufactured in a manner that violated New York common law,
and that these defects caused Riegel to suffer severe and permanent
injuries. The complaint raised a number of common-law claims. The
District Court held that the MDA pre-empted Riegel's claims of strict
liability; breach of implied warranty; and negligence in the design,
testing, inspection, distribution, labeling, marketing, and sale of the
catheter. App. to \newpage  Pet. for Cert. 68a; Complaint 3--4. It also
held that the MDA pre-empted a negligent manufacturing claim insofar as
it was not premised on the theory that Medtronic violated federal law.
App. to Pet. for Cert. 71a. Finally, the court concluded that the MDA
pre-empted Donna Riegel's claim for loss of consortium to the extent
it was derivative of the pre-empted claims. \emph{Id.,} at 68a; see also
\emph{id.,} at 75a.\footnotemark[2]

  The United States Court of Appeals for the Second Circuit affirmed
these dismissals. 451 F. 3d 104 (2006). The court concluded that
Medtronic was ``clearly subject to the federal, device-specific
requirement of adhering to the standards contained in its individual,
federally approved'' premarket approval application. \emph{Id.,} at
118. The Riegels' claims were pre-empted because they ``would,
if successful, impose state requirements that differed from, or added
to,'' the devicespecific federal requirements. \emph{Id.,} at 121. We
granted certiorari.\footnotemark[3] 551 U.~S. 1144 (2007).

\section{II}

  Since the MDA expressly pre-empts only state requirements ``different
from, or in addition to, any requirement applicable\dots tothe
device'' under federal law, \S~360k(a)(1), we must determine
whether the Federal Government has established requirements applicable
to Medtronic's catheter. If so, we must then determine whether
the Riegels' \newpage  common-law claims are based upon New York
requirements with respect to the device that are ``different from,
or in addition to,'' the federal ones, and that relate to safety and
effectiveness. \S~360k(a).

^2 The District Court later granted summary judgment to Medtronic on
those claims of Riegel it had found not pre-empted, viz., that Medtronic
breached an express warranty and was negligent in manufacturing because
it did not comply with federal standards. App. to Pet. for Cert. 90a.
It consequently granted summary judgment as well on Donna Riegel's
derivative consortium claim. \emph{Ibid.} The Court of Appeals affirmed
these determinations, and they are not before us.

^3 Charles Riegel having died, Donna Riegel is now petitioner on her own
behalf and as administrator of her husband's estate. \emph{Post,} p.
804. For simplicity's sake, the terminology of our opinion draws no
distinction between Charles Riegel and the Estate of Charles Riegel and
refers to the claims as belonging to the Riegels.

  We turn to the first question. In \emph{Lohr,} a majority of this
Court interpreted the MDA's pre-emption provision in a manner
``substantially informed'' by the FDA regulation set forth at 21 CFR
\S~808.1(d). 518 U. S., at 495; see also \emph{id.,} at 500--501.
That regulation says that state requirements are pre-empted ``only when
the Food and Drug Administration has established specific counterpart
regulations or there are other specific requirements applicable to a
particular device~.~.~.~.'' 21 CFR \S~808.1(d). Informed by
the regulation, we concluded that federal manufacturing and labeling
requirements applicable across the board to almost all medical devices
did not pre-empt the common-law claims of negligence and strict
liability at issue in \emph{Lohr.} The federal requirements, we said, were
not requirements specific to the device in question---they reflected
``entirely generic concerns about device regulation generally.'' 518
U. S., at 501. While we disclaimed a conclusion that general federal
requirements could never pre-empt, or general state duties never be
pre-empted, we held that no pre-emption occurred in the case at hand
based on a careful comparison between the state and federal duties at
issue. \emph{Id.,} at 500--501.

  Even though substantial-equivalence review under \S~510(k) is device
specific, \emph{Lohr} also rejected the manufacturer's contention that
\S~510(k) approval imposed device-specific ``requirements.'' We
regarded the fact that products entering the market through \S~510(k)
may be marketed only so long as they remain substantial equivalents of
the relevant pre-1976 devices as a qualification for an exemption rather
than a requirement. \emph{Id.,} at 493--494; see also \emph{id.,} at 513
(O'Connor, J., concurring in part and dissenting in part).

  Premarket approval, in contrast, imposes ``requirements'' under the
MDA as we interpreted it in \emph{Lohr.} Unlike gen\newpage eral labeling
duties, premarket approval is specific to individual devices. And it
is in no sense an exemption from federal safety review---it \emph{is}
federal safety review. Thus, the attributes that \emph{Lohr} found
lacking in \S~510(k) review are present here. While \S~510(k) is ``
‘focused on \emph{equivalence,} not safety,' '' \emph{id.,} at 493
(opinion of the Court), premarket approval is focused on safety, not
equivalence. While devices that enter the market through \S~510(k)
have ``never been formally reviewed under the MDA for safety or
efficacy,'' \emph{ibid.,} the FDA may grant premarket approval only
after it determines that a device offers a reasonable assurance of
safety and effectiveness, \S~360e(d). And while the FDA does not
`` ‘require' '' that a device allowed to enter the market as a
substantial equivalent ``take any particular form for any particular
reason,'' 518 U. S., at 493, the FDA requires a device that has
received premarket approval to be made with almost no deviations from
the specifications in its approval application, for the reason that
the FDA has determined that the approved form provides a reasonable
assurance of safety and effectiveness.

\section{III}

  We turn, then, to the second question: whether the Riegels'
common-law claims rely upon ``any requirement'' of New York law
applicable to the catheter that is ``different from, or in addition
to,'' federal requirements and that ``relates to the safety or
effectiveness of the device or to any other matter included in a
requirement applicable to the device.'' \S~360k(a). Safety and
effectiveness are the very subjects of the Riegels' common-law claims,
so the critical issue is whether New York's tort duties constitute
``requirements'' under the MDA.

\section{A}

  In \emph{Lohr,} five Justices concluded that common-law causes of action
for negligence and strict liability do impose ``requirement[s]''
and would be pre-empted by federal require\newpage ments specific to
a medical device. See 518 U. S., at 512 (opinion of O'Connor,
J., joined by Rehnquist, C. J., and \textsc{Scalia} and \textsc{Thomas,} JJ.);
\emph{id.,} at 503--505 (\textsc{Breyer,} J., concurring in part and concurring
in judgment). We adhere to that view. In interpreting two other
statutes we have likewise held that a provision pre-empting state
``requirements'' pre-empted common-law duties. \emph{Bates} v. \emph{Dow
Agrosciences LLC,} 544 U.~S. 431 (2005), found common-law actions
to be pre-empted by a provision of the Federal Insecticide, Fungicide,
and Rodenticide Act that said certain States `` ‘shall not impose or
continue in effect \emph{any requirements} for labeling or packaging in
addition to or different from those required under this subchapter.'
'' \emph{Id.,} at 443 (discussing 7 U.~S.~C. \S~136v(b); emphasis
added). \emph{Cipollone} v. \emph{Liggett Group, Inc.,} 505 U. S.
504 (1992), held common-law actions preempted by a provision of
the Public Health Cigarette Smoking Act of 1969, 15 U.~S.~C.
\S~1334(b), which said that ``[n]o requirement or prohibition based
on smoking and health shall be imposed under State law with respect
to the advertising or promotion of any cigarettes'' whose packages
were labeled in accordance with federal law. See 505 U. S., at 523
(plurality opinion); \emph{id.,} at 548--549 (\textsc{Scalia,} J., concurring
in judgment in part and dissenting in part).

  Congress is entitled to know what meaning this Court will assign
to terms regularly used in its enactments. Absent other indication,
reference to a State's ``requirements'' includes its common-law
duties. As the plurality opinion said in \emph{Cipollone,} common-law
liability is ``premised on the existence of a legal duty,'' and a
tort judgment therefore establishes that the defendant has violated a
state-law obligation. \emph{Id.,} at 522. And while the common-law
remedy is limited to damages, a liability award `` ‘can be, indeed
is designed to be, a potent method of governing conduct and controlling
policy.' '' \emph{Id.,} at 521.

  In the present case, there is nothing to contradict this normal
meaning. To the contrary, in the context of this leg\newpage islation
excluding common-law duties from the scope of preemption would make
little sense. State tort law that requires a manufacturer's catheters
to be safer, but hence less effective, than the model the FDA has
approved disrupts the federal scheme no less than state regulatory law
to the same effect. Indeed, one would think that tort law, applied
by juries under a negligence or strict-liability standard, is less
deserving of preservation. A state statute, or a regulation adopted
by a state agency, could at least be expected to apply cost-benefit
analysis similar to that applied by the experts at the FDA: How many
more lives will be saved by a device which, along with its greater
effectiveness, brings a greater risk of harm? A jury, on the other hand,
sees only the cost of a more dangerous design, and is not concerned
with its benefits; the patients who reaped those benefits are not
represented in court. As \textsc{Justice Breyer} explained in \emph{Lohr,} it
is implausible that the MDA was meant to ``grant greater power (to
set state standards ‘different from, or in addition to,' federal
standards) to a single state jury than to state officials acting through
state administrative or legislative lawmaking processes.'' 518 U. S.,
at 504. That perverse distinction is not required or even suggested by
the broad language Congress chose in the MDA,\footnotemark[4] and we will not turn
somersaults to create it. \newpage 

^4 The Riegels point to \S~360k(b), which authorizes the FDA to exempt
state ``requirements'' from pre-emption under circumstances that would
rarely be met for common-law duties. But a law that permits an agency to
exempt certain ``requirements'' from pre-emption does not suggest that
no other ``requirements'' exist. The Riegels also invoke \S~360h(d),
which provides that compliance with certain FDA orders ``shall not
relieve any person from liability under Federal or State law.'' This
indicates that some state-law claims are not pre-empted, as we held in
\emph{Lohr.} But it could not possibly mean that \emph{all} state-law claims
are not pre-empted, since that would deprive the MDA pre-emption clause
of all content. And it provides no guidance as to which state-law claims
are pre-empted and which are not.


\section{B}

  The dissent would narrow the pre-emptive scope of the term
``requirement'' on the grounds that it is ``difficult to believe
that Congress would, without comment, remove all means of judicial
recourse'' for consumers injured by FDAapproved devices. \emph{Post,} at
337 (opinion of \textsc{Ginsburg, J.})(internal quotation marks omitted).
But, as we have explained, this is exactly what a pre-emption clause
for medical devices does by its terms. The operation of a law enacted
by Congress need not be seconded by a committee report on pain of
judicial nullification. See, \emph{e. g., Connecticut Nat. Bank} v.
\emph{Germain,} 503 U.~S. 249, 253--254 (1992). It is not our job to
speculate upon congressional motives. If we were to do so, however, the
only indication available---the text of the statute---suggests that the
solicitude for those injured by FDA-approved devices, which the dissent
finds controlling, was overcome in Congress's estimation by solicitude
for those who would suffer without new medical devices if juries were
allowed to apply the tort law of 50 States to all innovations.\footnotemark[5]

  In the case before us, the FDA has supported the position taken by
our opinion with regard to the meaning of the statute. We have found
it unnecessary to rely upon that agency view because we think the
statute itself speaks clearly to the point at issue. If, however, we
had found the statute ambiguous and had accorded the agency's current
position deference, the dissent is correct, see \emph{post,} at 338, n.
8, that---inasmuch as mere \emph{Skidmore} deference would seemingly be
at issue---the degree of deference might be reduced by the fact that
the agency's earlier position was different. See \emph{Skidmore} v.
\emph{Swift \& Co.,} 323 U.~S. 134 (1944); \emph{United States} \newpage  v.
\emph{Mead Corp.,} 533 U.~S. 218 (2001); \emph{Good Samaritan Hospital} v.
\emph{Shalala,} 508 U.~S. 402, 417 (1993). But of course the agency's
earlier position (which the dissent describes at some length, \emph{post,}
at 337--338, and finds preferable) is even more compromised, indeed
deprived of all claim to deference, by the fact that it is no longer the
agency's position.

^5 Contrary to \textsc{Justice Stevens'} contention, \emph{post,} at 331
(opinion concurring in part and concurring in judgment), we do not
``advanc[e]'' this argument. We merely suggest that if one were to
speculate upon congressional purposes, the best evidence for that would
be found in the statute.

  The dissent also describes at great length the experience under the
FDCA with respect to drugs and food and color additives. \emph{Post,}
at 339--342. Two points render the conclusion the dissent seeks
to draw from that experience---that the pre-emption clause permits
tort suits---unreliable. (1) It has not been established (as the
dissent assumes) that no tort lawsuits are pre-empted by drug or
additive approval under the FDCA. (2) If, as the dissent believes, the
pre-emption clause permits tort lawsuits for medical devices just as
they are (by hypothesis) permitted for drugs and additives; and if,
as the dissent believes, Congress wanted the two regimes to be alike;
Congress could have applied the pre-emption clause to the entire FDCA.
It did not do so, but instead wrote a pre-emption clause that applies
only to medical devices.

\section{C}

  The Riegels contend that the duties underlying negligence,
strict-liability, and implied-warranty claims are not preempted even
if they impose `` ‘requirements,' '' because general common-law
duties are not requirements maintained `` ‘with respect to
devices.' '' Brief for Petitioner 34--36. Again, a majority
of this Court suggested otherwise in \emph{Lohr.} See 518 U. S., at
504--505 (opinion of \textsc{Breyer,} J.); \emph{id.,} at 514 (opinion
of O'Connor, J., joined by Rehnquist, C. J., and \textsc{Scalia} and
\textsc{Thomas,} JJ.).\footnotemark[6] And with good reason. The \newpage  language of
the statute does not bear the Riegels' reading. The MDA provides that
no State ``may establish or continue in effect \emph{with respect to a
device\dots any requirement}'' relating to safety or effectiveness
that is different from, or in addition to, federal requirements.
\S~360k(a) (emphasis added). The Riegels' suit depends upon New
York's ``continu[ing] in effect'' general tort duties ``with
respect to'' Medtronic's catheter. Nothing in the statutory text
suggests that the pre-empted state requirement must apply \emph{only} to
the relevant device, or only to medical devices and not to all products
and all actions in general.

^6 The opinions joined by these five Justices dispose of the Riegels'
assertion that \emph{Lohr} held common-law duties were too general to
qualify as duties ``with respect to a device.'' The majority opinion
in \emph{Lohr} also disavowed this conclusion, for it stated that the Court
did ``not believe that \newpage  [the MDA's] statutory and regulatory
language necessarily precludes\dots ‘general' state requirements
from ever being pre-empted~.~.~.~.'' 518 U. S., at 500.

  The Riegels' argument to the contrary rests on the text of an FDA
regulation which states that the MDA's pre-emption clause does not
extend to certain duties, including ``[s]tate or local requirements
of general applicability where the purpose of the requirement relates
either to other products in addition to devices (e. g., requirements
such as general electrical codes, and the Uniform Commercial Code
(warranty of fitness)), or to unfair trade practices in which the
requirements are not limited to devices.'' 21 CFR \S~808.1(d)(1).
Even assuming that this regulation could play a role in defining the
MDA's pre-emptive scope, it does not provide unambiguous support for
the Riegels' position. The agency's reading of its own rule is
entitled to substantial deference, see \emph{Auer} v. \emph{Robbins,} 519
U. S. 452, 461 (1997), and the FDA's view put forward in this case
is that the regulation does not refer to general tort duties of care,
such as those underlying the claims in this case that a device was
designed, labeled, or manufactured in an unsafe or ineffective manner,
Brief for United States as \emph{Amicus Curiae} 27--28. That is so,
according to the FDA, because the regulation excludes from pre-emption
requirements that relate only incidentally to medical devices, but not
other requirements. General tort \newpage  duties of care, unlike fire
codes or restrictions on trade practices, ``directly regulate'' the
device itself, including its design. \emph{Id.,} at 28. We find the
agency's explanation less than compelling, since the same could be
said of general requirements imposed by electrical codes, the Uniform
Commercial Code, or unfair-trade-practice law, which the regulation
specifically excludes from pre-emption.

  Other portions of 21 CFR \S~808.1, however, support the agency's
view that \S~808.1(d)(1) has no application to this case (though still
failing to explain why electrical codes, the Uniform Commercial Code,
or unfair-trade-practice requirements are different). Section 808.1(b)
states that the MDA sets forth a ``general rule'' pre-empting state
duties ``having the force and effect of law (whether established by
statute, ordinance, regulation, \emph{or court decision\\)~.~.~.~.''
(Emphasis added.) This sentence is far more comprehensible under the
FDA's view that \S~808.1(d)(1) has no application here than under the
Riegels' view. We are aware of no duties established by court decision
other than common-law duties, and we are aware of no common-law duties
that relate solely to medical devices.

  The Riegels' reading is also in tension with the regulation's
statement that adulteration and misbranding claims are pre-empted when
they ``ha[ve] the effect of establishing a substantive requirement
for a specific device, e. g., a specific labeling requirement'' that
is ``different from, or in addition to,'' a federal requirement.
\S~808.1(d)(6)(ii). Surely this means that the MDA would pre-empt
a jury determination that the FDA-approved labeling for a pacemaker
violated a state common-law requirement for additional warnings. The
Riegels' reading of \S~808.1(d)(1), however, would allow a claim for
tortious mislabeling to escape pre-emption so long as such a claim could
also be brought against objects other than medical devices.

  All in all, we think that \S~808.1(d)(1) can add nothing to our
analysis but confusion. Neither accepting nor rejecting the \newpage 
proposition that this regulation can properly be consulted to determine
the statute's meaning; and neither accepting nor rejecting the FDA's
distinction between general requirements that directly regulate and
those that regulate only incidentally; the regulation fails to alter
our interpretation of the text insofar as the outcome of this case is
concerned.

\section{IV}

  State requirements are pre-empted under the MDA only to the extent
that they are ``different from, or in addition to'' the requirements
imposed by federal law. \S~360k(a)(1). Thus, \S~360k does not
prevent a State from providing a damages remedy for claims premised
on a violation of FDA regulations; the state duties in such a case
``parallel,'' rather than add to, federal requirements. \emph{Lohr,}
518 U. S., at 495; see also \emph{id.,} at 513 (O'Connor, J., concurring
in part and dissenting in part). The District Court in this case
recognized that parallel claims would not be pre-empted, see App. to
Pet. for Cert. 70a--71a, but it interpreted the claims here to assert
that Medtronic's device violated state tort law notwithstanding
compliance with the relevant federal requirements, see \emph{id.,}
at 68a. Although the Riegels now argue that their lawsuit raises
parallel claims, they made no such contention in their briefs before the
Second Circuit, nor did they raise this argument in their petition for
certiorari. We decline to address that argument in the first instance
here.

\hrule

  For the foregoing reasons, the judgment of the Court of Appeals is

        \emph{Affirmed.}
