% Syllabus
% Reporter of Decisions

\setcounter{page}{148}

  Alleging losses after purchasing Charter Communications, Inc., common stock, petitioner filed suit against respondents and others under \S~10(b) of the Securities Exchange Act of 1934 and Securities and Exchange Commission (SEC) Rule 10b--5. Acting as Charter's customers and suppliers, respondents had agreed to arrangements that allowed Charter to mislead its auditor and issue a misleading financial statement affecting its stock price, but they had no role in preparing or disseminating the financial statement. Affirming the District Court's dismissal of respondents, the Eighth Circuit ruled that the allegations did not show that respondents made misstatements relied upon by the public or violated a duty to disclose. The court observed that, at most, respondents had aided and abetted Charter's misstatement, and noted that the private cause of action this Court has found implied in \S~10(b) and Rule 10b--5, \emph{Superintendent of Ins. of N. Y.} v. \emph{Bankers Life \& Casualty Co.,} 404 U.~S. 6, 13, n. 9, does not extend to aiding and abetting a \S~10(b) violation, see \emph{Central Bank of Denver, N. A.} v. \emph{First Interstate Bank of Denver, N. A.,} 511 U.~S. 164, 191. \emph{Held:} The \S~10(b) private right of action does not reach respondents because Charter investors did not rely upon respondents' statements or representations. Pp. 156--167.

  (a) Although \emph{Central Bank} prompted calls for creation of an express cause of action for aiding and abetting, Congress did not follow this course. Instead, in \S~104 of the Private Securities Litigation Reform Act of 1995 (PSLRA), it directed the SEC to prosecute aiders and abettors. Thus, the \S~10(b) private right of action does not extend to aiders and abettors. Because the conduct of a secondary actor must therefore satisfy each of the elements or preconditions for \S~10(b) liability, the plaintiff must prove, as here relevant, reliance upon a material misrepresentation or omission by the defendant. Pp. 156--158.

  (b) The Court has found a rebuttable presumption of reliance in two circumstances. First, if there is an omission of a material fact by one with a duty to disclose, the investor to whom the duty was owed need not provide specific proof of reliance. \emph{Affiliated Ute Citizens of Utah} v. \emph{United States,} 406 U.~S. 128, 153--154. Second, under the fraud-on\newpage the-market doctrine, reliance is presumed when the statements at issue become public. Neither presumption applies here: Respondents had no duty to disclose; and their deceptive acts were not communicated to the investing public during the relevant times. Petitioner, as a result, cannot show reliance upon any of respondents' actions except in an indirect chain that is too remote for liability. P. 159.

  (c) Petitioner's reference to so-called ``scheme liability'' does not, absent a public statement, answer the objection that petitioner did not in fact rely upon respondents' deceptive conduct. Were the Court to adopt petitioner's concept of reliance---\emph{i. e.,} that in an efficient market investors rely not only upon the public statements relating to a security but also upon the transactions those statements reflect---the implied cause of action would reach the whole marketplace in which the issuing company does business. There is no authority for this rule. Reliance is tied to causation, leading to the inquiry whether respondents' deceptive acts were immediate or remote to the injury. Those acts, which were not disclosed to the investing public, are too remote to satisfy the reliance requirement. It was Charter, not respondents, that misled its auditor and filed fraudulent financial statements; nothing respondents did made it necessary or inevitable for Charter to record the transactions as it did. The Court's precedents counsel against petitioner's attempt to extend the \S~10(b) private cause of action beyond the securities markets into the realm of ordinary business operations, which are governed, for the most part, by state law. See, \emph{e. g., Marine Bank} v. \emph{Weaver,} 455 U.~S. 551, 556. The argument that there could be a reliance finding if this were a common-law fraud action is answered by the fact that \S~10(b) does not incorporate common-law fraud into federal law, see, \emph{e. g., SEC} v. \emph{Zandford,} 535 U.~S. 813, 820, and should not be interpreted to provide a private cause of action against the entire marketplace in which the issuing company operates, cf. \emph{Blue Chip Stamps} v. \emph{Manor Drug Stores,} 421 U. S. 723, 733, n. 5. Petitioner's theory, moreover, would put an unsupportable interpretation on Congress' specific response to \emph{Central Bank} in PSLRA \S~104 by, in substance, reviving the implied cause of action against most aiders and abettors and thereby undermining Congress' determination that this class of defendants should be pursued only by the SEC. The practical consequences of such an expansion provide a further reason to reject petitioner's approach. The extensive discovery and the potential for uncertainty and disruption in a lawsuit could allow plaintiffs with weak claims to extort settlements from innocent companies. See, \emph{e. g., Blue Chip, supra,} at 740--741. It would also expose to such risks a new class of defendants---overseas firms with no other exposure to U. S. securities laws---thereby deterring them from doing business here, raising the cost of being a \newpage  publicly traded company under U. S. law, and shifting securities offerings away from domestic capital markets. Pp. 159--164.

  (d) Upon full consideration, the history of the \S~10(b) private right of action and the careful approach the Court has taken before proceeding without congressional direction provide further reasons to find no liability here. The \S~10(b) private cause of action is a judicial construct that Congress did not direct in the text of the relevant statutes. See, \emph{e. g., Lampf, Pleva, Lipkind, Prupis \& Petigrow} v. \emph{Gilbertson,} 501 U.~S. 350, 358--359. Separation of powers provides good reason for the nowsettled view that an implied cause of action exists only if the underlying statute can be interpreted to disclose the intent to create one, see, \emph{e. g., Alexander} v. \emph{Sandoval,} 532 U.~S. 275, 286--287. The decision to extend the cause of action is thus for the Congress, not for this Court. This restraint is appropriate in light of the PSLRA, in which Congress ratified the implied right of action after the Court moved away from a broad willingness to imply such private rights, see, \emph{e. g., Merrill Lynch, Pierce, Fenner \& Smith, Inc.} v. \emph{Curran,} 456 U.~S. 353, 381--382, and n. 66. It is appropriate for the Court to assume that when PSLRA \S~104 was enacted, Congress accepted the \S~10(b) private right as then defined but chose to extend it no further. See, \emph{e. g., Alexander, supra,} at 286--287. Pp. 164--166.

443 F. 3d 987, affirmed and remanded.

  \textsc{Kennedy,} J., delivered the opinion of the Court, in which \textsc{Roberts,} C. J., and \textsc{Scalia, Thomas,} and \textsc{Alito,} JJ., joined. \textsc{Stevens,} J., filed a dissenting opinion, in which \textsc{Souter} and \textsc{Ginsburg,} JJ., joined, \emph{post,} p. 167. \textsc{Breyer,} J., took no part in the consideration or decision of the case.

  \emph{Stanley M. Grossman} argued the cause for petitioner. With him on the briefs were \emph{Marc I. Gross} and \emph{Joshua B. Silverman.}

  \emph{Stephen M. Shapiro} argued the cause for respondents. With him on the brief were \emph{Andrew J. Pincus, Timothy S. Bishop, John P. Schmitz, Charles Rothfeld, J. Brett Busby, Oscar N. Persons, Susan E. Hurd, Stephen M. Sacks,} and \emph{John C. Massaro.}

  \emph{Deputy Solicitor General Hungar} argued the cause for the United States as \emph{amicus curiae} urging affirmance. \newpage  With him on the brief were \emph{Solicitor General Clement} and \emph{Kannon K. Shanmugam.}[[*]]

\footnotetext[*]{Briefs of \emph{amici curiae} urging reversal were filed for the State of Arkansas et al. by \emph{Dustin McDaniel,} Attorney General of Arkansas, and \emph{Stanley D. Bernstein,} and by the Attorneys General for their respective States as follows: \emph{Stuart Rabner} of New Jersey and \emph{Patrick C. Lynch} of Rhode Island; for the State of Ohio et al. by \emph{Marc Dann,} Attorney General of Ohio, \emph{Elise W. Porter,} Acting Solicitor General, \emph{Christopher R. Geidner} and \emph{Robert J. Krummen,} Deputy Solicitors, \emph{Beth A. Finnerty, Randall W.} \emph{Knutti,} and \emph{Andrea L. Seidt,} Assistant Attorneys General, by \emph{Greg Abbott,} Attorney General of Texas, and \emph{David C. Mattax,} and by the Attorneys General for their respective jurisdictions as follows: \emph{Talis J. Colberg} of Alaska, \emph{Terry Goddard} of Arizona, \emph{Richard Blumenthal} of Connecticut, \emph{Linda Singer} of the District of Columbia, \emph{Mark J. Bennett} of Hawaii, \emph{Lisa} \emph{Madigan} of Illinois, \emph{Thomas J. Miller} of Iowa, \emph{Gregory D. Stumbo} of Kentucky, \emph{G. Steven Rowe} of Maine, \emph{Douglas F. Gansler} of Maryland, \emph{Martha Coakley} of Massachusetts, \emph{Michael A. Fox} of Michigan, \emph{Lori Swanson} of Minnesota, \emph{Jim Hood} of Mississippi, \emph{Jeremiah W. (Jay) Nixon} of Missouri, \emph{Mike McGrath} of Montana, \emph{Catherine Cortez Masto} of Nevada, \emph{Kelly A. Ayotte} of New Hampshire, \emph{Gary King} of New Mexico, \emph{Andrew Cuomo} of New York, \emph{Wayne Stenehjem} of North Dakota, \emph{W. A. Drew Edmondson} of Oklahoma, \emph{Hardy Myers} of Oregon, \emph{Roberto J. Sánchez-Ramos} of Puerto Rico, \emph{Henry McMaster} of South Carolina, \emph{Robert E. Cooper, Jr.,} of Tennessee, \emph{Mark L. Shurtleff} of Utah, \emph{William H. Sorrell} of Vermont, \emph{Darrell V. McGraw, Jr.,} of West Virginia, and \emph{J. B. Van Hollen} of Wisconsin; for AARP et al. by \emph{Deborah Zuckerman, Jonathan W.} \emph{Cuneo, Robert J. Cynkar, Michael G. Lenett,} and \emph{Matthew Wiener;} for the American Association for Justice by \emph{Louis M. Bograd;} for the California State Teachers' Retirement System by \emph{Steven N. Williams} and \emph{Joseph W. Cotchett;} for Change to Win et al. by \emph{Patrick J. Szymanski;} for Former SEC Commissioners by \emph{Arthur R. Miller} and \emph{Meyer Eisenberg;} for the Los Angeles County Employees Retirement Association et al. by \emph{Stuart M. Grant, David L. Muir,} and \emph{PeterH.Mixon,} by \emph{Mr. Blumenthal,} Attorney General of Connecticut, and by \emph{Michael A. Cardozo;} for the New York State Teachers' Retirement System et al. by \emph{Max W. Berger;} for the North American Securities Administrators Association, Inc., by \emph{Alfred E. T. Rusch;} for the Honorable John Conyers, Jr., et al. by \emph{James Segel} and \emph{Lawranne Stewart;} and for James D. Cox et al. by \emph{Jill E. Fisch, pro se.}

Briefs of \emph{amici curiae} urging affirmance were filed for the American Bankers Association et al. by \emph{H. Rodgin Cohen, David H. Braff, Robert J.}}
