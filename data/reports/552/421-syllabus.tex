% Syllabus
% Reporter of Decisions

\setcounter{page}{421}

  One element of tax evasion under 26 U.~S.~C. \S~7201 is ``the
existence of a tax deficiency.'' \emph{Sansone} v. \emph{United States,}
380 U.~S. 343, 351. Petitioner Boulware was charged with criminal
tax evasion and filing a false income tax return for diverting funds
from a closely held corporation, HIE, of which he was the president,
founder, and controlling shareholder. To support his argument that
the Government could not establish the tax deficiency required to
convict him, Boulware sought to introduce evidence that HIE had no
earnings and profits in the relevant taxable years, so he in effect
received distributions of property that were returns of capital, up
to his basis in his stock, which are not taxable, see 26 U.~S.~C.
\S\S~301 and 316(a). Under \S~301(a), unless the Internal Revenue
Code requires otherwise, a ``distribution of property'' ``made
by a corporation to a shareholder with respect to its stock shall
be treated in the manner provided in [\S301(c)].'' Section 301(c)
provides that the portion of the distribution that is a ``dividend,''
as defined by \S~316(a), must be included in the recipient's gross
income; and the portion that is not a dividend is, depending on the
shareholder's basis for his stock, either a nontaxable return of
capital or a taxable capital gain. Section 316(a) defines ``dividend''
as a ``distribution'' out of ``earnings and profits.'' The District
Court granted the Government's \emph{in limine} motion to bar evidence
supporting Boulware's return-of-capital theory, relying on the Ninth
Circuit's \emph{Miller} decision that a diversion of funds in a criminal
tax evasion case may be deemed a return of capital only if the taxpayer
or corporation demonstrates that the distributions were intended to be
such a return. The court later found Boulware's proffer of evidence
insufficient under \emph{Miller} and declined to instruct the jury on
his theory. In affirming his conviction, the Ninth Circuit held that
Boulware's proffer was properly rejected under \emph{Miller} because he
offered no proof that the amounts diverted were intended as a return of
capital when they were made.

\emph{Held:}

  A distributee accused of criminal tax evasion may claim
return-ofcapital treatment without producing evidence that, when the
distribution occurred, either he or the corporation intended a return of
capital. Pp. 429--439. \newpage 

  (a) Tax classifications like ``dividend'' and ``return of
capital'' turn on a transaction's ``objective economic realities,''
not ``the particular form the parties employed.'' \emph{Frank Lyon
Co.} v. \emph{United States,} 435 U.~S. 561, 573. In economic
reality, a shareholder's informal receipt of corporate property
``may be as effective a means of distributing profits among
stockholders as the formal declaration of a dividend,'' \emph{Palmer}
v. \emph{Commissioner,} 302 U.~S. 63, 69, or as effective a means
of returning a shareholder's capital, see \emph{ibid}. Economic
substance remains the touchstone for characterizing funds that a
shareholder diverts before they can be recorded on a corporation's
books. Pp. 429--430.

  (b) \emph{Miller}'s view that a return-of-capital defense requires
evidence of a corresponding contemporaneous intent sits uncomfortably
not only with the tax law's economic realism, but also with the
particular wording of \S\S~301 and 316(a). As these sections are
written, the tax consequences of a corporation's distribution made
with respect to stock depend, not on anyone's purpose to return
capital or get it back, but on facts wholly independent of intent:
whether the corporation had earnings and profits, and the amount of the
taxpayer's basis for his stock. The \emph{Miller} court could claim no
textual hook for its contemporaneous intent requirement, but argued that
it avoided supposed anomalies. The court, however, mistakenly reasoned
that applying \S\S~301 and 316(a) in criminal cases unnecessarily
emphasizes the deficiency's amount while ignoring the willfulness
of the intent to evade taxes. Willfulness is an element of the crimes
because the substantive provisions defining tax evasion and filing
a false return expressly require it, see, \emph{e. g.,} \S~7201.
Nothing in \S\S~301 and 316(a) relieves the Government of the burden of
proving willfulness or impedes it from doing so if there is evidence of
willfulness. The \emph{Miller} court also erred in finding it troublesome
that, without a contemporaneous intent requirement, a shareholder
distributee would be immune from punishment if the corporation had no
earnings and profits but convicted if the corporation did have earning
and profits. An acquittal in the former instance would in fact result
merely from the Government's failure to prove an element of the
crime. The fact that a shareholder of a successful corporation may have
different tax liability from a shareholder of a corporation without
earnings and profits merely follows from the way \S\S~301 and 316(a)
are written and from \S~7201's tax deficiency requirement. Even if
there were compelling reasons to extend \S~7201 to cases in which
no taxes are owed, Congress, not the Judiciary, would have to do the
rewriting. Pp. 430--434.

  (c) \emph{Miller} also suffers from its own anomalies. First, \S\S~301
and 316 are odd stalks for grafting a contemporaneous intent
requirement. Correct application of their rules will often become
possible only at the end of the corporation's tax year, regardless of
the shareholder or corpo\newpage ration's understanding months earlier
when a particular distribution may have been made. Moreover, \S~301(a),
which expressly provides that distributions made with respect to stock
``shall be treated in the manner provided in [\S301(c)],'' ostensibly
provides for all variations of tax treatment of such distributions
unless a separate Code provision requires otherwise. Yet \emph{Miller}
effectively converts the section into one of merely partial coverage,
leaving the tax status of one class of distributions in limbo in
criminal cases. Allowing \S~61(a) of the Code, which defines gross
income, ``[e]xcept as otherwise provided,'' as ``all income from
whatever source derived,'' to step in where \S~301(a) has been pushed
aside would sanction yet another eccentricity: \S~301(a) would not
cover what it says it ``shall'' (distributions with respect to stock
for which no more specific provision is made), while \S~61(a) would
have to apply to what by its terms it should not (a receipt of funds
for which tax treatment is ``otherwise provided'' in \S~301(a)).
\emph{Miller} erred in requiring contemporaneous intent, and the Ninth
Circuit's judgment here, relying on \emph{Miller,} is likewise erroneous.
Pp. 434--436.

  (d) This Court declines to address the Government's argument
that the judgment should be affirmed on the ground that before any
distribution may be treated as a return of capital, it must first be
distributed to the shareholder ``with respect to\dots stock.''
The facts in this case have not been raked over with that condition in
mind, and any canvas of evidence and Boulware's proffer should be made
by a court familiar with the entire evidentiary record. Nor will the
Court take up in the first instance the question whether an unlawful
diversion may ever be deemed a ``distribution\dots with respect to
[a corporation's] stock.'' Pp. 436--439.

470 F. 3d 931, vacated and remanded.

  \textsc{Souter,} J., delivered the opinion for a unanimous Court.

  \emph{John D. Cline} argued the cause for petitioner. With him on the
briefs was \emph{C. Kevin Marshall.}

  \emph{Deanne E. Maynard} argued the cause for the United States. With her
on the brief were \emph{Solicitor General Clement, Acting Assistant Attorney
General Morrison, Deputy Solicitor General Dreeben, Alan Hechtkopf,
Karen Quesnel,} and \emph{S. Robert Lyons.\\[[*]]

^* \emph{John L. Pollok} and \emph{Joshua L. Dratel} filed a brief for the
National Association of Criminal Defense Lawyers as \emph{amicus curiae.}
