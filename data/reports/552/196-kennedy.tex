% Concurring in the Judgment
% Kennedy

\setcounter{page}{209}

  \textsc{Justice Kennedy,} with whom \textsc{Justice Breyer} joins as to Part II, concurring in the judgment.

  The Court's analysis, in my view, is correct in important respects; but my own understanding of the controlling principles counsels concurrence in the judgment and the expression of these additional observations. \newpage 

\section{I}

   When a state-mandated primary is used to select delegates to conventions or nominees for office, the State is bound not to design its ballot or election processes in ways that impose severe burdens on First Amendment rights of expression and political participation. See \emph{Kusper} v. \emph{Pontikes,} 414 U.~S. 51, 57--58 (1973); see also \emph{California Democratic Party} v. \emph{Jones,} 530 U.~S. 567, 581--582 (2000); cf. \emph{Lubin} v. \emph{Panish,} 415 U.~S. 709, 716 (1974); \emph{Bullock} v. \emph{Carter,} 405 U.~S. 134, 144 (1972); \emph{Gray} v. \emph{Sanders,} 372 U.~S. 368, 380 (1963). Respondents' objection to New York's scheme of nomination by convention is that it is difficult for those who lack party connections or party backing to be chosen as a delegate or to become a nominee for office. Were the state-mandatedand-designed nominating convention the sole means to attain access to the general election ballot there would be considerable force, in my view, to respondents' contention that the First Amendment prohibits the State from requiring a delegate selection mechanism with the rigidities and difficulties attendant upon this one. The system then would be subject to scrutiny from the standpoint of a ``reasonably diligent independent candidate,'' \emph{Storer} v. \emph{Brown,} 415 U. S. 724, 742 (1974). The Second Circuit took this approach. 462 F. 3d 161, 196 (2006).

  As the Court is careful to note, however, New York has a second mechanism for placement on the final election ballot. \emph{Ante,} at 201. One who seeks to be a justice of the New York Supreme Court may qualify by a petition process. The petition must be signed by the lesser of (1) 5 percent of the number of votes last cast for Governor in the judicial district or (2) either 3,500 or 4,000 voters (depending on the district). This requirement has not been shown to be an unreasonable one, a point respondents appear to concede. True, the candidate who gains ballot access by petition does not have a party designation; but the candidate is still considered by the voters.\newpage 

  The petition alternative changes the analysis. Cf. \emph{Munro} v. \emph{Socialist Workers Party,} 479 U.~S. 189, 199 (1986) (``It can hardly be said that Washington's voters are denied freedom of association because they must channel their expressive activity into a campaign at the primary as opposed to the general election'').

  This is not to say an alternative route to the general election exempts the delegate primary/nominating convention from all scrutiny. For instance, the Court in \emph{Bullock,} after determining that Texas' primary election filing fees were so ``patently exclusionary'' on the basis of wealth as to invoke strict scrutiny under the Equal Protection Clause, rejected the argument that candidate access to the general election without a fee saved the statute. 405 U. S., at 143--144, 146--147 (``[W]e can hardly accept as reasonable an alternative that requires candidates and voters to abandon their party affiliations in order to avoid the burdens of the filing fees''). But there is a dynamic relationship between, in this case, the convention system and the petition process; higher burdens at one stage are mitigated by lower burdens at the other. See \emph{Burdick} v. \emph{Takushi,} 504 U.~S. 428, 448 (1992) (\textsc{Kennedy,} J., dissenting) (``The liberality of a State's ballot access laws is one determinant of the extent of the burden imposed by the write-in ban; it is not, though, an automatic excuse for forbidding all write-in voting''); Persily, Candidates v. Parties: Constitutional Constraints on Primary Ballot Access Laws, 89 Geo. L. J. 2181, 2214--2216 (2001). And, though the point does not apply here, there are certain injuries (as in \emph{Bullock}) that are so severe they are unconstitutional no matter how minor the burdens at the other stage. As the Court recognized in \emph{Kusper,} moreover, there is an individual right to associate with the political party of one's choice and to have a voice in the selection of that party's candidate for public office. See 414 U. S., at 58. On the particular facts and circumstances of this case, then, I reach the same conclusion the Court does. \newpage 

\section{II}

  It is understandable that the Court refrains from commenting upon the use of elections to select the judges of the State's courts of general jurisdiction, for New York has the authority to make that decision. This closing observation, however, seems to be in order.

  When one considers that elections require candidates to conduct campaigns and to raise funds in a system designed to allow for competition among interest groups and political parties, the persisting question is whether that process is consistent with the perception and the reality of judicial independence and judicial excellence. The rule of law, which is a foundation of freedom, presupposes a functioning judiciary respected for its independence, its professional attainments, and the absolute probity of its judges. And it may seem difficult to reconcile these aspirations with elections.

  Still, though the Framers did not provide for elections of federal judges, most States have made the opposite choice, at least to some extent. In light of this longstanding practice and tradition in the States, the appropriate practical response is not to reject judicial elections outright but to find ways to use elections to select judges with the highest qualifications. A judicial election system presents the opportunity, indeed the civic obligation, for voters and the community as a whole to become engaged in the legal process. Judicial elections, if fair and open, could be an essential forum for society to discuss and define the attributes of judicial excellence and to find ways to discern those qualities in the candidates. The organized bar, the legal academy, public advocacy groups, a principled press, and all the other components of functioning democracy must engage in this process.

  Even in flawed election systems there emerge brave and honorable judges who exemplify the law's ideals. But it is unfair to them and to the concept of judicial independence if \newpage  the State is indifferent to a selection process open to manipulation, criticism, and serious abuse.

  Rule of law is secured only by the principled exercise of political will. If New York statutes for nominating and electing judges do not produce both the perception and the reality of a system committed to the highest ideals of the law, they ought to be changed and to be changed now. But, as the Court today holds, and for further reasons given in this separate opinion, the present suit does not permit us to invoke the Constitution in order to intervene.

\section{III}

  With these observations, I concur in the judgment of the Court.
