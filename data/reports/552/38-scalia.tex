% Concurring
% Scalia

\setcounter{page}{60}

  \textsc{Justice Scalia,} concurring.

  I join the opinion of the Court.

  In \emph{Rita} v. \emph{United States,} 551 U.~S. 338, 370--375 (2007)
(opinion concurring in part and concurring in judgment), I wrote
separately to state my view that any appellate review of sentences for
substantive reasonableness will necessarily result in a sentencing
scheme constitutionally indistinguishable from the mandatory Guidelines
struck down in \emph{United States} v. \emph{Booker,} 543 U.~S. 220
(2005). Whether a sentencing scheme uses mandatory Guidelines, a
``proportionality test'' for Guidelines variances, or a deferential
abuse-of-discretion standard, there will be some sentences upheld only
on the basis of additional judge-found facts.

  Although I continue to believe that substantivereasonableness review
is inherently flawed, I give \emph{stare decisis} effect to the statutory
holding of \emph{Rita.} The highly deferential standard adopted by the
Court today will result in far fewer unconstitutional sentences than
the proportionality standard employed by the Eighth Circuit. Moreover,
as I noted in \emph{Rita,} the Court has not foreclosed as-applied
constitutional challenges to sentences. The door therefore remains open
for a defendant to demonstrate that his sentence, whether inside or
outside the advisory Guidelines range, would not have been upheld but
for the existence of a fact found by the sentencing judge and not by the
jury.
