% Dissenting
% Thomas

\setcounter{page}{486}

  \textsc{Justice Thomas,} with whom \textsc{Justice Scalia} joins, dissenting.

  Petitioner essentially asks this Court to second-guess the fact-based
determinations of the Louisiana courts as to the reasons for a
prosecutor's decision to strike two jurors. The evaluation of a
prosecutor's motives for striking a juror is at bottom a credibility
judgment, which lies `` ‘peculiarly within a trial judge's
province.' '' \emph{Hernandez} v. \emph{New York,} 500 U.~S. 352, 365
(1991) (plurality opinion) (quoting \emph{Wainwright} v. \emph{Witt,} 469
U. S. 412, 428 (1985)); \emph{Hernandez, supra,} at 372 (O'Connor, J.,
concurring in judgment); \emph{ante,} at 477. ``[I]n the absence of
exceptional circumstances, we [should] defer to state-court factual
findings.'' \emph{Hernandez,} 500 U. S., at 366 (plurality opinion).
None of the evidence in the record as to jurors Jeffrey Brooks and
Elaine Scott demonstrates that the trial court clearly erred in finding
they were not stricken on the basis of race. Because the trial court's
determination was a ``permissible view of the evidence,'' \emph{id.,}
at 369, I would affirm the judgment of the Louisiana Supreme Court.

  The Court begins by setting out the ``deferential standard,''
\emph{ante,} at 479, that we apply to a trial court's resolution of
a \emph{Batson} v. \emph{Kentucky,} 476 U.~S. 79 (1986), claim, noting
that we will overturn a ruling on the question of discriminatory intent
only if it is ``clearly erroneous,'' \emph{ante,} at 477. Under this
standard, we ``will not reverse a lower court's \newpage  finding of
fact simply because we would have decided the case differently.''
\emph{Easley} v. \emph{Cromartie,} 532 U.~S. 234, 242 (2001) (internal
quotation marks omitted). Instead, a reviewing court must ask
``whether, ‘on the entire evidence,' it is ‘left with the
definite and firm conviction that a mistake has been committed.' ''
\emph{Ibid.} (quoting \emph{United States} v. \emph{United States Gypsum Co.,}
333 U.~S. 364, 395 (1948)).

  The Court acknowledges two reasons why a trial court ``has a pivotal
role in evaluating \emph{Batson} claims.'' \emph{Ante,} at 477. First,
the Court notes that the trial court is uniquely situated to judge the
prosecutor's credibility because the best evidence of discriminatory
intent `` ‘often will be the demeanor of the attorney who exercises
the challenge.' '' \emph{Ibid.} (quoting \emph{Hernandez, supra,} at 365
(plurality opinion)). Second, it recognizes that the trial court's
``firsthand observations'' of the juror's demeanor are of ``grea[t]
importance'' in determining whether the prosecutor's neutral basis
for the strike is credible. \emph{Ante,} at 477.

  The Court's conclusion, however, reveals that it is only paying
lipservice to the pivotal role of the trial court. The Court
second-guesses the trial court's determinations in this case merely
because the judge did not clarify which of the prosecutor's neutral
bases for striking Mr. Brooks was dispositive. But we have never
suggested that a reviewing court should defer to a trial court's
resolution of a \emph{Batson} challenge only if the trial court made
specific findings with respect to each of the prosecutor's proffered
race-neutral reasons. To the contrary, when the grounds for a trial
court's decision are ambiguous, an appellate court should not
presume that the lower court based its decision on an improper ground,
particularly when applying a deferential standard of review. See
\emph{Sprint/United Management Co.} v. \emph{Mendelsohn, ante,} at 386.

  The prosecution offered two neutral bases for striking Mr. Brooks:
his nervous demeanor and his stated concern about missing class. App.
444. The trial court, in rejecting \newpage  defendant's \emph{Batson}
challenge, stated only ``All right. I'm going [to] allow the
challenge. I'm going to allow the challenge.'' App. 445. The Court
concedes that ``the record does not show'' whether the trial court
made its determination based on Mr. Brooks' demeanor or his concern
for missing class, \emph{ante,} at 479, but then speculates as to what
the trial court \emph{might} have thought about Mr. Brooks' demeanor.
As a result of that speculation, the Court concludes that it ``cannot
presume that the trial judge credited the prosecutor's assertion
that Mr. Brooks was nervous.'' \emph{Ibid.} Inexplicably, however,
the Court concludes that it \emph{can} presume that the trial court
impermissibly relied on the prosecutor's supposedly pretextual concern
about Mr. Brooks' teaching schedule, even though nothing in the record
supports that interpretation over the one the Court rejects.

  Indeed, if the record suggests anything, it is that the judge was more
influenced by Mr. Brooks' nervousness than by his concern for missing
class. Following an exchange about whether his desire to get back to
class would make Mr. Brooks more likely to support a verdict on a lesser
included offense because it might avoid a penalty phase, defense counsel
offered its primary rebuttal to the prosecutor's proffered neutral
reasons. Immediately after argument on the nervousness point, the judge
ruled on the \emph{Batson} challenge, even interrupting the prosecutor to
do so:

      ``MR. VASQUEZ:\dots His main problem yesterday was the fact
    that he didn't know if he would miss some teaching time as a
    student teacher. The clerk called the school and whoever it was and
    the Dean said that wouldn't be a problem. He was told that this
    would go through the weekend, and he expressed that that was his
    only concern, that he didn't have any other problems.

      ``As far as him looking nervous, hell, everybody out here looks
    nervous. I'm nervous.

      ``MR. OLINDE: Judge, it's---\newpage 

      ``MR. VASQUEZ: Judge, that's---You know.

      ``MR. OLINDE: ---a question of this: It's a peremptory
    challenge. We need 12 out of 12 people. Mr. Brooks was very
    uncertain and very nervous looking and---

      ``THE COURT: All right. I'm going [to] allow the challenge.
    I'm going to allow the challenge.''

    App. 445.

\noindent Although this exchange is certainly not hard-and-fast evidence of the
trial court's reasoning, it undermines the Court's presumption that
the trial judge relied solely on Mr. Brooks' concern for missing
school.

  The Court also concludes that the trial court's determination lacked
support in the record because the prosecutor failed to strike two other
jurors with similar concerns. \emph{Ante,} at 483--484. Those jurors,
however, were never mentioned in the argument before the trial court,
nor were they discussed in the filings or opinions on any of the three
occasions this case was considered by the Louisiana Supreme Court.[[*]]
Petitioner failed to suggest a comparison with those two jurors in his
petition for certiorari, and apparently only discovered this ``clear
error'' in the record when drafting his brief before this Court. We
have no business overturning a conviction, years after the fact and
after extensive intervening litigation, based on arguments not presented
to the courts below. Cf. \emph{Miller-El} v. \emph{Dretke,} 545 U.~S. 231,
283 (2005) (\textsc{Thomas,} J., dissenting).

  Because I believe that the trial court did not clearly err in
rejecting petitioner's \emph{Batson} challenge with respect to Mr.
Brooks, I also must address the strike of Ms. Scott. The prosecution's
neutral explanation for striking Ms. Scott was \newpage  that she was
unsure about her ability to impose the death penalty. Like the claims
made about Mr. Brooks, there is very little in the record either to
support or to undermine the prosecution's asserted rationale for
striking Ms. Scott. But the trial court had the benefit of observing
the exchange between the prosecutor and Ms. Scott, and accordingly was
in the best position to judge whether the prosecutor's assessment of
her response was credible. When asked if she could consider the death
penalty, her first response was inaudible. App. 360. The trial
court, with the benefit of contextual clues not apparent on a cold
transcript, was better positioned to evaluate whether Ms. Scott was
merely soft-spoken or seemed hesitant in her responses. Similarly, a
firsthand observation of demeanor is the only thing that could give
sufficient content to Ms. Scott's ultimate response---``I think I
could,'' \emph{id.,} at 361---to determine whether the prosecution's
concern about her willingness to impose the death penalty was well
founded. Given the trial court's expertise in making credibility
determinations and its firsthand knowledge of the \emph{voir dire}
exchanges, it is entirely proper to defer to its judgment. Accordingly,
I would affirm the judgment below.

^* While the Court correctly observes that the Louisiana Supreme Court  
made a comparison between Mr. Brooks and unstricken white jurors,       
that is true only as to jurors Vicki Chauffe, Michael Sandras, and      
Arthur Yeager. 1998--1078, pp. 15--18 (La. 9/6/06), 942 So. 2d 484,   
495--496. The Court, on the other hand, focuses on Roland Laws and     
John Donnes, who were never discussed below in this context.            
