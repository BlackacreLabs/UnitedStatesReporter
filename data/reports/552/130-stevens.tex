Type: Dissenting
Author: Stevens

\setcounter{page}{140}

  \textsc{Justice Stevens,} with whom \textsc{Justice Ginsburg} joins, dissenting.

  Statutes of limitations generally fall into two broad categories: affirmative defenses that can be waived and so-called ``jurisdictional'' statutes that are not subject to waiver or equitable tolling. For much of our history, statutes of limitations in suits against the Government were customarily placed in the latter category on the theory that conditions attached to a waiver of sovereign immunity ``must be strictly observed and exceptions thereto are not to be implied.'' \emph{Soriano} v. \emph{United States,} 352 U.~S. 270, 276 (1957); see also \emph{Finn} v. \emph{United States,} 123 U.~S. 227, 232--233 (1887); \emph{Kendall} v. \emph{United States,} 107 U.~S. 123, 125--126 (1883). But that rule was ignored---and thus presumably abandoned---in \emph{Honda} v. \emph{Clark,} 386 U.~S. 484 (1967),\footnotemark[1] and \emph{Bowen} v. \emph{City of New York,} 476 U.~S. 467 (1986).\footnotemark[2]

  In \emph{Irwin} v. \emph{Department of Veterans Affairs,} 498 U. S. 89, 95--96 (1990), we followed the lead of \emph{Bowen} (and, by extension, \emph{Honda}), and explicitly replaced the \emph{Soriano} rule with a rebuttable presumption that equitable tolling rules ``applicable to suits against private defendants should also apply to suits against the United States.''\footnotemark[3] We acknowledged that

\footnotetext[1]{In \emph{Honda,} we concluded, as to petitioners' attempts to recover assets that had been seized upon the outbreak of hostilities with Japan, that it was ``consistent with the overall congressional purpose to apply a traditional equitable tolling principle, aptly suited to the particular facts of this case and nowhere eschewed by Congress.''386 U. S., at 501.}

\footnotetext[2]{In \emph{Bowen,} we permitted equitable tolling of the 60-day requirement for challenging the denial of disability benefits under the Social Security Act. We cautioned that ``we must be careful not to assume the authority to narrow the waiver that Congress intended, or construe the waiver unduly restrictively.'' 476 U. S., at 479 (citation and internal quotation marks omitted). \newpage  ``our previous cases dealing with the effect of time limits in suits against the Government [had] not been entirely consistent,'' 498 U. S., at 94, and we determined that ``a continuing effort on our part to decide each case on an ad hoc basis\dots would have the disadvantage of continuing unpredictability without the corresponding advantage of greater fidelity to the intent of Congress,'' \emph{id.,} at 95. We therefore crafted a background rule that reflected ``a realistic assessment of legislative intent,'' and also provided ``a practically useful principle of interpretation.''\emph{Ibid.}}

\footnotetext[3]{During the \emph{Irwin} oral arguments, several Members of the Court remarked on the need to choose between the \emph{Soriano} line of cases and the approach taken in cases like \emph{Bowen.} See Tr. of Oral Arg., O. T. 1990, No. 89--5867, pp. 25--26 (``Question: ‘[W]hat do you make of our cases which seem to go really in different directions. The \emph{Bowen} case, which was\newpage  unanimous and contains language in it that says statutory time limits are traditionally subject to equitable tolling, and other cases like maybe \emph{Soriano .~.~.} which point in the other direction[?]'''); see also \emph{id.,} at 8 (``Question: ‘\dots I thinkwesort of have to choose between \emph{Soriano} and \emph{Bowen,} don't you think?' '').}

  Our decision in \emph{Irwin} did more than merely ``mentio[n]'' \emph{Soriano, ante,} at 137; rather, we expressly declined to follow that case. We noted that the limitations language at issue in \emph{Irwin} closely resembled the text we had confronted in \emph{Soriano;} although we conceded that ``[a]n argument [could] undoubtedly be made'' that the statutes were distinguishable, we were ``not persuaded that the difference between them [was] enough to manifest a different congressional intent with respect to the availability of equitable tolling,'' 498 U. S., at 95. Having found the two statutes functionally indistinguishable, we nevertheless declined the Government's invitation to follow \emph{Soriano,} and we did not so much as cite \emph{Kendall} or \emph{Finn.} Instead, we adopted ``a more general rule to govern the applicability of equitable tolling in suits against the Government,'' 498 U. S., at 95, and we applied the new presumption in favor of equitable tolling to the case before us.\footnotemark[4] Nothing in the framing of our decision to adopt \newpage  a ``general rule'' to govern the availability of equitable tolling in suits against the Government, \emph{ibid.,} suggested a carveout for statutes we had already held ineligible for equitable tolling, pursuant to the approach that we had previously abandoned in \emph{Honda} and \emph{Bowen,} and definitively rejected in \emph{Irwin.}

\footnotetext[4]{In the years since we decided \emph{Irwin,} we have applied its rule in a number of statutory contexts. See, \emph{e. g., Scarborough} v. \emph{Principi,} 541 U.~S. 401, 420--423 (2004) (applying the rule of \emph{Irwin} and finding that an application for fees under the Equal Access to Justice Act, 28 U.~S.~C. \S~2412(d)(1)(A), should be permitted to be amended out of time). Most \newpage  significantly, in \emph{Franconia Associates} v. \emph{United States,} 536 U.~S. 129, 145 (2002), we affirmed, in the context of 28 U.~S.~C. \S~2501, the rule that ``limitations principles should generally apply to the Government ‘in the same way that' they apply to private parties'' (citing \emph{Irwin,} 498 U. S., at 95). Although the Government is correct that the question presented by \emph{Franconia} was when a claim accrued under \S~2501, our reliance on \emph{Irwin} undermines the majority's suggestion that \emph{Irwin} has no bearing on statutes that have previously been the subject of judicial construction.}

  Indeed, in his separate opinion in \emph{Irwin,} Justice White noted that the decision was not only inconsistent with our prior cases but also that it ``directly overrule[d]'' \emph{Soriano.} 498 U. S., at 98 (opinion concurring in part and concurring in judgment). Neither the Court's opinion nor my separate opinion disagreed with that characterization of our holding. The attempt of the Court today, therefore, to cast petitioner's argument as an entreaty to overrule \emph{Soriano,} as well as \emph{Kendall} and \emph{Finn}---and its response that ``[b]asic principles of \emph{stare decisis}\dots require us to reject this argument,'' \emph{ante,} at 139---has a hollow ring. If the doctrine of \emph{stare decisis} supplied a clear answer to the question posed by this case---or if the Government could plausibly argue that it had relied on \emph{Soriano} after our decision in \emph{Irwin}---I would join the Court's judgment, despite its unwisdom.\footnotemark[5] But I do not \newpage  agree with the majority's reading of our cases. It seems to me quite plain that \emph{Soriano} is no longer good law, and if there is in fact ambiguity in our cases, it ought to be resolved in favor of clarifying the law, rather than preserving an anachronism whose doctrinal underpinnings were discarded years ago.\footnotemark[6]

\footnotetext[5]{The majority points out quite rightly, \emph{ante,} at 139, that the doctrine of \emph{stare decisis} has `` ‘special force' '' in statutory cases. See \emph{Patterson} v. \emph{McLean Credit Union,} 491 U. S. 164, 172--173 (1989). But the doctrine should not prevent us from acknowledging when we have already overruled a prior case, even if we failed to say so explicitly at the time. In \emph{Rasul} v. \emph{Bush,} 542 U.~S. 466 (2004), for example, we explained that in \emph{Braden} v. \emph{30th Judicial Circuit Court of Ky.,} 410 U.~S. 484 (1973), we had overruled so much of \emph{Ahrens} v. \emph{Clark,} 335 U. S. 188 (1948), as found that the habeas petitioners' presence within the territorial reach of the district \newpage  court was a jurisdictional prerequisite. \emph{Braden} held, contrary to \emph{Ahrens,} that a prisoner's presence within the district court's territorial reach was \emph{not} an ``inflexible jurisdictional rule,'' 410 U. S., at 500. \emph{Braden} nowhere stated that it was overruling \emph{Ahrens,} although Justice Rehnquist began his dissent by noting: ``Today the Court overrules \emph{Ahrens} v. \emph{Clark.}'' 410 U. S., at 502. Thirty years later we acknowledged in \emph{Rasul} what was by then clear: \emph{Ahrens} was no longer good law.542 U. S., at 478--479, and n. 9.

Moreover, the logic of the ``special force'' of \emph{stare decisis} in the statutory context is that ``Congress remains free to alter what we have done,'' \emph{Patterson,} 491 U. S., at 172--173. But the amendment of an obscure statutory provision is not a high priority for a busy Congress, and we should remain mindful that enactment of legislation is by no means a cost-free enterprise.}

  With respect to provisions as common as time limitations, Congress, in enacting statutes, and judges, in applying them, ought to be able to rely upon a background rule of considerable clarity. \emph{Irwin} announced such a rule, and I would apply that rule to the case before us.\footnotemark[7] Because today's decision threatens to revive the confusion of our pre-\emph{Irwin} jurisprudence, I respectfully dissent.

\footnotetext[6]{See Holmes, The Path of the Law, 10 Harv. L. Rev. 457, 469 (1897) (``It is revolting to have no better reason for a rule of law than that so it was laid down in the time of Henry IV. It is still more revolting if the grounds upon which it was laid down have vanished long since, and the rule simply persists from blind imitation of the past'').}

\footnotetext[7]{The majority does gesture toward an application of \emph{Irwin,} contending that even if \emph{Irwin}'s rule is apposite, the presumption of congressional intent to allow equitable tolling is rebutted by this Court's ``definitive earlier interpretation'' of \S~2501, \emph{ante,} at 138. But the majority's application of the \emph{Irwin} rule is implausible, since \emph{Irwin} itself compared the language of \S~2501 with the limitations language of Title VII of the Civil Rights Act of 1964, and found that the comparison did \emph{not} reveal ``a different congressional intent with respect to the availability of equitable tolling,'' 498 U. S., at 95.}
