Author: Per Curiam
\setcounter{page}{117}
Type: Court

  \textsc{Per Curiam.}

  Respondent Maxwell Hoffman was convicted of firstdegree murder and sentenced to death. See \emph{State} v. \emph{Hoffman,} 123 Idaho 638, 851 P. 2d 934 (1993). Hoffman sought federal habeas relief on the grounds that, \emph{inter alia,} his counsel had been ineffective during both pretrial plea bargaining and the sentencing phase of his trial. The District Court, finding that Hoffman had received ineffective assistance of counsel during sentencing but not during plea bargaining, granted Hoffman's federal habeas petition in part and ordered the State of Idaho to resentence him. Civ. Action No. 94--0200--S--BLW (Mar. 30, 2002), App. to Pet. for Cert. 38, 65. The Court of Appeals for the Ninth Circuit affirmed the District Court's decision regarding ineffective\newpage assistance of counsel during sentencing,[[*]] but reversed with respect to the ineffective-assistance claim during plea negotiations. 455 F. 3d 926, 942 (2006). The Ninth Circuit thus granted the writ, ordering the District Court to direct the State either to release Hoffman or to ``offe[r] [him] a plea agreement with the ‘same material terms' offered in the original plea agreement.'' \emph{Id.,} at 943. The State sought, and we granted, certiorari. \emph{Post,} p. 1008.

  Hoffman now abandons his claim that counsel was ineffective during plea bargaining. See Respondent's Motion to Vacate Decision Below and Dismiss the Cause as Moot. He ``no longer seeks or desires the relief ordered by the Court of Appeals with respect to the plea offer.'' \emph{Id.,} at 3. Rather, Hoffman now ``wishes to withdraw his claim of ineffective assistance of counsel in connection with plea bargaining'' and asks this Court to dismiss his appeal with prejudice on that issue so that he may proceed with the resentencing ordered by the District Court. \emph{Ibid.}

  The State, in its response, notes that Hoffman's requested relief is ``virtually identical to the request made by the state in its Petition for Certiorari.'' Response to Respondent's Motion to Vacate Decision Below and Dismiss the Cause as Moot, p. 3. The State therefore agrees that the instant motion to vacate and dismiss with prejudice moots Hoffman's claim of ineffective assistance of counsel during plea negotiations and asks that the motion be granted.

  We grant respondent's motion. Because his claim for ineffective assistance of counsel during pretrial plea bargaining is moot, we vacate the judgment of the Court of Appeals to the extent that it addressed that claim. The case is remanded to the United States Court of Appeals for the Ninth \newpage  Circuit with directions that it instruct the United States District Court for the District of Idaho to dismiss the relevant claim with prejudice. \emph{Deakins} v. \emph{Monaghan,} 484 U. S. 193, 200--201 (1988); \emph{United States} v. \emph{Munsingwear, Inc.,} 340 U. S. 36, 39--40 (1950).

\footnotetext[*]{The State initially cross-appealed the District Court's grant      of Hoffman's habeas petition for ineffective assistance of counsel    at sentencing. The State, however, subsequently withdrew that           cross-appeal, leaving in place the District Court's order granting    habeas relief as to Hoffman's death sentence. 455 F. 3d 926, 931    (CA9 2006).}

\begin{flushright}\emph{It is so ordered.}\end{flushright}
