% Opinion of the Court
% Stevens

\setcounter{page}{40}

  \textsc{Justice Stevens} delivered the opinion of the Court.

  In two cases argued on the same day last Term we considered the
standard that courts of appeals should apply when reviewing the
reasonableness of sentences imposed by district judges. The first,
\emph{Rita} v. \emph{United States,} 551 U.~S. 338 (2007), involved a
sentence \emph{within} the range recommended by the Federal Sentencing
Guidelines; we held that when a district judge's discretionary
decision in a particular case accords with the sentence the United
States Sentencing Commission deems appropriate ``in the mine run
of cases,'' the court of appeals may presume that the sentence is
reasonable. \emph{Id.,} at 351.

  The second case, \emph{Claiborne} v. \emph{United States,} involved a
sentence \emph{below} the range recommended by the Guidelines, and
raised the converse question whether a court of appeals may apply a
``proportionality test,'' and require that a sen\newpage tence that
constitutes a substantial variance from the Guidelines be justified by
extraordinary circumstances. See \emph{Claiborne} v. \emph{United States,}
549 U.~S. 1016 (2006). We did not have the opportunity to answer
this question because the case was mooted by Claiborne's untimely
death. \emph{Claiborne} v. \emph{United States,} 551 U.~S. 87 (2007) \emph{(per
curiam).} We granted certiorari in the case before us today in order
to reach that question, left unanswered last Term. 551 U.~S. 1113
(2007). We now hold that, while the extent of the difference between
a particular sentence and the recommended Guidelines range is surely
relevant, courts of appeals must review all sentences---whether inside,
just outside, or significantly outside the Guidelines range---under a
deferential abuse-of-discretion standard. We also hold that the sentence
imposed by the experienced District Judge in this case was reasonable.

\section{I}

  In February or March 2000, petitioner Brian Gall, a second-year
college student at the University of Iowa, was invited by Luke
Rinderknecht to join an ongoing enterprise distributing a controlled
substance popularly known as ``ecstasy.''\footnotemark[1] Gall---who was then
a user of ecstasy, cocaine, and marijuana---accepted the invitation.
During the ensuing seven months, Gall delivered ecstasy pills, which he
received from Rinderknecht, to other conspirators, who then sold them to
consumers. He netted over \$30,000.

  A month or two after joining the conspiracy, Gall stopped using
ecstasy. A few months after that, in September 2000, he advised
Rinderknecht and other co-conspirators that he was withdrawing from the
conspiracy. He has not sold illegal drugs of any kind since. He has, in
the words of the District Court, ``self-rehabilitated.'' App. 75. He
graduated from the University of Iowa in 2002, and moved first to \newpage 
Arizona, where he obtained a job in the construction industry, and later
to Colorado, where he earned \$18 per hour as a master carpenter. He has
not used any illegal drugs since graduating from college.


^1 Ecstasy is sometimes called ``MDMA'' because its scientific name is
``methylenedioxymethamphetamine.''App. 24, 118.

  After Gall moved to Arizona, he was approached by federal law
enforcement agents who questioned him about his involvement in the
ecstasy distribution conspiracy. Gall admitted his limited participation
in the distribution of ecstasy, and the agents took no further action
at that time. On April 28, 2004---approximately 1½ years after
this initial interview, and 3½ years after Gall withdrew from the
conspiracy---an indictment was returned in the Southern District of
Iowa charging him and seven other defendants with participating in a
conspiracy to distribute ecstasy, cocaine, and marijuana, that began in
or about May 1996 and continued through October 30, 2002. The Government
has never questioned the truthfulness of any of Gall's earlier
statements or contended that he played any role in, or had any knowledge
of, other aspects of the conspiracy described in the indictment. When
he received notice of the indictment, Gall moved back to Iowa and
surrendered to the authorities. While free on his own recognizance, Gall
started his own business in the construction industry, primarily engaged
in subcontracting for the installation of windows and doors. In his
first year, his profits were over \$2,000 per month.

  Gall entered into a plea agreement with the Government, stipulating
that he was ``responsible for, but did not necessarily distribute
himself, at least 2,500 grams of [ecstasy], or the equivalent of at
least 87.5 kilograms of marijuana.'' \emph{Id.,} at 25. In the
agreement, the Government acknowledged that ``on or about September of
2000,'' Gall had communicated his intent to stop distributing ecstasy
to Rinderknecht and other members of the conspiracy. \emph{Ibid.} The
agreement further provided that recent changes in the Guidelines that
enhanced the recommended punishment for distributing ecstasy were not
applicable to Gall because he had with\newpage drawn from the conspiracy
prior to the effective date of those changes.

  In her presentence report, the probation officer concluded that Gall
had no significant criminal history; that he was not an organizer,
leader, or manager; and that his offense did not involve the use of
any weapons. The report stated that Gall had truthfully provided the
Government with all of the evidence he had concerning the alleged
offenses, but that his evidence was not useful because he provided
no new information to the agents. The report also described Gall's
substantial use of drugs prior to his offense and the absence of any
such use in recent years. The report recommended a sentencing range of
30 to 37 months of imprisonment.

  The record of the sentencing hearing held on May 27, 2005, includes a
``small flood'' of letters from Gall's parents and other relatives,
his fiance, neighbors, and representatives of firms doing business with
him, uniformly praising his character and work ethic. The transcript
includes the testimony of several witnesses and the District Judge's
colloquy with the assistant United States attorney (AUSA) and with
Gall. The AUSA did not contest any of the evidence concerning Gall's
law-abiding life during the preceding five years, but urged that ``the
guidelines are appropriate and should be followed,'' and requested
that the court impose a prison sentence within the Guidelines range.
\emph{Id.,} at 93. He mentioned that two of Gall's co-conspirators
had been sentenced to 30 and 35 months, respectively, but upon further
questioning by the District Court, he acknowledged that neither of them
had voluntarily withdrawn from the conspiracy.

  The District Judge sentenced Gall to probation for a term of 36
months. In addition to making a lengthy statement on the record, the
judge filed a detailed sentencing memorandum explaining his decision,
and provided the following statement of reasons in his written judgment:

      ``The Court determined that, considering all the fac tors under
18 U.~S.~C. 3553(a), the Defendant's explicit \newpage  withdrawal from
the conspiracy almost four years before the filing of the Indictment,
the Defendant's post-offense conduct, especially obtaining a college
degree and the start of his own successful business, the support of
fam ily and friends, lack of criminal history, and his age at the time
of the offense conduct, all warrant the sentence imposed, which was
sufficient, but not greater than nec essary to serve the purposes of
sentencing.'' \emph{Id.,} at 117.

  At the end of both the sentencing hearing and the sentencing
memorandum, the District Judge reminded Gall that probation, rather than
``an act of leniency,'' is a ``substantial restriction of freedom.''
\emph{Id.,} at 99, 125. In the memorandum, he emphasized:

    ``[Gall] will have to comply with strict reporting condi tions
    along with a three-year regime of alcohol and drug testing. He
    will not be able to change or make deci sions about significant
    circumstances in his life, such as where to live or work, which are
    prized liberty interests, without first seeking authorization from
    his Probation Officer or, perhaps, even the Court. Of course, the
    De fendant always faces the harsh consequences that await if he
    violates the conditions of his probationary term.'' \emph{Id.,} at
    125.

  Finally, the District Judge explained why he had concluded that the
sentence of probation reflected the seriousness of Gall's offense and
that no term of imprisonment was necessary:

      ``Any term of imprisonment in this case would be counter
effective by depriving society of the contribu tions of the Defendant
who, the Court has found, under stands the consequences of his criminal
conduct and is doing everything in his power to forge a new life. The
Defendant's post-offense conduct indicates neither that he will return
to criminal behavior nor that the Defend ant is a danger to society. In
fact, the Defendant's \newpage  post-offense conduct was not motivated
by a desire to please the Court or any other governmental agency, but
was the pre-Indictment product of the Defendant's own desire to lead a
better life.'' \emph{Id.,} at 125--126.

\section{II}

  The Court of Appeals reversed and remanded for resentencing. Relying
on its earlier opinion in \emph{United States} v. \emph{Claiborne,}
439 F. 3d 479 (CA8 2006), it held that a sentence outside of the
Guidelines range must be supported by a justification that `` ‘
``is proportional to the extent of the difference between the advisory
range and the sentence imposed.'' ' '' 446 F. 3d 884, 889 (CA8
2006) (quoting \emph{Claiborne,} 439 F. 3d, at 481, in turn quoting
\emph{United States} v. \emph{Johnson,} 427 F. 3d 423, 426--427 (CA7
2005)). Characterizing the difference between a sentence of probation
and the bottom of Gall's advisory Guidelines range of 30 months
as ``extraordinary'' because it amounted to ``a 100% downward
variance,'' 446 F. 3d, at 889, the Court of Appeals held that such
a variance must be---and here was not---supported by extraordinary
circumstances.

  Rather than making an attempt to quantify the value of the
justifications provided by the District Judge, the Court of Appeals
identified what it regarded as five separate errors in the District
Judge's reasoning: (1) He gave ``too much weight to Gall's
withdrawal from the conspiracy''; (2) given that Gall was 21 at the
time of his offense, the District Judge erroneously gave ``significant
weight'' to studies showing impetuous behavior by persons under the age
of 18; (3) he did not ``properly weigh'' the seriousness of Gall's
offense; (4) he failed to consider whether a sentence of probation
would result in ``unwarranted'' disparities; and (5) he placed ``too
much emphasis on Gall's post-offense rehabilitation.'' \emph{Id.,}
at 889--890. As we shall explain, we are not persuaded that these
factors, whether viewed separately or in the aggregate, are sufficient
to support the conclusion that the District \newpage  Judge abused his
discretion. As a preface to our discussion of these particulars,
however, we shall explain why the Court of Appeals' rule requiring
``proportional'' justifications for departures from the Guidelines
range is not consistent with our remedial opinion in \emph{United States}
v. \emph{Booker,} 543 U.~S. 220 (2005).

\section{III}

  In \emph{Booker} we invalidated both the statutory provision, 18
U.~S.~C. \S~3553(b)(1) (2000 ed., Supp. IV), which made the
Sentencing Guidelines mandatory, and \S~3742(e) (2000 ed. and Supp.
IV), which directed appellate courts to apply a \emph{de novo} standard of
review to departures from the Guidelines. As a result of our decision,
the Guidelines are now advisory, and appellate review of sentencing
decisions is limited to determining whether they are ``reasonable.''
Our explanation of ``reasonableness'' review in the \emph{Booker} opinion
made it pellucidly clear that the familiar abuse-of-discretion standard
of review now applies to appellate review of sentencing decisions. See
543 U. S., at 260--262; see also \emph{Rita,} 551 U. S., at 361--362
(\textsc{Stevens,} J., concurring).

  It is also clear that a district judge must give serious consideration
to the extent of any departure from the Guidelines and must explain his
conclusion that an unusually lenient or an unusually harsh sentence is
appropriate in a particular case with sufficient justifications. For
even though the Guidelines are advisory rather than mandatory, they are,
as we pointed out in \emph{Rita,} the product of careful study based on
extensive empirical evidence derived from the review of thousands of
individual sentencing decisions.\footnotemark[2] \emph{Id.,} at 349.\newpage 


^2 Notably, not all of the Guidelines are tied to this empirical
evidence. For example, the Sentencing Commission departed from the
empirical approach when setting the Guidelines range for drug offenses,
and chose instead to key the Guidelines to the statutory mandatory
minimum sentences that Congress established for such crimes. See
United States Sentencing Commission, Guidelines Manual \S~1A1.1 (Nov.
2006) (USSG). This decision, and its effect on a district judge's
authority to deviate from the \newpage  Guidelines range in a particular
drug case, is addressed in \emph{Kimbrough} v. \emph{United States, post,}
p. 85.

  In reviewing the reasonableness of a sentence outside the Guidelines
range, appellate courts may therefore take the degree of variance into
account and consider the extent of a deviation from the Guidelines. We
reject, however, an appellate rule that requires ``extraordinary''
circumstances to justify a sentence outside the Guidelines range. We
also reject the use of a rigid mathematical formula that uses the
percentage of a departure as the standard for determining the strength
of the justifications required for a specific sentence.

  As an initial matter, the approaches we reject come too close
to creating an impermissible presumption of unreasonableness for
sentences outside the Guidelines range. See \emph{id.,} at 354--355
(``The fact that we permit courts of appeals to adopt a presumption
of reasonableness does not mean that courts may adopt a presumption of
unreasonableness'').\footnotemark[3] Even the Government has acknowledged that
such a presumption would not be consistent with \emph{Booker.} See Brief
for United States in \emph{Rita} v. \emph{United States,} O. T. 2006, No.
06--5754, pp. 34--35.

  The mathematical approach also suffers from infirmities of
application. On one side of the equation, deviations from \newpage  the
Guidelines range will always appear more extreme---in percentage
terms---when the range itself is low, and a sentence of probation
will always be a 100% departure regardless of whether the Guidelines
range is 1 month or 100 years. Moreover, quantifying the variance as a
certain percentage of the maximum, minimum, or median prison sentence
recommended by the Guidelines gives no weight to the ``substantial
restriction of freedom'' involved in a term of supervised release or
probation. App. 95.


^3 Several Courts of Appeals had rejected such a presumption of
unreasonableness even prior to our decision in \emph{Rita.} See,
\emph{e. g., United States} v. \emph{Howard,} 454 F. 3d 700, 703 (CA7
2006) (``Although a sentence outside the range does not enjoy the
presumption of reasonableness that one within the range does, it does
not warrant a presumption of unreasonableness''); \emph{United States} v.
\emph{Matheny,} 450 F. 3d 633, 642 (CA6 2006) (``[T]his court's holding
that sentences within the advisory guideline range are presumptively
reasonable does not mean that sentences outside of that range are
presumptively unreasonable''); \emph{United States} v. \emph{Myers,} 439 F.
3d 415, 417 (CA8 2006) (``We have determined that a sentence imposed
within the guidelines range is presumptively reasonable. While it does
not follow that a sentence outside the guidelines range is unreasonable,
we review a district court's decision to depart from the appropriate
guidelines range for abuse of discretion'' (citation omitted)).

  We recognize that custodial sentences are qualitatively more
severe than probationary sentences of equivalent terms. Offenders
on probation are nonetheless subject to several standard conditions
that substantially restrict their liberty. See \emph{United States}
v. \emph{Knights,} 534 U.~S. 112, 119 (2001) (``Inherent in the very
nature of probation is that probationers ‘do not enjoy the absolute
liberty to which every citizen is entitled' '' (quoting \emph{Griffin}
v. \emph{Wisconsin,} 483 U.~S. 868, 874 (1987); internal quotation marks
omitted)).\footnotemark[4]

        Probationers may not leave the judicial district, move, or
change jobs without notifying, and in some cases receiving permission
from, their probation officer or the court. They must report regularly
to their probation officer, permit unannounced visits to their homes,
refrain from associating with any person convicted of a felony, and
refrain from excessive drinking. USSG \S~5B1.3. Most probationers
are also subject to individual ``special conditions'' imposed by the
court. Gall, for instance, may not patronize any establishment that
\newpage  derives more than 50% of its revenue from the sale of alcohol,
and must submit to random drug tests as directed by his probation
officer. App. 109.


^4 See also Advisory Council of Judges of National Council on Crime
and Delinquency, Guides for Sentencing 13--14 (1957) (``Probation is
not granted out of a spirit of leniency~.~.~.~. As the Wickersham
Commission said, probation is not merely ‘letting an offender off
easily'''); 1 N. Cohen, The Law of Probation and Parole \S~7:9
(2d ed. 1999) (``[T]he probation or parole conditions imposed on
an individual can have a significant impact on both that person
and society~.~.~.~. Often these conditions comprehensively
regulate significant facets of their day-to-day lives~.~.~.~.
They may become subject to frequent searches by government officials,
as well as to mandatory counseling sessions with a caseworker or
psychotherapist'').


  On the other side of the equation, the mathematical approach assumes
the existence of some ascertainable method of assigning percentages to
various justifications. Does withdrawal from a conspiracy justify more
or less than, say, a 30% reduction? Does it matter that the withdrawal
occurred several years ago? Is it relevant that the withdrawal was
motivated by a decision to discontinue the use of drugs and to lead a
better life? What percentage, if any, should be assigned to evidence
that a defendant poses no future threat to society, or to evidence that
innocent third parties are dependent on him? The formula is a classic
example of attempting to measure an inventory of apples by counting
oranges.\footnotemark[5]

  Most importantly, both the exceptional circumstances requirement
and the rigid mathematical formulation reflect a practice---common
among courts that have adopted ``proportional review''---of applying
a heightened standard of review to sentences outside the Guidelines
range. This is inconsistent with the rule that the abuse-of-discretion
standard of review applies to appellate review of all sentencing
decisions---whether inside or outside the Guidelines range.

  As we explained in \emph{Rita,} a district court should begin all
sentencing proceedings by correctly calculating the applicable
Guidelines range. See 551 U. S., at 347--348. As a matter of
administration and to secure nationwide consistency, the Guidelines
should be the starting point and the initial benchmark. The Guidelines
are not the only consideration, however. Accordingly, after giving
both parties an opportunity to argue for whatever sentence they deem
appropriate, the district judge should then consider all of the
\S~3553(a) \newpage  factors to determine whether they support the
sentence requested by a party.\footnotemark[6]


^5 Notably, when the Court of Appeals explained its disagreement with
the District Judge's decision in this case, it made no attempt to
quantify the strength of any of the mitigating circumstances.

      In so doing, he may not presume that the Guidelines range is
reasonable. See \emph{id.,} at 351. He must make an individualized
assessment based on the facts presented. If he decides that an
outside-Guidelines sentence is warranted, he must consider the extent
of the deviation and ensure that the justification is sufficiently
compelling to support the degree of the variance. We find it
uncontroversial that a major departure should be supported by a more
significant justification than a minor one. After settling on the
appropriate sentence, he must adequately explain the chosen sentence to
allow for meaningful appellate review and to promote the perception of
fair sentencing. \emph{Id.,} at 356--358.


^6 Section 3553(a) lists seven factors that a sentencing court must
consider. The first factor is a broad command to consider ``the nature
and circumstances of the offense and the history and characteristics of
the defendant.'' 18 U.~S.~C. \S~3553(a)(1). The second factor
requires the consideration of the general purposes of sentencing,
including:

  ``the need for the sentence imposed---

  ``(A) to reflect the seriousness of the offense, to promote respect
for the law, and to provide just punishment for the offense;

  ``(B) to afford adequate deterrence to criminal conduct;

  ``(C) to protect the public from further crimes of the defendant; and

  ``(D) to provide the defendant with needed educational or vocational
training, medical care, or other correctional treatment in the most
effective manner.'' \S~3553(a)(2).

  The third factor pertains to ``the kinds of sentences available,''
\S~3553(a)(3); the fourth to the Sentencing Guidelines; the
fifth to any relevant policy statement issued by the Sentencing
Commission; the sixth to ``the need to avoid unwarranted sentence
disparities,'' \S~3553(a)(6); and the seventh to ``the need to
provide restitution to any victim,'' \S~3553(a)(7). Preceding this
list is a general directive to ``impose a sentence sufficient, but not
greater than necessary, to comply with the purposes'' of sentencing
described in the second factor. \S~3553(a) (2000 ed., Supp. V).
The fact that \S~3553(a) explicitly directs sentencing courts to
consider the Guidelines supports the premise that district courts must
begin their analysis with the Guidelines and remain cognizant of them
throughout the sentencing process.\newpage 

  Regardless of whether the sentence imposed is inside or outside the
Guidelines range, the appellate court must review the sentence under an
abuse-of-discretion standard. It must first ensure that the district
court committed no significant procedural error, such as failing to
calculate (or improperly calculating) the Guidelines range, treating
the Guidelines as mandatory, failing to consider the \S~3553(a)
factors, selecting a sentence based on clearly erroneous facts, or
failing to adequately explain the chosen sentence---including an
explanation for any deviation from the Guidelines range. Assuming that
the district court's sentencing decision is procedurally sound, the
appellate court should then consider the substantive reasonableness
of the sentence imposed under an abuse-of-discretion standard. When
conducting this review, the court will, of course, take into account
the totality of the circumstances, including the extent of any variance
from the Guidelines range. If the sentence is within the Guidelines
range, the appellate court may, but is not required to, apply a
presumption of reasonableness. \emph{Id.,} at 347. But if the sentence
is outside the Guidelines range, the court may not apply a presumption
of unreasonableness. It may consider the extent of the deviation, but
must give due deference to the district court's decision that the
\S~3553(a) factors, on a whole, justify the extent of the variance. The
fact that the appellate court might reasonably have concluded that a
different sentence was appropriate is insufficient to justify reversal
of the district court.

  Practical considerations also underlie this legal principle. ``The
sentencing judge is in a superior position to find facts and judge their
import under \S~3553(a) in the individual case. The judge sees and
hears the evidence, makes credibility determinations, has full knowledge
of the facts and gains insights not conveyed by the record.''Brief
for Federal Public and Community Defenders et al. as \emph{Amici Curiae}
16. ``The sentencing judge has access to, and greater familiarity
with, the individual case and the individual defendant before \newpage 
him than the Commission or the appeals court.''\\Rita,} 551 U. S.,
at 357--358. Moreover, ``[d]istrict courts have an institutional
advantage over appellate courts in making these sorts of determinations,
especially as they see so many more Guidelines cases than appellate
courts do.'' \emph{Koon} v. \emph{United States,} 518 U.~S. 81, 98
(1996).\footnotemark[7]

  ``It has been uniform and constant in the federal judicial tradition
for the sentencing judge to consider every convicted person as an
individual and every case as a unique study in the human failings that
sometimes mitigate, sometimes magnify, the crime and the punishment to
ensue.'' \emph{Id.,} at 113.\footnotemark[8]

      The uniqueness of the individual case, however, does not change
the deferential abuse-of-discretion standard of review that applies to
all sentencing decisions. As we shall now explain, the opinion of the
Court of Appeals in this case does not reflect the requisite deference
and does not \newpage  support the conclusion that the District Court
abused its discretion.


^7 District judges sentence, on average, 117 defendants every year.
Administrative Office of United States Courts, 2006 Federal Court
Management Statistics 167. The District Judge in this case, Judge
Pratt, has sentenced over 990 offenders over the course of his career.
\emph{United States} v. \emph{Likens,} 464 F. 3d 823, 827, n. 1 (CA8 2006)
(Bright, J., dissenting). Only a relatively small fraction of these
defendants appeal their sentence on reasonableness grounds. See
\emph{Koon,} 518 U. S., at 98 (``In 1994, for example, 93.9% of Guidelines
cases were not appealed''); \emph{Likens,} 464 F. 3d, at 827, n. 1
(Bright, J., dissenting) (noting that the District Judge had sentenced
hundreds of defendants and that ``[w]e have reviewed only a miniscule
number of those cases''); cf. United States Sentencing Commission, 2006
Sourcebook of Federal Sentencing Statistics 135--152.

^8 It is particularly revealing that when we adopted an
abuse-ofdiscretion standard in \emph{Koon,} we explicitly rejected the
Government's argument that ``\\de novo} review of departure
decisions is necessary ‘to protect against unwarranted disparities
arising from the differing sentencing approaches of individual district
judges.''' 518 U. S., at 97 (quoting Brief for United States, O.
T. 1995, No. 94--1664, p. 12). Even then we were satisfied that a
more deferential abuse-of-discretion standard could successfully balance
the need to ``reduce unjustified disparities'' across the Nation and
``consider every convicted person as an individual.'' 518 U. S., at
113.

\section{IV}

  As an initial matter, we note that the District Judge committed no
significant procedural error. He correctly calculated the applicable
Guidelines range, allowed both parties to present arguments as to
what they believed the appropriate sentence should be, considered all
of the \S~3553(a) factors, and thoroughly documented his reasoning.
The Court of Appeals found that the District Judge erred in failing
to give proper weight to the seriousness of the offense, as required
by \S~3553(a)(2)(A), and failing to consider whether a sentence
of probation would create unwarranted disparities, as required by
\S~3553(a)(6). We disagree.

  Section 3553(a)(2)(A) requires judges to consider ``the need for the
sentence imposed\dots to reflect the seriousness of the offense, to
promote respect for the law, and to provide just punishment for the
offense.'' The Court of Appeals concluded that ``the district court
did not properly weigh the seriousness of Gall's offense'' because
it ``ignored the serious health risks ecstasy poses.'' 446 F. 3d,
at 890. Contrary to the Court of Appeals' conclusion, the District
Judge plainly did consider the seriousness of the offense. See, \emph{e.
g.,} App. 99 (``The Court, however, is bound to impose a sentence that
reflects the seriousness of joining a conspiracy to distribute MDMA
or Ecstasy''); \emph{id.,} at 122.\footnotemark[9] It is true that the District
\newpage  Judge did not make specific reference to the (unquestionably
significant) health risks posed by ecstasy, but the prosecutor did not
raise ecstasy's effects at the sentencing hearing. Had the prosecutor
raised the issue, specific discussion of the point might have been in
order, but it was not incumbent on the District Judge to raise every
conceivably relevant issue on his own initiative.

^9 The District Judge also gave specific consideration to the fact---not
directly taken into account by the Guidelines---that Gall netted \$30,000
from his participation in the conspiracy. He noted, however: ``[T]his
fact can be viewed from different perspectives. On the one hand, [Gall]
should be punished for profiting from a criminal scheme~.~.~.~. On
the other hand, [Gall], who is from a working-class family and has few
financial resources, decided to turn his back on what, for him, was a
highly profitable venture~.~.~.~. The Court can not consider, for
the purposes of sentencing, one side of the financial aspect of the
offense conduct without considering the other.'' App. 123--124, n.
3.

  The Government's legitimate concern that a lenient sentence for
a serious offense threatens to promote disrespect for the law is
at least to some extent offset by the fact that seven of the eight
defendants in this case have been sentenced to significant prison terms.
Moreover, the unique facts of Gall's situation provide support for
the District Judge's conclusion that, in Gall's case, ``a sentence
of imprisonment may work to promote not respect, but derision, of the
law if the law is viewed as merely a means to dispense harsh punishment
without taking into account the real conduct and circumstances involved
in sentencing.'' \emph{Id.,} at 126.

  Section 3553(a)(6) requires judges to consider ``the need to avoid
unwarranted sentence disparities among defendants with similar records
who have been found guilty of similar conduct.'' The Court of Appeals
stated that ``the record does not show that the district court
considered whether a sentence of probation would result in unwarranted
disparities.'' 446 F. 3d, at 890. As with the seriousness of the
offense conduct, avoidance of unwarranted disparities was clearly
considered by the Sentencing Commission when setting the Guidelines
ranges. Since the District Judge correctly calculated and carefully
reviewed the Guidelines range, he necessarily gave significant weight
and consideration to the need to avoid unwarranted disparities.

  Moreover, as we understand the colloquy between the District Judge and
the AUSA, it seems that the judge gave specific attention to the issue
of disparity when he inquired about the sentences already imposed by a
different judge on two of Gall's codefendants. The AUSA advised the
District \newpage  Judge that defendant Harbison had received a 30-month
sentence and that Gooding had received 35 months. The following colloquy
then occurred:

      ``THE COURT:\dots You probably know more about this than
    anybody. How long did those two stay in the conspiracy, and did they
    voluntarily withdraw?

      ``MR. GRIESS: They did not.

      ``THE COURT: They did not?

      ``MR. GRIESS: They did not voluntarily withdraw. And they were in
    the conspiracy, I think, for a shorter period of time, but at the
    very end.

      ``THE COURT: Okay. Thank you.

      ``MR. GRIESS: A significant difference there, Your Honor, is
    that they were in the conspiracy after the guidelines changed and,
    therefore, were sentenced at a much higher level because of that.''

    App. 88.

  A little later Mr. Griess stated: ``The last thing I want to talk
about goes to sentencing disparity~.~.~.~. Obviously, the Court
is cognizant of that and wants to avoid any unwarranted sentencing
disparities.'' \emph{Id.,} at 89. He then discussed at some length the
sentence of 36 months imposed on another codefendant, Jarod Yoder, whose
participation in the conspiracy was roughly comparable to Gall's.
Griess voluntarily acknowledged three differences between Yoder and
Gall: Yoder was in the conspiracy at its end and therefore was sentenced
under the more severe Guidelines, he had a more serious criminal
history, and he did not withdraw from the conspiracy.

  From these facts, it is perfectly clear that the District Judge
considered the need to avoid unwarranted disparities, but also
considered the need to avoid unwarranted \emph{similarities} among other
co-conspirators who were not similarly situated. The District Judge
regarded Gall's voluntary withdrawal as a reasonable basis for giving
him a less severe sentence than the three codefendants discussed
with the \newpage  AUSA, who neither withdrew from the conspiracy nor
rehabilitated themselves as Gall had done. We also note that neither
the Court of Appeals nor the Government has called our attention to a
comparable defendant who received a more severe sentence.

  Since the District Court committed no procedural error, the only
question for the Court of Appeals was whether the sentence was
reasonable---\\i. e.,} whether the District Judge abused his discretion
in determining that the \S~3553(a) factors supported a sentence of
probation and justified a substantial deviation from the Guidelines
range. As we shall now explain, the sentence was reasonable. The Court
of Appeals' decision to the contrary was incorrect and failed to
demonstrate the requisite deference to the District Judge's decision.

\section{V}

  The Court of Appeals gave virtually no deference to the District
Court's decision that the \S~3553(a) factors justified a significant
variance in this case. Although the Court of Appeals correctly stated
that the appropriate standard of review was abuse of discretion, it
engaged in an analysis that more closely resembled \emph{de novo} review
of the facts presented and determined that, in its view, the degree of
variance was not warranted.

  The Court of Appeals thought that the District Court ``gave too
much weight to Gall's withdrawal from the conspiracy because the
court failed to acknowledge the significant benefit Gall received from
being subject to the 1999 Guidelines.''\footnotemark[10]446 F. 3d, at 889.
This criticism is flawed in that it ignores the critical relevance
of Gall's voluntary withdrawal, a circumstance that distinguished
his conduct not only from that of all his codefendants, but from the
vast ma\newpage jority of defendants convicted of conspiracy in federal
court. The District Court quite reasonably attached great weight to
the fact that Gall voluntarily withdrew from the conspiracy after
deciding, on his own initiative, to change his life. This lends strong
support to the District Court's conclusion that Gall is not going
to return to criminal behavior and is not a danger to society. See
18 U.~S.~C. \S\S~3553(a)(2)(B), (C). Compared to a case where the
offender's rehabilitation occurred after he was charged with a crime,
the District Court here had greater justification for believing Gall's
turnaround was genuine, as distinct from a transparent attempt to build
a mitigation case.

^10 The Court of Appeals explained that under the current Guidelines,
which treat ecstasy more harshly, Gall's base offense level would have
been 32, eight levels higher than the base offense level imposed under
the 1999 Guidelines.

  The Court of Appeals thought the District Judge ``gave significant
weight to an improper factor'' when he compared Gall's sale of
ecstasy when he was a 21-year-old adult to the ``impetuous and
ill-considered'' actions of persons under the age of 18. 446 F. 3d,
at 890. The appellate court correctly observed that the studies cited
by the District Judge do not explain how Gall's ``specific behavior
in the instant case was impetuous or ill-considered.'' \emph{Ibid.}

  In that portion of his sentencing memorandum, however, the judge
was discussing the ``character of the defendant,'' not the nature
of his offense. App. 122. He noted that Gall's criminal history
included a ticket for underage drinking when he was 18 years old and
possession of marijuana that was contemporaneous with his offense
in this case. In summary, the District Judge observed that all of
Gall's criminal history, ``including the present offense, occurred
when he was twenty-one-years old or younger'' and appeared ``to
stem from his addictions to drugs and alcohol.'' \emph{Id.,} at 122,
123. The District Judge appended a long footnote to his discussion
of Gall's immaturity. The footnote includes an excerpt from our
opinion in \emph{Roper} v. \emph{Simmons,} 543 U.~S. 551, 569 (2005),
which quotes a study stating that a lack of maturity and an undeveloped
sense of responsibility are qualities that `` ‘often result in
impetuous and ill-considered actions.' '' \newpage  The District Judge
clearly stated the relevance of these studies in the opening and closing
sentences of the footnote:

      ``Immaturity at the time of the offense conduct is not an
    inconsequential consideration. Recent studies on the development of
    the human brain conclude that human brain development may not become
    complete until the age of twenty-five~.~.~.~. [T]he recent
    [National Institutes of Health] report confirms that there is no
    bold line demarcating at what age a person reaches full maturity.
    While age does not excuse behavior, a sen tencing court should
    account for age when inquiring into the conduct of a defendant.''
    App. 123, n. 2.

  Given the dramatic contrast between Gall's behavior before he
joined the conspiracy and his conduct after withdrawing, it was not
unreasonable for the District Judge to view Gall's immaturity at the
time of the offense as a mitigating factor, and his later behavior
as a sign that he had matured and would not engage in such impetuous
and illconsidered conduct in the future. Indeed, his consideration of
that factor finds support in our cases. See, \emph{e. g., Johnson} v.
\emph{Texas,} 509 U.~S. 350, 367 (1993) (holding that a jury was free to
consider a 19-year-old defendant's youth when determining whether
there was a probability that he would continue to commit violent acts in
the future and stating that `` ‘youth is more than a chronological
fact. It is a time and condition of life when a person may be most
susceptible to influence and to psychological damage' '' (quoting
\emph{Eddings} v. \emph{Oklahoma,} 455 U.~S. 104, 115 (1982))).

  Finally, the Court of Appeals thought that, even if Gall's
rehabilitation was dramatic and permanent, a sentence of probation
for participation as a middleman in a conspiracy distributing 10,000
pills of ecstasy ``lies outside the range of choice dictated by the
facts of the case.''446 F. 3d, at 890 (internal quotation marks
omitted). If the Guidelines were still mandatory, and assuming the
facts did not justify a Guidelines-based downward departure, this
would provide a \newpage  Guidelines state that probation alone is not an
appropriate sentence for comparable offenses.\footnotemark[11] But the Guidelines
are not mandatory, and thus the ``range of choice dictated by the facts
of the case'' is significantly broadened. Moreover, the Guidelines
are only one of the factors to consider when imposing sentence, and
\S~3553(a)(3) directs the judge to consider sentences other than
imprisonment.

  We also note that the Government did not argue below, and has not
argued here, that a sentence of probation could never be imposed for
a crime identical to Gall's. Indeed, it acknowledged that probation
could be permissible if the record contained different---but in our
view, no more compelling---mitigating evidence. Tr. of Oral Arg.
37--38 (stating that probation could be an appropriate sentence, given
the exact same offense, if ``there are compelling family circumstances
where individuals will be very badly hurt in the defendant's family if
no one is available to take care of them'').

  The District Court quite reasonably attached great weight to Gall's
self-motivated rehabilitation, which was undertaken not at the direction
of, or under supervision by, any court, but on his own initiative. This
also lends strong support to the conclusion that imprisonment was not
necessary to deter Gall from engaging in future criminal conduct or to
protect the public from his future criminal acts. See 18 U.~S.~C.
\S\S~3553(a)(2)(B), (C).

  The Court of Appeals clearly disagreed with the District Judge's
conclusion that consideration of the \S~3553(a) factors justified a
sentence of probation; it believed that the circumstances presented
here were insufficient to sustain such a marked deviation from the
Guidelines range. But it is not for the Court of Appeals to decide \emph{de
novo} whether the justification for a variance is sufficient or the
sentence reasonable. On abuse-of-discretion review, the Court of Appeals
should have given due deference to the District Court's reasoned
\newpage  and reasonable decision that the \S~3553(a) factors, on the
whole, justified the sentence. Accordingly, the judgment of the Court of
Appeals is reversed.


^11 Specifically, probation is not recommended under the Guidelines when
the applicable Guidelines range is outside Zone A of the sentencing
table \newpage 


\begin{flushright}\emph{It is so ordered.}\end{flushright}
