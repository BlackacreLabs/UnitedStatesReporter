% Concurring in the Judgment
% Thomas

\setcounter{page}{260}

  \textsc{Justice Thomas,} with whom \textsc{Justice Scalia} joins, concurring in the judgment.

  I agree with the Court that petitioner alleges a cognizable claim under \S~502(a)(2) of the Employee Retirement Income Security Act of 1974 (ERISA), 29 U.~S.~C. \S~1132(a)(2), but it is ERISA's text and not ``the kind of harms that concerned [ERISA's] draftsmen'' that compels my decision. \emph{Ante,} at 256. In \emph{Massachusetts Mut. Life Ins. Co.} v. \emph{Russell,} 473 U.~S. 134 (1985), the Court held that \S~409 of ERISA, 29 U.~S.~C. \S~1109, read together with \S~502(a)(2), authorizes recovery only by ``the plan as an entity,'' 473 U. S., at 140, and does not permit individuals to bring suit when they do not seek relief on behalf of the plan, \emph{id.,} at 139--144. The majority accepts \emph{Russell}'s fundamental holding, but reins in the Court's further suggestion in \emph{Russell} that suits under \S~502(a)(2) are meant to ``protect the entire plan,'' rather than ``the rights of an individual beneficiary.'' \emph{Ante,} at 253--254 (internal quotation marks omitted); see \emph{Russell, supra,} at 142. The majority states that emphasizing the ``entire plan'' was a sensible application of \S\S~409 and 502(a)(2) in the historical context of defined benefit plans, but that the subsequent proliferation of defined contribution plans has rendered \emph{Russell}'s dictum inapplicable to most modern cases. \newpage  \emph{Ante,} at 255--256. In concluding that a loss suffered by a participant's defined contribution plan account because of a fiduciary breach ``creates the kind of harms that concerned the draftsmen of \S~409,'' the majority holds that \S~502(a)(2) authorizes recovery for plan participants such as petitioner. \emph{Ante,} at 256.

  Although I agree with the majority's holding, I write separately because my reading of \S\S~409 and 502(a)(2) is not contingent on trends in the pension plan market. Nor does it depend on the ostensible ``concerns'' of ERISA's drafters. Rather, my conclusion that petitioner has stated a cognizable claim flows from the unambiguous text of \S\S~409 and 502(a)(2) as applied to defined contribution plans. Section 502(a)(2) states that ``[a] civil action may be brought'' by a plan ``participant, beneficiary or fiduciary,'' or by the Secretary of Labor, to obtain ``appropriate relief'' under \S409. 29 U.~S.~C. \S~1132(a)(2). Section 409(a) provides that ``[a]ny person who is a fiduciary with respect to a \emph{plan}\dots shall be personally liable to make good to such \emph{plan} any losses to the \emph{plan} resulting from each [fiduciary] breach, and to restore to such \emph{plan} any profits of such fiduciary which have been made through use of assets of the \emph{plan} by the fiduciary .~.~.~.'' 29 U.~S.~C. \S~1109(a) (emphasis added).

  The plain text of \S~409(a), which uses the term ``plan'' five times, leaves no doubt that \S~502(a)(2) authorizes recovery only for the plan. Likewise, Congress' repeated use of the word ``any'' in \S~409(a) clarifies that the key factor is whether the alleged losses can be said to be losses ``to the plan,'' not whether they are otherwise of a particular nature or kind. See, \emph{e. g., Ali} v. \emph{Federal Bureau of Prisons, ante,} at 219 (noting that the natural reading of ``any'' is ``one or some indiscriminately of whatever kind'' (internal quotation marks omitted)). On their face, \S\S~409(a) and 502(a)(2) permit recovery of \emph{all} plan losses caused by a fiduciary breach.\newpage 

  The question presented here, then, is whether the losses to petitioner's individual 401(k) account resulting from respondents' alleged breach of their fiduciary duties were losses ``to the plan.'' In my view they were, because the assets allocated to petitioner's individual account were plan assets. ERISA requires the assets of a defined contribution plan (including ``gains and losses'' and legal recoveries) to be allocated for bookkeeping purposes to individual accounts within the plan for the beneficial interest of the participants, whose benefits in turn depend on the allocated amounts. See 29 U.~S.~C. \S~1002(34) (defining a ``defined contribution plan'' as a ``plan which provides for an individual account for each participant and for benefits based solely upon the amount contributed to the participant's account, and any income, expenses, gains and losses, and any forfeitures of accounts of other participants which may be allocated to such participant's account''). Thus, when a defined contribution plan sustains losses, those losses are reflected in the balances in the plan accounts of the affected participants, and a recovery of those losses would be allocated to one or more individual accounts.

  The allocation of a plan's assets to individual accounts for bookkeeping purposes does not change the fact that all the assets in the plan remain plan assets. A defined contribution plan is not merely a collection of unrelated accounts. Rather, ERISA requires a plan's combined assets to be held in trust and legally owned by the plan trustees. See 29 U.~S.~C. \S~1103(a) (providing that ``all assets of an employee benefit plan shall be held in trust by one or more trustees''). In short, the assets of a defined contribution plan under ERISA constitute, at the very least, the sum of all the assets allocated for bookkeeping purposes to the participants' individual accounts. Because a defined contribution plan is essentially the sum of its parts, losses attributable to the account of an individual participant are necessarily ``losses to \newpage  the plan'' for purposes of \S~409(a). Accordingly, when a participant sustains losses to his individual account as a result of a fiduciary breach, the plan's aggregate assets are likewise diminished by the same amount, and \S~502(a)(2) permits that participant to recover such losses on behalf of the plan.[[*]]

\footnotetext[*]{Of course, a participant suing to recover benefits on behalf of the plan is not entitled to monetary relief payable directly to him; rather, any recovery must be paid to the plan.}
