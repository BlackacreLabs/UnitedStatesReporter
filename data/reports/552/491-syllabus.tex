% Syllabus
% Reporter of Decisions

\setcounter{page}{491}

  In the \emph{Case Concerning Avena and Other Mexican Nationals}
(\\Mex.} v. \emph{U. S.\\), 2004 I. C. J. 12 (\emph{Avena}), the
International Court of Justice (ICJ) held that the United States had
violated Article 36(1)(b) of the Vienna Convention on Consular Relations
(Vienna Convention or Convention) by failing to inform 51 named Mexican
nationals, including petitioner Medellín, of their Vienna Convention
rights. The ICJ found that those named individuals were entitled to
review and reconsideration of their U. S. state-court convictions
and sentences regardless of their failure to comply with generally
applicable state rules governing challenges to criminal convictions.
In \emph{Sanchez-Llamas} v. \emph{Oregon,} 548 U.~S. 331---issued after
\emph{Avena} but involving individuals who were not named in the \emph{Avena}
judgment---this Court held, contrary to the ICJ's determination,
that the Convention did not preclude the application of state default
rules. The President then issued a memorandum (President's Memorandum
or Memorandum) stating that the United States would ``discharge its
international obligations'' under \emph{Avena} ``by having State courts
give effect to the decision.''

  Relying on \emph{Avena} and the President's Memorandum, Medellín filed
a second Texas state-court habeas application challenging his state
capital murder conviction and death sentence on the ground that he had
not been informed of his Vienna Convention rights. The Texas Court of
Criminal Appeals dismissed Medellín's application as an abuse of the
writ, concluding that neither \emph{Avena} nor the President's Memorandum
was binding federal law that could displace the State's limitations on
filing successive habeas applications. \emph{Held: } Neither \emph{Avena} nor
the President's Memorandum constitutes directly enforceable federal
law that pre-empts state limitations on the filing of successive habeas
petitions. Pp. 504--532.

  1. The \emph{Avena} judgment is not directly enforceable as domestic law
in state court. Pp. 504--523.

  (a) While a treaty may constitute an international commitment, it
is not binding domestic law unless Congress has enacted statutes
implementing it or the treaty itself conveys an intention that it be
``selfexecuting'' and is ratified on that basis. See, \emph{e. g.,
Foster} v. \emph{Neilson,} 2 Pet. 253, 314. The \emph{Avena} judgment
creates an international law obligation on the part of the United
States, but it is not automatically binding \newpage  domestic law because
none of the relevant treaty sources---the Optional Protocol, the United
Nations Charter, or the ICJ Statute---creates binding federal law in the
absence of implementing legislation, and no such legislation has been
enacted.

  The most natural reading of the Optional Protocol is that it is a bare
grant of jurisdiction. The Protocol says nothing about the effect of an
ICJ decision, does not commit signatories to comply therewith, and is
silent as to any enforcement mechanism. The obligation to comply with
ICJ judgments is derived from Article 94 of the U. N. Charter, which
provides that ``[e]ach\dots Member\dots undertakes to comply with
the [ICJ's] decision\dots in any case to which it is a party.''
The phrase ``undertakes to comply'' is simply a commitment by member
states to take future action through their political branches. That
language does not indicate that the Senate, in ratifying the Optional
Protocol, intended to vest ICJ decisions with immediate legal effect in
domestic courts.

  This reading is confirmed by Article 94(2)---the enforcement
provision---which provides the sole remedy for noncompliance: referral
to the U. N. Security Council by an aggrieved state. The provision of
an express diplomatic rather than judicial remedy is itself evidence
that ICJ judgments were not meant to be enforceable in domestic
courts. See \emph{Sanchez}-\\Llamas,} 548 U. S., at 347. Even this
``quintessentially \emph{international} remed[y],'' \emph{id.,} at
355, is not absolute. It requires a Security Council resolution,
and the President and Senate were undoubtedly aware that the United
States retained the unqualified right to exercise its veto of any such
resolution. Medellín's construction would eliminate the option of
noncompliance contemplated by Article 94(2), undermining the ability of
the political branches to determine whether and how to comply with an
ICJ judgment.

  The ICJ Statute, by limiting disputes to those involving nations,
not individuals, and by specifying that ICJ decisions have no binding
force except between those nations, provides further evidence that
the \emph{Avena} judgment does not automatically constitute federal law
enforceable in U. S. courts. Medellín, an individual, cannot be
considered a party to the \emph{Avena} decision. Finally, the United
States' interpretation of a treaty ``is entitled to great weight,''
\emph{Sumitomo Shoji America, Inc.} v. \emph{Avagliano,} 457 U.~S. 176,
184--185, and the Executive Branch has unfailingly adhered to its
view that the relevant treaties do not create domestically enforceable
federal law. Pp. 504--514.

  (b) The foregoing interpretive approach---parsing a treaty's text to
determine if it is self-executing---is hardly novel. This Court has long
looked to the language of a treaty to determine whether the President
who negotiated it and the Senate that ratified it intended that the
treaty \newpage  automatically create domestically enforceable federal
law. See, \emph{e. g., Foster, supra.} Pp. 514--516.

  (c) The Court's conclusion that \emph{Avena} does not by itself
constitute binding federal law is confirmed by the ``postratification
understanding'' of signatory countries. See \emph{Zicherman} v.
\emph{Korean Air Lines Co.,} 516 U.~S. 217, 226. There are currently 47
nations that are parties to the Optional Protocol and 171 nations that
are parties to the Vienna Convention. Yet neither Medellín nor his
\emph{amici} have identified a single nation that treats ICJ judgments as
binding in domestic courts. The lack of any basis for supposing that
any other country would treat ICJ judgments as directly enforceable as
a matter of its domestic law strongly suggests that the treaty should
not be so viewed in our courts. See \emph{Sanchez-Llamas,} 548 U. S., at
343--344, and n. 3.

  The Court's conclusion is further supported by general principles of
interpretation. Given that the forum state's procedural rules govern
a treaty's implementation absent a clear and express statement to
the contrary, see, \emph{e. g., id.,} at 351, one would expect the
ratifying parties to the relevant treaties to have clearly stated any
intent to give ICJ judgments such effect. There is no statement in the
Optional Protocol, the U. N. Charter, or the ICJ Statute that supports
this notion. Moreover, the consequences of Medellín's argument give
pause: neither Texas nor this Court may look behind an ICJ decision and
quarrel with its reasoning or result, despite this Court's holding
in \emph{Sanchez-Llamas} that ``[n]othing in the [ICJ's] structure or
purpose\dots suggests that its interpretations were intended to be
conclusive on our courts,'' \emph{id.,} at 354. Pp. 516--519.

  (d) The Court's holding does not call into question the ordinary
enforcement of foreign judgments. An agreement to abide by the result
of an international adjudication can be a treaty obligation like any
other, so long as the agreement is consistent with the Constitution. In
addition, Congress is up to the task of implementing non-self-executing
treaties, even those involving complex commercial disputes. Medellín
contends that domestic courts generally give effect to foreign
judgments, but the judgment Medellín asks us to enforce is hardly
typical: It would enjoin the operation of state law and force the State
to take action to ``review and reconside[r]'' his case. Foreign
judgments awarding injunctive relief against private parties, let alone
sovereign States, ``are not generally entitled to enforcement.''
1 Restatement (Third) of Foreign Relations Law of the United States
\S~481, Comment \emph{b,} p. 595 (1986). Pp. 519--523.

  2. The President's Memorandum does not independently require the
States to provide review and reconsideration of the claims of the 51
\newpage  Mexican nationals named in \emph{Avena} without regard to state
procedural default rules. Pp. 523--532.

  (a) The President seeks to vindicate plainly compelling interests in
ensuring the reciprocal observance of the Vienna Convention, protecting
relations with foreign governments, and demonstrating commitment to the
role of international law. But those interests do not allow the Court to
set aside first principles. The President's authority to act, as with
the exercise of any governmental power, ``must stem either from an act
of Congress or from the Constitution itself.'' \emph{Youngstown Sheet \&
Tube Co.} v. \emph{Sawyer,} 343 U.~S. 579, 585.

  Justice Jackson's familiar tripartite scheme provides the accepted
framework for evaluating executive action in this area. First, ``[w]hen
the President acts pursuant to an express or implied authorization of
Congress, his authority is at its maximum, for it includes all that
he possesses in his own right plus all that Congress can delegate.''
\emph{Id.,} at 635 (Jackson, J., concurring). Second, ``[w]hen the
President acts in absence of either a congressional grant or denial of
authority, he can only rely upon his own independent powers, but there
is a zone of twilight in which he and Congress may have concurrent
authority, or in which its distribution is uncertain.'' \emph{Id.,} at
637. In such a circumstance, Presidential authority can derive support
from ``congressional inertia, indifference or quiescence.'' \emph{Ibid.}
Finally, ``[w]hen the President takes measures incompatible with the
expressed or implied will of Congress, his power is at its lowest
ebb,'' and the Court can sustain his actions ``only by disabling the
Congress from acting upon the subject.'' \emph{Id.,} at 637--638.
Pp. 523--525.

  (b) The United States marshals two principal arguments in favor of
the President's authority to establish binding rules of decision
that pre-empt contrary state law. The United States argues that the
relevant treaties give the President the authority to implement the
\emph{Avena} judgment and that Congress has acquiesced in the exercise of
such authority. The United States also relies upon an ``independent''
international dispute-resolution power. We find these arguments, as well
as Medellín's additional argument that the President's Memorandum
is a valid exercise of his ``Take Care'' power, unpersuasive. Pp.
525--532.

  (i) The United States maintains that the President's Memorandum is
implicitly authorized by the Optional Protocol and the U. N. Charter.
But the responsibility for transforming an international obligation
arising from a non-self-executing treaty into domestic law falls
to Congress, not the Executive. \emph{Foster,} 2 Pet., at 315. It
is a fundamental constitutional principle that `` ‘[t]he power
to make the necessary laws is in Congress; the power to execute in
the President.''' \emph{Hamdan} v. \emph{Rumsfeld,} 548 U.~S. 557,
591. A non-self-executing treaty, by defini\newpage tion, is one that
was ratified with the understanding that it is not to have domestic
effect of its own force. That understanding precludes the assertion
that Congress has implicitly authorized the President---acting on
his own---to achieve precisely the same result. Accordingly, the
President's Memorandum does not fall within the first category of
the \emph{Youngstown} framework. Indeed, because the non-self-executing
character of the relevant treaties not only refutes the notion that
the ratifying parties vested the President with the authority to
unilaterally make treaty obligations binding on domestic courts, but
also implicitly prohibits him from doing so, the President's assertion
of authority is within \emph{Youngstown}'s third category, not the first
or even the second.

  The United States maintains that congressional acquiescence requires
that the President's Memorandum be given effect as domestic law. But
such acquiescence is pertinent when the President's action falls
within the second \emph{Youngstown} category, not the third. In any
event, congressional acquiescence does not exist here. Congress'
failure to act following the President's resolution of prior ICJ
controversies does not demonstrate acquiescence because in none of
those prior controversies did the President assert the authority to
transform an international obligation into domestic law and thereby
displace state law. The United States' reliance on the President's
``related'' statutory responsibilities and on his ``established
role'' in litigating foreign policy concerns is also misplaced. The
President's statutory authorization to represent the United States
before the United Nations, the ICJ, and the U. N. Security Council
speaks to his \emph{international} responsibilities, not to any unilateral
authority to create domestic law.

  The combination of a non-self-executing treaty and the lack of
implementing legislation does not preclude the President from acting to
comply with an international treaty obligation by other means, so long
as those means are consistent with the Constitution. But the President
may not rely upon a non-self-executing treaty to establish binding rules
of decision that pre-empt contrary state law. Pp. 525--530.

  (ii) The United States also claims that---independent of the United
States' treaty obligations---the Memorandum is a valid exercise of
the President's foreign affairs authority to resolve claims disputes.
See, \emph{e. g., American Ins. Assn.} v. \emph{Garamendi,} 539 U.~S. 396,
415. This Court's claims-settlement cases involve a narrow set
of circumstances: the making of executive agreements to settle civil
claims between American citizens and foreign governments or foreign
nationals. They are based on the view that ``a systematic, unbroken,
executive practice, long pursued to the knowledge of the Congress
and never before questioned,'' can ``raise a presumption that the
[action] had been [taken] in pursuance of its consent.'' \emph{Dames
\& Moore} v. \emph{Regan,} 453 U.~S. 654, \newpage  686. But ``[p]ast
practice does not, by itself, create power.'' \emph{Ibid.} The
President's Memorandum---a directive issued to state courts that would
compel those courts to reopen final criminal judgments and set aside
neutrally applicable state laws---is not supported by a ``particularly
longstanding practice.'' The Executive's limited authority to settle
international claims disputes pursuant to an executive agreement cannot
stretch so far. Pp. 530--532.

  (iii) Medellín's argument that the President's Memorandum is
a valid exercise of his power to ``Take Care'' that the laws be
faithfully executed, U. S. Const., Art. II, \S3, fails because the
ICJ's decision in \emph{Avena} is not domestic law. P. 532.

223 S. W. 3d 315, affirmed.

  \textsc{Roberts,} C. J., delivered the opinion of the Court, in which
\textsc{Scalia,} \textsc{Kennedy, Thomas,} and \textsc{Alito,} JJ., joined.
\textsc{Stevens,} J., filed an opinion concurring in the judgment, \emph{post,}
p. 533. \textsc{Breyer,} J., filed a dissenting opinion, in which \textsc{Souter}
and \textsc{Ginsburg, JJ.,} joined, \emph{post,} p. 538.

  \emph{Donald Francis Donovan} argued the cause for petitioner. With him
on the briefs were \emph{Carl Micarelli} and \emph{Catherine M. Amirfar.}

  \emph{Solicitor General Clement} argued the cause for the United States
as \emph{amicus curiae} urging reversal. With him on the brief were
\emph{Assistant Attorney General Fisher, Deputy Solicitor General Dreeben,
Irving L. Gornstein,} and \emph{Robert J. Erickson.}

  \emph{R. Ted Cruz,} Solicitor General of Texas, argued the cause for
respondent. With him on the brief were \emph{Greg Abbott,} Attorney
General, \emph{Kent C. Sullivan,} First Assistant Attorney General, \emph{Eric
J. R. Nichols,} Deputy Attorney General, \emph{Sean D. Jordan,} Deputy
Solicitor General, and \emph{Kristofer S. Monson, Daniel L. Geyser,} and
\emph{Adam W. Aston,} Assistant Solicitors General.[[*]]

^* Briefs of \emph{amici curiae} urging reversal were filed for the
Government of the United Mexican States by \emph{Sandra L. Babcock;} for
the American Bar Association by \emph{Karen J. Mathis} and \emph{Jeffrey L.
Bleich;} for Foreign Sovereigns by \emph{Asim M. Bhansali, Steven A.
Hirsch, Craig Smyser,} and \emph{Jason Luong;} for Former United States
Diplomats by \emph{Harold Hongju Koh, Don} \newpage  \emph{ald B. Ayer, Charles
R. A. Morse,} and \emph{Christian G. Vergonis;} and for Ambassador L.
Bruce Laingen et al. by \emph{Daniel C. Malone.}

^   Briefs of \emph{amici curiae} urging affirmance were filed for the
Commonwealth of Virginia et al. by \emph{Robert F. McDonnell,} Attorney
General of Virginia, \emph{William E. Thro,} State Solicitor General,
\emph{Stephen R. McCullough,} Deputy State Solicitor General, and \emph{William
C. Mims,} Chief Deputy Attorney General, and by the Attorneys General
for their respective jurisdictions as follows: \emph{Troy King} of Alabama,
\emph{Talis J. Colberg} of Alaska, \emph{Terry Goddard} of Arizona, \emph{Dustin
McDaniel} of Arkansas, \emph{Edmund G. Brown, Jr.,} of California, \emph{John
W. Suthers} of Colorado, \emph{Joseph R. Biden III} of Delaware, \emph{Bill
McCollum} of Florida, \emph{Thurbert E. Baker} of Georgia, \emph{Lawrence
Wasden} of Idaho, \emph{Steve Carter} of Indiana, \emph{Paul J. Morrison} of
Kansas, \emph{Gregory D. Stumbo} of Kentucky, \emph{Jim Hood} of Mississippi,
\emph{Jeremiah W. (Jay) Nixon} of Missouri, \emph{Mike McGrath} of Montana,
\emph{Catherine Cortez Masto} of Nevada, \emph{Roy Cooper} of North Carolina,
\emph{Wayne Stenehjem} of North Dakota, \emph{W. A. Drew Edmondson} of
Oklahoma, \emph{Hardy Myers} of Oregon, \emph{Thomas W. Corbett, Jr.,} of
Pennsylvania, \emph{Roberto J. Sánchez-Ramos} of Puerto Rico, \emph{Henry D.
McMaster} of South Carolina, \emph{Lawrence E. Long} of South Dakota,
\emph{Robert E. Cooper, Jr.,} of Tennessee, \emph{Mark L. Shurtleff} of Utah,
and \emph{Rob McKenna} of Washington; for Constitutional and International
Law Scholars by \emph{Ernest A. Young} and \emph{Edward C. Dawson;} for Former
Senior Officials of the Department of Justice by \emph{Charles J. Cooper}
and \emph{Brian Stuart Koukoutchos;} for the Washington Legal Foundation et
al. by \emph{Daniel J. Popeo} and \emph{Richard A. Samp;} and for Randy and
Sandra Ertman et al. by \emph{Kent S. Scheidegger.}

^   Briefs of \emph{amici curiae} were filed for the European Union et
al. by \emph{S. Adele Shank} and \emph{John B. Quigley;} for EarthRights
International by \emph{Judith Brown Chomsky;} for International Court of
Justice Experts by \emph{Lori Fisler Damrosch} and \emph{Charles Owen Verrill,
Jr.;} and for the Mountain States Legal Foundation by \emph{William Perry
Pendley.}
