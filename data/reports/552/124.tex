% Opinion of the Court
% Thomas

\setcounter{page}{215}

  \textsc{Justice Thomas} delivered the opinion of the Court.

  This case concerns the scope of 28 U.~S.~C. \S~2680, which carves out certain exceptions to the United States' waiver of sovereign immunity for torts committed by federal employees. Section 2680(c) provides that the waiver of sovereign\starpage\ immunity does not apply to claims arising from the detention of property by ``any officer of customs or excise or any other law enforcement officer.'' Petitioner contends that this clause applies only to law enforcement officers enforcing customs or excise laws, and thus does not affect the waiver of sovereign immunity for his property claim against officers of the Federal Bureau of Prisons (BOP). We conclude that the broad phrase ``any other law enforcement officer'' covers all law enforcement officers. Accordingly, we affirm the judgment of the Court of Appeals upholding the dismissal of petitioner's claim. 
\section{I}

  Petitioner Abdus-Shahid M. S. Ali was a federal prisoner at the United States Penitentiary in Atlanta, Georgia, from 2001 to 2003. In December 2003, petitioner was scheduled to be transferred to the United States Penitentiary Big Sandy (USP Big Sandy) in Inez, Kentucky. Before being transferred, he left two duffle bags containing his personal property in the Atlanta prison's Receiving and Discharge Unit to be inventoried, packaged, and shipped to USP Big Sandy. Petitioner was transferred, and his bags arrived some days later. Upon inspecting his property, he noticed that several items were missing. The staff at USP Big Sandy's Receiving and Discharge Unit told him that he had been given everything that was sent, and that if things were missing he could file a claim. Many of the purportedly missing items were of religious and nostalgic significance, including two copies of the Qur'an, a prayer rug, and religious magazines. Petitioner estimated that the items were worth \$177.

  Petitioner filed an administrative tort claim. In denying relief, the agency noted that, by his signature on the receipt form, petitioner had certified the accuracy of the inventory listed thereon and had thereby relinquished any future claims relating to missing or damaged property. Petitioner then filed a complaint alleging, \emph{inter alia,} violations of the\starpage  Federal Tort Claims Act (FTCA), 28 U.~S.~C. \S\S~1346, 2671 \emph{et seq.} The BOP maintained that petitioner's claim was barred by the exception in \S~2680(c) for property claims against law enforcement officers. The District Court agreed and dismissed petitioner's FTCA claim for lack of subjectmatter jurisdiction. Petitioner appealed.

  The Eleventh Circuit affirmed, agreeing with the District Court's interpretation of \S~2680(c). 204 Fed. Appx. 778, 779--780 (2006) \emph{(per curiam).} In rejecting petitioner's arguments, the Court of Appeals relied on this Court's broad interpretation of \S~2680(c)'s ``detention'' clause in \emph{Kosak} v. \emph{United States,} 465 U. S. 848, 854--859 (1984), on decisions by other Courts of Appeals, and on its own decision in\emph{Schlaebitz} v. \emph{United States Dept. of Justice,} 924 F. 2d 193, 195 (1991) \emph{(per curiam)} (holding that United States Marshals, who were allegedly negligent in releasing a parolee's luggage to a third party, were ``law enforcement officers'' under \S~2680(c)). See 204 Fed. Appx., at 779--780.

  We granted certiorari, 550 U.~S. 968 (2007), to resolve the disagreement among the Courts of Appeals as to the scope of \S~2680(c).\footnotemark[1]

\section{II}

  In the FTCA, Congress waived the United States' sovereign immunity for claims arising out of torts committed\starpage  by federal employees. See 28 U.~S.~C. \S~1346(b)(1). As relevant here, the FTCA authorizes ``claims against the United States, for money damages\dots for injury or loss of property\dots caused by the negligent or wrongful act or omission of any employee of the Government while acting within the scope of his office or employment.'' \emph{Ibid.} The FTCA exempts from this waiver certain categories of claims. See \S\S~2680(a)--(n). Relevant here is the exception in subsection (c), which provides that \S~1346(b) shall not apply to ``[a]ny claim arising in respect of the assessment or collection of any tax or customs duty, or the detention of any goods, merchandise, or other property by any officer of customs or excise or any other law enforcement officer.''\S~2680(c).

\footnotetext[1]{The Eleventh Circuit joined five other Courts of Appeals in construing \S~2680(c) to encompass all law enforcement officers. See \emph{Bramwell} v. \emph{Bureau of Prisons,} 348 F. 3d 804, 806--807 (CA9 2003); \emph{Chapa} v. \emph{United States Dept. of Justice,} 339 F. 3d 388, 390 (CA5 2003) \emph{(per curiam); Hatten} v. \emph{White,} 275 F. 3d 1208, 1210 (CA10 2002); \emph{Cheney} v. \emph{United States,} 972 F. 2d 247, 248 (CA8 1992) \emph{(per curiam); Ysasi} v. \emph{Rivkind,} 856 F. 2d 1520, 1525 (CA Fed. 1988). Five other Courts of Appeals reached the contrary conclusion, interpreting the clause as limited to officers performing customs or excise functions.See \emph{ABC} v. \emph{DEF,} 500 F. 3d 103, 107 (CA2 2007); \emph{Dahler} v. \emph{United States,} 473 F. 3d 769, 771--772 (CA7 2007) \emph{(per curiam); Andrews} v. \emph{United States,} 441 F. 3d 220, 227 (CA4 2006); \emph{Bazuaye} v. \emph{United States,} 83 F. 3d 482, 486 (CADC 1996); \emph{Kurinsky} v. \emph{United States,} 33 F. 3d 594, 598 (CA6 1994).}

  This case turns on whether the BOP officers who allegedly lost petitioner's property qualify as ``other law enforcement officer[s]'' within the meaning of \S~2680(c).\footnotemark[2] Petitioner argues that they do not because ``any other law enforcement officer'' includes only law enforcement officers acting in a customs or excise capacity. Noting that Congress referenced customs and excise activities in both the language at issue and the preceding clause in \S~2680(c), petitioner argues that the entire subsection is focused on preserving the United States' sovereign immunity only as to officers enforcing those laws.

  Petitioner's argument is inconsistent with the statute's language.\footnotemark[3] The phrase ``\emph{any} other law enforcement officer''\starpage  suggests a broad meaning. \emph{Ibid.} (emphasis added). We have previously noted that ``[r]ead naturally, the word `any' has an expansive meaning, that is, ‘one or some indiscriminately of whatever kind.' ''\emph{United States} v. \emph{Gonzales,} 520 U.~S. 1, 5 (1997) (quoting Webster's Third New International Dictionary 97 (1976)). In \emph{Gonzales,} we considered a provision that imposed an additional sentence for firearms used in federal drug trafficking crimes and provided that such additional sentence shall not be concurrent with `` ‘any other term of imprisonment.' '' 520 U. S., at 4 (quoting 18 U.~S.~C. \S~924(c)(1) (1994 ed.); emphasis deleted). Notwithstanding the subsection's initial reference to federal drug trafficking crimes, we held that the expansive word ``any'' and the absence of restrictive language left ``no basis in the text for limiting'' the phrase ``any other term of imprisonment'' to federal sentences. 520 U. S., at 5. Similarly, in \emph{Harrison} v. \emph{PPG Industries, Inc.,} 446 U.~S. 578 (1980), the Court considered the phrase ``any other final action'' in amendments to the Clean Air Act. The Court explained that the amendments expanded a list of Environmental Protection Agency Administrator actions by adding two categories of actions: actions under a specifically enumerated statutory provision, and ``any other final action'' under the Clean Air Act. \emph{Id.,} at 584 (emphasis deleted). Focusing on Congress' choice of the word ``any,'' the Court ``discern[ed] no uncertainty in the meaning of the phrase, ‘any other final action,' '' and emphasized that the statute's ``expansive language offer[ed] no indi\starpage cation whatever that Congress intended'' to limit the phrase to final actions similar to those in the specifically enumerated sections.\emph{Id.,} at 588--589.

\footnotetext[2]{We assume, without deciding, that the BOP officers ``detained'' Ali's property and thus satisfy \S~2680(c)'s ``arising in respect of\dots detention'' requirement. The Court of Appeals held that the ``detention'' clause was satisfied, and petitioner expressly declined to raise the issue on certiorari.See 204 Fed. Appx. 778, 779--780 (CA11 2006) \emph{(per curiam);} Brief for Petitioner 10--11, n. 9.}

\footnotetext[3]{We consider this question for the first time in this case. Petitioner argues that this Court concluded in \emph{Kosak} v. \emph{United States,} 465 U.~S. 848 (1984), that the phrase ``any other law enforcement officer'' is ambiguous. Reply Brief for Petitioner 4. In that case, the Court construed a portion\starpage  of the same clause at issue here, but the decision had no bearing on the meaning of ``any other law enforcement officer.'' 465 U. S., at 853--862 (holding that ``detention'' encompasses claims resulting from negligent handling or storage). Indeed, the Court expressly declined to reach the issue. \emph{Id.,} at 852, n. 6 (``We have no occasion in this case to decide what kinds of ‘law-enforcement officer[s],' other than customs officials, are covered by the exception'' (alteration in original)). Petitioner's reliance on the footnote as concluding that the phrase is ambiguous reads too much into the Court's reservation of a question that was not then before it.}

  We think the reasoning of \emph{Gonzales} and \emph{Harrison} applies equally to the expansive language Congress employed in 28 U.~S.~C. \S~2680(c). Congress' use of ``any'' to modify ``other law enforcement officer'' is most naturally read to mean law enforcement officers of whatever kind.\footnotemark[4] The word ``any'' is repeated four times in the relevant portion of \S~2680(c), and two of those instances appear in the particular phrase at issue: ``\emph{any} officer of customs or excise or \emph{any} other law enforcement officer.'' (Emphasis added.) Congress inserted the word ``any'' immediately before ``other law enforcement officer,'' leaving no doubt that it modifies that phrase. To be sure, the text's references to ``tax or customs duty'' and ``officer[s] of customs or excise'' indicate that Congress intended to preserve immunity for claims arising from an officer's enforcement of tax and customs laws. The text also indicates, however, that Congress intended to preserve immunity for claims arising from the detention of property, and\starpage  there is no indication that Congress intended immunity for ose claims thto turn on the type of law being enforced.

\footnotetext[4]{Of course, other circumstances may counteract the effect of expansive modifiers. For example, we have construed an ``any'' phrase narrowly when it included a term of art that compelled that result. See \emph{Circuit City Stores, Inc.} v. \emph{Adams,} 532 U.~S. 105, 115--116 (2001) (construing ``any other class of workers engaged in\dots commerce,'' 9 U.~S.~C. \S~1, narrowly based on the Court's previous interpretation of ``in commerce'' as a term of art with a narrower meaning). We also have construed such phrases narrowly when another term in the provision made sense only under a narrow reading, see \emph{United States} v. \emph{Alvarez-Sanchez,} 511 U.~S. 350, 357--358 (1994) (limiting ``any law-enforcement officer'' to federal officers because the statute's reference to ``delay'' made sense only with respect to federal officers), and when a broad reading would have implicated sovereignty concerns, see \emph{Raygor} v. \emph{Regents of Univ. of Minn.,} 534 U.~S. 533, 541--542 (2002) (applying the ``clear statement rule'' applicable to waivers of sovereign immunity to construe the phrase ``all civil actions'' to exclude a category of claims, ``even though nothing in the statute expressly exclude[d]'' them). None of the circumstances that motivated our decisions in these cases is present here.}

  Petitioner would require Congress to clarify its intent to cover all law enforcement officers by adding phrases such as ``performing any official law enforcement function,'' or ``without limitation.'' But Congress could not have chosen a more all-encompassing phrase than ``any other law enforcement officer'' to express that intent. We have no reason to demand that Congress write less economically and more repetitiously.

  Recent amendments to \S~2680(c) support the conclusion that ``any other law enforcement officer'' is not limited to officers acting in a customs or excise capacity. In the Civil Asset Forfeiture Reform Act of 2000, Congress added subsections (c)(1)--(c)(4) to 28 U.~S.~C. \S~2680. \S~3(a), 114 Stat. 211. As amended, \S~2680(c) provides that the \S~1346(b) waiver of sovereign immunity, notwithstanding the exception at issue in this case, applies to: ``[A]ny claim based on injury or loss of goods, merchandise, or other property, while in the possession of any officer of customs or excise or any other law enforcement officer, if---``(1) the property was seized for the purpose of forfeiture under any provision of Federal law providing for the forfeiture of property other than as a sentence imposed upon conviction of a criminal offense; ``(2) the interest of the claimant was not forfeited; ``(3) the interest of the claimant was not remitted or mitigated (if the property was subject to forfeiture); and ``(4) the claimant was not convicted of a crime for which the interest of the claimant in the property was subject to forfeiture under a Federal criminal forfeiture law.''

  The amendment does not govern petitioner's claim because his property was not ``seized for the purpose of forfeiture,'' \starpage as required by paragraph (1). Nonetheless, the amendment is relevant because our construction of ``any other law enforcement officer'' must, to the extent possible, ensure that the statutory scheme is coherent and consistent. See \emph{Robinson} v. \emph{Shell Oil Co.,} 519 U.~S. 337, 340 (1997) (citing \emph{United States} v. \emph{Ron Pair Enterprises, Inc.,} 489 U.~S. 235, 240 (1989)). The amendment canceled the exception---and thus restored the waiver of sovereign immunity---for certain seizures of property based on \emph{any} federal forfeiture law.See 28 U.~S.~C. \S2680(c)(1) (excepting property claims if ``the property was seized for the purpose of forfeiture under \emph{any provision of Federal law} providing for the forfeiture of property'' (emphasis added)).

  Under petitioner's interpretation, only law enforcement officers enforcing customs or excise laws were immune under the prior version of \S~2680(c). Thus, on petitioner's reading, the amendment's only effect was to restore the waiver for cases in which customs or excise officers, or officers acting in such a capacity, enforce forfeiture laws. This strikes us as an implausible interpretation of the statute. If that were Congress' intent, it is not apparent why Congress would have restored the waiver with respect to the enforcement of \emph{all} civil forfeiture laws instead of simply those related to customs or excise. Petitioner's interpretation makes sense only if we assume that Congress went out of its way to restore the waiver for cases in which customs or excise officers, or officers acting in such a capacity, enforce forfeiture laws unrelated to customs or excise. But petitioner fails to demonstrate that customs or excise officers, or officers acting in such a capacity, ever enforce civil forfeiture laws unrelated to customs or excise, much less that they do so with such frequency that Congress is likely to have singled them out in the amendment.\footnotemark[5] It seems far more likely that Congress\starpage  restored the waiver for officers enforcing \emph{any} civil forfeiture law because, in its view, \emph{all} such officers were covered by the exception to the waiver prior to the amendment.

\footnotetext[5]{\textsc{Justice Kennedy'}s dissent (hereinafter the dissent) argues that, during border searches, customs and excise officers ``routinely'' enforce civil\starpage  forfeiture laws unrelated to customs or excise. \emph{Post,} at 239--240. But the examples the dissent provides do not support that assertion. The dissent maintains that a customs officer who seizes material defined as contraband under 49 U.~S.~C. \S~80302 \emph{et seq.} is one such example. \emph{Post,} at 240. But a customs officer's authority to effect a forfeiture of such contraband derives from a specific customs law.See 19 U.~S.~C. \S~1595a(c)(1)(C). Similarly, the dissent suggests that a Drug Enforcement Administration (DEA) agent ``assisting a customs official'' in a border search who seizes drug-related contraband under 21 U.~S.~C. \S~881 is acting in a ``traditional revenue capacity.'' \emph{Post,} at 240. But that argument is based on the assumption that an officer who assists in conducting a border search acts in a customs capacity even if he is not a customs officer and is not enforcing a customs law. That assumption, far from self-evident, only underscores the difficulty that would attend any attempt to define the contours of the implied limitation on \S~2680(c)'s reach proposed by petitioner and embraced by the dissent. ``Acting in a customs or excise capacity'' is not a selfdefining concept, and not having included such a limitation in the statute's language, Congress of course did not provide a definition. Finally, the dissent points out that a customs or excise officer might effect a forfeiture of currency or monetary instruments under 31 U.~S.~C. \S~5317(c). \emph{Post,} at 240. But \S~5317(c) is hardly a civil forfeiture law unrelated to customs or excise. See \S~5317(c)(2) (2000 ed., Supp. \textsc{V}) (authorizing forfeiture of property involved in a violation of, \emph{inter alia,} \S~5316 (2000 ed.), which sets forth reporting requirements for exporting and importing monetary instruments).}

  Against this textual and structural evidence that ``any other law enforcement officer'' does in fact mean any other law enforcement officer, petitioner invokes numerous canons of statutory construction. He relies primarily on \emph{ejusdem generis,} or the principle that ``when a general term follows a specific one, the general term should be understood as a reference to subjects akin to the one with specific enumeration.'' \emph{Norfolk \& Western R. Co.} v. \emph{Train Dispatchers,} 499 U.~S. 117, 129 (1991). In petitioner's view, ``any officer of customs or excise or any other law enforcement officer'' should be read as a three-item list, and the final, catchall\starpage  phrase ``any other law enforcement officer'' should be limited to officers of the same nature as the preceding specific phrases.

  Petitioner likens his case to two recent cases in which we found the canon useful. In \emph{Washington State Dept. of Social and Health Servs.} v. \emph{Guardianship Estate of Keffeler,} 537 U.~S. 371, 375 (2003), we considered the clause ``execution, levy, attachment, garnishment, or other legal process'' in 42 U.~S.~C. \S~407(a). Applying \emph{ejusdem generis,} we concluded that ``other legal process'' was limited to legal processes of the same nature as the specific items listed. 537 U. S., at 384--385. The department's scheme for serving as a representative payee of the benefits due to children under its care, while a ``legal process,'' did not share the common attribute of the listed items, viz., ``utilization of some judicial or quasi-judicial mechanism\dots by which control over property passes from one person to another in order to discharge'' a debt. \emph{Id.,} at 385. Similarly, in \emph{Dolan} v. \emph{Postal Service,} 546 U.~S. 481 (2006), the Court considered whether an exception to the FTCA's waiver of sovereign immunity for claims arising out of the `` ‘loss, miscarriage, or negligent transmission of letters or postal matter' '' barred a claim that mail negligently left on the petitioner's porch caused her to slip and fall. \emph{Id.,} at 485 (quoting 28 U.~S.~C. \S~2680(b)). Noting that ``loss'' and ``miscarriage'' both addressed ``failings in the postal obligation to deliver mail in a timely manner to the right address,'' 546 U. S., at 487, the Court concluded that ``negligent transmission'' must be similarly limited, \emph{id.,} at 486--489, and rejected the Government's argument that the exception applied to ``all torts committed in the course of mail delivery,'' \emph{id.,} at 490.

  Petitioner asserts that \S~2680(c), like the clauses at issue in \emph{Keffeler} and \emph{Dolan,} `` ‘presents a textbook \emph{ejusdem generis} scenario.' '' Brief for Petitioner 15 (quoting \emph{Andrews} v. \emph{United States,} 441 F. 3d 220, 224 (CA4 2006)). We disagree.\starpage  The structure of the phrase ``any officer of customs or excise or any other law enforcement officer'' does not lend itself to application of the canon. The phrase is disjunctive, with one specific and one general category, not---like the clauses at issue in \emph{Keffeler} and \emph{Dolan}---a list of specific items separated by commas and followed by a general or collective term. The absence of a list of specific items undercuts the inference embodied in \emph{ejusdem generis} that Congress remained focused on the common attribute when it used the catchall phrase.Cf. \emph{United States} v. \emph{Aguilar,} 515 U.~S. 593, 615 (1995) (\textsc{Scalia,} J., concurring in part and dissenting in part) (rejecting the canon's applicability to an omnibus clause that was ``one of\dots several distinct and independent prohibitions'' rather than ``a general or collective term following a list of specific items to which a particular statutory command is applicable'').

  Moreover, it is not apparent what common attribute connects the specific items in \S~2680(c). Were we to use the canon to limit the meaning of ``any other law enforcement officer,'' we would be required to determine the relevant limiting characteristic of ``officer of customs or excise.'' In \emph{Jarecki} v. \emph{G. D. Searle \& Co.,} 367 U.~S. 303 (1961), for example, the Court invoked \emph{noscitur a sociis} in limiting the scope of the term `` ‘discovery' '' to the common characteristic it shared with `` ‘exploration' '' and `` ‘prospecting.' '' \emph{Id.,} at 307. The Court noted that all three words in conjunction ``describe[d] income-producing activity in the oil and gas and mining industries.'' \emph{Ibid.} Here, by contrast, no relevant common attribute immediately appears from the phrase ``officer of customs or excise.'' Petitioner suggests that the common attribute is that both types of officers are charged with enforcing the customs and excise laws. But we see no reason why that should be the relevant characteristic as opposed to, for example, that officers of that type are commonly involved in the activities enumerated in the statute: the as\starpage sessment and collection of taxes and customs duties and the detention of property.

  Petitioner's appeals to other interpretive principles are also unconvincing. Petitioner contends that his reading is supported by the canon \emph{noscitur a sociis,} according to which `` ‘a word is known by the company it keeps.' '' \emph{S. D. Warren Co.} v. \emph{Maine Bd. of Environmental Protection,} 547 U.~S. 370, 378 (2006). But the cases petitioner cites in support of applying \emph{noscitur a sociis} involved statutes with stronger contextual cues. See \emph{Gutierrez} v. \emph{Ada,} 528 U.~S. 250, 254--258 (2000) (applying the canon to narrow the relevant phrase, ``any election,'' where it was closely surrounded by six specific references to gubernatorial elections); \emph{Jarecki, supra,} at 306--309 (applying the canon to narrow the term ``discoveries'' to discoveries of mineral resources where it was contained in a list of three words, all of which applied to the oil, gas, and mining industries and could not conceivably all apply to any other industry). Here, although customs and excise are mentioned twice in \S~2680(c), nothing in the overall statutory context suggests that customs and excise officers were the exclusive focus of the provision. The emphasis in subsection (c) on customs and excise is not inconsistent with the conclusion that ``any other law enforcement officer'' sweeps as broadly as its language suggests.

  Similarly, the rule against superfluities lends petitioner sparse support. The construction we adopt today does not necessarily render ``any officer of customs or excise'' superfluous; Congress may have simply intended to remove any doubt that officers of customs or excise were included in ``law enforcement officer[s].'' See \emph{Fort Stewart Schools} v. \emph{FLRA,} 495 U.~S. 641, 646 (1990) (noting that ``technically unnecessary'' examples may have been ``inserted out of an abundance of caution''). Moreover, petitioner's construction threatens to render ``any other law enforcement officer'' superfluous because it is not clear when, if ever, ``other law enforcement \starpage officer[s]'' act in a customs or excise capacity.In any event, we do not woodenly apply limiting principles every time Congress includes a specific example along with a general phrase.See \emph{Harrison,} 446 U. S., at 589, n. 6 (rejecting an argument that \emph{ejusdem generis} must apply when a broad interpretation of the clause could render the specific enumerations unnecessary).

  In the end, we are unpersuaded by petitioner's attempt to create ambiguity where the statute's text and structure suggest none. Had Congress intended to limit \S~2680(c)'s reach as petitioner contends, it easily could have written ``any other law enforcement officer \emph{acting in a customs or excise capacity.}'' Instead, it used the unmodified, all-encompassing phrase ``any other law enforcement officer.'' Nothing in the statutory context requires a narrowing construction---indeed, as we have explained, the statute is most consistent and coherent when ``any other law enforcement\starpage\ officer'' is read to mean what it literally says. See \emph{Norfolk \& Western R. Co.,} 499 U. S., at 129 (noting that interpretive canons must yield ``when the whole context dictates a different conclusion''). It bears emphasis, moreover, that \S~2680(c), far from maintaining sovereign immunity for the entire universe of claims against law enforcement officers, does so only for claims ``arising in respect of'' the ``detention'' of property. We are not at liberty to rewrite the statute to reflect a meaning we deem more desirable.\footnotemark[7] Instead, we must give effect to the text Congress enacted: Section 2680(c) forecloses lawsuits against the United States for the unlawful detention of property by ``any,'' not just ``some,'' law enforcement officers.


\footnotetext[6]{As an example of ``other law enforcement officer[s]'' acting in an excise or customs capacity, petitioner cites \emph{Formula One Motors, Ltd.} v. \emph{United States,} 777 F. 2d 822, 823--824 (CA2 1985) (holding that the seizure of a vehicle still in transit from overseas by DEA agents who searched it for drugs was ``sufficiently akin to the functions carried out by Customs officials to place the agents' conduct within the scope of section 2680(c)''). But it is not clear that the agents in that case were acting in an excise or customs capacity rather than in their ordinary capacity as law enforcement agents. It seems to us that DEA agents searching a car for drugs are acting in their capacity as officers charged with enforcing the Nation's drug laws, not the customs or excise laws.}

  Similarly, the dissent notes that 14 U.~S.~C. \S~89(a) authorizes Coast Guard officers to enforce customs laws. \emph{Post,} at 233. But the very next subsection of \S~89 provides that Coast Guard officers effectively \emph{are} customs officers when they enforce customs laws. See \S~89(b)(1) (providing that Coast Guard officers ``insofar as they are engaged, pursuant to the authority contained in this section, in enforcing any law of the United States shall\dots be deemed to be acting as agents of the particular executive department\dots charged with the administration of the particular law''). As a result, a Coast Guard officer enforcing a customs law is a customs officer, not some ``other law enforcement officer.''

\section{III}

  For the reasons stated, the judgment of the Court of Appeals for the Eleventh Circuit is

\begin{flushright}
	\emph{Affirmed.}
\end{flushright}

\footnotetext[7]{Congress, we note, did provide an administrative remedy for lost property claimants like petitioner. Federal agencies have authority under 31 U.~S.~C. \S~3723(a)(1) to settle certain ``claim[s] for not more than \$1,000 for damage to, or loss of, privately owned property that\dots is caused by the negligence of an officer or employee of the United States Government acting within the scope of employment.'' The BOP has settled more than 1,100 such claims in the last three years. Brief for Respondents 41, n. 17.}
