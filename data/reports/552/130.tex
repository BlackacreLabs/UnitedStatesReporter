% Court
% Breyer

\setcounter{page}{132}

  \textsc{Justice Breyer} delivered the opinion of the Court.

  The question presented is whether a court must raise on its own the timeliness of a lawsuit filed in the Court of Federal Claims, despite the Government's waiver of the issue. We hold that the special statute of limitations governing the Court of Federal Claims requires that \emph{sua sponte} consideration.

\section{I}

  Petitioner John R. Sand \& Gravel Company filed an action in the Court of Federal Claims in May 2002. The complaint explained that petitioner held a 50-year mining lease on certain land. And it asserted that various Environmental Protection Agency activities on that land (involving, \emph{e. g.,} the building and moving of various fences) amounted to an unconstitutional taking of its leasehold rights.

  The Government initially asserted that petitioner's several claims were all untimely in light of the statute providing that ``[e]very claim of which the United States Court of Federal Claims has jurisdiction shall be barred unless the petition thereon is filed within six years after such claim first accrues.'' 28 U.~S.~C. \S~2501. Later, however, the Government effectively conceded that certain claims were timely. See App. 37a--39a (Government's pretrial brief). The Government subsequently won on the merits.See 62 Fed. Cl. 556, 589 (2004).\newpage 

  Petitioner appealed the adverse judgment to the Court of Appeals for the Federal Circuit. See 457 F. 3d 1345, 1346 (2006). The Government's brief said nothing about the statute of limitations, but an \emph{amicus} brief called the issue to the court's attention. See \emph{id.,} at 1352. The court considered itself obliged to address the limitations issue, and it held that the action was untimely. \emph{Id.,} at 1353--1360. We subsequently agreed to consider whether the Court of Appeals was right to ignore the Government's waiver and to decide the timeliness question.550 U.~S. 968 (2007).

\section{II}

  Most statutes of limitations seek primarily to protect defendants against stale or unduly delayed claims. See, \emph{e. g., United States} v. \emph{Kubrick,} 444 U.~S. 111, 117 (1979). Thus, the law typically treats a limitations defense as an affirmative defense that the defendant must raise at the pleadings stage and that is subject to rules of forfeiture and waiver. See Fed. Rules Civ. Proc. 8(c)(1), 12(b), 15(a); \emph{Day} v. \emph{McDonough,} 547 U.~S. 198, 202 (2006); \emph{Zipes} v. \emph{Trans World Airlines, Inc.,} 455 U.~S. 385, 393 (1982). Such statutes also typically permit courts to toll the limitations period in light of special equitable considerations. See, \emph{e. g., Rotella} v. \emph{Wood,} 528 U.~S. 549, 560--561 (2000); \emph{Zipes, supra,} at 393; see also \emph{Cada} v. \emph{Baxter Healthcare Corp.,} 920 F. 2d 446, 450--453 (CA7 1990).

  Some statutes of limitations, however, seek not so much to protect a defendant's case-specific interest in timeliness as to achieve a broader system-related goal, such as facilitating the administration of claims, see, \emph{e. g., United States} v. \emph{Brockamp,} 519 U. S. 347, 352--353 (1997), limiting the scope of a governmental waiver of sovereign immunity, see, \emph{e. g., United States} v. \emph{Dalm,} 494 U.~S. 596, 609--610 (1990), or promoting judicial efficiency, see, \emph{e. g., Bowles} v. \emph{Russell,} 551 U.~S. 205, 210--213 (2007). The Court has often read the time limits of these statutes as more absolute, say, as requir\newpage ing a court to decide a timeliness question despite a waiver, or as forbidding a court to consider whether certain equitable considerations warrant extending a limitations period. See, \emph{e. g., id.,} at 212--213; see also \emph{Arbaugh} v. \emph{Y \&HCorp.,} 546 U. S. 500, 514 (2006). As convenient shorthand, the Court has sometimes referred to the time limits in such statutes as ``jurisdictional.'' See, \emph{e. g., Bowles, supra,} at 210.

  This Court has long interpreted the court of claims limitations statute as setting forth this second, more absolute, kind of limitations period.

\subsection{A}

  In \emph{Kendall} v. \emph{United States,} 107 U.~S. 123 (1883), the Court applied a predecessor of the current 6-year bar to a claim that had first accrued in 1865 but that the plaintiff did not bring until 1872. \emph{Id.,} at 124; see also Act of Mar. 3, 1863, \S~10, 12 Stat. 767 (Rev. Stat. \S~1069). The plaintiff, a former Confederate States employee, had asked for equitable tolling on the ground that he had not been able to bring the suit until Congress, in 1868, lifted a previously imposed legal disability.See 107 U. S., at 124--125. But the Court denied the request. \emph{Id.,} at 125--126. It did so not because it thought the equities ran against the plaintiff, but because the statute (with certain listed exceptions) did not permit tolling. Justice Harlan, writing for the Court, said the statute was ``jurisdiction[al],'' that it was not susceptible to judicial ``engraft[ing]'' of unlisted disabilities such as ``sickness, surprise, or inevitable accident,'' and that ``\emph{it [was] the duty of the court to raise the [timeliness] question whether it [was] done by plea or not.}'' \emph{Ibid.} (emphasis added).

  Four years later, in \emph{Finn} v. \emph{United States,} 123 U. S. 227 (1887), the Court found untimely a claim that had originally been filed with a Government agency, but which that agency had then voluntarily referred by statute to the Court of Claims. \emph{Id.,} at 229--230 (citing Act of June 25, 1868, \S~7, 15 Stat. 76); see also Rev. Stat. \S\S~1063--1065. That Government reference, it might have been argued, amounted to a \newpage  waiver by the Government of any limitations-based defense. Cf. \emph{United States} v. \emph{Lippitt,} 100 U.~S. 663, 669 (1880) (reserving the question of the time bar's application in such circumstances). The Court nonetheless held that the long (over 10-year) delay between the time the claim accrued and the plaintiff's filing of the claim before the agency made the suit untimely. \emph{Finn,} 123 U. S., at 232. And as to any argument of Government waiver or abandonment of the time-bar defense, Justice Harlan, again writing for the Court, said that the ordinary legal principle that ``limitation\dots is a defence [that a defendant] must plead\dots \emph{has no application to suits in the Court of Claims against the United States.}''\emph{Id.,} at 232--233 (emphasis added).

  Over the years, the Court has reiterated in various contexts this or similar views about the more absolute nature of the court of claims limitations statute.See \emph{Soriano} v. \emph{United States,} 352 U. S. 270, 273--274 (1957); \emph{United States} v. \emph{Greathouse,} 166 U. S. 601, 602 (1897); \emph{United States} v. \emph{New York,} 160 U.~S. 598, 616--619 (1896); \emph{De Arnaud} v. \emph{United States,} 151 U.~S. 483, 495--496 (1894).

\subsection{B}

  The statute's language has changed slightly since \emph{Kendall} was decided in 1883, but we do not see how any changes in language make a difference here. The only arguably pertinent linguistic change took place during the 1948 recodification of Title 28. See \S~2501, 62 Stat. 976. Prior to 1948, the statute said that ``[e]very claim .~.~. \emph{cognizable by} the Court of Claims, shall be forever barred'' unless filed within six years of the time it first accrues. Rev. Stat. \S~1069 (emphasis added); see also Act of Mar. 3, 1911, \S~156, 36 Stat. 1139 (reenacting the statute without any significant changes). Now, it says that ``[e]very claim of which'' the Court of Federal Claims ``\emph{has jurisdiction} shall be barred'' unless filed within six years of the time it first accrues.28 U.~S.~C. \S~2501 (emphasis added).\newpage 

  This Court does not ``presume'' that the 1948 revision ``worked a change in the underlying substantive law ‘unless an intent to make such a change is clearly expressed.' '' \emph{Keene Corp.} v. \emph{United States,} 508 U.~S. 200, 209 (1993) (quoting \emph{Fourco Glass Co.} v. \emph{Transmirra Products Corp.,} 353 U.~S. 222, 227 (1957); alterations omitted); see also H. R. Rep. No. 308, 80th Cong., 1st Sess., 1--8 (1947) (hereinafter Rep. No. 308) (revision sought to codify, not substantively modify, existing law); Barron, The Judicial Code: 1948 Revision, 8 F. R. D. 439 (1948) (same). We can find no such expression of intent here. The two linguistic forms (``cognizable by''; ``has jurisdiction'') mean about the same thing. See Black's Law Dictionary 991 (4th ed. 1951) (defining ``jurisdiction'' as ``the authority by which courts and judicial officers take \emph{cognizance of} and decide cases'' (emphasis added)); see also Black's Law Dictionary 1038 (3d ed. 1933) (similarly using the term ``cognizance'' to define ``jurisdiction''). Nor have we found any suggestion in the Reviser's Notes or anywhere else that Congress intended to change the prior meaning.See Rep. No. 308, at A192 (Reviser's Note); Barron, \emph{supra,} at 446 (Reviser's Notes specify where change was intended). Thus, it is not surprising that nearly a decade \emph{after} the revision, the Court, citing \emph{Kendall,} again repeated that the statute's limitations period was ``jurisdiction[al]'' and not susceptible to equitable tolling.See \emph{Soriano, supra,} at 273--274, 277.

\section{III}

  In consequence, petitioner can succeed only by convincing us that this Court has overturned, or that it should now overturn, its earlier precedent.

\subsection{A}

  We cannot agree with petitioner that the Court already has overturned the earlier precedent. It is true, as petitioner points out, that in \emph{Irwin} v. \emph{Department of Veterans} \newpage  \emph{Affairs,} 498 U.~S. 89 (1990), we adopted ``a more general rule'' to replace our prior ad hoc approach for determining whether a Government-related statute of limitations is subject to equitable tolling---namely, ``that the same rebuttable presumption of equitable tolling applicable to suits against private defendants should also apply to suits against the United States.'' \emph{Id.,} at 95--96. It is also true that \emph{Irwin,} using that presumption, found equitable tolling applicable to a statute of limitations governing employment discrimination claims against the Government. See \emph{id.,} at 96; see also 42 U.~S.~C. \S~2000e--16(c) (1988 ed.). And the Court noted that this civil rights statute was linguistically similar to the court of claims statute at issue here.See \emph{Irwin, supra,} at 94--95.

  But these few swallows cannot make petitioner's summer. That is because \emph{Irwin} dealt with a different limitations statute. That statute, while similar to the present statute in language, is unlike the present statute in the key respect that the Court had not previously provided a definitive interpretation. Moreover, the Court, while mentioning a case that reflects the particular interpretive history of the court of claims statute, namely, \emph{Soriano, supra,} says nothing at all about overturning that or any other case in that line. See 498 U. S., at 94--95. Courts do not normally overturn a long line of earlier cases without mentioning the matter. Indeed, \emph{Irwin} recognized that it was announcing a general prospective rule, see \emph{id.,} at 95, which does not imply revisiting past precedents.

  Finally, \emph{Irwin} adopted a ``\emph{rebuttable} presumption'' of equitable tolling. \emph{Ibid.} (emphasis added). That presumption seeks to produce a set of statutory interpretations that will more accurately reflect Congress' likely meaning in the mine run of instances where it enacted a Government-related statute of limitations. But the word ``rebuttable'' means that the presumption is not conclusive. Specific statutory lan\newpage guage, for example, could rebut the presumption by demonstrating Congress' intent to the contrary. And if so, a definitive earlier interpretation of the statute, finding a similar congressional intent, should offer a similarly sufficient rebuttal.

  Petitioner adds that in \emph{Franconia Associates} v. \emph{United} \emph{States,} 536 U.~S. 129 (2002), we explicitly considered the court of claims limitations statute, we described the statute as ``unexceptional,'' and we cited \emph{Irwin} for the proposition ``that limitations principles should generally apply to the Government in the same way that they apply to private parties.'' 536 U. S., at 145 (internal quotation marks omitted). But we did all of this in the context of rejecting an argument by the Government that the court of claims statute embodies a special, earlier-than-normal, rule as to when a claim first accrues. \emph{Id.,} at 144--145. The quoted language thus refers only to the statute's claims-accrual rule and adds little or nothing to petitioner's contention that \emph{Irwin} overruled our earlier cases---a contention that we have just rejected.

\subsection{B}

  Petitioner's argument must therefore come down to an invitation now to reject or to overturn \emph{Kendall, Finn, Soriano,} and related cases. In support, petitioner can claim that \emph{Irwin} and \emph{Franconia} represent a turn in the course of the law and can argue essentially as follows: The law now requires courts, when they interpret statutes setting forth limitations periods in respect to actions against the Government, to place greater weight upon the equitable importance of treating the Government like other litigants and less weight upon the special governmental interest in protecting public funds. Cf. \emph{Irwin, supra,} at 95--96. The older interpretations treated these interests differently. Those older cases have consequently become anomalous. The Government is unlikely to have relied significantly upon those earlier cases. Hence the Court should now overrule them.

  \newpage Basic principles of \emph{stare decisis,} however, require us to reject this argument. Any anomaly the old cases and \emph{Irwin} together create is not critical; at most, it reflects a different judicial assumption about the comparative weight Congress would likely have attached to competing legitimate interests. Moreover, the earlier cases lead, at worst, to different interpretations of different, but similarly worded, statutes; they do not produce ``unworkable'' law. See \emph{United States} v. \emph{International Business Machines Corp.,} 517 U.~S. 843, 856 (1996) (internal quotation marks omitted); \emph{California} v. \emph{FERC,} 495 U.~S. 490, 499 (1990). Further, \emph{stare decisis} in respect to statutory interpretation has ``special force,'' for ``Congress remains free to alter what we have done.'' \emph{Patterson} v. \emph{McLean Credit Union,} 491 U.~S. 164, 172--173 (1989); see also \emph{Watson} v. \emph{United States, ante,} at 82--83. Additionally, Congress has long acquiesced in the interpretation we have given.See \emph{ibid.; Shepard} v. \emph{United States,} 544 U.~S. 13, 23 (2005).

  Finally, even if the Government cannot show detrimental reliance on our earlier cases, our reexamination of wellsettled precedent could nevertheless prove harmful. Justice Brandeis once observed that ``in most matters it is more important that the applicable rule of law be settled than that it be settled right.'' \emph{Burnet} v. \emph{Coronado Oil \& Gas Co.,} 285 U.~S. 393, 406 (1932) (dissenting opinion). To overturn a decision settling one such matter simply because we might believe that decision is no longer ``right'' would inevitably reflect a willingness to reconsider others. And that willingness could itself threaten to substitute disruption, confusion, and uncertainty for necessary legal stability. We have not found here any factors that might overcome these considerations.

\section{IV}

  The judgment of the Court of Appeals is affirmed.

\begin{flushright}\emph{It is so ordered.}\end{flushright}
