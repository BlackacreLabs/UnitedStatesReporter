% Concurring in Part and Concurring in the Judgement
% Roberts

\setcounter{page}{257}

  \textsc{Chief Justice Roberts,} with whom \textsc{Justice Kennedy} joins, concurring in part and concurring in the judgment.

  In the decision below, the Fourth Circuit concluded that the loss to LaRue's individual plan account did not permit him to ``serve as a legitimate proxy for the plan in its entirety,'' thus barring him from relief under \S~502(a)(2) of the Employee Retirement Income Security Act of 1974 (ERISA), 29 U.~S.~C. \S~1132(a)(2). 450 F. 3d 570, 574 (2006). The Court today rejects that reasoning. See \emph{ante,} at 252, 255--256. I agree with the Court that the Fourth Circuit's analysis was flawed, and join the Court's opinion to that extent.

  The Court, however, goes on to conclude that \S~502(a)(2) does authorize recovery in cases such as the present one. See \emph{ante,} at 255--256. It is not at all clear that this is true. LaRue's right to direct the investment of his contributions was a right granted and governed by the plan. See \emph{ante,} at 250--251. In this action, he seeks the benefits that would otherwise be due him if, as alleged, the plan carried out his investment instruction. LaRue's claim, therefore, is a claim for benefits that turns on the application and interpretation of the plan terms, specifically those governing investment options and how to exercise them.

  It is at least arguable that a claim of this nature properly lies only under \S~502(a)(1)(B) of ERISA. That provision allows a plan participant or beneficiary ``to recover benefits due to him under the terms of his plan, to enforce his rights under the terms of the plan, or to clarify his rights to future benefits under the terms of the plan.'' 29 U.~S.~C. \S~1132(a)(1)(B). It is difficult to imagine a more accurate description of LaRue's claim. And in fact claimants have filed suit under \S~502(a)(1)(B) alleging similar benefit denials in violation of plan terms.See, \emph{e. g., Hess} v. \emph{Reg-Ellen Machine Tool Corp.,} 423 F. 3d 653, 657 (CA7 2005) (allegation made under \S~502(a)(1)(B) that a plan administrator wrong\newpage fully denied instruction to move retirement funds from employer's stock to a diversified investment account).

  If LaRue may bring his claim under \S~502(a)(1)(B), it is not clear that he may do so under \S~502(a)(2) as well. Section 502(a)(2) provides for ``appropriate'' relief. Construing the same term in a parallel ERISA provision, we have held that relief is not ``appropriate'' under \S~502(a)(3) if another provision, such as \S~502(a)(1)(B), offers an adequate remedy. See \emph{Varity Corp.} v. \emph{Howe,} 516 U.~S. 489, 515 (1996). Applying the same rationale to an interpretation of ``appropriate'' in \S~502(a)(2) would accord with our usual preference for construing the ``same terms [to] have the same meaning in different sections of the same statute,'' \emph{Barnhill} v. \emph{Johnson,} 503 U.~S. 393, 406 (1992), and with the view that ERISA in particular is a `` ‘comprehensive and reticulated statute' '' with ``carefully integrated civil enforcement provisions,'' \emph{Massachusetts Mut. Life Ins. Co.} v. \emph{Russell,} 473 U.~S. 134, 146 (1985) (quoting \emph{Nachman Corp.} v. \emph{Pension Benefit Guaranty Corporation,} 446 U.~S. 359, 361 (1980)). In a variety of contexts, some Courts of Appeals have accordingly prevented plaintiffs from recasting what are in essence planderived benefit claims that should be brought under \S~502(a)(1)(B) as claims for fiduciary breaches under \S~502(a)(2). See, \emph{e. g., Coyne \& Delany Co.} v. \emph{Blue Cross \& Blue Shield of Va., Inc.,} 102 F. 3d 712, 714 (CA4 1996). Other Courts of Appeals have disagreed with this approach.See, \emph{e. g., Graden} v. \emph{Conexant Systems Inc.,} 496 F. 3d 291, 301 (CA3 2007).

  The significance of the distinction between a \S~502(a)(1)(B) claim and one under \S~502(a)(2) is not merely a matter of picking the right provision to cite in the complaint. Allowing a \S~502(a)(1)(B) action to be recast as one under \S~502(a)(2) might permit plaintiffs to circumvent safeguards for plan administrators that have developed under \S~502(a)(1)(B). Among these safeguards is the requirement, recognized by almost all the Courts of Appeals, see \emph{Fallick} v. \emph{Nationwide\newpage Mut. Ins. Co.,} 162 F. 3d 410, 418, n. 4 (CA6 1998) (citing cases), that a participant exhaust the administrative remedies mandated by ERISA \S~503, 29 U.~S.~C. \S~1133, before filing suit under \S~502(a)(1)(B).[[*]] Equally significant, this Court has held that ERISA plans may grant administrators and fiduciaries discretion in determining benefit eligibility and the meaning of plan terms, decisions that courts may review only for an abuse of discretion.\emph{Firestone Tire \& Rubber Co.} v. \emph{Bruch,} 489 U.~S. 101, 115 (1989).

  These safeguards encourage employers and others to undertake the voluntary step of providing medical and retirement benefits to plan participants, see \emph{Aetna Health Inc.} v. \emph{Davila,} 542 U. S. 200, 215 (2004), and have no doubt engendered substantial reliance interests on the part of plans and fiduciaries. Allowing what is really a claim for benefits under a plan to be brought as a claim for breach of fiduciary duty under \S~502(a)(2), rather than as a claim for benefits due ``under the terms of the plan,'' \S~502(a)(1)(B), may result in circumventing such plan terms.

  I do not mean to suggest that these are settled questions. They are not. Nor are we in a position to answer them. LaRue did not rely on \S~502(a)(1)(B) as a source of relief, and the courts below had no occasion to address the argument, raised by an \emph{amicus} in this Court, that the availability of relief under \S~502(a)(1)(B) precludes LaRue's fiduciary breach claim. See Brief for ERISA Industry Committee as \emph{Amicus Curiae} 13--30. I simply highlight the fact that the Court's determination that the present claim may be brought under \S~502(a)(2) is reached without considering whether the possible availability of relief under \S~502(a)(1)(B) alters that conclusion. See, \emph{e. g., United Parcel Service, Inc.} v. \emph{Mitchell,} 451 U.~S. 56, 60, n. 2 (1981) (noting general reluctance to consider arguments raised only by an \emph{amicus} and not consid\newpage ered by the courts below). In matters of statutory interpretation, where principles of \emph{stare decisis} have their greatest effect, it is important that we not seem to decide more than we do. I see nothing in today's opinion precluding the lower courts on remand, if they determine that the argument is properly before them, from considering the contention that LaRue's claim may proceed only under \S~502(a)(1)(B). In any event, other courts in other cases remain free to consider what we have not---what effect the availability of relief under \S~502(a)(1)(B) may have on a plan participant's ability to proceed under \S~502(a)(2).

\footnotetext[*]{Sensibly, the Court leaves open the question whether exhaustion may be required of a claimant who seeks recovery for a breach of fiduciary duty under \S~502(a)(2).See \emph{ante,} at 253, n. 3.}
