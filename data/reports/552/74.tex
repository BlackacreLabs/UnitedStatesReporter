% Court
% Souter

\setcounter{page}{76}

  \textsc{Justice Souter} delivered the opinion of the Court.

  The question is whether a person who trades his drugs for a gun
``uses'' a firearm ``during and in relation to\dots [a]
drug trafficking crime'' within the meaning of 18 U.~S.~C.
\S~924(c)(1)(A).\footnotemark[1] We hold that he does not.

\section{I}

\subsection{A}

  Section 924(c)(1)(A) sets a mandatory minimum sentence, depending
on the facts, for a defendant who, ``during and in relation to any
crime of violence or drug trafficking crime[,]\dots uses or carries
a firearm.''\footnotemark[2] The statute leaves the term ``uses'' undefined,
though we have spoken to it twice before.

  \emph{Smith} v. \emph{United States,} 508 U.~S. 223 (1993), raised the
converse of today's question, and held that ``a criminal who trades
his firearm for drugs ‘uses' it during and in relation to a drug
trafficking offense within the meaning of \S~924(c)(1).'' \emph{Id.,}
at 241. We rested primarily on the ``ordinary or natural meaning''
of the verb in context, \emph{id.,} at 228, and understood its common
range as going beyond employment as a weapon: ``it is both reasonable
and normal to say that petitioner ‘used' his MAC--10 in his drug
trafficking offense by trading it for cocaine,'' \emph{id.,} at 230.

  Two years later, the issue in \emph{Bailey} v. \emph{United States,} 516
U. S. 137 (1995), was whether possessing a firearm kept near the scene
of drug trafficking is ``use'' under \S~924(c)(1). We looked again to
``ordinary or natural'' meaning, \emph{id.,} at 145, and decided that
mere possession does not amount to ``use'': ``\S~924(c)(1) requires
evidence sufficient to show an \emph{active employment} of the firearm by
the defendant, a use that makes \newpage  the firearm an operative factor
in relation to the predicate offense,'' \emph{id.,} at 143.\footnotemark[3]

^1 Formerly 18 U.~S.~C. \S~924(c)(1) (1994 ed.).

^2 Any violation of \S~924(c)(1)(A), for example, demands a
mandatory minimum sentence of five years.See 18 U.~S.~C.
\S~924(c)(1)(A)(i). If the firearm is brandished, the minimum goes up
to 7 years, see \S~924(c)(1)(A)(ii); if the firearm is discharged,
the minimum jumps to 10 years, see \S~924(c)(1)(A)(iii).

\subsection{B}

  This third case on the reach of \S~924(c)(1)(A) began to take shape
when petitioner, Michael A. Watson, told a Government informant that he
wanted to acquire a gun. On the matter of price, the informant quoted no
dollar figure but suggested that Watson could pay in narcotics. Next,
Watson met with the informant and an undercover law enforcement agent
posing as a firearms dealer, to whom he gave 24 doses of oxycodone
hydrocholoride (commonly, OxyContin) for a .50-caliber semiautomatic
pistol. When law enforcement officers arrested Watson, they found the
pistol in his car, and a later search of his house turned up a cache of
prescription medicines, guns, and ammunition. Watson said he got the
pistol ``to protect his other firearms and drugs.'' App. C to Pet.
for Cert. 11a.

  A federal grand jury indicted him for distributing a Schedule II
controlled substance and for ``using'' the pistol during and in
relation to that crime, in violation of \S~924(c)(1)(A).\footnotemark[4] Watson
pleaded guilty across the board, reserving the right to challenge
the factual basis for a \S~924(c)(1)(A) conviction and the added
consecutive sentence of 60 months for using the gun. The Court of
Appeals affirmed,191 Fed. Appx. 326 (CA5 2006) \emph{(per curiam),}
on Circuit precedent foreclosing any argument that Watson had not
``used'' a firearm, see \emph{id.,} at 327 (citing \emph{United States} v.
\emph{Ulloa,} 94 F. 3d 949 (CA5 1996), and \emph{United States} v. \emph{Zuniga,}
18 F. 3d 1254 (CA5 1994)). \newpage 


  We granted certiorari to resolve a conflict among the Circuits
on whether a person ``uses'' a firearm within the meaning of 18
U.~S.~C. \S~924(c)(1)(A) when he trades narcotics to obtain a
gun.\footnotemark[5] 549 U.~S. 1251 (2007). We now reverse.

^3 In 1998, Congress responded to \emph{Bailey} by amending \S~924(c)(1).
The amendment broadened the provision to cover a defendant who, ``in
furtherance of any [crime of violence or drug trafficking] crime,
possesses a firearm.'' 18 U.~S.~C. \S~924(c)(1)(A). The
amendment did not touch the ``use'' prong of \S~924(c)(1).

^4 The grand jury also indicted Watson as a felon in possession of a
firearm, in violation of \S~922(g)(1). This count referred to the five
firearms found in Watson's house, but not the pistol he got for the
narcotics.

\section{II}

\subsection{A}

  The Government's position that Watson ``used'' the pistol under
\S~924(c)(1)(A) by receiving it for narcotics lacks authority in either
precedent or regular English. To begin with, neither \emph{Smith} nor
\emph{Bailey} implicitly decides this case. While \emph{Smith} held that
firearms may be ``used'' in a barter transaction, even with no violent
employment, see 508 U. S., at 241, the case addressed only the
trader who swaps his gun for drugs, not the trading partner who ends
up with the gun. \emph{Bailey,} too, is unhelpful, with its rule that a
gun must be made use of actively to satisfy \S~924(c)(1)(A), as ``an
operative factor in relation to the predicate offense.'' 516 U. S.,
at 143. The question here is whether it makes sense to say that Watson
employed the gun at all; \emph{Bailey} does not answer it. \newpage 

^5 Compare \emph{United States} v. \emph{Cotto,} 456 F. 3d 25 (CA1 2006)
(trading drugs for a firearm constitutes ``use'' of the firearm under
\S~924(c)(1)(A)); \emph{United States} v. \emph{Sumler,} 294 F. 3d 579 (CA3
2002) (same); \emph{United States} v. \emph{Ramirez-Rangel,} 103 F. 3d 1501
(CA9 1997) (same); \emph{United States} v. \emph{Ulloa,} 94 F. 3d 949 (CA5
1996) (same); \emph{United States} v. \emph{Cannon,} 88 F. 3d 1495 (CA8 1996)
(same), with \emph{United States} v. \emph{Montano,} 398 F. 3d 1276 (CA11
2005) \emph{(per curiam)} (defendant did not ``use'' a firearm within
the meaning of \S~924(c)(1)(A) when he traded drugs for a firearm);
\emph{United States} v. \emph{Stewart,} 246 F. 3d 728 (CADC 2001) (same);
\emph{United States} v. \emph{Warwick,} 167 F. 3d 965 (CA6 1999) (same);
\emph{United States} v. \emph{Westmoreland,} 122 F. 3d 431 (CA7 1997) (same).
The Fourth Circuit has held that a defendant ``used'' a firearm where
he gave cocaine base to a compatriot in exchange for assistance in
obtaining a gun. See \emph{United States} v. \emph{Harris,} 39 F. 3d 1262
(1994). Subsequent unpublished opinions in that Circuit have relied on
\emph{Harris} for the proposition that the receipt of a firearm in exchange
for drugs constitutes use of the firearm. See, \emph{e. g., United States}
v. \emph{Belcher,} No. 98--4845, 1999 WL 1080103 (Nov. 29, 1999) \emph{(per
curiam).}


  With no statutory definition or definitive clue, the meaning of the
verb ``uses'' has to turn on the language as we normally speak it,
see, \emph{e. g., Lopez} v. \emph{Gonzales,} 549 U.~S. 47, 53 (2006);
\emph{Asgrow Seed Co.} v. \emph{Winterboer,} 513 U.~S. 179, 187 (1995);
\emph{FDIC} v. \emph{Meyer,} 510 U.~S. 471, 476 (1994); there is no other
source of a reasonable inference about what Congress understood when
writing or what its words will bring to the mind of a careful reader.
So, in \emph{Smith} we looked for ``everyday meaning,'' 508 U. S.,
at 228, revealed in phraseology that strikes the ear as ``both
reasonable and normal,'' \emph{id.,} at 230. See also \emph{Bailey, supra,}
at 145. This appeal to the ordinary leaves the Government without much
of a case.

  The Government may say that a person ``uses'' a firearm simply by
receiving it in a barter transaction, but no one else would. A boy
who trades an apple to get a granola bar is sensibly said to use the
apple, but one would never guess which way this commerce actually flowed
from hearing that the boy used the granola.Cf. \emph{United States} v.
\emph{Stewart,} 246 F. 3d 728, 731 (CADC 2001) (``[W]hen a person pays
a cashier a dollar for a cup of coffee in the courthouse cafeteria,
the customer has not used the coffee. He has only used the dollar
bill''). So, when Watson handed over the drugs for the pistol, the
informant or the agent\footnotemark[6] ``used'' the pistol to get the drugs, just
as \emph{Smith} held, but regular speech would not say that Watson himself
used the pistol in the trade. ``A seller does not ‘use' a buyer's
consideration,'' \emph{United States} v. \emph{Westmoreland,} 122 F. 3d
431, 436 (CA7 1997), and the Government's contrary position recalls
another case; \emph{Lopez, supra,} at 56, rejected the Government's
interpretation of 18 U.~S.~C. \S~924(c)(2) because ``we do not
normally speak or write the Government's way.''\footnotemark[7] \newpage 

^6 The record does not say which.

^7 Dictionaries confirm the conclusion. ``Use'' is concededly
``elastic,'' \emph{Smith} v. \emph{United States,} 508 U.~S. 223,
241 (1993) (\textsc{Scalia,} J., dissenting), but none of its standard
definitions stretch far enough to reach Watson's \newpage  conduct,
see, \emph{e. g.,} Webster's New International Dictionary of the
English Language 2806 (2d ed. 1939) (``to employ''); The Random House
Dictionary of the English Language 2097 (2d ed. 1987) (to ``apply to
one's own purposes''; ``put into service; make use of''); Black's
Law Dictionary 1541 (6th ed. 1990) (``[t]o avail oneself of;\dots to
utilize''); see also \emph{Smith, supra,} at 228--229 (listing various
dictionary definitions).

\subsection{B}

  The Government would trump ordinary English with two arguments. First,
it relies on \emph{Smith} for the pertinence of a neighboring provision,
18 U.~S.~C. \S~924(d)(1), which authorizes seizure and forfeiture
of firearms ``intended to be used in'' certain criminal offenses
listed in \S~924(d)(3). Some of those offenses involve receipt of a
firearm,\footnotemark[8] from which the Government infers that ``use'' under
\S~924(d) necessarily includes receipt of a gun even in a barter
transaction. \emph{Smith} is cited for the proposition that the term must
be given the same meaning in both subsections, and the Government urges
us to import ``use'' as ``receipt in barter'' into \S~924(c)(1)(A).

  We agree with the Government that \S~924(d) calls for attention; the
reference to intended use in a receipt crime carries some suggestion
that receipt can be ``use'' (more of a hint, say, than speaking of
intended ``use'' in a crime defined as exchange). But the suggestion
is a tepid one and falls short of supporting what is really an attempt
to draw a conclusion too specific from a premise too general.

  The \emph{Smith} majority rested principally on ordinary speech in
reasoning that \S~924(c)(1) extends beyond use as a weapon and
includes use as an item of barter, see 508 U. S., at 228--230,
and the \emph{Smith} opinion looks to \S~924(d) only for its light
on that conclusion. It notes that the ``intended to be used''
clause of \S924(d)(1) refers to offenses where ``the firearm is
\newpage  \emph{not} used as a weapon but instead as an item of barter or
commerce,'' \emph{id.,} at 234, with the implication that Congress
intended ``use'' to reach commercial transactions, not just gun
violence, in \S~924(d) generally, see \emph{id.,} at 234--235. It
was this breadth of treatment that led the \emph{Smith} majority to say
that, ``[u]nless we are to hold that using a firearm has a different
meaning in \S~924(c)(1) than it does in \S~924(d)---and clearly we
should not---we must reject petitioner's narrow interpretation.''
\emph{Id.,} at 235 (citation omitted); see also \emph{Bailey, supra,}
at 146 (``[U]sing a firearm should not have a different meaning in
\S~924(c)(1) than it does in \S~924(d)'' (internal quotation marks
omitted)).

^8 See, \emph{e. g.,} 18 U.~S.~C. \S922(j) (prohibiting, \emph{inter
alia,} the receipt of a stolen firearm in interstate commerce);
\S~924(b) (prohibiting, \emph{inter alia,} the receipt of a firearm in
interstate commerce with the intent to commit a felony).

  The Government overreads \emph{Smith.} While the neighboring provision
indicates that a firearm is ``used'' nonoffensively, and supports the
conclusion that a gun can be ``used'' in barter, beyond that point
its illumination fails. This is so because the utility of \S~924(d)(1)
is limited by its generality and its passive voice; it tells us a gun
can be ``used'' in a receipt crime, but not whether both parties to a
transfer use the gun, or only one, or which one. The nearby subsection
(c)(1)(A), however, requires just such a specific identification. It
provides that a person who uses a gun in the circumstances described
commits a crime, whose perpetrator must be clearly identifiable in
advance.

  The agnosticism on the part of \S~924(d)(1) about who does the using
is entirely consistent with common speech's understanding that the
first possessor is the one who ``uses'' the gun in the trade, and
there is thus no cause to admonish us to adhere to the paradigm of a
statute ``as a symmetrical and coherent regulatory scheme, .~.~.
in which the operative words have a consistent meaning throughout,''
\emph{Gustafson} v. \emph{Alloyd Co.,} 513 U.~S. 561, 569 (1995), or to
invoke the ``standard principle of statutory construction\dots that
identical words and phrases within the same statute should normally
be given the same meaning,'' \emph{Powerex Corp.} v. \emph{Reliant Energy
Services, Inc.,} 551 U.~S. 224, 232 (2007). Subsections \newpage 
(d)(1) and (c)(1)(A) as we read them are not at odds over the verb
``use''; the point is merely that in the two subsections the common
verb speaks to different issues in different voices and at different
levels of specificity. The provisions do distinct jobs, but we do not
make them guilty of employing the common verb inconsistently.\footnotemark[9]

\subsection{C}

  The second effort to trump regular English is the claim that failing
to treat receipt in trade as ``use'' would create unacceptable
asymmetry with \emph{Smith.} At bottom, this atextual policy critique says
it would be strange to penalize one side of a gun-for-drugs exchange but
not the other: ``[t]he danger to society is created not only by the
person who brings the firearm to the drug transaction, but also by the
drug dealer who takes the weapon in exchange for his drugs during the
transaction,'' Brief for United States 23.

  The position assumes that \emph{Smith} must be respected, and we join
the Government at least on this starting point. A difference of opinion
within the Court (as in \emph{Smith}) does not keep the door open for
another try at statutory construction, where \emph{stare decisis} has
``special force [since] the legislative power is implicated, and
Congress remains free to alter what we have done.'' \emph{Patterson} v.
\emph{McLean Credit Union,} 491 U.~S. 164, 172--173 (1989). What is
more, in 14 years Congress has taken no step to modify \emph{Smith}'s
holding, and this long congressional acquiescence ``has enhanced even
the \newpage  usual precedential force'' we accord to our interpretations
of statutes, \emph{Shepard} v. \emph{United States,} 544 U.~S. 13, 23
(2005).

^9 For that matter, the Government's argument that ``use'' must
always have an identical meaning in \S\S~924(c)(1)(A) and 924(d)(1)
would upend \emph{Bailey} v. \emph{United States,} 516 U.~S. 137 (1995).
One of the relevant predicate offenses referred to by \S~924(d)(1)
is possession of ``any stolen firearm\dots [in] interstate or
foreign commerce.'' 18 U.~S.~C. \S922(j). If we were to hold that
all criminal conduct covered by the ``intended to be used'' clause in
\S~924(d)(1) is ``use'' for purposes of \S~924(c)(1)(A), it would
follow that mere possession is use. But that would squarely conflict
with our considered and unanimous decision in \emph{Bailey} that ``
‘use' must connote more than mere possession of a firearm.'' 516
U. S., at 143.

  The problem, then, is not with the sturdiness of \emph{Smith} but with
the limited malleability of the language \emph{Smith} construed, and
policy-driven symmetry cannot turn ``receipt-intrade'' into ``use.''
Whatever the tension between the prior result and the outcome here, law
depends on respect for language and would be served better by statutory
amendment (if Congress sees asymmetry) than by racking statutory
language to cover a policy it fails to reach.

  The argument is a peculiar one, in fact, given the Government's
take on the current state of \S~924(c)(1)(A). It was amended after
\emph{Bailey} and now prohibits not only using a firearm during and in
relation to a drug trafficking crime, but also possessing one ``in
furtherance of'' such a crime. 18 U.~S.~C. \S~924(c)(1)(A); see n.
3, \emph{supra.} The Government is confident that ``a drug dealer who
takes a firearm in exchange for his drugs generally will be subject to
prosecution'' under this new possession prong.Brief for United States
27; see Tr. of Oral Arg. 41 (Watson's case ``could have been charged
as possession''); cf. \emph{United States} v. \emph{Cox,} 324 F. 3d 77,
83, n. 2 (CA2 2003) (``For defendants charged under \S~924(c) after
[the post-\emph{Bailey}] amendment, trading drugs for a gun will probably
result in\dots possession [in furtherance of a drug trafficking
crime]''). This view may or may not prevail, and we do not speak to
it today, but it does leave the appeal to symmetry underwhelming in a
contest with the English language, on the Government's very terms.

\hrule

  Given ordinary meaning and the conventions of English, we hold that
a person does not ``use'' a firearm under \S~924(c)(1)(A) when he
receives it in trade for drugs. The judgment of the Court of Appeals is
reversed, and the case is remanded for further proceedings consistent
with this opinion.

\begin{flushright}\emph{It is so ordered.}\end{flushright}
