% Court
% Souter

\setcounter{page}{424}

  \textsc{Justice Souter} delivered the opinion of the Court.

  Sections 301 and 316(a) of the Internal Revenue Code set the
conditions for treating certain corporate distributions as returns of
capital, nontaxable to the recipient. 26 U.~S.~C. \S\S~301, 316(a)
(2000 ed. and Supp. V). The question here is whether a distributee
accused of criminal tax evasion may claim return-of-capital treatment
without producing evidence that either he or the corporation intended
a capital return when the distribution occurred. We hold that no such
showing is required.

\section{I}

  ``[T]he capstone of [the] system of sanctions\dots calculated to
induce\dots fulfillment of every duty under the income tax law,''
\emph{Spies} v. \emph{United States,} 317 U.~S. 492, 497 (1943), is 26
U.~S.~C. \S~7201, making it a felony willfully to ``attemp[t]
in any manner to evade or defeat any tax imposed by'' the Code.\footnotemark[1]
One element of tax evasion under \S7201 is ``the existence of a
tax deficiency,'' \emph{Sansone} v. \emph{United States,} 380 U. S.
343, 351 (1965); see also \emph{Lawn} v. \emph{United States,} 355 U. S.
339, 361 (1958),\footnotemark[2] which the Government must prove beyond a
reasonable doubt, see \emph{ibid.} (``[O]f course, a conviction upon
a charge of attempting to evade assessment of income taxes by the
filing of a fraudulent return cannot stand in the absence of proof of a
deficiency'').

  Any deficiency determination in this case will turn on \S\S~301 and
316(a) of the Code. According to \S~301(a), unless another provision of
the Code requires otherwise, a ``distri\newpage  bution of property''
that is ``made by a corporation to a shareholder with respect to its
stock shall be treated in the manner provided in [\S~301(c)].'' Under
\S~301(c), the portion of the distribution that is a ``dividend,''
as defined by \S~316(a), must be included in the recipient's gross
income; and the portion that is not a dividend is, depending on the
shareholder's basis for his stock, either a nontaxable return of
capital or a gain on the sale or exchange of stock, ordinarily taxable
to the shareholder as a capital gain. Finally, \S~316(a) defines
``dividend'' as

    ``any distribution of property made by a corporation to its
    shareholders---

      ``(1) out of its earnings and profits accumulated afte February r
    28, 1913, o                                                        r

      ``(2) out of its earnings and profits of the taxable year
    (computed as of the close of the taxable year without diminution by
    reason of any distributions made during the taxable year), without
    regard to the amount of the earnings and profits at the time the
    distribution was made.''

\noindent Sections 301 and 316(a) together thus make the existence of ``earnings
and profits''\footnotemark[3] the decisive fact in determining the tax
consequences of distributions from a corporation to a shareholder with
respect to his stock. This requirement of ``relating the tax status of
corporate distributions to earnings and profits is responsive to a felt
need for protecting returns of capital from tax.'' 4 Bittker \& Lokken
¶ 92.1.1, at 92--3.

^1 A related provision, 26 U.~S.~C. \S~7206(1), criminalizes the
willful filing of a tax return believed to be materially false. See n.
9, \emph{infra.}

^2 ``[T]he elements of \S~7201 are willfulness[,] the existence
of a tax deficiency,\dots and an affirmative act constituting an
evasion or attempted evasion of the tax.'' \emph{Sansone} v. \emph{United
States,} 380 U.~S. 343, 351 (1965). The Courts of Appeals have
divided over whether the Government must prove the tax deficiency is
``substantial,'' see \emph{United States} v. \emph{Daniels,} 387 F. 3d
636, 640--641, and n. 2 (CA7 2004) (collecting cases); we do not
address that issue here.

\section{II}

  In this criminal tax proceeding, petitioner Michael Boulware was
charged with several counts of tax evasion and \newpage  filing a false
income tax return, stemming from his diversion of funds from Hawaiian
Isles Enterprises (HIE), a closely held corporation of which he was
the president, founder, and controlling (though not sole) shareholder.
At trial,\footnotemark[4] the United States sought to establish that Boulware had
received taxable income by ``systematically divert[ing] funds from HIE
in order to support a lavish lifestyle.'' 384 F. 3d 794, 799 (CA9
2004). The Government's evidence showed that

^3 Although the Code does not ``comprehensively define ‘earnings and
profits,' '' 4 B. Bittker \& L. Lokken, Federal Taxation of Income,
Estates and Gifts ¶ 92.1.3, p. 92--6 (3d ed. 2003) (hereinafter
Bittker \& Lokken), the ``[p]rovisions of the Code and regulations
relating to earnings and profits ordinarily take taxable income as the
point of departure,'' \emph{id.,} at 92--9.


    ``[Boulware] gave millions of dollars of HIE money to his
    girlfriend\dots and millions of dollars to his wife .~.~.
    without reporting any of this money on his personal income tax
    returns~.~.~.~. [H]e siphoned off this money primarily by
    writing checks to employees and friends and having them return the
    cash to him, by diverting payments by HIE customers, by submitting
    fraudulent invoices to HIE, and by laundering HIE money through
    companies in the Kingdom of Tonga and Hong Kong.'' \emph{Ibid.}

\noindent In defense, Boulware sought to introduce evidence that HIE had no
retained or current earnings and profits in the relevant taxable
years, with the consequence (he argued) that he in effect received
distributions of property that must have been returns of capital, up to
his basis in his stock. See \S~301(c)(2). Because the return of capital
was nontaxable, the argument went, the Government could not establish
the tax deficiency required to convict him. \newpage 

^4 The trial at issue in this case was actually Boulware's second
trial on \S\S~7201 and 7206(1) charges, his convictions on those counts
in an earlier trial having been vacated by the Ninth Circuit for reasons
not at issue here, see 384 F. 3d 794 (2004). In that earlier trial,
Boulware was also convicted of conspiracy to make false statements to a
federally insured financial institution, in violation of 18 U.~S.~C.
\S~371. The Ninth Circuit affirmed Boulware's conspiracy conviction
that first time around, however, so the present trial did not include a
conspiracy charge.

  The Government moved \emph{in limine} to bar evidence in support
of Boulware's return-of-capital theory, on the grounds of
``irrelevan[ce] in [this] criminal tax case,'' App. 20. The Government
relied on the Ninth Circuit's decision in \emph{United States} v.
\emph{Miller,} 545 F. 2d 1204 (1976), in which that court held that in a
criminal tax evasion case, a diversion of funds may be deemed a return
of capital only after ``some demonstration on the part of the taxpayer
and/or the corporation that such [a distribution was] intended to be
such a return,'' \emph{id.,} at 1215. Boulware, the Government argued,
had offered to make no such demonstration. App. 21.

  The District Court granted the Government's motion, and when
Boulware sought ``to present evidence of [HIE's] alleged
over-reporting of income, and an offer of proof relating to the issue
of\dots dividends,'' \emph{id.,} at 135, the District Court
denied his request. The court said that ``[n]ot only would much
of [his proffered] evidence be excludable as expert legal opinion,
it is plainly insufficient under the Miller case,'' \emph{id.,} at
138, and accordingly declined to instruct the jury on Boulware's
return-of-capital theory. The jury rejected his alternative defenses
(that the diverted funds were nontaxable corporate advances or loans, or
that he used the moneys for corporate purposes), and found him guilty on
nine counts, four of tax evasion and five of filing a false return.

  The Ninth Circuit affirmed. 470 F. 3d 931 (2006). It acknowledged
that ``imposing an intent requirement creates a disconnect between
civil and criminal liability,'' but thought that under \emph{Miller,}
``the characterization of diverted corporate funds for civil tax
purposes does not dictate their characterization for purposes of a
criminal tax evasion charge.'' 470 F. 3d, at 934. The court held
the test in a criminal case to be ``whether the defendant has willfully
attempted to evade the payment or assessment of a tax.'' \emph{Ibid.}
Because Boulware `` ‘presented no concrete proof that the amounts
were considered, intended, or recorded on the corporate rec\newpage ords
as a return of capital at the time they were made,' '' \emph{id.,} at
935 (quoting \emph{Miller, supra,} at 1215), the Ninth Circuit held that
Boulware's proffer was ``properly rejected\dots asinadequate,''
470 F. 3d, at 935.

  Judge Thomas concurred because the panel was bound by \emph{Miller,}
but noted that ``\emph{Miller}---and now the majority opinion---hold
that a defendant may be criminally sanctioned for tax evasion without
owing a penny in taxes to the government.'' 470 F. 3d, at 938.
That, he said, not only ``indicate[s] a logical fallacy, but is
in flat contradiction with the tax evasion statute's requirement
.~.~. of a tax deficiency.'' \emph{Ibid.} (internal quotation marks
omitted).\footnotemark[5]

  We granted certiorari, 551 U.~S. 1191 (2007), to resolve a split
among the Courts of Appeals over the application of \S\S~301 and 316(a)
to informally transferred or diverted corporate funds in criminal tax
proceedings.\footnotemark[6] We now vacate and remand.

^5 Judge Thomas went on to say that the Government would prevail
even without \emph{Miller}'s rule because, in his view, Boulware's
diversions were ``unlawful,'' and the return-of-capital rules would
not apply to diversions made for unlawful purposes. See 470 F. 3d, at
938--939.

^6 As noted, the Ninth Circuit holds that \S\S~301 and 316(a) are not
to be consulted in a criminal tax evasion case until the defendant
produces evidence of an intent to treat diverted funds as a return
of capital at the time it was made. See 470 F. 3d 931 (2006) (case
below). By contrast, the Second Circuit allows a criminal defendant to
invoke \S\S~301 and 316(a) without evidence of a contemporaneous intent
to treat such moneys as returns of capital. See \emph{United States}
v. \emph{Bok,} 156 F. 3d 157, 162 (1998) (``[I]n return of capital
cases, a taxpayer's intent is not determinative in defining the
taxpayer's conduct''). Meanwhile, the Third, Sixth, and Eleventh
Circuits arguably have taken the position that \S\S~301 and 316(a)
are altogether inapplicable in criminal tax cases involving informal
distributions. See \emph{United States} v. \emph{Williams,} 875 F. 2d 846,
850--852 (CA11 1989); \emph{United States} v. \emph{Goldberg,} 330 F. 2d 30,
38 (CA3 1964); \emph{Davis} v. \emph{United States,} 226 F. 2d 331, 334--335
(CA6 1955); but see Brief for Petitioner 16 (``[T]hese cases can be
read to address the allocation of the burden of proof on the return of
capital issue, rather than the applicable substantive principles'').
\newpage 

\section{III}

\subsection{A}

  The colorful behavior described in the allegations requires a
reminder that tax classifications like ``dividend'' and ``return of
capital'' turn on ``the objective economic realities of a transaction
rather than\dots the particular form the parties employed,''
\emph{Frank Lyon Co.} v. \emph{United States,} 435 U.~S. 561, 573 (1978);
a ``given result at the end of a straight path is not made a different
result\dots byfollowingadevious path,'' \emph{Minnesota Tea Co.} v.
\emph{Helvering,} 302 U.~S. 609, 613 (1938).\footnotemark[7] As for distributions
with respect to stock, in economic reality a shareholder's informal
receipt of corporate property ``may be as effective a means of
distributing profits among stockholders as the formal declaration
of a dividend,'' \emph{Palmer} v. \emph{Commissioner,} 302 U.~S. 63,
69 (1937), or as effective a means of returning a shareholder's
capital, see \emph{ibid.} Accordingly, ``[a] distribution to a shareholder
in his capacity as such\dots \newpage  is subject to \S~301 even
though it is not declared in formal fashion.'' B. Bittker \& J.
Eustice, Federal Income Taxation of Corporations and Shareholders ¶
8.05[1], pp. 8--36 to 8--37 (6th ed. 1999) (hereinafter Bittker
\& Eustice); see also Gardner, The Tax Consequences of Shareholder
Diversions in Close Corporations, 21 Tax L. Rev. 223, 239 (1966)
(hereinafter Gardner) (``Sections 316 and 301 do not require any formal
path to be taken by a corporation in order for those provisions to
apply'').

^7 We have also recognized that ``[t]he legal right of a taxpayer to
decrease the amount of what otherwise would be his taxes, or altogether
avoid them, by means which the law permits, cannot be doubted.''
\emph{Gregory} v. \emph{Helvering,} 293 U.~S. 465, 469 (1935). The rule
is a two-way street: ``while a taxpayer is free to organize his affairs
as he chooses, nevertheless, once having done so, he must accept the
tax consequences of his choice, whether contemplated or not,~.~.~.
and may not enjoy the benefit of some other route he might have chosen
to follow but did not,'' \emph{Commissioner} v. \emph{National Alfalfa
Dehydrating \& Milling Co.,} 417 U.~S. 134, 149 (1974); see also
\emph{id.,} at 148 (referring to ``the established tax principle that a
transaction is to be given its tax effect in accord with what actually
occurred and not in accord with what might have occurred''); \emph{Founders
Gen. Corp.} v. \emph{Hoey,} 300 U.~S. 268, 275 (1937) (``To make the
taxability of the transaction depend upon the determination whether
there existed an alternative form which the statute did not tax would
create burden and uncertainty''). The question here, of course, is
not whether alternative routes may have offered better or worse tax
consequences, see generally Isenbergh, Review: Musings on Form and
Substance in Taxation, 49 U. Chi. L. Rev. 859 (1982); rather, it is
``whether what was done\dots was the thing which the statute[, here
\S\S~301 and 316(a),] intended,'' \emph{Gregory, supra,} at 469.


  There is no reason to doubt that economic substance remains the
right touchstone for characterizing funds received when a shareholder
diverts them before they can be recorded on the corporation's books.
While they ``never even pass through the corporation's hands,''
Bittker \& Eustice ¶ 8.05[9], at 8--51, even diverted funds may be
seen as dividends or capital distributions for purposes of \S\S~301
and 316(a), see \emph{Truesdell} v. \emph{Commissioner,} 89 T. C. 1280
(1987) (treating diverted funds as ``constructive'' distributions
in civil tax proceedings). The point, again, is that ``taxation
is not so much concerned with the refinements of title as it is with
actual command over the property taxed---the actual benefit for which
the tax is paid.'' \emph{Corliss} v. \emph{Bowers,} 281 U.~S. 376, 378
(1930); see also \emph{Griffiths} v. \emph{Commissioner,} 308 U.~S. 355, 358
(1939).\footnotemark[8]

\section{B}

  \emph{Miller}'s view that a criminal defendant may not treat
a distribution as a return of capital without evidence of a
cor\newpage responding contemporaneous intent sits uncomfortably not only
with the tax law's economic realism, but with the particular wording
of \S\S~301 and 316(a), as well. As those sections are written, the
tax consequences of a ``distribution by a corporation with respect to
its stock'' depend, not on anyone's purpose to return capital or to
get it back, but on facts wholly independent of intent: whether the
corporation had earnings and profits, and the amount of the taxpayer's
basis for his stock. Cf. \emph{Truesdell} v. \emph{Commissioner,} Internal
Revenue Service (IRS) Action on Decision 1988--25, 1988 WL 570761
(Sept. 12, 1988) (recommendation regarding acquiescence), IRS Non
Docketed Service Advice Review, 1989 WL 1172952 (Mar. 15, 1989) (reply
to request for reconsideration) (``[I]ntent is irrelevant\dots .
[E]very distribution made with respect to a shareholder's stock is
taxable as ordinary income, capital gain, or not at all pursuant to
section 301(c) dependent upon the corporation's earnings and profits
and the shareholder's stock basis. The determination is computational
and not dependent upon intent'').

^8 Thus in the period between this Court's decisions in
\emph{Commissioner} v. \emph{Wilcox,} 327 U.~S. 404 (1946) (holding
embezzled funds to be nontaxable to the embezzler), and \emph{James} v.
\emph{United States,} 366 U.~S. 213 (1961) (overruling \emph{Wilcox,} holding
embezzled funds to be taxable income), the Government routinely
argued that diverted funds were ``constructive distributions,''
taxable to the recipient as dividends. See generally Gardner 237
(``While \emph{Wilcox} was good law, the safest way to insure that both
the corporation and the shareholder would be taxed on their respective
gain from the diverted funds was to label them dividends''); 4 Bittker
\& Lokken ¶ 92.2(7), at 92--23, n. 37.

  When the \emph{Miller} court went the other way, needless to say, it
could claim no textual hook for the contemporaneous intent requirement,
but argued for it as the way to avoid two supposed anomalies. First,
the court thought that applying \S\S~301 and 316(a) in criminal
cases unnecessarily emphasizes the exact amount of deficiency while
``completely ignor[ing] one essential element of the crime charged:
the willful intent to evade taxes~.~.~.~.'' 545 F. 2d, at
1214. But there is an analytical mistake here. Willfulness is an
element of the crimes charged because the substantive provisions
defining tax evasion and filing a false return expressly require it,
see \S~7201 (``[a]ny person who willfully attempts~.~.~.~'');
\S~7206(1) (``[w]illfully makes and subscribes~.~.~.~'').
The element of willfulness is addressed at trial by requiring the
Government to prove it. Nothing in \S\S~301 and 316(a) as written (that
is, without an intent requirement) relieves the Government of this
burden of proving willfulness or impedes it from doing \newpage  so if
evidence of willfulness is there. Those two sections as written simply
address a different element of criminal evasion, the existence of a
tax deficiency, and both deficiency and willfulness can be addressed
straightforwardly (in jury instructions or bench findings) without
tacking an intent requirement onto the rule distinguishing dividends
from capital returns.

  Second, the \emph{Miller} court worried that if a defendant could claim
capital treatment without showing a corresponding and contemporaneous
intent,

    ``[a] taxpayer who diverted funds from his close corporation when
    it was in the midst of a financial difficulty and had no earnings
    and profits would be immune from punishment (to the extent of his
    basis in the stock) for failure to report such sums as income; while
    that very same taxpayer would be convicted if the corporation had
    experienced a successful year and had earnings and profits.'' 545
    F. 2d, at 1214.

``Such a result,'' said the court, ``would constitute an extreme
example of form over substance.'' \emph{Ibid.} The Circuit thus
assumed that a taxpayer like Boulware could be convicted of evasion with
no showing of deficiency from an unreported dividend or capital gain.

  But the acquittal that the author of \emph{Miller} called form trumping
substance would in fact result from the Government's failure to prove
an element of the crime. There is no criminal tax evasion without
a tax deficiency, see \emph{supra,} at 424,\footnotemark[9] and there is no
deficiency owing to a distribution (re\newpage ceived with respect to a
corporation's stock) if a corporation has no earnings and profits
and the value distributed does not exceed the taxpayer-shareholder's
basis for his stock. Thus the fact that a shareholder distributee
of a successful corporation may have different tax liability from a
shareholder of a corporation without earnings and profits merely follows
from the way \S\S~301 and 316(a) are written (to distinguish dividend
from capital return), and from the requirement of tax deficiency for a
\S~7201 crime. Without the deficiency there is nothing but some act
expressing the will to evade, and, under \S~7201, acting on ``bad
intentions, alone, [is] not punishable,'' \emph{United States} v.
\emph{D'Agostino,} 145 F. 3d 69, 73 (CA2 1998).

^9 Boulware was also convicted of violating \S~7206(1), which makes
it a felony ``[w]illfully [to] mak[e] and subscrib[e] any return,
statement, or other document, which contains or is verified by a written
declaration that it is made under the penalties of perjury, and which
[the taxpayer] does not believe to be true and correct as to every
material matter.'' He argues that if the Ninth Circuit erred, its
error calls into question not only his \S~7201 conviction, but his
\S~7206(1) conviction as well. Brief for Petitioner 15--16.
Although the Courts of Appeals are unanimous in holding that \S~7206(1)
``does not require the prosecution to prove the existence of \newpage  a
tax deficiency,'' \emph{United States} v. \emph{Tarwater,} 308 F. 3d 494,
504 (CA6 2002); see also \emph{United States} v. \emph{Peters,} 153 F. 3d 445,
461 (CA7 1998) (collecting cases), it is arguable that ``the nature
and character of the funds received can be critical in determining
whether\dots \S~7206(1) has been violated, [even if] proof of a tax
deficiency is unnecessary,'' 1 I. Comisky, L. Feld, \& S. Harris, Tax
Fraud \& Evasion ¶ 2.03[5], p. 21 (2007); see also Brief for Petitioner
15--16. The Government does not argue that Boulware's \S\S~7201
and 7206(1) convictions should be treated differently at this stage
of the proceedings, however, and we will accede to the Government's
working assumption here that the \S\S~7201 and 7206(1) convictions
stand or fall together.

  It is neither here nor there whether the \emph{Miller} court was
justified in thinking it would improve things to convict more of the
evasively inclined by dropping the deficiency requirement and finding
some other device to exempt returns of capital.\footnotemark[10] Even if there were
compelling reasons to extend \newpage  \S~7201 to cases in which no taxes
are owed, it bears repeating that ``[t]he spirit of the doctrine which
denies to the federal judiciary power to create crimes forthrightly
admonishes that we should not enlarge the reach of enacted crimes by
constituting them from anything less than the incriminating components
contemplated by the words used in the statute,'' \emph{Morissette} v.
\emph{United States,} 342 U.~S. 246, 263 (1952) (opinion for the Court by
Jackson, J.) (footnote omitted). If \S~301, \S~316(a), or \S~7201
could stand amending, Congress will have to do the rewriting.

^10 ``A better [method of exempting returns of capital from taxation]
could no doubt be devised.'' 4 Bittker \& Lokken ¶ 92.1.1, at 92--3;
see \emph{ibid.} (suggesting, for example, that ``all receipts from a
corporation could be treated as taxable income, and a correction for
any resulting overtaxation could be made in computing gain or loss when
stock is sold, exchanged, or becomes worthless''); see also Andrews,
``Out of its Earnings and Profits'': Some Reflections on the Taxation
of Dividends, 69 Harv. L. Rev. 1403, 1439 (1956) (criticizing the
earnings and profits concept ``[a]s a device for separating income from
return of capital,'' and suggesting that ``[d]istributions which ought
to be treated as return of capital [could] be brought within the concept
of a partial liquidation by special provision'').

\section{C}

  Not only is \emph{Miller} devoid of the support claimed for it, but
it suffers the demerit of some anomalies of its own. First and
most obviously, \S\S~301 and 316 are odd stalks for grafting a
contemporaneous intent requirement, given the fact that the correct
application of their rules will often become known only at the end
of the corporation's tax year, regardless of the shareholder's
or corporation's understanding months earlier when a particular
distribution may have been made. Section 316(a)(2) conditions treating
a distribution as a constructive dividend by reference to earnings and
profits, and earnings and profits are to be ``computed as of the close
of the taxable year\dots without regard to the amount of the earnings
and profits at the time the distribution was made.'' A corporation may
make a deliberate distribution to a shareholder, with everyone expecting
a profitable year and considering the distribution to be a dividend,
only to have the shareholder end up liable for no tax if the company
closes out its tax year in the red (so long as the shareholder's basis
covers the distribution); when such facts are clear at the time the
reporting forms and returns are filed,\footnotemark[11] the shareholder \newpage 
does not violate \S~7201 by paying no tax on the moneys received,
intent being beside the point. And since intent to make a distribution
a taxable one cannot control, it would be odd to condition nontaxable
return-of-capital treatment on contemporaneous intent, when the statute
says nothing about intent at all.

^11 Sometimes these facts are not clear, and in certain circumstances
a corporation may be required to assume it is profitable. For example,
the instructions to IRS Form 1099--DIV provide that when a corporation
is unsure whether it has sufficient earnings and profits at the end
of the taxable year to cover a distribution to shareholders, ``the
entire payment \newpage  must be reported as a dividend.'' See
http://www.irs.gov/pub/irs-pdf/i1099div.pdf (as visited Feb. 15, 2008,
and available in Clerk of Court's case file).

  The intent interpretation is strange for another reason, too (a reason
in some tension with the Ninth Circuit's assumption that an unreported
distribution without contemporaneous intent to return capital will
support a conviction for evasion). The text of \S~301(a) ostensibly
provides for all variations of tax treatment of distributions received
with respect to a corporation's stock unless a separate provision of
the Code requires otherwise. Yet \emph{Miller} effectively converts the
section into one of merely partial coverage, with the result of leaving
one class of distributions in a tax status limbo in criminal cases.
That is, while \S~301(a) expressly provides that distributions made by
a corporation to a shareholder with respect to its stock ``shall be
treated in the manner provided in [\S~301(c)],'' under \emph{Miller,}
a distribution from a corporation without earnings and profits would
fail to be a return of capital for lack of contemporaneous intent to
treat it that way; but to the extent that distribution did not exceed
the taxpayer's basis for the stock (and thus become a capital gain),
\S~301(a) would leave the distribution unaccounted for.

  It is no answer to say that \S~61(a) of the Code would step in where
\S~301(a) has been pushed out. Although \S~61(a) defines gross income,
``[e]xcept as otherwise provided,'' as ``all income from whatever
source derived,'' the plain text of \S~301(a) does provide otherwise
for distributions made with respect to stock. So using \S~61(a) as a
stopgap would only sanction yet another eccentricity: \S~301(a) would
be held not to cover what its text says it ``shall'' (the class of
distribu\newpage  tions made with respect to stock for which no other more
specific provision is made), while \S~61(a) would need to be applied
to what by its terms it should not be (a receipt of funds for which tax
treatment is ``otherwise provided'' in \S~301(a)).


  The implausibility of a statutory reading that either creates
a tax limbo or forces resort to an atextual stopgap is all the
clearer from the Ninth Circuit's discussion in this case of its
own understanding of the consequences of \emph{Miller}'s rule: the
court openly acknowledged that ``imposing an intent requirement
creates a disconnect between civil and criminal liability,'' 470
F. 3d, at 934. In construing distribution rules that draw no
distinction in terms of criminal or civil consequences, the disparity
of treatment assumed by the Court of Appeals counts heavily against its
contemporaneous intent construction (quite apart from the Circuit's
understanding that its interpretation entails criminal liability for
evasion without any showing of a tax deficiency).

  \emph{Miller} erred in requiring a contemporaneous intent to treat the
receipt of corporate funds as a return of capital, and the judgment of
the Court of Appeals here, relying on \emph{Miller,} is likewise erroneous.

\section{IV}

  The Government has raised nothing that calls for affirmance in the
face of the Court of Appeals's reliance on \emph{Miller.} The United
States does not defend differential treatment of criminal and civil
cases, see Brief for United States 24, and it thus stops short of
fully defending the Ninth Circuit's treatment. The Government's
argument, instead, is that we should affirm under the rule that before
any distribution may be treated as a return of capital (or, by a
parity of reasoning, a dividend), it must first be distributed to
the shareholder ``with respect to\dots stock.'' \emph{Id.,} at
19 (internal quotation marks omitted). The taxpayer's intent,
the Government says, may be relevant to this limiting condition,
and Boul\newpage ware never expressly claimed any such intent. See
\emph{ibid.} (``[I]ntent is\dots relevant to whether a payment is a
‘distribution\dots with respect to [a corporation's] stock'
''); but see Tr. of Oral Arg. 44 (``[J]ust to be clear, the Government
is arguing for an objective test here'').

  The Government is of course correct that ``with respect to . .
. stock'' is a limiting condition in \S~301(a). See \emph{supra,}
at 424--425.\footnotemark[12] As the Government variously says, it requires
that ``the distribution of property by the corporation be made to
a shareholder because of his ownership of its stock,'' Brief for
United States 16; and that `` ‘an amount paid by a corporation to
a shareholder [be] paid to the shareholder in his capacity as such,'
'' \emph{ibid.} (quoting 26 CFR \S~1.301--1(c) (2007); emphasis
deleted).

  This, however, is not the time or place to home in on the ``with
respect to\dots stock'' condition. Facts with a bearing on it
may range from the distribution of stock ownership\footnotemark[13] \newpage 
to conditions of corporate employment (whether, for example, a
shareholder's efforts on behalf of a corporation amount to a good
reason to treat a payment of property as salary). The facts in this case
have yet to be raked over with the stock ownership condition in mind,
since \emph{Miller} seems to have pretermitted a full consideration of the
defensive proffer, and if consideration is to be given to that condition
now, the canvas of evidence and Boulware's proffer should be made by a
court familiar with the whole evidentiary record.\footnotemark[14]

^12 Another limiting condition is that the diversion of funds must be a
``distribution'' in the first place (regardless of the ``with respect
to stock'' limitation), see \emph{supra,} at 429--430, though the
Government is content to assume that \S~301(a)'s ``distribution''
language is capacious enough to cover the diversions involved here, and
that if Boulware bears the burden of production in going forward with
the defense that the funds he received constituted a ``distribution''
within the meaning of \S~301(a), see n. 14, \emph{infra,} that burden
has been met. Nor does the Government dispute that Boulware offered
sufficient evidence of his basis and HIE's lack of earnings and
profits. See Brief for United States 34, n. 11.

^13 See, \emph{e. g., Truesdell} v. \emph{Commissioner,} IRS Non Docketed
Service Advice Review, 1989 WL 1172952 (Mar. 15, 1989) (``We believe
a corporation and its shareholders have a common objective---to
earn a profit for the corporation to pass onto its shareholders.
Especially where the corporation is wholly owned by one shareholder,
the corporation becomes the alter ego of the shareholder in his profit
making capacity~.~.~.~. [B]y passing corporate funds to himself
as shareholder, a sole shareholder is acting in pursuit of these
common objectives''). We note, however, that although Boulware
was not a sole shareholder, the Tax Court has taken it as ``well
settled that a distribution of corporate earnings to shareholders
may constitute a dividend,'' and so a return of capital as well,
``notwithstanding that it is not in proportion to stockholdings.''
\emph{Dellinger} v. \emph{Commissioner,} \newpage  32 T. C. 1178, 1183
(1959); see \emph{ibid.} (noting that because other stockholders did
not complain when a taxpayer received unequal property, ``under the
circumstances they must be deemed to have ratified the distribution'');
see also \emph{Crowley} v. \emph{Commissioner,} 962 F. 2d 1077 (CA1 1992);
\emph{Lengsfield} v. \emph{Commissioner,} 241 F. 2d 508 (CA5 1957);
\emph{Baird} v. \emph{Commissioner,} 25T. C. 387 (1955); \emph{Thielking} v.
\emph{Commissioner,} 53 TCM 746 (1987), ¶ 87,227 P--H Memo TC.

  As a more specific version of its ``with respect to\dots stock''
position, the Government says that the diversions of corporate funds to
Boulware were in fact unlawful, see Brief for United States 34--37;
see also n. 5, \emph{supra,} and it argues that \S\S~301 and 316(a)
are inapplicable to illegal transfers, see Brief for United States
34--37; see also \emph{D'Agostino,} 145 F. 3d, at 73 (``[T]he ‘no
earnings and profits, no income' rule would not necessarily apply in
a case of \emph{unlawful} diversion, such \newpage  as embezzlement, theft,
a violation of corporate law, or an attempt to defraud third party
creditors'' (emphasis in original)); see also n. 8, \emph{supra}. The
Government goes so far as to claim that ``[t]he only rational basis
for the jury's judgment was a conclusion that [Boulware] unlawfully
diverted the funds.'' Brief for United States 37.

^14 Boulware does not dispute that he bears the burden of producing
some evidence to support his return-of-capital theory, including
evidence that the corporation lacked earnings and profits and that he
had sufficient basis in his stock to cover the distribution. See Tr.
of Oral Arg. 53. He instead argues that, as to the ``with respect
to\dots stock'' requirement, it suffices to show ``[t]hat he
is a stockholder, and that he did not receive this money in any
nonstockholder capacity.'' \emph{Id.,} at 57. The Government, for
its part, on the authority of \emph{Holland} v. \emph{United States,} 348
U. S. 121 (1954), and \emph{Bok,} 156 F. 3d, at 163--164, argues
that Boulware must offer more evidence than that. We express no view
on that issue here, just as we decline to consider the more general
question whether the Second Circuit's rule in \emph{Bok,} which places
on the criminal defendant the burden to produce evidence in support of
a return-of-capital theory, is authorized by \emph{Holland} and consistent
with \emph{Sandstrom} v. \emph{Montana,} 442 U.~S. 510 (1979), and
related cases.

  But we decline to take up the question whether an unlawful diversion
may ever be deemed a ``distribution\dots with respect to [a
corporation's] stock,'' a question which was not considered by
the Ninth Circuit. We do, however, reject the Government's current
characterization of the jury verdict in Boulware's case. True, the
jurors were not moved by Boulware's suggestion that the diversions
were corporate advances or loans, or that he was using the funds for
corporate purposes. But the jury was not asked, and cannot be said
to have answered, whether Boulware breached any fiduciary duty as a
controlling shareholder, unlawfully diverted corporate funds to defraud
his wife, or embezzled HIE's funds outright.

\section{V}

  Sections 301 and 316(a) govern the tax consequences of constructive
distributions made by a corporation to a shareholder with respect to
its stock. A defendant in a criminal tax case does not need to show a
contemporaneous intent to treat diversions as returns of capital before
relying on those sections to demonstrate no taxes are owed. The judgment
of the Court of Appeals is vacated, and the case is remanded for further
proceedings consistent with this opinion.

\begin{flushright}\emph{It is so ordered.}\end{flushright}
